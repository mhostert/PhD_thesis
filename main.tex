\documentclass[openany,twoside,frontopenright,chaprunninghead]{ip3thesis}
\input{preamble}
%%% commands %%%%%%%%%%%%%%%%%%%%%%%%%%%%%%%%%%%%%%%%%%%%%%%
\newcommand{\marrow}[5]{%
    \fmfcmd{style_def marrow#1
    expr p = drawarrow subpath (1/4, 3/4) of p shifted 6 #2 withpen pencircle scaled 0.4;
    label.#3(btex #4 etex, point 0.5 of p shifted 6 #2);
    enddef;}
    \fmf{marrow#1,tension=0}{#5}}

\newcommand{\refeq}[1]{Eq.~(\ref{#1})}
\newcommand{\refeqs}[2]{Eqs.~(\ref{#1})~and~(\ref{#2})}
\newcommand{\refeqss}[3]{Eqs.~(\ref{#1}), (\ref{#2})~and~(\ref{#3})}
\newcommand{\reffig}[1]{Fig.~\ref{#1}}
\newcommand{\reffigs}[2]{Figs.~\ref{#1}~and~\ref{#2}}
\newcommand{\refsec}[1]{Section~\ref{#1}}
\newcommand{\refapp}[1]{Appendix~\ref{#1}}
\newcommand{\reftab}[1]{Table~\ref{#1}}
\newcommand{\refref}[1]{Ref.~\cite{#1}}
\newcommand{\refrefs}[2]{Refs.~\cite{#1}~and~\cite{#2}}

\def\abelian{abelian}
\def\nonabelian{non-abelian}
\def\lagrangian{lagrangian}
\def\eg{\emph{e.g.}}
\def\ie{\emph{i.e.}}
\def\aka{\emph{a.k.a.}}

\newcommand{\subsubsubsection}[1]{~\\[-2mm]\noindent {\textbf{#1}}\\[-5mm]\nopagebreak[0]}

\newcommand{\Dslash}{\ensuremath D\hspace{-0.24cm}\raisebox{1pt}{/}\hspace{0.05cm}}
\newcommand{\Xslash}{\ensuremath X\hspace{-0.26cm}\raisebox{1pt}{/}\hspace{0.06cm}}
\newcommand{\Wslash}{\ensuremath W\hspace{-0.3cm}\raisebox{1pt}{/}\hspace{0.06cm}}
\newcommand{\dslash}{\ensuremath \partial\hspace{-0.18cm}\raisebox{1pt}{/}\hspace{0.04cm}}
\newcommand{\gp}{\ensuremath g^\prime}
\newcommand{\nutrident}[4]{$\nu_{#1} \to \nu_{#2} {#3}^+ {#4}^-$}
\newcommand{\nubartrident}[4]{$\overline{\nu}_{#1} \to \overline{\nu}_{#2} {#3}^- {#4}^+$}
\newcommand*{\pbar}[1]{\accentset{(-)}{#1}}

%%%%%%% A few editorial macros. %%%%%%%

\definecolor{light-gray}{gray}{0.65}
\newcommand{\filler}[1]{{\color{light-gray}{#1}}}

\newcounter{CommentCount}
\setcounter{CommentCount}{1}

\newcommand{\marcom}[2]{\textsuperscript{\textcolor{#1}{\theCommentCount}}\marginpar{\textsuperscript{\textcolor{#1}{\theCommentCount}}\textcolor{#1}{{\small#1: #2}}}\stepcounter{CommentCount}}

\newcommand{\newtext}[2]{\textcolor{#1}{\ul{#2}}}

%%%%%%%%%%%%%%%%%%%%%%% Physics names

\def\minerva{MINER$\nu$A\ }
\def\sw{s_\textsc{w}}

\def\mphi{m_{\Phi}}
\def\mhiggs{m_{H}}
\def\lphi{\lambda_{\Phi}}
\def\lhiggs{\lambda_{H}}


\usepackage[toc,page]{appendix}


\begin{document}
\title{Hidden neutrino physics}
\subtitle{Rare processes, new forces, and dark sectors}
\author{Matheus Hostert}
\researchgroup{Institute for Particle Physics Phenomenology}
\maketitlepage*

\begin{abstract}
%
	This is some abstract about this thesis.
%
\end{abstract}


\begin{dedication*}
%
B\'arbara, Humberto and Luciana.
%
\end{dedication*}

\disableprotrusion
\tableofcontents*
\listoffigures
\listoftables
\enableprotrusion

\begin{declaration*}
%
	The work in this thesis is based on research carried out in the Department of
	Physics at Durham University. No part of this thesis has been
	submitted elsewhere for any degree or qualification.
%
\end{declaration*}

\begin{acknowledgements*}

	I would like to thank my supervisor Silvia Pascoli for her guidance and for her vision as a supervisor. I am lucky to have been your student and to have had the opportunities you gave me. I am also greatly thankful to Peter Ballett, without whom this thesis would have been but a boring collection of pages. Thank you for your patience and for sharing a bit of your crystal clear thinking with me.
	
	\vspace{2ex}
	Carlos Arg\"uelles, Pedro Machado and Renata Zukanovich Funchal, thank you for the inspiration and support. My collaborators Yuber Perez Gonzalez, Zahra Tabrizi and Yu-Dai Tsai, thank you for all the hard work that made this thesis possible.

	\vspace{2ex}
	Looking further back, I also want to thank D\'ebora Peres Menezes and Marcus Benghi Pinto for introducing me to scientific research so early in my career and understanding the importance of doing so. Nuclear physics and symmetry ideas that at the time seemed so foreign to me have reappered in my own research several times.
	
	\vspace{2ex}
	To my family, thank you for supporting my decisions and for being unconditional allies. I would be nothing without the education of my mother and the wisdom of my father. B\^e, you are simply the \emph{best}.
	
	\vspace{2ex}
	Madeline, your support and sense of humour have been truly out of this world. Thank you for being such a great partner and for filling life with laughter.
	
\end{acknowledgements*}

%\dedicationtext{This thesis is dedicated to}


\cleardoublepage

\begin{epigraph*}
% 
% \begin{center}
% \centering 
% \includegraphics[width=1\textwidth]{img/Jabberwocky.jpg}
% 
% 	%"It seems very pretty," she said when she had finished it, "but it's rather hard to understand!" (You see she didn't like to confess, even to herself, that she couldn't make it out at all.) "Somehow it seems to fill my head with ideas—only I don't exactly know what they are!
% 	"It seems very pretty, [...] somehow it seems to fill my head with ideas—only I don't exactly know what they are!
% 	\source{Through the Looking-Glass, and What Alice Found There}{Lewis Carroll}
% \end{center}
\end{epigraph*} 


\chapter*{Preface}

These are exciting times in high-energy physics. The Standard Model (SM), our most powerful and well-tested theory of particles and their interactions, triumphs across the experimental landscape and proves to be much more robust than anticipated. Yet, as we will see throughout this thesis, once confronted with some of the simplest questions about the Universe, it provides unsatisfactory answers and, most surprisingly, little theoretical guidance on what may lie beyond. While there is no guarantee that the questions we ask are indeed ``good'' ones, to many, the limitations of the SM attest to the need of a new paradigm in particle physics. Whatever it may be, it makes the series of negative results in particle physics all the more thrilling. Uncertain moments like the one we live are frequent in the history of physics. For many times we saw established theoretical expectations and increasingly fine-tuned models making way for elegant theories like Special Relativity, for new particles such as the neutrino and for new ideas like spontaneous symmetry breaking (SSB).

In practice, of course, such grandiose endeavors are reduced to much less noble but no less important efforts. The frequency of negative results and the need to over-constrain our models make the search for new physics a true exercise in patience. Nevertheless, it is in a persistent and curious spirit that this thesis stands. The common theme of the work developed here is that of \emph{hidden neutrino physics}, both in the sense of rare neutrino processes and of hidden sectors beyond the SM. Hidden, dark or secret are all terms associated with new particles that hide not at high energies with $\mathscr{O}(1)$ couplings, but rather at lower scales with couplings to the SM that are much smaller than one. With the exciting theoretical and experimental prospects in neutrino and dark matter physics, we believe that it is all the more timely to think harder about what neutrinos, hidden particles of their own, may teach us about this type of hidden physics. 

We start this thesis by reminding ourselves of the main features and limitations of the SM in this chapter, and delving into unique aspects of neutrino physics in the next. This way, we hope to set the scene and motivate the direction we pursue in the following chapters. Indeed, the processes of neutrino trident production and neutrino-electron scattering we discuss in Chapter 3 are a testimony of the unique properties of neutrinos and the precision achieved in experimental neutrino physics. As an application of these measurements, we will see in Chapter 4 how such signatures probe new gauge interactions much weaker than those in the SM ($g \lesssim 10^{-3}$). The fact that neutrinos are the only neutral fermions in the SM also has interesting consequences for mass generation and naturally connects them to neutral hidden sectors. We explore this connection with a specific model at hand in Chapter 5, scrutinizing its phenomenology at short-baselines in Chapter 6. Finally, in Chapter 7 we look towards the far future and highlight the potential of an entry-level neutrino factory in neutrino physics.

\chapter{Introduction}
\begin{flushright}
	``It really tied the room together.''
	\\--- Jeff Lebowski
\end{flushright}
%%%%%%%%%%%%%%%%%%%%%%%%%%%%%%%%%%%%%%%%%%%%%%%%%%%%
\graphicspath{{}{introduction/}{Diagrams/}}

\section{The Standard Model}

The Standard Model (SM) of particle physics is a Yang-Mills theory~\cite{Yang:1954ek} of strong, weak and electromagnetic (EM) particle interactions based on an $SU(3) \times SU(2) \times U(1)$ local gauge symmetry. The first remarkable aspect of the theory is in the fact that it relies on the same idea that explains Maxwell's equations, the principle of gauge invariance. In this way, it is hard to pin down the official conception of the SM, although widely associated with the works of Sheldon L. Glashow~\cite{Glashow:1961tr}, Steven Weinberg~\cite{Weinberg:1967tq} and Abdus Salam~\cite{Salam:1968rm}. Unconcerned with quarks and the strong force, they proposed a spontaneously broken $SU(2) \times U(1)$ local gauge symmetry for leptons, which already reflected most of what we know about the electroweak (EW) interactions nowadays. In fact, the spontaneous broken symmetry that was used already predicted the existence of a charged massive vector boson, the $W^\pm$, a neutral massive vector boson, the $Z$, and of a massless generator of the unbroken $U(1)_{\rm EM}$ group, the photon $\gamma$. Beyond unifying the weak and EM forces, the breaking through the Higgs mechanism~\cite{Higgs:1964ia,Higgs:1964pj} implied that an additional scalar particle, the Higgs boson $H$, had to exist. This last prediction was experimentally validated after the discovery of a neutral scalar boson at the LHC in 2012~\cite{Chatrchyan:2012xdj,Aad:2012tfa}, the last SM particle to be experimentally observed.

The strong force had a much richer and more turbulent history. The quark model, developed by Murray Gell-Mann and George Zweig~\cite{Zweig:1981pd,GellMann:1964nj} in 1964, had great success in explaining the growing number of hadronic resonances found by experiments. However, it was not until asymptotic freedom was discovered in non-Abelian gauge theories~\cite{Gross:1973id,Politzer:1973fx} that quantum chromodynamics (QCD) was really born. QCD is an $SU(3)$ local gauge theory describing the interaction of quarks and gluons, and is vastly different from any other theory we will encounter in this thesis. Its uniqueness is best exemplified through color confinement, the property that colored particles must always be present in bound colorless states, called hadrons. For QCD, confinement is guaranteed below the scale $\Lambda_{\rm QCD} \approx 250$ MeV, below which strong processes are non-perturbative. This is to be contrasted with asymptotic freedom, where the strong interactions between quarks and gluons become asymptotically weaker at higher energies. The presence of new degrees of freedom other than quarks and gluons at low energies, namely the hadrons, is a clear evidence of a phase transition and makes QCD a unique topic within the SM. At times we will refer to known results in this theory, but it usually has little bearings on electroweak physics.

\subsection{Fields and symmetries}

We now set out for a more precise definition of the SM field content, discussing some details of local gauge invariance. All fermion fields in the SM are Weyl fields of either definite left-handed (LH) or right-handed (RH) chirality. An equivalent statement is that SM fields are eigenvectors of $\gamma_5$: $\gamma_5 \psi_R = \psi_R$ for RH, and $\gamma_5 \psi_L = - \psi_L$ for LH fields. This is an important feature that allows us to work with 2 component Weyl spinors and makes explicitly manifest the chiral nature of weak interactions. The LH field content and their representation under the different gauge groups is shown in \reftab{tab:SMcharges}. Note that only LH particles transform non-trivially under $SU(2)_L$. Also shown is the Higgs field $H$, a complex scalar field, doublet under $SU(2)$. As we will see in the next section, $H$ is responsible for the breaking of $SU(2)_L \times U(1)_Y \to U(1)_{\rm EM}$.
%
\renewcommand{\arraystretch}{1.4}%
\begin{table}[t]
 \begin{tabular}{lccccccccccc}
 \hline
    & $Q_L^\alpha$& $L^\alpha$ & $\overline{u_R^\alpha\vphantom{d}}$ & $\overline{d_R^\alpha}$ & $\overline{e_R^\alpha\vphantom{d}}$ & &$H$ & & $G$ & $W$ & $B$\\
    \hline
  SU$(3)_c$ & $\bm{3}$ & $\bm{1}$& $\overline{\bm{3}}$ & $\overline{\bm{3}}$ & $\bm{1}$ & & $\bm{1}$ & & $\bm{8}$ & $\bm{1}$ & $\bm{1}$ \\
  SU$(2)_L$& $\bm{2}$ & $\bm{2}$ & $\bm{1}$ & $\bm{1}$& $\bm{1}$& & $\bm{2}$ & & $\bm{1}$ & $\bm{3}$ & $\bm{1}$ \\
  U$(1)_Y$ & $1/3$ & $-1$ & $-4/3$ & $2/3$ & $2$ & & $1$ & & $0$ & $0$ & $1$ \\
  \hline
 \end{tabular}
 \caption[SM field content.]{The representation of the left-handed Weyl fields, the complex scalar and gauge bosons under each gauge group of the SM. For $U(1)_Y$, the charge is shown instead. All fermions carry a flavour index $\alpha = e, \mu$ or $\tau$.\label{tab:SMcharges}}
\end{table}
\renewcommand{\arraystretch}{1.0}%
%
From the observed EM charges $Q_{\rm EM}$, the $SU(2)_L$ isospin $T_3$, and by virtue of the Gell-Mann-Nishijima formula~\cite{Nakano:1953zz,Gell-Mann:1956iqa}
%
\begin{equation}
 Q_{\rm EM} = T_3 + \frac{Y}{2},
\end{equation}
%
the hypercharge $Y$ of each SM field is fixed. The \emph{local} gauge transformation of the matter fields are given by
\begin{equation}
\psi  \to \exp{i g \theta^a(x) T^a } \psi,
\end{equation}
where $g$ is the gauge coupling constant, $a$ counts the number of generators $T^a$, and $\theta^a(x)$ are arbitrary parameters that depend on space-time coordinates $x^\mu$. To achieve local gauge invariance, we require the following gauge fields associated with each group:
%
\begin{equation}
 SU(3)_C: \{G_1 (x), \cdots, G_8 (x)\}, \quad SU(2)_L:  \{W_1(x), W_2(x), W_3(x)\}, \quad U(1)_Y: B(x),
\end{equation}
%
corresponding to the eight gluons, the $SU(2)$ gauge fields and the hypercharge field. Note that the number of gauge fields matches the number of generators in each group, \eg\ for $SU(N)$ there are $N^2 -1$ generators. For the original $SU(2)\times U(1)$ theory, this implied that in addition to the charged gauge fields, which explained Fermi's theory for beta decays, and the observed massless photon, there must have been an additional neutral gauge field corresponding to some linear combination of $W^3$ of $SU(2)_L$ and $B$ of $U(1)_Y$. This striking prediction was in fact first confirmed by Gargamelle through the observation of accelerator neutrinos scattering into final states with no charged leptons~\cite{Hasert:1973ff}. 

The generators of a given symmetry group equipped with commutators form a Lie Algebra, obeying $[T^a, T^b] = i f^{abc} T_c$, with $f^{abc}$ being the group structure constant. In the special case $f^{abc} = 0$, the generators commute and the group is said to be Abelian, like in the case of $U(1)_Y$. Otherwise, the group is non-Abelian and the theory displays a much richer underlying dynamics. The SM is a non-Abelian theory, since its symmetry group contains direct products of two $SU(N)$ groups with $N > 1$. We now illustrate how to build a gauge-invariant Yang-Mills Lagrangian, like that of the SM. Take an $a$-dimensional Yang-Mills theory and define $\bm{\theta} = T^a \theta^a$ and $U = e^{i g \bm{\theta}}$. We can now perform gauge transformations on the relevant matter fields $\psi$, gauge fields $\bm{A}_\mu = T_a A_\mu(x)^a$ and derivatives of matter fields as follows
%
\begin{equation}
\psi \to U \psi, \quad \bm{A}_\mu \to U \bm{A}_\mu U^{-1} - \frac{i}{g} (\partial_\mu U) U^{-1}, \quad \partial_\mu \psi \to U \partial_\mu \psi +  \psi (\partial_\mu U).
\end{equation}
%
As we can see, the last term is not invariant due to the local character of the gauge transformations. To preserve gauge invariance, a covariant derivative, transforming as $D_\mu \to U D_\mu U^{-1}$, now replaces the ordinary derivative. It is defined as 
\begin{equation}
D_\mu = \partial_\mu + ig \bm{A},\quad \text{such that} \quad D_\mu\psi \to U D_\mu\psi, \,\,\implies\,\,\overline{\psi} i\slashed{D}\psi \to \overline{\psi} i \slashed{D} \psi. 
\end{equation}
%
The invariant term above is the fermion kinetic term. Beyond fermion propagation, it is the main way to describe fermion-gauge interactions in the SM. In particular, the full covariant derivative in the SM is given by
%
\begin{equation}
 D_\mu = \partial_\mu + ig \,W_\mu^a\tau_a + i\frac{Y}{2} g^\prime \,B_\mu + i\frac{g_s}{2} \,G_\mu^b\lambda_b,
\end{equation}
%
where $\tau_a = \sigma_a/2$ are the generators built from Pauli matrices acting on the doublets of $SU(2)_L$, and $\lambda_b$ the generators built from the Gell-Mann matrices acting on the triplet representations of $SU(3)_c$. This also fixes the notation for the gauge couplings in the SM. Finally, the gauge invariant kinetic terms for the gauge bosons are 
%
\begin{equation}
\mathscr{L}_{\rm gauge} = -\frac{1}{4} G^{a}_{\mu\nu} G_a^{\mu\nu} - \frac{1}{4} W^{a}_{\mu\nu} W_a^{\mu\nu} -\frac{1}{4} B_{\mu\nu} B^{\mu\nu},
\end{equation}
%
where $F^a_{\mu \nu} = \partial_\mu F^a_\nu - \partial_\nu F^a_{\mu} - g_F f^{abc} F_{b\,\mu} F_{c\,\nu}$ with $g_F$ the relevant gauge coupling. The kinetic term in Abelian theories concern only the propagation of gauge bosons, however, for non-Abelian groups the term proportional to $g_F$ in $F^a_{\mu \nu}$ introduces interactions among the gauge bosons proportional to $g$ and $g^2$. Therefore, a non-Abelian theory is already an interacting theory without the addition of any matter fields.

\subsection{Spontaneous symmetry breaking}

So far we have only discussed the gauge and fermionic content of the SM. The scalar sector is, in fact, quite special. The only scalar particle, the Higgs boson, is responsible for spontaneously breaking $SU(2)_L\times U(1)_Y$ to $U(1)_{\rm EM}$ after it acquires a non-zero vacuum expectation value (vev). This introduces a mass scale in the theory which, apart from dimensionless couplings, sets the scale of EW physics. Note that because it is a scalar particle, a non-zero vev does not violate the symmetries of space-time, namely Lorentz invariance. The Higgs is a complex scalar field and a doublet under  $SU(2)_L$, and so we can write
%
\begin{equation}
  H =  \frac{1}{\sqrt{2}} \left( \begin{matrix}  G_1^+ + i G_2^+ \\  h^0 + i G_3^0 \end{matrix} \right) =  \frac{e^{i G_a \tau^a}}{\sqrt{2}} \left( \begin{matrix} 0 \\  h \end{matrix} \right).
\end{equation}
%
The Higgs Lagrangian reads
%
\begin{equation}
\mathscr{L}_{\rm Higgs} \supset \left( D^\mu H \right)^\dagger \left( D_\mu H \right) - V(H), \quad {\rm with} \quad V(H) = \mu^2 H^\dagger H + \lambda \left( H^\dagger H \right)^2,	
\end{equation}
%
where $\mu^2$ has mass dimension 2, being the only dimensionful parameter in the SM. If $\mu^2 < 0$, minimizing the potential $V(H)$ requires $\bra{0} H \ket{0} = \left( 0, \,\, v/\sqrt{2} \right)^T$, where $v^2 = - \mu^2 /\lambda$ is the vev chosen to lie in the real and neutral direction. We now can then expand around the true vacuum of the theory by redefining the fields $G_a \to G_a/v$ and $h \to h + v$. At this point, a rewriting of the potential reveals the mass and interactions of every component of the scalar doublet. Note, however, that it contains no mass terms for $G_1$, $G_2$ and $G_3$. These are the Goldstone bosons of the theory, and although they are massless, they do possess interactions with the scalar and gauge boson fields. One way to understand their role is to perform an $SU(2)_L \times U(1)_Y$ gauge transformation in our Lagrangian such that the resulting Higgs doublet reads
%
\begin{equation}
H \to e^{- i G_a \tau^a/v} H =  \frac{1}{\sqrt{2}}\left( \begin{matrix}  0 \\ h + v \end{matrix} \right).
\end{equation}
%
This transformation must also be applied to the gauge fields, fixing the gauge. This particular choice is rather convenient and is known as the unitary gauge. We then find
%
\begin{align}
 \mathscr{L}_{\rm Higgs} &\supset - \frac{1}{2} m_h^2 h^2 - \lambda v h^3 - \frac{\lambda}{4} h^4  \nonumber\\ &\qquad\qquad +  M_\textsc{w}^2 W_\mu^\dagger W^\mu \left[ 1 + \frac{2 h}{v} + \frac{h^2}{v^2}\right] + \frac{M_\textsc{z}^2}{2} Z_\mu Z^\mu \left[ 1 + \frac{2 h}{v} + \frac{h^2}{v^2}\right],
\end{align}
%
where $m_h = \sqrt{2\lambda}\, v = 125.18 \pm 0.16$ GeV~\cite{PDG}. Most importantly, after SSB, the Higgs kinetic term has given us three massive and one massless vector bosons, defined as
\begin{equation}
 W^\pm_\mu = \frac{1}{\sqrt{2}} \left(W^1_{\mu} \mp i W^2_{\mu} \right), \quad Z_\mu = c_\textsc{w} W^3_\mu - s_\textsc{w} B_\mu, \quad A_\mu = c_\textsc{w} B_\mu - s_\textsc{w} W^3_\mu.
\end{equation}
% 
where $s_\textsc{w}$ ($c_\textsc{w}$) is the sine (cosine) of the weak angle, defined by $c_\textsc{w} = g/\sqrt{g^2 + g^{\prime\,2}}$. These fields correspond to the mediators of the weak charged-current (CC) interactions ($M_\textsc{W} = g v/2 = 80.387\pm 0.016$ GeV~\cite{PDG}), of the weak neutral-current (NC) interactions ($M_\textsc{z} = M_\textsc{w}/c_\textsc{w} = 91.1876\pm0.0021$ GeV~\cite{ALEPH:2005ab}) and the massless photon $A_\mu$, mediator of the unbroken EM interactions. Their interactions with the Higgs boson are also shown in the triple and quadruple vertex terms above. The interactions with matter are obtained from the fermion kinetic terms, where the charged, neutral and electromagnetic currents are defined and written as
%
\begin{equation*}
\mathscr{L}_{\rm NC} = e \,{J}_\mu^\gamma A^\mu + \frac{g}{c_\textsc{w}}\, {J}_\mu^Z Z^\mu, \quad {J}_\mu^\gamma =  \overline{\psi} Q_{\rm EM} \gamma_\mu \psi, \quad %
{J}_\mu^Z = \overline{\psi} \gamma_\mu \left[ \left( \frac{T_3}{2} - Q_{\rm EM} s_\textsc{w}^2 \right) - \frac{T_3}{2} \gamma^5\right] \psi,
\end{equation*}
\begin{equation}
\mathscr{L}_{\rm CC} = \frac{g}{\sqrt{2}}  \, \left( {J}_\mu^+ W^{\mu \,+} + {J}_\mu^- W^{\mu \,-} \right), \quad
%
{J}_\mu^+ = \frac{1}{2} \overline{\psi}_u \gamma_\mu \left( 1 - \gamma^5\right) \psi_d + {\rm h.c.},
\end{equation}
%
where $\psi \in \{ \nu_L, e_L, u_L, d_L, e_R, u_R, d_R \}$, and $\psi_{u,\,d}$ denoting fermions with $T_3 = \pm 1/2$. From the weak currents we note two important aspects: \emph{i)} weak interactions indeed violate parity with a $V-A$ structure, \emph{ii)} charged-current interactions are purely LH as they should be since no RH fields are charged under $SU(2)_L$. After SSB, only the EM current is conserved $\partial^\mu J_\mu^\gamma = 0$. 

In the discussion above, we fixed the gauge of the SM to simplify the EW Lagrangian. This is not necessary and, in fact, another possibility is to keep all terms involving the Goldstone fields $G_a$ and eliminate off-diagonal kinetic terms of the type $Z_\mu\partial^\mu G_3$ by introducing the following gauge breaking Lagrangian to the SM
\begin{equation}
 \mathscr{L}_{\rm R_\xi} = -\frac{\left(\partial_\mu A^\mu\right)^2}{2 \xi_\gamma} - \frac{\left(\partial_\mu Z^\mu + \xi_Z M_Z G_3^0 \right)^2}{2 \xi_Z} - \frac{\left|\partial_\mu W^{\mu\,-} + i \xi_W M_W G^- \right|^2}{2 \xi_W}.
\end{equation}
This is known as the $R_\xi$ gauge, where the explicit dependence on the gauge breaking parameters $\xi$ serves as a useful diagnostic of gauge invariance in physical observables. The Lorentz gauge is recovered for $\xi = 0$ and the Feynman-'t Hooft gauge with $\xi = 1$. This method to fix the gauge played an important role in the development of the SM. First introduced by Ludvig Fadeev and Victor Popov~\cite{Faddeev:1967fc}, this provided a recipe to perform calculations in gauge theories without the ambiguity of the gauge symmetry. In practice, one must also add unphysical ghost fields to guarantee the unitarity of the theory. These only appear in loop processes and we will not encounter them again in this thesis. An additional advantage of fixing the gauge in this way is that it allows us to trace the Goldstone degrees of freedom. The pseudo-scalar fields $G^\pm$ and $G_3$ end up behaving very similarly to the $W^\pm$ and $Z$ gauge bosons. In fact, at high-energies it can be shown that the Goldstone bosons are equivalent to the longitudinal polarization states of their respective gauge bosons~~\cite{Cornwall:1974km,LlewellynSmith:1973yud}. This is known as the Goldstone boson equivalence theorem, and it turns out to be very important to understand processes like $W_L \, W_L$ and $Z_L \,Z_L$ scattering. At very high-energies and without the Higgs boson, such processes grow indefinitely ($\sigma \propto s$), spoiling the unitarity of the $S$-matrix. The fact that this problem was solved by including contributions from $h$ exchange provided a no-lose theorem for the LHC: either the Higgs boson would be discovered, or new physics must appear to unitarize these processes. 

\subsection{Fermion masses}

The EW sector is also responsible for the generation of fermion masses in the SM. As noted before, all LH fermions in the SM are $SU(2)_L$ doublets, just like the Higgs. This allows us to construct the so-called Yukawa terms,
%
\begin{equation}
 \mathscr{L}_{\rm Yukawa}=  y^e_{\alpha\,\beta}  \left(\overline{L}^\alpha H\right) e_R^\beta +  y^u_{\alpha\,\beta}  \left(\overline{Q}^\alpha_L \tilde{H} \right) u_R^\beta + y^d_{\alpha\,\beta}  \left(\overline{Q}_L^\alpha H \right) d_R^\beta  + \,\, {\rm h.c.},
\end{equation}
%
where we defined the charge-parity (CP) conjugated Higgs field $\tilde{H} = i \sigma_2 H^* = ( h^0 + i G_3^0, \,\, G_1^- - i G_2^-  )^T$ and included all three families of fermions by promoting $y^\psi \to \bm{y}^\psi$ to a $3\times3$ matrix. . After SSB, these interaction terms endow charged-leptons and quarks with a dirac mass term of the form
\begin{equation}
 m_\psi \overline{\psi} \psi = m_\psi \left( \overline{\psi}_L \psi_R +\overline{\psi}_R \psi_L \right),\quad {\rm with} \quad m_\psi = \frac{y_\psi \, v}{2},
\end{equation}
where $\psi_{L,\,R} = P_{L,\, R} \, \psi = (1 \mp \gamma_5)\, \psi/2 $ are the chiral projections of the fermion field $\psi$.

In the quark sector, the Yukawa matrix is off-diagonal and the different generations mix. The physical quark masses are found after rotating the up and down quarks, left and right, as $u_{L,\, R}^\alpha = \left(V_{L,\,R}^{u} \right)_{\alpha i}^* u^i_{L,\, R}$ and $d_{L,\, R}^\alpha = \left(V_{L,\,R}^{d} \right)_{\alpha i}^* d^i_{L,\, R}$. The diagonal mass matrix is then $\bm{m}^{u,\,d} = \bm{V}_L^{u,\,d} \bm{y}_{u,\,d} \bm{V}_R^{u,\, d\,\dagger} v/\sqrt{2}$. Note that after this procedure we cannot help but introduce mixing in the charged current. This defines the Cabbibo-Kobayashi-Maskawa (CKM) matrix~\cite{Cabibbo:1963yz,Kobayashi:1973fv}, $\bm{V}_{\rm CKM} = \bm{V}_L^u \bm{V}_L^{d\, \dagger}$. The CKM matrix is nearly diagonal, so the mixing between quark flavour and mass eigenstates is small. From the unitarity of the rotation matrices, neutral currents remain invariant 
\begin{equation}
\sum_{\alpha,\, \beta} \overline{\psi}_L^\alpha \,\Gamma^\mu \, \psi_L^\beta = \sum_{i,\,j} \overline{\psi}_L^i \, \left( \sum_{\alpha,\, \beta} (V_L)_{\alpha i} (V_L^*)_{\beta j}\right) \,\Gamma^\mu\, \psi_L^j =  \sum_{i,\,j} \overline{\psi}_L^i \, \,\Gamma^\mu\, \psi_L^j,
\end{equation}
%
where $\Gamma^\mu$ are the neutral-current couplings and gamma matrices.  Crucially, $\Gamma^\mu$ has no flavour dependence and so the SM forbids flavour changing neutral currents (FCNC). This mechanism was first proposed by Glashow, Illiopoulous and Maiani to explain why decays of the type $K^0 \to \mu \mu$ were unobserved. Famously referred to as the GIM mechanism, this relies on the flavour universal nature of SM neutral currents and on the unitarity of the CKM. As we will see, this mechanism also plays an important role in the neutrino sector and in many extensions of the SM.
%
\begin{figure}
 \includegraphics[width=0.85\textwidth]{SM_zoo3.pdf}
 \caption{Artistic rendering of the particles in the Standard Model. \label{fig:SM_diagram}}
\end{figure}
%

Summarizing, the SM particles are illustrated in \reffig{fig:SM_diagram} and the full SM Lagrangian is simply
%
\begin{equation}
 \mathscr{L}_{\rm SM} = \mathscr{L}_{\rm gauge} \,+\, \mathscr{L}_{\rm Higgs}   \,+\, \mathscr{L}_{\rm Yukawa} \,+\, \sum_{\alpha, \Psi}\, \overline{\Psi^\alpha} i \slashed{D} \Psi^\alpha, 
\end{equation}
%
where $\alpha$ is a flavour index and $\Psi \in \{L, Q_L, e_R, u_R, d_R \}$. We saw how a single parameter with massive dimensions in the scalar potential of the SM leads to SSB. This is then ``propagated'' to the rest of the SM through the Higgs kinetic terms and Yukawa couplings. At this point it is possible to appreciate two problems with the SM mass generation mechanism. Firstly, it implies that all Yukawa couplings are just parameters to be inferred from the measured masses of particles. That is, the SM makes no statements and provides no explanations as to why the Yukawas that we observe in nature are what they are. This is known as the flavour puzzle and is equivalent to asking what explains the different observed fermion masses. This problem is aggravated when we consider that neutrinos do have masses and that the leptonic mixing is drastically different from the one in the quark sector. This leads us to the second problem with the SM mass generation. The SM predicts exaclty massless neutrinos in the absence of $\nu_R$ fields. This is perhaps the biggest motivation for studying neutrino physics at present. 

Before moving on to more speculative topics, a few comments are in order. EW SSB seems to be, as far as we know, a real phenomenon. It explains why the symmetries of the SM were so well hidden the first place: true symmetries of Nature seem to not be shared by the vacuum. While the evidence for EW SSB comes mainly from studying fundamental particles, its consequences do not concern only particle physics. EW physics helps us understand the past and future of our own Universe. In the early Universe, at hight temperatures, it is expected that the EW symmetry is restored~\cite{Kirzhnits:1972ut,Dolan:1973qd,Weinberg:1974hy}. If this is the case, the EW phase transition provides a unique test of the Higgs mechanism and points to a completely different Universe from our own, where finite temperature effects and non-perturbative physics play a major role. In addition, we have no reason to expect the vacuum structure of the Universe to be as simple as describe above. After all, the stability of our own vacuum is not even guaranteed within the SM~\cite{Cabibbo:1979ay,Degrassi:2012ry}. Radiative corrections to the Higgs self-coupling $\lambda$ alter the shape of the scalar potential and imply we may live in a local, rather than global, minimum of the potential. For these reasons, studying the Higgs sector, confirming that it generates all fermion masses in the SM and why it seemingly fails to do so in the case of the neutrino are all questions worth pursuing.


%%%%%%%%%%%%%%%%%%%%%%%%%%%%%%%%%%%%%%%%%%%%%%%%%%%%%%%%%%%%%%%%%%%%%%%%%%%%%%%%
%
% BSM!!!!!!
%
%%%%%%%%%%%%%%%%%%%%%%%%%%%%%%%%%%%%%%%%%%%%%%%%%%%%%%%%%%%%%%%%%%%%%%%%%%%%%%%

\section{Evidence for Beyond the Standard Model physics}

The most important building aspects and building blocks of the SM have been laid out above. Now, a different question will concern us: is this theory sufficient to explain fundamental particles and their interactions? In this section we will list what we believe to be the most important hints and evidences that this is not the case. We have already stumbled upon a few problems of the SM, but even before that one must already suspect that the SM is not a final theory. It does not explain gravity. This tells us that the SM should be treated as an effective theory valid up until the Planck mass $M_{\rm Pl} = \left(\hbar c/ 2 G_{\rm Newton}\right)^{1/2} \approx 10^{19}$ GeV, where the effects of gravity are expected to be large~\footnote{This is a naive expectation based on the observation that the Schwarzschild radius $\ell_s = 2 G_{\rm Newton} m/c^2$ and the Compton wavelength $\ell_c = h /mc$ of a particle become comparable at $m \approx M_{\rm Pl}$.}. This very fact already brings us to one of the most debated evidences for beyond the Standard Model (BSM) physics.

\paragraph{The hierarchy problem} The lack of evidence for new physics at the LHC can be argued to be more than just unfortunate. If no new physics is indeed present between the EW and the Planck scale, then the cut-off of the SM, beyond which the effective field theory is no longer valid, is $\Lambda = M_{\rm Pl}$. This implies that unless symmetries are at play, terms of dimensions $d$ are suppressed or are of the order of $\Lambda^{4-d}$. However, in the SM $m_h^2 \ll \Lambda^2$, suggesting a fine-tuning of many orders of magnitude. Quantum corrections to the Higgs mass, $m_h^2 = m^2_{\rm bare} + \delta m_h^2$, are dominated by the top quark and go as $\delta m_h^2 = y_t^2 \Lambda^2 / 8 \pi^2$. This quadratically divergent result implies that to obtain the observed light Higgs mass, whatever new physics that may appear at the scale $\Lambda$ (possibly even below $M_{\rm Pl}$) must cancel the fermion loops to order $m_h^2/ \Lambda^2$. In other words, the matching condition for the renormalization of $m_h^2$ parameter becomes fine-tuned to order $m_h^2/ \Lambda^2$ in the presence of such cut-off. Supersymmetric theories are notorious candidates to solve this problem, but so far we are yet to find any evidence for them. One may argue that indeed there exists a ``desert'' between the EW and the Planck scale, and that some miraculous mechanism is at play in quantum gravity that may solve the fine-tuning problem. In that case, a solution to all following items in this list must be found at that scale, or somewhere outside the realm of particle physics.

\paragraph{The strong-CP problem} The QCD Lagrangian admits the following field-strength contraction term
%
\begin{equation}
 \mathscr{L} \supset \frac{\theta \alpha_s}{8\pi} G_{\mu\nu}^a \tilde{G}_a^{\mu\nu},\quad {\rm where }  \quad \tilde{G}^a_{\mu\nu} = \frac{\epsilon_{\mu\nu\rho\sigma}}{2} {G}^{a\, \rho\sigma}.
\end{equation} 
This can be shown to be a surface term (a total divergence in the action) and can be neglected in perturbative calculations. Nevertheless, this term induces CP violation in the strong sector via non-perturbative effects, leading to a large electric dipole moment for free neutrons~\cite{Crewther:1979pi}, which is orders of magnitude above the experimental upper limits~\cite{Afach:2015sja}. The most popular scenario to explain the smallness of $\theta$ is the Peccei-Quinn symmetry~\cite{Peccei:1977ur}, a global chiral $U(1)$. The breaking of this symmetry leads to the prediction of a Goldstone boson, the axion.

\paragraph{Matter-antimatter asymmetry} The observed baryon asymmetry of the Universe contradicts the standard Cosmology, which predicts that matter and anti-matter were created in equal amounts in the Big Bang. A good measure of this effect is $\eta_B = (n_B - n_{\overline{B}})/n_\gamma$, where the difference between the number density of baryons and anti-baryons is normalized to the photon number density $n_\gamma$, rendering $\eta_B$ insensitive to the expansion of the Universe. This is measured to be extremely small, $\eta_B = \approx 6 \times 10^{-10}$ with baryons being the dominant component. The SM does not provide enough source of CP violation to explain this phenomenon. Popular scenarios to explain this are EW baryogenesis and leptogenesis. The latter relies on the CP violation from the lepton sector, which is later translated into a baryon asymmetry through non-perturbative spharelon processes that violate total $B+L$ number. This is relevant for neutrino physics, since heavy right-handed neutrinos may realize leptogenesis. 

\paragraph{Dark matter} In the 1930's, Fritz Zwicky measured the velocity dispersion of galaxies in the Coma cluster~\cite{Zwicky:1933gu}, and applied the Virial theorem to show that the matter inferred from its luminosity was insufficient to hold the cluster together. Alongside the pioneering work of Vera Rubin on galaxy rotation curves in the 70's~\cite{Rubin:1970zza}, these observations showed that the gravitational potential in astrophysical scales is much deeper than the one extrapolated from luminous matter. Already at the time, astronomers would refer to the source of this additional gravitational influence as Dark Matter (DM). As astrophysics and cosmology evolved, concrete evidence for DM continued to build up. Now, it is present at a variety of scales, from the precise measurements of the cosmic microwave background (CMB)~\cite{Akrami:2018vks}, the matter distribution in galaxy cluster mergers~\cite{Clowe:2006eq}, and the observed large scale structure of the Universe~\cite{Blumenthal:1984bp}. In fact, from the CMB power spectrum we can infer the DM density today as~\cite{Akrami:2018vks}
\begin{equation}
 \Omega_{\rm DM} h^2 = 0.1200\pm0.0012
\end{equation}
with $\Omega_{\rm DM} = \rho_{\rm DM}/\rho_c$ the energy density of DM in units of the critical density $\rho_c \approx 10^{-26}$ kg/m$^3$, and $h = H_0/(100$ km s$^{-1}$/Mpc$^{-1}) = 0.674\pm0.005$ the scaled Hubble expansion rate. This is roughly five times larger than the density of baryons, understood as all other non-relativistic matter. The latter is also measured through the relative abundances of light elements during Big Bang Nucleosynthesis (BBN)~\cite{Cooke:2013cba}, where DM plays no role, and provides further evidence for the non-baryonic nature of DM.

The nature of DM is not yet understood and many possibilities are under discussion. Modified gravity models explain local astrophysical observations, but struggle to explain all CMB datasets and X-ray observations of mergers of galaxy clusters~\cite{Famaey:2011kh}. Primordial black holes~\cite{Barack:2018yly} have also been put forward as DM candidates and have triggered great interest due to their connection to the detection of gravitational waves. However, the most popular hypothesis at this point remains that DM is made of new particles. This new state better be neutral, to have evaded our detection, and sufficiently long-lived, so that it may linger until today after its production in the early Universe. The fluid of such particles would have to display negligible pressure and viscosity, and to have been created cold so as to help form clumpy structures in the Universe through its gravitational pull. This points us to a particle that is massive, collisionless and, yet, very abundant today. Most notably, DM models have often focused on the possibility of a weakly-interacting massive particle (WIMP). In this paradigm, DM particles, say $\chi$, are produced in the early Universe through its weak interactions with the SM plasma. At later times, approximately at temperatures of the order of the DM mass $m_{\chi}$, DM production stops and annihilation into SM particles dominates. As the Universe cools and expands, the DM gas is diluted and annihilation is no longer effective, \emph{freezing-out} the DM population at around $T \approx m_{\chi}/20$. In particular, the relic density obtained in this mechanism is of the order $\Omega_\chi H_0^2 \approx 0.1 \,{\rm pb}/ \sigma$, where $\sigma$ stands for the thermally averaged cross section of $\chi$ annihilation into SM particles. The fact that $\sigma\approx 1$ pb allows to reproduce the current DM density and is of the order of typical weak cross sections (as in mediated by weak bosons) is known as the WIMP-miracle. WIMP DM is a collisionless and thermal candidate, although DM candidates that are non-thermal, or collisionless, or both exist.

\paragraph{Neutrino masses} One of the most important evidences for BSM physics is the fact that neutrinos have mass. This comes from the plethora of measurements of neutrino oscillations, which we study in the next chapter. Put simply, neutrino oscillation data requires at least 2 non-degenerate massive neutrinos. Although one might argue that this is solved by the mere addition of at least 2 RH singlet states to the SM, this simple extension would require additional theoretical ingredients and experimental confirmation. For instance, such states would be the only SM particle to admit a Majorana mass term of the type $M \,\overline{\nu^c}_R \nu_R$, and unless new symmetries are introduced, there is no reason to expect that $M$ is exactly zero. Therefore, neutrino masses are the first evidence of physics beyond the SM observed in controlled laboratory conditions.


\section{Extending the Standard Model}

While we lack a single compelling evidence for the direct detection of a new particle, we can interpret many of the problems outlined above as evidence for new states. The existence of DM and neutrino masses, in particular, suggests (but does not require) that these new states may be a new sector of electromagnetically neutral particles, secluded to their own dark sector. Here, the SM provides little guidance on their masses or symmetries. For this reason, many possibilities to such models exist. We can be guided by experimental anomalous results, by increased elegance in our theories, or by complete agnosticism. In this section we follow take a stronger preference for the latter approach, and set out to discuss the many possibilities through which secluded states may couple to the SM. Much of what we discuss here narrows down the scope of the models we will work with throughout this thesis.

Dark sectors arise in theories where new states, say dark matter particles, are secluded and are not charged under the SM group. These particles may have simple or complicated dynamics in their lair, but their only connection to our SM world is through small \emph{portal couplings}~\cite{Boehm:2003hm,*Boehm:2003ha,Alexander:2016aln}. This hypothesis is compelling because it explains why a large fraction of the Universe is invisible to us (in the form of dark states), and why the SM appears so self-contained. The modular and hidden aspect of this point of view is indeed frightening, but this is is not the first time we have encountered it. When Wolfgang Pauli proposed the existence of the neutrino to explain the continuous spectra of electrons from beta decays, he had in fact stumbled upon a key player of a new hidden sector. Initially, in Pauli's own words, this particle was believed to be ``impossible to detect''. It was later clear that this may not be the case if the 4-fermion theory for beta decays by Enrico Fermi was correct. We now know that to be true, where the fact that the neutrino had escaped detection up until that point is explained by the smallness of the fermi coupling constant~\cite{PDG}
\begin{equation}
 \frac{G_F}{\sqrt{2}} = \frac{g^2}{8 M_\textsc{w}^2} = 1.1663787(6) \times 10^{−5}\,\, {\rm GeV}^{−2}.
\end{equation}
In this case, the hidden sector can be thought of as the neutrino, and the weak interactions are our small portal couplings to it. In fact, to study such hidden sector, we had to overcome the smallness of $G_F$ with large exposure experiments at low energies, such as in the detection of neutrinos at just a few meters away from a nuclear reactor by Cowan and Reiness in 1956. Of course, this was not enough to understand the whole picture. As it turns out, $G_F$ is only an effective coupling, and its smallness is simply due to the large masses of the mediators of the weak force. The rest culminates in the construction of the SM. 

With this familiar analogy, we hope that the solution to the problems in the previous sections may display some similar features. Of course, the smallness of portal couplings may not always be due to the high mediator masses. It may arise from a mere accident of the theory, if one believes in such things, from large separation of scales, or from other mechanisms. One realization of small couplings is nicely exemplified by kinetic mixing. In theories with heavy fermions charged both under a new $U(1)$ group and hypercharge, the low energy effects of the new $U(1)$ come in through loop-diagrams like those shown in \reffig{fig:Dark_sectors}. The loop suppression then explains why $\epsilon$, the kinetic mixing parameter, is small $\epsilon \lesssim 10^{-2}$.
%
\begin{figure}[t]
 \includegraphics[width=0.95\textwidth]{Dark_sectors.pdf}
 \caption[A possibility for a portal coupling and why it is small.]{A possibility for a portal between the Standard Model and a dark sector. Fermions $f$ charged under both the hypercharge group and a new $U(1)$ group generate kinetic mixing at loop level between $B^\mu$ and the new boson $X^\mu$. In the approximate formula for $\epsilon$, $\mu^2$ stands for the renormalization scale and $g_X$ to the gauge coupling of the new $U(1)$. \label{fig:portals_diagram}}
\end{figure}
%

To understand the possible links between dark sectors and the SM, we would like to understand all possible ways in which dark and SM particles can interact. One way to tackle this question is to build effective field theories, where one studies all operators which are allowed by the content and symmetries of the SM. The idea is to construct a series of $d>4$ operators in $1/\Lambda^{d-4}$, where $\Lambda$ is the scale of the new physics. This approach thrives on its generality, but can become complicated very quickly with growing $d$. Most importantly, the scale $\Lambda$ is assumed to be large, so that all new degrees of freedom have been integrated out of the theory. This is suitable for extensions involving particles which are very heavy, but the series is no longer well defined for new physics that is light and kinematically accessible at our experiments. In this case, the kinematics of the new particles play a role, forcing us to write down the field content and symmetry group of the new physics. This is the approach we describe in what follows.

\subsection{Portals to hidden sectors}

We would like our SM extensions to follow specific guiding principles and organize them in a meaningful way. One way to do so is to study all the low dimension neutral operators that the SM has to offer. In contrast to effective field theories, we want renormalizable operators with $d<4$ and that are preferably gauge invariant. As it turns only a few such operators exist, which we usually refer to as \emph{portals}. We dedicate this section to presenting these, as well as the most popular operators that have also been associated with portal couplings, but that are not renormalizable or gauge-invariant.
%
\begin{figure}[t]
 \includegraphics[width=0.95\textwidth]{Dark_portals.pdf}
 \caption[Diagramatic representation of the portal couplings discussed.]{Diagramatic representation of all portal couplings discussed here. From left to right, the top row shows the neutrino, vector and $d=4$ Higgs portal. The bottom row shows the super-renormalizable $d=3$ Higgs portal, the fermionic portal and the non-renormalizable pseudo-scalar portal. \label{fig:all_portals}}
\end{figure}
%

\paragraph{Neutrino portal} Arguably the most motivated portal, this $d=5/2$ operator can be written as
\begin{equation}
\left( \overline{L}^\alpha \cdot \tilde{H}\right).%= \left( \left(\nu^\alpha_L\right)_a \,\, \left(e_L\right)_a \right)
\end{equation}
%
Any fermion field which couples to this operator acquires couplings to SM neutrinos. This typically induces off-diagonal mass terms in the Lagrangian, leading to mixing between the new species and all massive neutrinos in the broken phase of the SM. The new particle is then commonly referred to as \emph{heavy neutral lepton} or right-handed neutrino, though its chirality is a matter of convention. The smallness of this coupling is usually associated to a difference of scales between the EW vev, and the new mass scales of the heavy neutral lepton. We will study such a model in detail in the next chapter.

\paragraph{Vector portal} Any new vector particle $X^\mu$ from an Abelian gauge group may couple to the $d=2$ field strength of the SM hypercharge
\begin{equation}
B_{\mu\nu},
\end{equation}
through its own field strength tensor $X^{\mu\nu}$. The resulting term, $B_{\mu\nu} X^{\mu\nu}$, is a off-diagonal kinetic term for the massive bosons and is sometimes called the \emph{kinetic mixing} operator. This may arise from heavy fermion loops that are integrated out, or through the simultaneous presence of the two Abelian groups across all scales~\footnote{Grand unification clearly forbids such possibility at the highest scales, but this remains, after all, a hypothesis.}. To work in a basis of physical states with diagonal kinetic terms, where the propagators are in their standard form, one usually performs a field redefinition. If much lighter than the EW scale, the new vector particle couples primarily to the EM current, hence the name dark photon. If heavy, it can also couple to the NC and is therefore referred to as a dark $Z$. Models with the term 
%
\begin{equation}
 Z_\mu X^\mu 
\end{equation}
%
also come up in the literature, where it is said that \emph{mass-mixing} between the new vector particle and the SM $Z$ exists. This term is not gauge invariant, but may arise in the broken phase of BSM theories with additional doublet scalars, like in two-Higgs-doublet models (2HDM). In this case, several charged degrees of freedom typically appear and experimental constraints tend to be more severe.

\paragraph{Higgs portal} New scalar particles can couple to the $d=2$ bilinear 
%
\begin{equation}
 H^\dagger H,
\end{equation}
%
the only renormalizable portal with no free Lorentz or spinor indices. In this case there are two possibilities for a scalar to couple to the SM, depending on its charges. We can write  $H^\dagger H \, S^\dagger S$ for a charged, or $H^\dagger H\, S $ for a singlet complex scalar. The latter term is the only super-renormalizable operator connecting the SM fields to new physics which is allowed. Beyond important consequences for EW SSB, these operators typically inherit the Higgs couplings to matter fields, and may be hard to search for due to the smallness of the SM Yukawa couplings. Remarkably, this extension can also have consequences to the hierarchy problem, as the new scalar also contributes to the Higgs self-energy~\cite{Craig:2013xia}.

%
\begin{figure}[t]
 \includegraphics[width=0.8\textwidth]{Triangles.pdf}
 \caption[Anomalous triangle diagrams.]{Anomalous triangle diagrams. The fermion loop is reversed on the right. \label{fig:triangles}}
\end{figure}
%
\paragraph{Fermionic currents}

A whole set of (EM) neutral operators in the SM come from the fermionic currents
\begin{equation}
 J^\mu = \overline{\Psi} \gamma^\mu \Psi,
\end{equation}
where $\Psi \in \{Q_L, L, u_R, d_R, \ell_R\}$. These are not gauge invariant, in general, and will generally require new gauge symmetries to be useful as a portal to the dark sector. The SM currents can be associated with new conserved charges, which in turn may be regarded as a global or promoted to a local gauge symmetry. In the latter case, the new conserved charge is said to be gauged under a local symmetry and we can introduce an additional gauge bosons, potentially massive if the symmetry is broken. Here, we must also require that it be anomaly-free. This means that the symmetry must be conserved not only classically, but also at loop level. Various types of anomalies exist, but of most interest in gauge extensions of the SM are the chiral gauge anomalies. These can be calculated from the amplitudes of the triangle diagrams shown in \reffig{fig:triangles}, and are proportional to 
%
\begin{equation}
 {\rm Tr}\left[ (T^a T^b + T^b T^a) T^c\right],
\end{equation}
%
where $T^a$ is the generator of the symmetry group of the attached gauge boson in the relevant representation and only left-handed particles and anti-particles are running in the loop. The simplest case is the one of the $\left[U(1)\right]^3$ chiral anomaly, where it reduces to the requirement that $\sum_\psi Q_\psi^3 = 0$ for all fermions $\psi$ charged under the Abelian group. In the SM, baryon number $B$, lepton number $L$ and the individual lepton number $L_\alpha$ are all accidentally conserved quantities~\footnote{Beyond the SM, $L_\alpha$ is already violated at tree-level by the small observed neutrino masses.}. Non-perturbative effects, however, violate $B$ and $L_\alpha$, and these quantities are no longer conserved. Nevertheless, $B-L$ and the combinations $L_\alpha - L_\beta$ are preserved in these processes and can be taken to be a \emph{global} symmetry of the SM. At loop-level, however, $B-L$ is violated by the chiral triangle diagrams and only $L_\alpha - L_\beta$ provide anomaly-free choices. We explore these currents in Abelian extensions of the SM in Chapter 4.

The above exhausts the minimal possibilities for SM portals that lead to renormalizable operators to new physics. Nevertheless, for completeness, we will also comment on a well-motivated non-renormalizable operator that is also frequently discussed in the context of light new physics.


\paragraph{Pseudo-scalar} The Peccei-Quinn solution to the strong CP problem predicts the existence of a new pseudo-scalar $a$, the axion. This is the pseudo-Goldstone boson from the breaking of the global $U(1)_{\rm PQ}$ \emph{axial} symmetry at a scale $f_a$. Such particle would then acquire couplings to gauge bosons and SM fermions $\psi$ as in
%
\begin{equation}
 c_1 \, \frac{a}{f_a} G_{\mu\nu}^a \widetilde{G}^{\mu\nu}_{a} + c_2 \, \frac{a}{f_a} W_{\mu\nu}^a \widetilde{W}^{\mu\nu}_{a} + c_3 \, \frac{a}{f_a} B_{\mu\nu} \widetilde{B}^{\mu\nu} +  \sum_\psi c_4^\psi \, \frac{\partial_\mu a}{f_a} \overline{\psi} \gamma^\mu \gamma^5 \psi,
\end{equation}
%
where $c_i$ are model-dependent couplings and are typically linear dependent. Axion particles from models that solve the strong CP problem, commonly referred to as the QCD axions, are not the only possibility. In fact, light pseudo-scalar from the breaking of new symmetries at higher energies provide a well-motivated target for study and go under the name axion-like-particle (ALP). In this case, the relation between the ALP mass and its couplings is less restricted. 


\chapter{Neutrino physics}
\begin{flushright}
	``''
	\\--- Madeline Line
\end{flushright}
\graphicspath{{}{theory/}}


Neutrinos provide the most compelling evidence for physics beyond the SM. In this chapter we explore why and discuss some of the most important properties of neutrinos for their experimental study.

Neutrinos are purely weakly interacting fields

Neutrinos -- Pauli + Madame Wu + Freinberg numus + Lederman

\section{Oscillations}

\section{Sources}

To find a source of neutrinos, all we have to do is to look for environments where the Weak force is prominently manifested. A natural candidate, as we have seen in the discovery of the neutrino, are nuclear reactors. Fortunately, the list does not stop there. Abundant neutrino sources include the Sun,

\section{Scattering}

Neutrino cross sections are an invaluable observable to understand the Weak force and to search for new physics. In conventional oscillation experiments, we probe the neutral and charged-current interactions of neutrinos with electrons and hadronic matter. 

\section{Mass mechanisms}

\begin{figure}[t]
\centering
\includegraphics[width=\textwidth]{seesaw_mechanisms.pdf}
\caption[The tree-level UV completions of the Weinberg operator.]{The only three UV completions of the $d=5$ Weinberg operator with their respective contributions to light neutrino masses.\label{fig:seesaw_mechanisms}}
\end{figure}


In \reffig{fig:seesaw_mechanisms}, we show these unique tree-level completions of the Weinberg operator.

For a review on low-scale models see \cite{Boucenna:2014zba}.

\subsection{Conventional seesaws}

\subsection{Low scale seesaw variants}

\subsection{Radiative masses}

The most straightforward extension of the SM which can generate neutrino masses at loop level is perhaps the so-called Zee-Babu model.

\chapter{Neutrino trident scattering}
%%%%%%%%%%%%%%%%%%%%%%%%%%%%%%%%%%%%%%%%%%%%%%%%%%%%
\graphicspath{{}{tridentSM/figs/}{tridentSM/}}

\section{Introduction}
\label{sec:intro}

The Standard Model (SM) has been confronted with a variety of experimental data and has so far emerged as an impressive phenomenological description of nature, except in the neutrino sector. The observation of neutrino flavour oscillations by solar, atmospheric, reactor and accelerator neutrino experiments over the last 50 years has revealed the existence of neutrino mass and flavour mixing, making necessary the first significant extension of the SM.

The precise determination of the neutrino mixing parameters as well as the search for the neutrino mass ordering and leptonic CP violation drive both present and future accelerator neutrino experiments. To accomplish these tasks, these experiments rely on state-of-the-art near detectors, made of heavy materials, located a few hundred meters downstream of the neutrino source and subjected to a high intensity beam. Their main purpose is to ensure high precision measurements at a far detector by reducing the systematic uncertainties related to neutrino fluxes, charged-current (CC) and neutral-current (NC) cross sections and backgrounds.  
%
The high beam luminosity they are subjected to (about $10^{21}$ protons on target) and their relatively large fiducial mass of high-$Z$ materials (typically 100~ton) make these detectors ideal places to investigate rare neutrino-nucleus interactions ($\sigma\lesssim 10^{-44}$~cm${}^2$), such as neutrino trident scattering. 

Trident events are processes predicted by the SM as the result of (anti)neutrino-nucleus scattering with the production of a charged lepton pair \cite{Czyz:1964zz,Lovseth:1971vv,Fujikawa:1971nx,Brown:1971qr,Koike:1971tu}, $\pbar{\nu}_{\alpha}+{\cal{H}} \to \pbar{\nu}_{\alpha \,{\rm{or}} \, \kappa(\beta)} + \ell_{\beta}^- + \ell_\kappa^+ +{\cal{H}}$, $\{\alpha,\beta,\kappa\}\in \{e,\mu,\tau\}$\footnote{Throughout the manuscript we will consider ${\alpha,\beta, \kappa}$ as flavour indexes.} where $\cal{H}$ denotes a hadronic target. Depending on the (anti)neutrino and charged lepton flavours in the final-state, the process will be mediated by the $Z^0$ boson, $W$ boson or both. Coherent interactions between (anti)neutrinos and the atomic nuclei are expected to dominate these processes as long as the momentum transferred $Q$ is significantly smaller than the inverse of the nuclear size \cite{Czyz:1964zz}. For larger momentum transfers diffractive and deep-inelastic scattering become increasingly relevant \cite{Magill:2016hgc}.
%
Although this process exists for all combinations of same-flavour or mixed flavour charged-lepton final-states, to this day only the $\nu_\mu$-induced dimuon mode, $\pbar{\nu}_\mu + {\cal{H}} \to \pbar{\nu}_\mu  + \mu^+ + \mu^- + {\cal{H}}$, has been observed. The first measurement of this trident signal performed by CHARM II~\cite{Geiregat:1990gz} is also the one with the largest statistics: 55 signal events in a beam of neutrinos and antineutrinos with $\langle E_\nu \rangle \approx 20$ GeV. Other measurements by CCFR~\cite{Mishra:1991bv} and NuTeV~\cite{Adams:1998yf} at larger energies soon followed.

As the measurement of trident events may provide a sensitive test of the weak sector~\cite{Brown:1973ih} as well as placing constraints on physics beyond the SM~\cite{Mishra:1991bv,Gaidaenko:2000hg,Altmannshofer:2014pba,Kaneta:2016uyt,Ge2017,Magill:2017mps,Falkowski:2018dmy} it is relevant to investigate how to probe it further at current and future neutrino experiments. Atmospheric neutrinos, for instance, may provide a feasible measurement of the dimuon channel, as pointed out in \refref{Ge2017}\footnote{The authors of \refref{Ge2017} have performed the full calculation of the trident process and made their code publicly available.}. Other trident modes were also re\-cog\-ni\-zed to be relevant by the authors of Ref.~\cite{Magill:2016hgc} who calculated the cross sections for trident production in all possible flavour combinations and estimated the number of events expected for the DUNE and SHiP experiments. They used the Equivalent Photon Approximation (EPA)~\cite{Belusevic:1987cw} to compute the cross section in the coherent and diffractive regimes of the scattering. The EPA, however, is known to breakdown for final state electrons~\cite{Kozhushner:1962aa, Shabalin:1963aa, Czyz:1964zz} leading, as we will demonstrate here, to an overestimation of the cross section that in some cases is by more than 200\%. 

In this work, we present a unified treatment of the coherent and diffractive trident calculation beyond the EPA for all modes. We then compute the number and distribution of events expected in each mode at various near detectors, devoting particular attention to the case of liquid argon (LAr) detectors, as they are expected to lead the field of precision neutrino scattering measurements over the next few decades thanks to their excellent tracking and calorimetry capabilities. Finally, we address the likely backgrounds that may hinder these experimental searches --- a question that we believe to be of utmost importance given the rarity of the process, and one that has been omitted in earlier sensitivity studies \cite{Magill:2016hgc,Altmannshofer:2014pba}. 

This paper is organized as follows. In Sec.~\ref{sec:xsec}, we explain how to correctly calculate the trident SM cross sections, comparing our results to the EPA and explicitly showing the breakdown of this approximation. In Sec.~\ref{sec:LAr}, we discuss the trident event rates and kinematic distributions at the near detectors of several present and future neutrino oscillation experiments based on LAr technology: the three detectors of the Short-Baseline Neutrino (SBN) Program at Fermilab~\cite{SBNproposal} and the near detector for the long-baseline Deep Underground Neutrino Experiment (DUNE)~\cite{Acciarri:2016ooe,DUNECDRvolII}, also located at Fermilab. We also consider the potential gains from an optimistic future facility: a 100 t LAr detector subject to the novel low-systematics neutrino beam of the Neutrinos from STORed Muons ($\nu$STORM) project~\cite{Soler:2015ada,nuSTORM2017}. 
%
We discuss the sources of background events at these facilities, providing a GENIE-level analysis \cite{Andreopoulos2009} of how to reduce these backgrounds and assessing the impact they are expected to have on the trident measurement. 
%
In Sec.~\ref{sec:others}, we discuss other near detectors that use more conventional technologies: the Interactive Neutrino GRID (INGRID)~\cite{Abe:2011xv,Abe:2015biq,Abe:2016fic,Abe:2016tez}, the on-axis iron near detector for T2K at J-PARC, as well as three detectors at the Neutrino at the Main Injector (NuMI) beamline at Fermilab, the one for the Main INjector ExpeRiment $\nu$-A (MINER$\nu$A)~\cite{Altinok:2017xua,MINERvA:2017} and the near detectors for the Main Injector Oscillation Search (MINOS)~\cite{Adamson:2014pgc,AlpernBoehm} and the Numi Off-axis $\nu_e$ Appearance (NO$\nu$A) experiment~\cite{Wang:Biao,sanchez_mayly_2018_1286758}. 
%
Finally, in Sec.~\ref{sec:conc} we  present our last remarks and conclusions.

%%%%%%%%%%%%%%%%%%%%%%%%%%%%%%%%%%%%%%%%%%%%%%%%%%%%
\section{Trident Production Cross Section}\label{sec:xsec}

In this section we consider neutrino trident production in the SM, defined as the process where a (anti)neutrino scattering off a hadronic system ${\cal H}$ produces a pair of same-flavour or mixed flavour charged leptons 
%
\begin{equation}
\pbar{\nu}_{\alpha}(p_1) \,+\, {\cal H}(P) \,\to\, \pbar{\nu}_{\alpha\, {\rm or}\, \kappa(\beta)} (p_2) \,+\, \ell_\beta^- (p_4)  \,+\, \ell_\kappa^+ (p_3) \,+\, {\cal H}(P^\prime),
\label{eq:indices}
\end{equation}
%
where $\beta (\kappa)$ corresponds to the flavour index of the negative (positive) charged lepton in both neutrino and antineutrino cases. 
%
Neutrino trident scattering can be divided into three regimes depending on the nature of the hadronic target: coherent, diffractive and deep inelastic, when the neutrino scatters off the nuclei, nucleons and quarks, respectively. At the energies relevant for neutrino oscillation experiments, the deep inelastic scattering contribution amounts at most to 1\% of the total trident production cross section \cite{Magill:2016hgc} and we will not consider it further.

The cross section for trident production has been calculated before in the literature, both in the context of the $V-A$ theory~\cite{Czyz:1964zz,Lovseth:1971vv,Fujikawa:1971nx} and in the SM~\cite{Brown:1973ih}, while the EPA treatment was developed in Refs.~\cite{Kozhushner:1962aa,Shabalin:1963aa,Belusevic:1987cw}. Most calculations have focused on the coherent channels \cite{Czyz:1964zz,Lovseth:1971vv,Fujikawa:1971nx,Brown:1973ih,Belusevic:1987cw} but the diffractive process has been considered in \cite{Czyz:1964zz,Lovseth:1971vv}. More recently, calculations using the EPA have been performed for coherent scattering with a dimuon final-state \cite{Altmannshofer:2014pba}, and for all combinations of hadronic targets and flavours of final-states in \cite{Magill:2016hgc}. While the EPA is expected to agree reasonably well with the full calculation for coherent channels with dimuon final-states, the assumptions of this approximation are invalid for the coherent process with electrons in the final-state \cite{Kozhushner:1962aa,Shabalin:1963aa,Czyz:1964zz}. 
%
For this reason, we perform the full $2\to 4$ calculation without the EPA in a manner applicable to any hadronic target, following a similar approach to Refs.~\cite{Czyz:1964zz,Lovseth:1971vv}. Our treatment of the cross section allows us to quantitatively assess the breakdown of the EPA in both coherent and diffractive channels for all final-state flavours, an issue we come back to in Sec.~\ref{sec:EPAbreakdown}.

We write the total cross section for neutrino trident production off a nucleus ${\cal N}$ with $Z$ protons and $(A-Z)$ neutrons as the sum
%
\begin{equation}
\sigma_\mathrm{\nu {\cal N}} = \sigma_\mathrm{\nu c} +\sigma_\mathrm{\nu d}\, ,
\end{equation}
where $\sigma_\mathrm{\nu c}$ ($\sigma_\mathrm{\nu d}$) is the coherent (diffractive) part of the cross section. 
%
%%%%%%%%%%%% SM DIAGRAMS %%%%%%%%%%%%%%%%
\unitlength = 0.9mm
\begin{figure}[t]
% \centering\includegraphics[width=\textwidth]{tridentSM/figs/SM-trident.pdf}
\centering\includegraphics[width=\textwidth]{tridentSM/figs/Neutrino_trident_production.pdf}
\caption{Diagrams contributing to the neutrino trident process in the four-point interaction limit of the Standard Model.  
\label{fig:Tdiagrams}}
\end{figure}

%%%%%%%%%%%%%%%%%%%%%%%%%%%%%%%%%%%%%%%%%
%
The relevant diagrams for these processes in the coherent or diffractive 
regimes involve the boson $Z^0$, $W$ or both mediators, depending on the particular mode. In the four-point interaction limit, depicted in \reffig{fig:Tdiagrams}, these reduce to only two contributions\footnote{An additional diagram involving a $WW\gamma$ vertex has also been neglected, since it is of order $1/M_W^4$.}, one where the photon couples to the negatively and one to the positively charged lepton. 
In Table \ref{tab:tridentmodes} we present the processes that will be considered in this work as well as the SM contributions present in each. Although our formalism applies also to processes with final-state $\tau$ leptons, the increased threshold makes them irrelevant for the experiments of interest in this study and we do not consider them further.
%
The trident amplitude for a coherent (${\rm X=c}$) or diffractive (${\rm X=d}$) scattering regime can be written as
%
\begin{equation}
i \mathcal{M} = \mathrm{L}^\mu (\{p_i\},q) \, \frac{-ig_{\mu \nu}}{q^2} \, \mathrm{H}_{\rm X}^{\nu}(P,P^{\prime})\, ,
\end{equation}
%
where $\{p_i\}=\{p_2,p_3,p_4\}$ is the set of outgoing leptonic momenta. $ \mathrm{L}^\mu (\{p_i\},q)$ is the total leptonic amplitude 
%
\begin{align}
\mathrm{L}^\mu & \equiv - \frac{ie G_F}{\sqrt{2}}[\bar{u}(p_2)\gamma^\tau(1-\gamma_5)u(p_1)] \times 
\bar{u}(p_4)\left[\gamma_\tau(V_{\alpha\beta\kappa}-A_{\alpha\beta\kappa}\gamma_5)\frac{1}{(\slashed{q}-\slashed{p}_3-m_3)}\gamma^\mu \right . \nonumber \\ 
& \left . + \gamma^\mu \frac{1}{(\slashed{p}_4-\slashed{q}-m_4)} \gamma_\tau (V_{\alpha\beta\kappa}-A_{\alpha\beta\kappa}\gamma_5) \right] v(p_3)\, ,
\label{eq:Lmu}
\end{align}
%
and $\mathrm{H}_{\rm X}^{\nu}(P,P^{\prime})$ is the total hadronic amplitude
\begin{align}
H_{\rm X}^\nu  &\equiv \langle {\cal H}(P) \vert J_\mathrm{E.M.}^\nu (q^2)\vert {\cal H}(P^\prime)\rangle\, ,
\label{eq:Hmu}
\end{align}
%
with  $q \equiv P - P^\prime$ denoting the transferred momentum, $m_3$ ($m_4$) the positively (negatively) charged lepton mass, $V_{\alpha\beta\kappa}(A_{\alpha\beta\kappa})\equiv g_{V}^{\beta}(g_A^{\beta})\delta_{\beta\kappa}+\delta_{\alpha\beta} \,(\beta=\alpha \, \mathrm{or} \; \kappa)$ the vector (axial) couplings, depending on the channel and have labels in accordance to Eq. (\ref{eq:indices}), and ${J}^\nu_{\rm{E.M.}} (q^2)$ the electromagnetic current for the hadronic system ${\cal H}$ (a nucleus or a nucleon).
%
\begin{table}[t]
\begin{center}
\begin{tabular}{|cc|}
\toprule\toprule
\bf (Anti)Neutrino &  \bf SM Contributions \\
\midrule\midrule
$\pbar{\nu}_\mu\, {\cal H} \to \pbar{\nu}_\mu\, \mu^- \mu^+\,  {\cal H}$  & CC + NC \\
$\pbar{\nu}_\mu\, {\cal H} \to \pbar{\nu}_e\,  e^\pm \mu^\mp\, {\cal H}$  & CC\\
$\pbar{\nu}_\mu\, {\cal H} \to \pbar{\nu}_\mu\,  e^- e^+\, {\cal H}$ & NC\\
$\pbar{\nu}_e\, {\cal H} \to \pbar{\nu}_e\,  e^- e^+\, {\cal H}$ &  CC + NC \\
$\pbar{\nu}_e\, {\cal H} \to \pbar{\nu}_\mu\,  \mu^\pm e^\mp\, {\cal H}$ & CC \\
$\pbar{\nu}_e\, {\cal H} \to \pbar{\nu}_e\,  \mu^- \mu^+\, {\cal H}$ & NC \\
\bottomrule
\bottomrule
\end{tabular}
\end{center}
\caption{\label{tab:tridentmodes} (Anti)Neutrino trident processes considered in this paper.}
\end{table}
%

We can write the differential cross section as
\begin{align}
\frac{\dd^2 \sigma_{\nu  {\rm X}}}{\dd Q^2 \dd \hat{s}}= \frac{1}{32  \pi^2(s-M_{\cal H}^2)^2}\frac{\mathrm{H}_{\rm X}^{\mu\nu}\mathrm{L}_{\mu\nu}}{Q^4}\, ,
\end{align}
where $s = (p_1 + P)^2$, $\hat{s} \equiv 2\,(p_1 \vdot q)$, $Q^2 = -q^2$ and $M_{\cal H}$ is the mass of the hadronic target. We have also introduced the hadronic tensor $\mathrm{H}_{\rm X}^{\mu \nu}$
%
\begin{align}
\mathrm{H}_{\rm X}^{\mu\nu} &\equiv \overline{\sum_{\rm{spins}}}  \left(\mathrm{H}_{\rm X}^\mu\right)^* \mathrm{H}_{\rm X}^\nu.
%
\end{align}
The two scattering regimes in which the hadronic tensor is computed will be discussed in more detail in Sec.~\ref{sec:had}. The leptonic tensor, $\mathrm{L}^{\mu \nu}$, integrated over the phase space of the three final-state leptons, $\dd^{3} \Pi \left(p_1 + q; \{p_i\}\right)$, and merely summed over final and initial spins is given by
%
\begin{equation}
\mathrm{L}^{\mu \nu} (p_1, q) \equiv  \int \dd ^{3} \Pi \left(p_1 + q; \{p_i\}\right) \left( \sum_{\rm{spins}} \left(  \mathrm{L}^\mu \right)^*  \mathrm{L}^\nu  \right)\, .
\label{eq:Lmunu}
\end{equation}
We can use $\mathrm{L}^{\mu \nu}$ to define two scalar functions, one related to the longitudinal ($\mathrm{L}_{\mathrm{L}}$) and the other to the transverse ($\mathrm{L}_{\mathrm{T}}$) polarization of the exchanged photon
\begin{equation}
\mathrm{L}_{\mathrm{T}} = -\frac{1}{2}\left( g^{\mu \nu} - \frac{4Q^2}{\hat{s}^2} p_1^\mu p_1^\nu \right) \mathrm{L}_{\mu \nu}, \quad \mathrm{and} \quad \mathrm{L_{L}} =  \frac{4Q^2}{\hat{s}^2} p_1^\mu p_1^\nu \mathrm{L}_{\mu \nu}\, .\label{eq:LT_LL}
\end{equation}
%
This allows us to write the differential cross section as a sum of a longitudinal and a transverse contribution \cite{Hand:1963bb} as follows
%
\begin{align}
\frac{\dd^2 \sigma_{\nu  {\rm X}}}{\dd Q^2 \dd \hat{s}} &= \frac{1}{32 \pi^2} \frac{1}{\hat{s}\,Q^2} \left [ h_{\rm X}^\mathrm{T}(Q^2, \hat{s}) \, \sigma^\mathrm{T}_{\nu \gamma} (Q^2, \hat{s}) + h_{\rm X}^\mathrm{L}(Q^2, \hat{s}) \, \sigma^\mathrm{L}_{\nu \gamma} (Q^2, \hat{s}) \right] \, ,\label{eq:full_diff_xsec}
\end{align}
%
where we have defined two functions for the flux of longitudinal and transverse virtual photons 
%
\begin{subequations}
\label{eq:splitting_function}
\begin{align}
%
h_{\rm X}^\mathrm{T}(Q^2, \hat{s})  &\equiv \frac{2}{(E_\nu M_{\cal H})^2} \left[ p_{1 \mu} p_{1\nu} -\frac{\hat{s}^2}{4 Q^2} \, g_{\mu \nu} \right]\mathrm{H}_{\rm X}^{\mu \nu} , \quad \text{and} \\ \quad
h_{\rm X}^\mathrm{L}(Q^2, \hat{s})  &\equiv \frac{1}{(E_\nu M_{\cal H})^2} \, p_{1\mu} p_{1\nu}\, \mathrm{H}_{\rm X}^{\mu \nu}\, ,
%
\end{align}
%
\end{subequations}
%
and two leptonic neutrino-photon cross sections associated with them\footnote{Note that we include a factor of $1/2$ in $\sigma^\mathrm{T}_{\nu \gamma}$ to match the polarization averaging of the on-shell cross section: $\sigma_{\nu \gamma}^{\rm on-shell} = \frac{1}{2 \hat{s}} \left( \overline{\sum}_r (\epsilon_r^\mu)^* \epsilon^\nu_r \, {\rm L}_{\mu\nu} \right) \big\vert_{Q^2=0} = \frac{1}{4 \hat{s}} \left( - g^{\mu\nu} L_{\mu\nu}\right) \big\vert_{Q^2=0} = \frac{{\rm L_T}}{2 \hat{s}}\big\vert_{Q^2=0} = \sigma_{\nu\gamma}^\text{T}(0,\hat{s})$.}
%
\begin{equation}
\sigma^\mathrm{T}_{\nu \gamma} (Q^2, \hat{s})  = \frac{\mathrm{L_T}}{2 \hat{s}}\, , \quad \mathrm{and} \quad
\sigma^\mathrm{L}_{\nu \gamma} (Q^2, \hat{s})  = \frac{\mathrm{L_L}}{\hat{s}}\, .
\end{equation}
%
The kinematically allowed region in the $(Q^2,\hat{s})$ plane can be obtained by considering the full four-body phase space, as in~\cite{Czyz:1964zz,Lovseth:1971vv,Fujikawa:1971nx}. The limits for such physical region are given by
\begin{subequations}\label{eq:qslimts}
	\begin{align}
		Q_{\rm min}^2&=\frac{M_{\cal H} \hat{s}^2}{2E_\nu(2E_\nu M_{\cal H}-\hat{s})},&\  
        Q_{\rm max}^2&=\hat{s}-m_L^2,\label{eq:qlimts}\\
        \hat{s}_{\rm min}&=\frac{E_\nu}{2E_\nu + M_{\cal H}}\left[m_L^2+2E_\nu M_{\cal H} -\Delta\right]&\  
        \hat{s}_{\rm max}&=\frac{E_\nu}{2E_\nu + M_{\cal H}}\left[m_L^2+2E_\nu M_{\cal H} +\Delta\right],\label{eq:shatlimts}
	\end{align}
\end{subequations}
%
with $m_L\equiv m_3+m_4$, and
%
\begin{align*}
	\Delta \equiv \sqrt{(2E_\nu M_{\cal H}-m_L^2)^2-4M_{\cal H}^2 m_L^2}\,.
\end{align*}
%
Let us emphasize that \refeq{eq:full_diff_xsec} is an exact decomposition, and does not rely on any approximation of the process. In the following section, we will show how to calculate the flux functions $h_{\rm X}^\mathrm{T}$ and $h_{\rm X}^\mathrm{L}$ from Eq.~\ref{eq:splitting_function} in different scattering regimes. The total cross section for the process can then be computed by finding $\sigma^\mathrm{L}_{\nu \gamma}$ and $\sigma^\mathrm{T}_{\nu \gamma}$ from Eqs.~(\ref{eq:Lmu}), (\ref{eq:Lmunu}) and (\ref{eq:LT_LL}) and integrating over all allowed values of $Q^2$ and $\hat{s}$. Note that $\sigma^\mathrm{L}_{\nu \gamma}$ and $\sigma^\mathrm{T}_{\nu \gamma}$ are universal functions for a given leptonic process and need only to be computed once.

%%%%%%%%%%%%%%%%%%%%%%%%%%%%%%%%%%%%%%%%%%%%%%%%%%%%%%%%%%%%%%
\subsection{Hadronic Scattering Regimes}\label{sec:had}

Depending on the magnitude of the virtuality of the photon, $Q = \sqrt{-q^2}$, the hadronic current can contribute in different ways to the trident process. Thus, given the decomposition in \refeq{eq:full_diff_xsec}, the change in the hadronic treatment translates to computing the flux factors $h_{\rm X}^\mathrm{T}$ and $h_{\rm X}^\mathrm{L}$ for each scattering regime.  From those flux factors, $\sigma_{\nu\mathrm{c}}$ and $\sigma_{\nu\mathrm{d}}$ can be calculated.

%%%%%%%%%%%%%%%%%%%%%%%%%%%%%%%%%%%%%%%%%%%%%%%%%%
\subsubsection{Coherent Regime (${\rm H}^{\mu \nu}_{\rm c}$)}

In the coherent scattering regime the incoming neutrino interacts with the whole nucleus without resolving its substructure. For this to occur frequently, we need small values of $Q$. Despite the relatively large neutrino energies in contemporary neutrino beams, this is still allowed for trident.

In this regime, the hadronic tensor $\mathrm{H}^{\mu\nu}_\mathrm{c}$ for a ground state spin-zero nucleus of charge $Z e$ can be written in terms 
of the nuclear electromagnetic form factor $F(Q^2)$, discussed in more detail in Appendix~\ref{app:formfactors}, as
%
\begin{equation}
\mathrm{H}^{\mu \nu}_\mathrm{c} =  4Z^2 e^2 \left| F (Q^2)\right|^2 \left(P^\mu - \frac{q^\mu}{2}\right) \left(P^\nu - \frac{q^\nu}{2}\right).
\end{equation}
%
From Eq.~\ref{eq:splitting_function}, we find that the transverse and longitudinal flux functions for the coherent regime are
%
\begin{subequations}\label{eq:hcoh}
\begin{align}
h^\mathrm{T}_\mathrm{c}(Q^2, \hat{s})  &=  8 Z^2 e^2   \left(1 - \frac{\hat{s}}{2E_\nu M} - \frac{\hat{s}^2}{4 E_\nu^2 Q^2}\,\right) |F (Q^2)|^2\, , \\
h^\mathrm{L}_\mathrm{c}(Q^2, \hat{s})  &=  4 Z^2 e^2  \left(1 - \frac{\hat{s}}{4E_\nu M}\right)^2 |F (Q^2)|^2\, ,
\end{align}
\end{subequations}
where $E_\nu$ is the energy of the incoming neutrino and $M$ is the nuclear mass. For a fixed value of $\hat{s}$ in the physical region, the $h^{\rm T}_{\rm c}$ flux function becomes zero at $Q_{\rm min}$ while the longitudinal component does not. This different behaviour can be seen explicitly in their definitions, Eqs. \eqref{eq:hcoh}, as the terms in the parenthesis in $h^{\rm T}_{\rm c}$ cancel each other at $Q_{\rm min}$. This does not occur for $h^{\rm L}_{\rm c}$ since the physical values of $\hat{s}$ are always smaller than $E_\nu M$ in this hadronic regime. 
Due to this fact, $Q_{\rm min}$, which according to  Eq.\ \eqref{eq:qlimts} depends on  both the  neutrino energy and target material, can be approximated to
\begin{align*}
	Q_{\rm min} \approx \frac{\hat{s}}{2E_{\nu}},
\end{align*}
which only depends on the incoming neutrino energy. On the other hand, as $Q$ becomes large, the flux functions $h^{T,L}$ become quite similar, $h^{\rm T}_{\rm c}\approx 2 h^{\rm L}_{\rm c}$, and favour small values of $\hat{s}$. After some critical value of the virtuality $Q$, $h^{\rm T, L}_{\rm c}$ become negligible due to the nuclear form factor. The $Q$ value at which this happens depends on the target material, but not on the incoming neutrino energy. For instance, in the case of an Ar target
the flux functions basically vanish for $Q\gtrsim 250$ MeV.

The final cross sections for coherent neutrino trident production on Argon can be seen in \reffig{fig:coh_xsec}. Despite thresholds being important for the behaviour of these cross sections for GeV neutrino energies, we can see that mixed channels quickly become the most important due to their CC nature. At large energies one can then rank the cross sections from largest to smallest as CC, CC+NC, and NC only channels. Nevertheless, one must be aware of the fact that the cross sections are dominated by low $Q^2$ even at large energies, leading to large effects due to the final-state lepton masses as discussed in \cite{Magill:2016hgc}.

\begin{figure}[t]
\centering
\includegraphics[width=\textwidth]{figs/Xsec_4PS_coh.pdf}
\caption{Cross sections for coherent neutrino trident production on $^{40}$Ar (left) and  $^{208}$Pb (right) normalized to $\sigma_0 =  Z^2\, 10^{-44}$ cm$^2$. The full (dashed) lines correspond to the scattering of an incoming $\nu_\mu$ ($\nu_e$) produced by the NC (light-blue), CC (purple), and CC+NC (orange) SM interactions. \label{fig:coh_xsec}}
\end{figure}


%%%%%%%%%%%%%%%%%%%%%%%%%%%%%%%%%%%%%%%%%%%%%%%%%%%%
\subsubsection{Diffractive Regime ($\mathrm{H}^{\mu\nu}_\mathrm{d}$)}

At larger $Q^2$, the neutrino interacts with the individual nucleons of the nucleus. In this diffractive regime $\mathrm{H}^{\mu\nu}_\mathrm{d}$ is given by the sum of the contributions of the two types of nucleons: protons ($\mathrm{N=p}$) and neutrons ($\mathrm{N=n}$), so
\begin{equation}
\mathrm{H}^{\mu \nu}_\mathrm{d} (P, P^\prime) = Z\, \mathrm{H}^{\mu \nu}_\mathrm{p} (P, P^\prime)+
(A-Z)\,\mathrm{H}^{\mu \nu}_\mathrm{n}(P, P^\prime)\, ,
\end{equation}
where each $\mathrm{H}^{\mu \nu}_\mathrm{N}$ is the square of the matrix element of the nucleon electromagnetic current summed over final and averaged over initial spins. Neglecting second class currents, the matrix elements take the form
\begin{equation}
\bra{\mathrm{N}(P^\prime)} {J}^\mu_{\rm{E.M.}} (Q^2) \ket{\mathrm{N} (P) } = e \, \overline{u}_\mathrm{N} (P^\prime) \left[ \gamma^\mu F^\mathrm{N}_1(Q^2) - i \frac{\sigma^{\mu \nu} q_{\nu}}{2 M_{\rm N}} F^\mathrm{N}_2(Q^2) \right] u_\mathrm{N}(P)\, ,
\end{equation}
%
with $F^\mathrm{N}_{1,2}(Q^2)$ the Dirac and Pauli form factors, respectively. The hadronic tensors are then given by \cite{Kniehl:1990iv}
%
\begin{equation}
\mathrm{H}^{\mu \nu}_\mathrm{N} = e^2 \left[ 4 \, H_1^\mathrm{N}(Q^2) \left(P^\mu - \frac{q^\mu}{2}\right)\left(P^\nu - \frac{q^\nu}{2}\right) - H_2^\mathrm{N}(Q^2) \left( Q^2 g^{\mu \nu} + q^\mu q^\nu \right) \right]\, ,
\end{equation}
%
where the  $H_1^\mathrm{N}(Q^2)$ and $H_2^\mathrm{N}(Q^2)$ form factors, functions of $F^\mathrm{N}_{1,2}(Q^2)$, are given in Appendix~\ref{app:formfactors}. The flux functions in the diffractive regime can then be calculated as
\begin{subequations}\label{eq:dcoh}
\begin{align}
h^\mathrm{T}_\mathrm{N}(Q^2, \hat{s})  &=  8 \, e^2 \left[ \left(1 - \frac{\hat{s}}{2E_\nu M_{\rm N}} - \frac{\hat{s}^2}{4 E_\nu^2 Q^2 }\,\right) H_1^\mathrm{N}(Q^2) + \frac{\hat{s}^2}{8E_\nu^2 M_{\rm N}^2}  H_2^\mathrm{N}(Q^2)\right ]\, ,\label{eq:hTdiff}\\
%
h^\mathrm{L}_\mathrm{N}(Q^2, \hat{s})  &=  4 e^2 \, \left[ \left(1-\frac{\hat{s}}{4 E_\nu M_{\rm N}} \right)^2 H_1^\mathrm{N}(Q^2)  - \frac{\hat{s}^2}{16 E_\nu^2 M_{\rm N}^2} H_2^\mathrm{N}(Q^2)\right]\, .\label{eq:hLdiff}
\end{align}
\end{subequations}
%
In the case of the proton, the flux functions $h^{\rm T, L}_{\rm p}$ have some unique features given the presence of both electric and magnetic contributions. Specifically, the transverse function is non-zero at $Q=Q_{\rm min}$ for a fixed $\hat{s}$, due to the additional term proportional to $H_2^{\rm p}$. Indeed, for large values of $\hat{s}$, the $H_2^{\rm p}$ term dominates the transverse function. An opposite behaviour occurs for the longitudinal component. There, the $H_1^{\rm p}$ term dominates over the second term for all physical values of $\hat{s}$, $Q$, and for any incoming neutrino energy. On the other hand, the flux functions of the neutron, which have only 
the magnetic moment contribution, have somewhat different characteristics. While 
$h^{\rm T}_{\rm n}$ behaves similarly to $h^{\rm T}_{\rm p}$, that is, it is dominated by the second term for large values of $\hat{s}$, $h^{\rm L}_{\rm n}$ is zero at $Q_{\rm min}$ due to the exact cancellation between the $H_{1,2}^{\rm n}$ terms. This cancellation is not evident from Eq.\ ~\eqref{eq:hLdiff}; however, simplifying the longitudinal component for the neutron case, one finds
\begin{align*}
	h^\mathrm{L}_\mathrm{n}(Q^2, \hat{s})  &=4 e^2 \left(1+\frac{Q^2}{4M_{\rm n}^2}\right)\frac{Q^2}{4 M_{\rm N}^2}\left( 1 - \frac{\hat{s}}{2 E_\nu M_{\rm N}} - \frac{\hat{s}^2}{4 E_\nu^2 Q^2} \right) \left| F^\mathrm{n}_2(Q^2) \right|^2,
\end{align*}
which is zero for $Q=Q_{\rm min}$. Also, this shows why $h^{\rm L}_{\rm p}$ does not 
vanish at $Q_{\rm min}$ since there we have the additional contribution of the electric component. 

When the neutrino interacts with an individual nucleon inside the nucleus, one must be aware of the nuclear effects at play. One such effect is Pauli blocking, a suppression of neutrino-nucleon interactions due to the Pauli exclusion principle. Modelling the nucleus as an ideal Fermi gas of protons and neutrons, one can take Pauli blocking effects into account by requiring that the hit nucleon cannot be in a state which is already occupied \cite{Brown:1971qr}. This requirement is implemented in our calculations by a simple replacement of the differential diffractive cross section
\begin{align*}
\frac{\dd^2 \sigma_{\nu  \mathrm{d}}}{\dd Q^2 \dd \hat{s}}\to f (|\vec{q}|) \, \frac{\dd^2 \sigma_{\nu  \mathrm{d}}}{\dd Q^2 \dd \hat{s}},
\end{align*}
where $|\vec{q}|$ is the magnitude of the transferred three-momentum in the lab frame. In particular, following \cite{Brown:1971qr}, assuming an equal density of neutrons and protons, we have
%
\begin{equation}
f (|\vec{q}|) = \begin{cases} \displaystyle
                    \frac{3}{2} \frac{|\vec{q}|}{2 \, k_F} - \frac{1}{2} \left( \frac{|\vec{q}|}{2 \, k_F} \right)^3 ,\, &\mathrm{if }\;\; |\vec{q}| < 2\, k_F\, ,\\
                    1,\, &\mathrm{if }\;\; |\vec{q}| \geq 2 \, k_F\, ,
                \end{cases}
\end{equation}
%
where $k_F$ is the Fermi momentum of the gas, taken to be $235$ MeV. This is a rather low value of $k_F$ and the assumption of equal density of neutrons and protons must be taken with care for heavy nuclei. We refrain from trying to model any additional nuclear effects as we believe that this is the dominant effect on the total diffractive rate, particularly when requiring no hadronic activity in the event. The net result is a reduction of the diffractive cross section by about $50\%$ for protons and $20\%$ for neutrons. Unless clearly stated otherwise, we always include Pauli blocking in our calculations.

Our final cross sections for this regime can be seen in \reffig{fig:dif_xsec}. One can clearly see that the neutron contribution is subdominant, and that, up to factors of $Z^2$, the proton one is comparable to the coherent cross section. Note that now the typical values of $Q^2$ are much larger than in the coherent regime and the impact of the final-state lepton masses is much smaller. 

\begin{figure}[t]
\centering\includegraphics[width=\textwidth]{figs/Xsec_4PS_diff.pdf}
\caption{Cross sections for diffractive neutrino trident production on neutrons (left) and protons (right), including Pauli blocking effects as described in the text, normalized to $\sigma_0 =  10^{-44}$ cm$^2$. The full (dashed) lines correspond to the scattering of an incoming $\nu_\mu$ ($\nu_e$) produced by the NC (light-blue), CC (purple), and CC+NC (orange) SM interactions. \label{fig:dif_xsec}}
\end{figure}


%%%%%%%%%%%%%%%%%%%%%%%%%%%%%%%%%%%%%%%%%%%%%%%%%%%%
\subsection{Breakdown of the EPA \label{sec:EPAbreakdown}}

In order to understand the breakdown of the EPA in the neutrino trident case, let us first remind briefly the reader about the Weizs\"acker--Williams method of equivalent photons in Quantum Electrodynamics (QED)~\cite{vonWeizsacker:1934nji,Williams:1934ad}, and the main reason for its validity in that theory. The EPA, first introduced by E.\ Fermi~\cite{Fermi:1924tc}, is based on a simple principle: when an ultra-relativistic particle $P_i$ approaches a charged system $C_s$, like a nucleus, it will perceive the electromagnetic fields as nearly transverse, similar to the fields of a pulse of radiation, {\it i.e.},  as an on-shell photon. Therefore, it is possible to obtain an approximate total cross section for the inelastic scattering process producing a set of final particles $P_f$, $\sigma_{\rm t}(P_i + C_s \to P_f + C_s)$, by computing the scattering of the incoming particle with a real photon integrated over the energy spectrum of the off-shell photons,
%
\begin{align}
	\sigma_{\rm t}(P_i + C_s \to P_f + C_s)\approx\int\, dP(Q^2,\hat{s})\,\sigma_\gamma(P_i + \gamma \to P_f; \hat{s}, Q^2 = 0),
\end{align}
where the photo-production cross section for the process $P_i + \gamma \to P_f$, 
$\sigma_\gamma(P_i + \gamma \to P_f; \hat{s}, Q^2 = 0)$, 
depends on the center-of-mass energy of the $P_i$--photon system, $\sqrt{\hat{s}}$. Here $dP(Q^2,\hat{s})$ corresponds to the energy spectrum of the virtual photons, that is, the probability of emission of a virtual photon with transferred four-momentum $Q^2$ resulting in an center-of-mass energy $\sqrt{\hat{s}}$.
For trident scattering off a nuclear target, this probability can be approximated by~\cite{Belusevic:1987cw,Altmannshofer:2014pba}
\begin{align}\label{eq:GenEPA}
	dP(Q^2,\hat{s})=\frac{Z^2e^2}{4\pi^2}|F (Q^2)|^2\,\frac{d\hat{s}}{\hat{s}}\,\frac{dQ^2}{Q^2}\, .
\end{align}
A crucial fact in QED is that the cross section $\sigma_\gamma^{\rm QED}(P_i + \gamma \to P_f; \hat{s},0)$ is inversely proportional to $\hat{s}$,
\begin{align*}
	\sigma_\gamma^{\rm QED}(P_i + \gamma \to P_f; \hat{s},0) \propto \frac{1}{\hat{s}}\,.
\end{align*}
We see clearly that small values of $\hat{s}$ and consequently of the transferred four-momentum $Q^2$ dominate the cross section. Hence, the on-shell contribution is much more significant 
than the off-shell one, so the EPA will be valid and give the correct cross  section 
estimate for any QED process. 

Now, let us consider the case of neutrino trident production. In this case, the equivalent-photon cross section in the four-point interaction limit has a completely opposite dependence on the center-of-mass energy; it is \emph{proportional} to $\hat{s}$,
\begin{align*}
	\sigma_\gamma^{\rm FL}(P_i + \gamma \to P_f; \hat{s}, 0)\propto G_{\rm F}^2\, \hat{s}\, .
\end{align*}
This dependence is a manifestation of the unitarity violation in the Fermi theory. Therefore, we can see that for weak processes larger values of $\hat{s}$, and, consequently, larger values of $Q^2$ are more significant \cite{Kozhushner:1962aa, Shabalin:1963aa}.  The EPA is then generally not valid for the neutrino trident production, as the virtual photon contribution dominates over the real one. Nevertheless, one may wonder if there is a situation  in which the EPA can give a reasonable estimate for a neutrino trident process. 
As noticed in the early literature \cite{Kozhushner:1962aa, Shabalin:1963aa}, the presence of the nuclear form factor introduces a cut in the transferred momentum which, in turn, makes the EPA applicable for the specific case of the dimuon channel in the coherent regime. Let us discuss this in more detail. 

Recalling our exact decomposition, \refeq{eq:full_diff_xsec}, it is necessary to consider two assumptions for implementing the EPA \cite{Kozhushner:1962aa}:
%
\begin{enumerate} 
%
\item The longitudinal polarization contribution to the cross section can be neglected, i.e., $\sigma_{\nu\gamma}^\mathrm{L}(Q^2,\hat{s})\approx 0$;
%
\item The transverse polarization contribution to the cross section can be taken to be on-shell, i.e., $\sigma^\text{T}_{\nu\gamma}(Q^2,\hat{s}) \approx \sigma^\text{T}_{\nu\gamma}(0,\hat{s})$. 
%
\end{enumerate}
%
Assuming for now that these approximations hold, we can find a simplified expression for the coherent neutrino-target process, described by Eqs.~(\ref{eq:full_diff_xsec}) and (\ref{eq:hcoh}), in terms of the photon-neutrino cross section\footnote{An analogous expression can be obtained for the diffractive regime from Eq.~(\ref{eq:dcoh}).}:
%
\begin{align}     
\sigma_\text{EPA} = \frac{Z^2e^2}{4\pi^2}\int_{m_L^2}^{\hat{s}_{\rm max}} \frac{d\hat{s}}{\hat{s}}\,
\sigma^\mathrm{T}_{\nu\gamma}(0,\hat{s})
\int_{(\hat{s}/2E_\nu)^2}^{Q^2_{\rm max}}\frac{|F (Q^2)|^2}{Q^4} \left[ Q^2(1-y) - M_{\cal H}^2y^2\right]dQ^2\, , 
\end{align}
%
where we introduced the fractional change of the nucleus energy $y$, defined as $\hat{s} = (s-M_{\cal H}^2)y$, and the integration limits can be obtained from \eqref{eq:qslimts} after considering that $m_L^2\ll E_\nu M_{\cal H}$. Keeping only the leading terms in the small parameter $y$ \cite{Belusevic:1987cw}, we recover the EPA applied to the neutrino trident case
%
\begin{align} \label{eq:EPA_bad}
\sigma_\text{EPA} = \int \sigma^\mathrm{T}_{\nu\gamma}(0,\hat{s}) \, dP(Q^2,\hat{s})\, ,
\end{align}
%
where $dP(Q^2,\hat{s})$ is given in Eq.~(\ref{eq:GenEPA}). The EPA in the form of \refeq{eq:EPA_bad} has been used in trident calculations for the coherent dimuon channel \cite{Altmannshofer:2014pba} as well as for coherent mixed- and electron-flavour trident modes and diffractive trident modes \cite{Magill:2016hgc}.  Using our decomposition, we can explicitly compute both $\sigma^\mathrm{L}_{\nu \gamma}$ and $\sigma^\mathrm{T}_{\nu \gamma}$ and verify if the EPA conditions are satisfied for any channel and, if they are not, quantify the error introduced by making this approximation. For that purpose, we will compare the results of the full calculation, \refeq{eq:full_diff_xsec}, with the EPA results, \refeq{eq:EPA_bad}, by computing the following ratios in the physical region of the $(Q,\hat{s})$ plane,
\begin{align}\label{eq:ratios}
		\frac{\sigma^{\rm L}(Q^2,\hat{s})\,h_{\rm c}^{\rm L}(Q^2,\hat{s})}{\sigma^{\rm T}(Q^2,\hat{s})\,h_{\rm c}^{\rm T}(Q^2,\hat{s})}\, , \quad \frac{\sigma^\mathrm{T}_{\nu\gamma}(Q^2,\hat{s})}{\sigma^\mathrm{T}_{\nu\gamma}(0,\hat{s})}\, .
\end{align} 
The first ratio in Eq.\ \eqref{eq:ratios} will indicate where the longitudinal contribution can be neglected compared to the transverse one; while, the second ratio will show where the transverse contribution behaves as an on-shell photon. 

As an illustration of the general behaviour, we show in Fig.\ \ref{fig:4PSvsEPA} those ratios 
of cross sections for an incoming $\nu_\mu$ of fixed energy $E_\nu=3$ GeV colliding coherently with an $^{40}$Ar target, for the dielectron (left panels), mixed  (middle panels) and dimuon  (right panels)
channels. On the top panels of Fig.\ \ref{fig:4PSvsEPA} we see that the longitudinal component can be neglected for $Q\lesssim m_\alpha$, for the dielectron and dimuon channels, $\alpha=e,\mu$, while in the mixed case there is a much less pronounced hierarchy between the transverse and longitudinal components. On the bottom panels we have the comparison between on-shell and off-shell transverse photo-production cross sections. Again, we find that the EPA is only valid for $Q \lesssim m_\alpha$ for the dielectron and dimuon channels. For the mixed case, there is only a very small region in $Q < 10^{-2}$\,GeV for which the off-shell transverse cross section is comparable to the on-shell one. This relative suppression of the off-shell cross section can be understood by noticing that $Q$ enters the lepton propagators, suppressing the process for $Q \gtrsim m_\alpha$. For mixed channels it is then the smallest mass scale ($m_e$) that dictates the fall-off of the matrix element in $Q$, whilst the heaviest mass ($m_\mu$) defines the phase space boundaries, rendering most of this phase space incompatible with the EPA assumptions.   
%
\begin{figure}[t]
\centering
\includegraphics[width=\textwidth]{figs/4PS_vs_EPA.pdf}%
\caption{\label{fig:4PSvsEPA} Comparison between the full calculation of the trident production 
coherent cross section and the EPA in the kinematically allowed region of the $(Q,\hat{s})$ plane for an incoming $\nu_\mu$ with fixed energy $E_\nu=3$ GeV colliding with an $^{40}$Ar target. 
The left, middle and right panels correspond to the dielectron, mixed and dimuon final-states, respectively. The top panels correspond to the comparison between the longitudinal and transverse contributions while the bottom ones show the ratio between the transverse cross sections computed for an specific value of $Q$ with the cross section for an on-shell photon. The thick black dashed lines correspond to the cut in the $Q^2$ integration at $\Lambda_{\rm QCD}^2/ A^{2/3}$, and the shadowed region around these lines account for a variation of $20\%$ in the value of this cut. The purple dashed lines are for $Q=m_\alpha$, $\alpha=e,\mu$ for the unmixed cases.}
\end{figure}

These results explicitly show that the EPA is, in principle, not suitable for any neutrino trident process as it can overestimate the cross section quite substantially by treating the photo-production cross section at large $Q^2$ as on-shell. However, as previously mentioned, in the coherent regime the nuclear form factor introduces a strong suppression for large values of $Q^2$. In general, this dominates the behaviour of the cross sections for values of $Q^2$ smaller than the purely kinematic limit, $Q^2_{\rm max}$, and of the order of $\Lambda_{\rm QCD}/ A^{1/3}\approx 0.06$ GeV for coherent scattering on $^{40}$Ar. In the dimuon case, the latter scale happens to be smaller than the charged lepton masses, implying that the region where the EPA breaks down is heavily suppressed due to the nuclear form factor. The same cannot be said about coherent trident channels involving electrons, as the nuclear form factor suppression happens for much larger values of $Q$ than the EPA breakdown. Furthermore, for diffractive scattering the nucleon form factors suppress the cross sections only for much larger $Q$ values, $Q\approx 0.8$ GeV. The effective range of integration then includes a significant region where the EPA assumptions are invalid, leading to an overestimation of the diffractive cross section for every process regardless of the flavours of their final-state charged leptons. 

%
\begin{figure}[t]
	\centering
	\includegraphics[width=\textwidth]{figs/XSec_ratio.pdf}%
	\caption{\label{fig:comparison4PS_EPA} 
    Ratio $\mathcal{R}$ of the trident cross section calculated using 
    the EPA to the full four-body calculation. 
    Left panel: Ratio in the coherent regime on $^{40}$Ar. The full curves correspond to the central value of $Q_{\rm cut}$, and the upper (lower) boundary corresponds to a choice 100 times larger ($20\%$ smaller). 
Right  panel: Ratio in the diffractive regime for scattering on protons, where the full curves corresponds to the central value of $1.0$ GeV, and the upper (lower) boundary corresponds to a choice 100 times larger ($20\%$ smaller); we have taken the lower limit in the integration on $Q$ to match the choice of the coherent regime and we do not include Pauli blocking in these curves. A guide to the eye at $\mathcal{R} = 1$ is also shown.}
\end{figure}

In some calculations, artificial cuts have been imposed on the range of $Q^2$, affecting the validity of the EPA. In Ref. \cite{Magill:2016hgc}, it is claimed that to avoid double counting between different regimes, an artificial cut must be imposed, lowering the upper limit of integration in $Q^2$. Ref.~\cite{Magill:2016hgc} chooses a value of $Q^{\rm cut}_{\rm max} = \Lambda_{\rm QCD}/ A^{1/3}$ in the coherent regime (black thick dashed lines in Fig.\ \ref{fig:4PSvsEPA}), and $Q^{\rm cut}_{\rm min}= {\rm max}\left( \Lambda_{\rm QCD}/ A^{1/3}, \hat{s}/2E_\nu\right)$ and $Q^{\rm cut}_{\rm max} = 1.0$ GeV in the diffractive regime. We believe that no such cut is required on physical grounds\footnote{It should be noted that the coherent and diffractive regimes have different phase space boundaries and that the form factors should guarantee their independence.}, and their presence will impact the EPA cross section quite dramatically. Let us first consider the dimuon case in the coherent regime, where the EPA assumptions hold reasonably well in the relevant parts of phase space. By introducing a value for $Q^{\rm cut}_{\rm max}$ we would be decreasing the total relevant phase space for the process, reducing the total cross section. Therefore, despite the EPA tendency to overestimate the cross section in this channel, an artificial cut in $Q^2$ can actually lead to an underestimation of the cross section. In the electron channels, where the EPA breakdown is much more dramatic, we can expect that the overestimation of the cross section by the EPA is reduced by the cut $Q^{\rm cut}_{\rm max}$. In fact, one way to improve the EPA for the dielectron channel is to artificially cut on the $Q^2$ integral around the region where the ap\-pro\-xi\-ma\-tion breaks down \cite{Frixione:1993yw}. This cut does then improve the coherent EPA calculation by decreasing the overestimation of the cross section. However, an energy independent cut cannot provide a good estimate of the cross section over all values of $E_\nu$. To illustrate our point and to quantify the errors induced by the EPA, we show on the left panel of \reffig{fig:comparison4PS_EPA} the ratio $\mathcal{R}$ of the trident cross section calculated using the EPA with an artificial cut at $Q^2_\text{cut}$, as performed in \cite{Magill:2016hgc}, to the full calculation used in this work as a function of the incoming neutrino energy:
%
 \begin{equation}
 \mathcal{R} = \frac{\sigma_{\rm EPA} (E_\nu) \vert_{Q_{\rm cut}}}{\sigma_{\rm 4PS} (E_\nu)}\,.
 \end{equation}
%
 In this plot we vary the artificial cut on $Q^2$ around the choice of \cite{Magill:2016hgc} (shown as the central dashed line) in two ways. First we reduce it by $20 \%$, and then increase it by a large factor, recovering the case with no $Q^2$ cut. From this, our conclusions about the validity of the approximation are confirmed, and it becomes evident that the trident coherent cross section is very sensitive to the choice of $Q^2_{\mathrm{cut}}$. In particular, the EPA with all the assumptions that lead to \refeq{eq:EPA_bad} and the absence of a $Q^2$ cut can lead to an overestimation of all trident channels, including the dimuon one. Once the cut is implemented, however, the approximation becomes better for the dimuon channel, but still unacceptable for the electron ones. It is also clear that an energy independent cut cannot give the correct cross section at all energies. This is particularly troublesome for detectors subjected to a neutrino flux covering a wide energy range such as the near detectors for DUNE and  MINOS or MINER$\nu$A. Moreover, \refeq{eq:EPA_bad} fails at low energies, and generally, overestimates the coherent cross sections by at least  200\%. At these energies, one must be wary of the additional approximations in \refeq{eq:EPA_bad} regarding the integration limits and the small $y$ limit.     

On the right panel of \reffig{fig:comparison4PS_EPA} we illustrate what happens in the diffractive regime, where the nucleon form factors impact the cross section at much larger values of $Q^2$ and have a slower fall-off. We see that the diffractive cross section is dramatically overestimated over the full range of $E_\nu$ considered and for any trident mode. The discrepancy is particularly important for $E_\nu \lesssim$ 5 GeV and larger than in the coherent regime by at least an order of magnitude\footnote{There are some differences in the treatment of the hadronic system between the EPA calculation in \cite{Magill:2016hgc} and the one presented here. However, these differences are of the order 10\% to 20\%. Note also that we do not implement any Pauli blocking when calculating $\mathcal{R}$ to avoid ambiguities over the choice of the range of $Q^2$.}. We also see that the cuts on $Q^2$ impact the EPA calculation much less dramatically, and that its use is unlikely to yield the correct result.

Given these problems with both coherent and diffractive cross section calculations due to the breakdown of the EPA for trident production, in what follows we will use the complete four-body calculation.

%%%%%%%%%%%%%%%%%%%%%%%%%%%%%%%%%%%%%%%%%%%%%%%%%%%%
\subsection{Coherent versus Diffractive Scattering in Trident Production}
\label{subsec:cohdiff}

%
\begin{figure}[t]
	\centering
	\includegraphics[width=\textwidth]{figs/Ratio_CDvsT.pdf}%
	\caption{\label{fig:RatioCDvsT} On the left (right) panel we show the
    ratio of the coherent (full lines) and the diffractive (dashed lines) contributions to the total trident cross section for an incoming flux of $\nu_\mu$($\nu_e$) as a function of $E_\nu$ for an 
    $^{40}$Ar target.}
\end{figure}
%

Let us  now comment on the significance of  the coherent and diffractive contributions to the total cross for the different trident channels. 
In Fig.\ \ref{fig:RatioCDvsT} we present the ratio of the coherent and the diffractive scattering cross sections to the total cross section for an $^{40}$Ar target for an incoming $\nu_\mu$ (left) and $\nu_e$ (right)  neutrino. We can see that the coherent regime dominates at all neutrino energies when there is an electron in the final-state, especially in the dielectron case. 
This can be explained by noting that the $Q^2$ necessary to create an electron pair is smaller than the one needed to create a muon; thus, coherent scattering is more likely to occur for this mode. Conversely, 
as one needs larger momentum transferred to produce a muon (either accompanied by an electron 
or another muon) the  diffractive regime becomes more likely in these modes, as we can explicitly 
see in Fig.\ \ref{fig:RatioCDvsT}. 
Because of this effect the diffractive contribution is $\lesssim$ 10\%, except for the 
dimuon channel where it can be between $30$ and $40$\% in most of the energy region.
Furthermore, when we compare the two incoming types of neutrinos, we see that for an incoming $\nu_\mu$ the diffractive contribution is larger than the coherent one in the range $0.3\ {\rm GeV}\lesssim E_\nu \lesssim 0.8$ GeV, while for an incoming $\nu_e$ this never happens. 
This difference can be explained by the fact that CC and NC contributions  are simultaneously present for the scattering of an initial $\nu_\mu$ creating a muon pair, whereas
for an initial $\nu_e$ creating a muon pair, we will only have the NC contribution, see Table \ref{tab:tridentmodes}.

An important difference between the coherent and diffractive regimes will be in their hadronic signatures in the detector. Neutrino trident production is usually associated with zero hadronic energy at the vertex, a feature that proved very useful in reducing backgrounds in previous measurements. Whilst this is a natural assumption for the coherent regime, it need not be the case in the diffractive one. In fact, in the latter it is likely that the struck nucleon is ejected from the nucleus in a significant fraction of events with $Q$ exceeding the nuclear binding energy
%
\footnote{The peak of our diffractive $Q^2$ distributions happens at around $Q \approx 300$ MeV, much beyond the typical binding energy for Ar (see \refapp{app:distributions}). Without Pauli suppression, however, we expect this value to drop.}. Since the dominant diffractive contribution comes from scattering on protons, these could then be visible in the detector if their energies are above threshold. On the other hand, the struck nucleon is subject to many nuclear effects which may significantly affect the hadronic signature, such as interactions of the struck nucleon in the nuclear medium as well as reabsorption. Our calculation of Pauli blocking, for example, shows large suppressions ($\sim 50\%$) precisely in the low $Q^2$ region, usually associated with no hadronic activity. This then raises the question of how well one can predict the hadronic signatures of diffractive events given the difficulty in modelling the nuclear environment. We therefore do not commit to an estimate of the number of diffractive events that would have a coherent-like hadronic signature, but merely point out that this might introduce additional uncertainties in the calculation, especially in the $\mu^+ \mu^-$ channel where the diffractive contribution is comparable to the coherent one. Finally, from now on we will refer to the number of trident events with no hadronic activity as coherent-like, where this number can range from coherent only to coherent plus all diffractive events. 



%%%%%%%%%%%%%%%%%%%%%%%%%%%%%%%%%%%%%%%%%%%%%%%%%%%%
\section{Trident Events in LAr Detectors}
\label{sec:LAr}

In this section we calculate the total number of expected  trident events for some present and future LAr detectors with different fiducial masses, total exposures and beamlines. In Table~\ref{tab:LAr} we specify the values used for each set-up and in Fig.~\ref{fig:LAr} we show the total production cross section for each neutrino trident mode of Table~\ref{tab:tridentmodes}
as well as the neutrino fluxes as a function of $E_\nu$ at the position of each experiment.  

%%%%%%%%%%%%%%%%%%%%%%%%%%%%%%%%%%%%%%%%%%%%%%%%%%%%
\subsection{Event Rates}
\label{subsec:rates}

The total number of trident events, $N^{\text{\Neptune}}_{\rm X}$, expected for a given trident mode at any detector is written as  
\begin{eqnarray}
\label{eq:nevents}
N^{\text{\Neptune}}_{\rm X}={\rm Norm}\times\int dE_\nu \, \sigma_{\nu {\rm X}}(E_\nu) \frac{d\phi_{\nu}(E_\nu)}{dE_\nu}\epsilon(E_\nu)\,,
\end{eqnarray}
where $\sigma_{\nu {\rm X}}$ can be the trident total (${\rm X}={\cal N}$), coherent ($\mathrm{X=c}$) or diffractive ($\mathrm{X=d}$) cross sections 
for a given mode, $\phi_{\nu}$ is the flux of the incoming neutrino and $\epsilon(E_\nu)$ is the efficiency of detection of the charged leptons. In the calculations of this section, we assume an efficiency of $100\%$\footnote{See \refsec{subsec:kine} for a discussion on the detection efficiencies for trident events and backgrounds.}.
%
The normalization is calculated as 
 %
$${\rm Norm}= {\rm{Exposure}}~[{\rm{POT}}] \times \frac{{\rm{Fiducial~Detector~Mass}\times N_A}}{m_{\rm T}} \left[{\rm{target~particles}}\right],$$
%%
where $m_{\rm T}$ is the molar mass of the target particle and $N_A$ is Avogadro's number.
%
Two features of the cross sections are important for the event rate calculation: 
threshold effects, especially for channels involving muons in the final-state,
and cross section's growth with energy. In particular, we expect higher trident event rates for experiments with higher energy neutrino beams. 

We start our study with the three detectors of the SBN program, one of which, $\mu$BooNE, is already installed and taking data at Fermilab. These three LAr time projection chamber detectors are located along the Booster Neutrino Beam line which is by now a well-understood source, having the focus of active research for over 15 years. 
%
Although the number of trident events expected in these detectors is rather low, they may offer one of the first opportunities to study trident events in LAr, as well as to better understand their backgrounds in this medium and to devise improved analysis techniques.
%
After that we study the proposed near detector for DUNE. This turns out to be the most important LAr detector for trident production since it will provide the highest number of events in both neutrino and antineutrino modes. 
%
Finally, having in mind the novel flavour composition of neutrino beams from muon facilities, we investigate trident rates at a 100~t LAr detector for the $\nu$STORM project. This last facility could offer a very well understood neutrino beam with as many electron neutrinos as muon antineutrinos from muon decays, creating new possibilities for trident scattering measurements.
%%
%%
\begin{table}[t]
\begin{center}
\scalebox{0.9}{
\begin{tabular}{|c|c|c|c|c|}
%\hline
\hline\hline
\bf Experiment& \bf Baseline (m) & \bf Total Exposure (POT) & \bf Fiducial Mass (t) & \bf $\mathbf{E_\nu}$ (GeV)\\\hline\hline
SBND & 110 & $6.6\times 10^{20}$ & 112 & $0-3$\\\hline
$\mu$BooNE & 470 & $1.32\times 10^{21}$ & 89& $0-3$\\\hline
ICARUS & 600 & $6.6\times 10^{20}$ & 476 & $0-3$\\\hline
DUNE & 574  & $12.81~(12.81)\times 10^{21}$& 50 & $0-40$ \\\hline
$\nu$STORM & 50  & $10^{21}$ & 100& $0-6$\\\hline\hline
\end{tabular}}
\end{center}
\caption{\label{tab:LAr} Summary of the LAr detectors set-up and values assumed in our calculations.
The POT numbers are given for a neutrino (antineutrino) beam.}
\end{table}

\begin{figure}[t]
\centering
\includegraphics[width=1.\textwidth]{figs/LAr_f+xsec.pdf}
\caption{\label{fig:LAr}Energy distribution of the neutrino fluxes at the position of 
the LAr detectors DUNE (top left, \cite{DUNE:flux}), SBND (top right,\cite{SBNproposal}) and $\nu$STORM (bottom left, \cite{nuSTORM2017}) and of the cross sections for the various trident modes (bottom right). The fluxes at $\mu$BooNE and ICARUS are similar to the one shown for SBND when normalized over distance.}
\end{figure}

%%%%%%%%%%%%%%%%%%%%%%%%%%%%%%%%%%%%%%%%%%%%%%%%%%%%
\subsubsection{The SBN Program}
\label{subsubsec:SBND}

The SBN Program at Fermilab is a joint endeavour by three collaborations ICARUS, $\mu$BooNE and 
SBND to perform searches for eV-sterile neutrinos and study neutrino-Ar cross sections \cite{SBNproposal}. As can be seen in Tab.~\ref{tab:LAr}, SBND has the shortest baseline (110 m) and therefore the largest neutrino fluxes (shown in Fig.~\ref{fig:LAr} and taken from Fig. 3 of \cite{SBNproposal}). The largest detector, ICARUS, is also the one with the longest baseline (600 m) and consequently subject to the lowest neutrino fluxes.
%
The ratio between the fluxes at the different detectors are  $\phi_{\mu\rm{BooNE}}/\phi_{\rm{SBND}}=5$\% and $\phi_{\rm{ICARUS}}/\phi_{\rm{SBND}}=3$\%.
%
The neutrino beam composition is about 93\% of $\nu_\mu$,  6\% of $\overline\nu_\mu$ and  
$1\%$ of $\nu_e+\overline{\nu}_e$. 

Considering the difference in fluxes and the total number of targets in each of these 
detectors, one can estimate the following ratios of trident events: 
${N^\text{\Neptune}_{\mu\rm{BooNE}}}/{N^\text{\Neptune}_{\rm{SBND}}}\sim 8$\% and ${N^\text{\Neptune}_{\rm{ICARUS}}}/{N^\text{\Neptune}_{\rm{SBND}}}\sim 10$\%. Unfortunately, 
since the fluxes are peaked at a rather low energy ($E_\nu \lesssim 1$ GeV), where the trident  
cross sections are still quite small ($\lesssim 10^{-42}$ cm$^2$) we expect very few 
trident events produced.
%
The exact number of trident events for those detectors according to our calculations is 
presented in Tab.~\ref{tab:LArrates}. For each trident channel the first (second) row
shows the number of coherent (diffractive) events. As expected, less than a total 
of 20 events across all channels can be detected by SBND, and a negligible rate of events is expected at $\mu$BooNE and ICARUS. 

%%%%%%%%%%%%%%%%%%%%%%%%%%%%%%%%%%%%%%%%%%%%%%%%%%%%
\subsubsection{DUNE Near Detector}
\label{subsubsec:DUNE}

\begin{table}[t]
\begin{center}
\scalebox{0.9}{
\begin{tabular}{|cccccc|}
\hline\hline
		\bf Channel & \bf SBND& \bf $\mu$BooNE & \bf ICARUS & \bf DUNE ND &\bf  $\nu$STORM ND \\ \hline \hline
		$\nu_\mu\to\nu_e e^+ \mu^-$& $10$ &$0.7$ &$1$ &$2844 ~ (235)$ & $159$ \\
        &$1$ &$0.08$ &$0.1$ &$369 ~ (33)$& $18$\\\hline
        $\overline\nu_\mu\to\overline\nu_e e^- \mu^+$&$0.4$ &$0.02$ &$0.04$ &$122~(2051)$ & $23$\\
        &$0.04$ &$0.003$ &$0.004$ &$16~(262)$ & $3$\\\hline
	$\nu_e\to\nu_\mu e^- \mu^+$& $0.05$ &$0.003$ &$0.004$ &$22~(7)$ & $9$\\
        &$0.008$ &$0.0005$ &$0.0008$ &$5~(1)$ & $2$\\\hline
	$\overline\nu_e\to\overline\nu_\mu e^+ \mu^-$& $0.005$ &$0.0003$ &$0.0005$ &$5~(14)$ & $-$\\
    &$0.001$ &$0.0001$ &$0.0001$ &$1~(3)$ & $-$\\\hline
    \hline\hline
    {$\rm{Total} \ e^\pm \mu^\mp$}& $10$ &$0.7$ &$1$ &$2993~(2307)$ & $191$ \\
    &$1$ &$0.1$ &$0.1$ &$391~(299)$ & $23$\\\hline
    \hline
		$\nu_\mu\to\nu_\mu e^+ e^-$& $6$ &$0.4$ &$0.7$ &$913~(58)$ & $73$ \\
        &$0.2$ &$0.04$ &$0.02$ &$57~(5)$ & $3$\\\hline
        $\overline\nu_\mu\to\overline\nu_\mu e^- e^+$& $0.2$ &$0.01$ &$0.02$ &$34~(695)$ & $9$\\
        &$0.01$ &$0.001$ &$0.002$ &$2~(41)$ & $0.5$\\\hline
	$\nu_e\to\nu_e e^- e^+$&$0.2$ &$0.01$ &$0.02$ &$50~(13)$ & $32$ \\
    &$0.01$ &$0.001$ &$0.002$ &$4~(1)$ & $2$\\\hline
	$\overline\nu_e\to\overline\nu_e e^+ e^-$&$0.02$ &$0.001$ &$0.002$ &$10~(34)$ & $-$ \\
    &$0.0009$ &$0.0001$ &$0.0002$ &$1~(2)$ & $-$\\\hline
    \hline\hline
    ${\rm{Total}}\  e^+ e^-$& $6$ &$0.4$ &$0.7$ &$1007~(800)$ & $114$\\
    &$0.2$ &$0.0$ &$0.02$ &$64~(49)$ & $6$\\\hline
    \hline    
		$\nu_\mu\to\nu_\mu \mu^+ \mu^-$& $0.4$ &$0.03$ &$0.04$ &$271~(32)$ & $9$ \\
        &$0.3$ &$0.03$ &$0.04$ &$135~(14)$ & $5$\\\hline
        $\overline\nu_\mu\to\overline\nu_\mu \mu^- \mu^+$& $0.01$ &$0.001$ &$0.001$ &$14~(177)$ & $2$\\
        &$0.01$ &$0.0009$ &$0.001$ &$7~(93)$ & $1$\\\hline
 $\nu_e\to\nu_e \mu^+ \mu^-$    &$0.002$ &$0.0001$ &$0.0001$ &$1~(0.5)$ & $0.4$\\  
 &$0.001$ &$0.0001$ &$0.0001$ &$0.5~(0.2)$ & $0.2$\\\hline
        $\overline\nu_e\to\overline\nu_e \mu^+ \mu^-$&$0.0002$ &$0.0000$ &$0.0000$ &$0.3~(0.9)$ & $-$\\
        &$0.0001$ &$0.0000$ &$0.0000$ &$0.1~(0.3)$ & $-$\\\hline
        \hline\hline
    ${\rm{Total}} \ \mu^+ \mu^-$ &$0.4$ &$0.0$ &$0.0$ &$286~(210)$ & $11$ \\
    &$0.3$ &$0.0$ &$0.0$ &$143~(108)$ & $6$\\
    \hline\hline
\end{tabular}}
\end{center}
\caption{\label{tab:LArrates}Total number of \textbf{coherent} (top row) and \textbf{diffractive} (bottom row) trident events expected at different LAr experiments for a given channel.
The numbers in parentheses are for the antineutrino running mode, when present. These calculations  
considered a detector efficiency of 100\%. }
\end{table}

The DUNE experiment will operate with neutrino as well as antineutrino LBNF beams produced by 
directing a 1.2 MW beam of protons onto a fixed target \cite{Acciarri:2016ooe,DUNECDRvolII}. 
The design of the near detector is not finalised, but the current designs favour a mixed technology  detector combining a LAr TPC with a larger tracker module.  In this work, we will assume that DUNE ND is a LAr detector located at $574$ m from the target with a fiducial mass of 50~t \cite{WeberTalk}. As the trident event rate scales with the density of the target, any tracker module will not significantly influence the total event rate, and does not feature in our estimates; although, its presence is assumed to improve reconstruction of final-state muons. Our estimates can be easily scaled for the final design by using \refeq{eq:nevents}.

For the first 6 years of data taking (3 years in the neutrino plus 3 years in the antineutrino 
mode) the collaboration expects $1.83\times 10^{21}$~POT/year with  a plan to upgrade the beam after the 6th year for 2 extra years in each beam mode  with double exposure, making a total of $1.83 \times(3+2\times2)\times 10^{21}~{\rm{POT}}$ for each mode \cite{DUNE:exposure}. We will 
assume the total 10-year exposure in our calculations.
%
. as the relevant fluxes at the DUNE ND location (see Fig.~\ref{fig:LAr}). The beam composition of the neutrino (antineutrino) beam is about 96\% $\nu_\mu$ ($\overline\nu_\mu$), 4\%  $\overline\nu_\mu$ ($\nu_\mu$) and 1\% $\nu_e+\overline\nu_e$.
 
The number of trident events for DUNE ND can be found in Tab.~\ref{tab:LArrates}. 
The numbers in parentheses correspond to antineutrino beam mode.
Note that although the trident cross sections are the same 
for neutrinos and antineutrinos, the fluxes are a bit lower for the antineutrino beam, as a consequence we predict a lower event rate for this beam\footnote{A similar difference will apply to the processes constituting the background to the trident process, although there is an additional suppression in many channels due to the lower antineutrino cross sections.}.
%
Due to the much higher energy and wider energy range of the neutrino fluxes at DUNE ND, as compared to the SBN detectors, DUNE can observe a considerable number of trident events, about 300 times the number of trident events expected for SBND just in the neutrino mode. Moreover, the subdominant component of 
each beam mode will also contribute to the signal. For example, we expect to observe $2051$ trident events in the $\overline{\nu}_\mu\to\overline{\nu}_e e^- \mu^+$ channel in the antineutrino mode. However, we also expect 
$235$ events in the $\nu_\mu\to\nu_e e^+ \mu^-$ channel produced by 
the subdominant component of $\nu_\mu$ in the antineutrino beam.
%
We have considered 100\% detection efficiency here, however, we will see in Sec.~\ref{subsec:bck} that after implementing hadronic vetos, detector thresholds and kinematical cuts to substantially reduce the background we expect an efficiency of about 47\%-65\% on coherent tridents, depending on the channel (see Tab.~\ref{tab:DUNE_ND_NU_BG}).

The mixed flavour trident channel is the one with the highest statistics (more than 6000 events adding 
neutrino and antineutrino beam modes), 11\% of which are produced by diffractive scattering. The dielectron channel comes next with a total of a bit more than 1900 events, 5\%  of which are produced by diffractive scattering. Although the  dimuon channel is the less copious one, with only about 
750 events produced, almost 34\% of these events are produced by a diffractive process.
This can be understood by recalling our discussions in Sec.~\ref{subsec:cohdiff}.

Finally, we note that a dedicated high-energy run at DUNE has been mooted, to be undertaken after the full period of data collecting for the oscillation analysis. Thanks to the higher energies of the beam, this has the potential to see a significant number of neutrino tridents, provided it can collect enough POTs.  

%%%%%%%%%%%%%%%%%%%%%%%%%%%%%%%%%%%%%%%%%%%%%%%%%%%%
\subsubsection{$\nu$STORM}
\label{subsubsec:nuSTORM}
In this section we study the trident rates for a possible LAr detector for the proposed 
$\nu$STORM experiment \cite{Soler:2015ada,nuSTORM2017}. The $\nu$STORM facility 
is based on a neutrino factory-like design and has the goal to search for sterile neutrinos and study neutrino nucleus cross sections \cite{Adey:2014rfv}. Although this proposal is in its early days, $\nu$STORM has the potential to make cross section measurements with unprecedented precision. In its current design, $120$-GeV protons are used to produce pions from a fixed target with the pions subsequently decaying into muons and neutrinos. The muons are captured in a storage ring and during repeated passes around the ring they decay to produce neutrinos.
%
Consequently, the storage ring is an intense source of three types of neutrino
flavours: $\nu_\mu$ from $\pi^+$ and $K^+$ decays, which will be more than $99\%$ of the total flux, $\nu_e$ and $\overline\nu_\mu$ from recirculated muon decays which will comprise less than $1\%$ of the total flux. An important point, however, is that the neutrinos coming from the pion and kaon decays can be separated by event timing from the ones produced by the stored muons. This distinction allows the $\nu_\mu$ flux to be studied almost independently from the $\overline{\nu}_\mu$ and $\nu_e$ flux. In addition, it implies after the initial flash of meson-derived events, that the flux consists of as many electron neutrinos as muon antineutrinos. We will assume a LAr detector for $\nu$STORM at a baseline of 50\,m with 100\,t of fiducial mass with an exposure of $10^{21}$ POT. The neutrino fluxes, assuming 
a central $\mu^+$ momentum of $3.8$~GeV/c in the storage ring, are taken from Ref.~\cite{nuSTORM2017} and are 
shown in Fig.~\ref{fig:LAr}.

In Tab.~\ref{tab:LArrates}, we show the results of our calculations for $\nu$STORM. 
More than $97\%$ of the events from the incoming $\nu_\mu$ are from pion decays and only less 
than $3\%$ from kaon decays. Since we only consider the decay of mesons with positive charges and we expect neutral and wrong charge contamination to be small, we do not have trident events from incoming $\overline\nu_e$.
%
The total number of mixed flavour, dielectron and dimuon channel events is, respectively,
214, 120 and 17, much less than what can be achieved at the larger neutrino energies available at the DUNE ND. The novel flavour structure of the beam does enhance the contribution of $\nu_e$ induced tridents with respect to the $\pbar{\nu}_\mu$ ones, but this contribution only becomes dominant for the $e^+e^-$ tridents in the muon decay events. Finally, we emphasize that the experimental design parameters for $\nu$STORM are far from definite. Increasing the energy of stored muons and the size of the detector are both viable options which could significantly enhance the rates we present.

%
%%%%%%%%%%%%%%%%%%%%%%%%%%%%%%%%%%%%%%%%%%%%%%%%%%%%
\subsection{Kinematical Distributions at DUNE ND}
%
\label{subsec:kine}

In this section we explore the trident signal in more detail, showing some relevant kinematical distributions for coherent and diffractive events. For concreteness, and due to its large number of events, we choose to focus on the DUNE ND, only commenting slightly on the signal at the lower energies of SBN and $\nu$STORM. The observables we calculate are the invariant mass of the charged leptons $m^2_{\ell^+ \ell^-}$, their separation angle $\Delta \theta$ and their individual energies $E_\pm$. The flux convolved distributions of these observables are shown for the DUNE ND in neutrino mode in \reffig{fig:DUNE_ND_dist}. In these plots, we sum all trident channels with a given undistinguishable final-state proportionally to their rates, although $\nu_\mu$ initiated processes always dominate. The coherent and diffractive contributions are shown separately and on the same axes, but we do not worry about their relative normalization. Other potentially interesting quantities are the angle between the cone formed by the two charged leptons and the beam, $\alpha_C$, and the angle of each charged lepton with respect to the beam direction, $\theta_\pm$.  These additional observables are explored in \refapp{app:distributions}. We also report the distributions of the momentum transfer to the hadronic system, $Q^2$. Although this is not a directly measurable quantity, it is a strong discriminant between the coherent and diffractive processes. We do not present the antineutrino distributions here, but they are qualitatively similar.

Perhaps one of the most valuable tools for background suppression in the measurement of the $\mu^+\mu^-$ trident signal at CHARM~II, CCFR and NuTeV \cite{Geiregat:1990gz,Mishra:1991bv,Adams:1998yf} was the smallness of the invariant mass $m^2_{\ell^+ \ell^-}$. This feature, shown here on the top row of \reffig{fig:DUNE_ND_dist}, is also present at lower energies, where the distributions become even more peaked at lower values; although, the diffractive events tend to be have a more uniform distribution in this variable. This is also true for the angular separation $\Delta \theta$, where coherent dimuon tridents tends to be quite collimated, with $90\%$ of events having $\Delta \theta < 20^\circ$, whilst diffractive ones are less so, with only $47\%$ of events surviving the cut. This difference is much less pronounced for mixed and dielectron channels, where only half of our coherent events obey $\Delta \theta < 20^\circ$, when $37\%$ of diffractive events do so.

%
\begin{figure}[H]
\centering
\includegraphics[width=\textwidth]{figs/DUNE_nu_3horn_mll_theta_E.pdf}
\caption{Flux convolved neutrino trident production distributions for DUNE ND in neutrino mode. In purple we show the coherent contribution in $^{40}$Ar and in blue the diffractive contribution from protons as targets only (including Pauli blocking). The coherent and diffractive distributions are normalized independently. The relative importance of each contribution as a function of $E_\nu$
can be seen in Fig.~\ref{fig:RatioCDvsT}.
%
\label{fig:DUNE_ND_dist}}
\end{figure}
%


An interesting feature of same flavour tridents induced by a neutrino (antineutrino) is that the negative (positive) charged lepton tends to be slightly more energetic than its counterpart, whilst for mixed tridents muons tend to carry away most of the energy. These considerations are also reflected in the angular distributions. The most energetic particle is also the more forward one. For instance, in mixed neutrino induced tridents, $\sim 80 \%$ of the $\mu^-$ are expected to be within $10^\circ$ of the beam direction, whilst only $\sim 35 \%$ of their $e^+$ counterparts do so (see \refapp{app:distributions} for additional distributions).


Finally, we mention that detection thresholds can also be important for trident channels with electrons in the final-state. Assuming, for example, a detection threshold for muons and electromagnetic (EM) showers of 30 MeV in LAr, we end up with efficiencies of (99\%, 71\%, 77\%, 86\%) for ($\mu^+ \mu^-$, $e^+ e^-$, $e^+ \mu^-$, $e^- \mu^+$) coherent tridents. These efficiencies become (96\%, 91\%, 93\%, 96\%) for diffractive tridents, dropping for $\mu^+\mu^-$ and increasing for all others. For comparison, at the lower neutrino energies of SBND and assuming the same detection thresholds, the efficiencies for coherent and diffractive tridents are slightly lower, (97\%, 57\%, 67\%, 77\%) and (90\%, 81\%, 85\%, 90\%) respectively.

%%%%%%%%%%%%%%%%%%%%%%%%%%%%%%%%%%%%%%%%%%%%%%%%%%%%
\subsection{Background Estimates for Neutrino Trident in LAr}
\label{subsec:bck}

The study of any rare process is a struggle against both systematic uncertainties in the event rates and unavoidable background processes. True dilepton signatures are naturally rare in neutrino scattering experiments, but with modest rates of particle misidentification a non-trivial background arises. In this section we estimate the background to trident processes in LAr and its impact on the trident measurement. We perform our analysis only for DUNE ND, in neutrino and antineutrino mode, but our results are expected to be broadly applicable to other LAr detectors. We have generated a sample of $1.1 \times 10^6$ background events using GENIE \cite{Andreopoulos2009} for incident electron and muon flavour neutrinos and antineutrinos. It is worth noting, however, that this event sample will in fact be smaller than the total number of neutrino interactions expected in the DUNE ND. 
%
Our goal, therefore, will be to demonstrate that with modest analysis cuts background levels can be suppressed significantly such that they become comparable to or smaller than the signals we are looking for. In the absence of events that satisfy our background definition, we argue that the frequency of that type of event is less than one in $1.1\times 10^6$ interactions of the corresponding initial neutrino.  

To account for misreconstruction in the detector, we implement resolutions as a gaussian smear around the true MC energies and angles. We assume relative energy resolutions as $\sigma/E = 15\%/\sqrt{E}$ for $e/\gamma$ showers and protons, and $6\%/\sqrt{E}$ for charged pions and muons. Angular resolutions are assumed to be $1^\circ$ for all particles (proton angles are never smeared in our analysis). The detection thresholds are a crucial part of the analysis, since for many channels one ends up with very soft electrons. We take thresholds to be $30$ MeV for muons and $e/\gamma$ showers kinetic energy, $21$ MeV for protons and $100$ MeV for $\pi^{\pm}$ \cite{DUNECDRvolII}.

%%%%%%%%%%%%%%%%%%%%%%%%%%%%%%%%%%%%%%%%%%%%%%%%%%%%
\subsubsection{Background Candidates}
\label{subsubsec:misID}
We focus on three final-state charged lepton combinations: $\mu^+\mu^-$, $\mu^\pm e^\mp$ and $e^+e^-$. Genuine production of these states is possible in background processes, but usually rare, deriving from meson resonances or other prompt decays. The majority of the background is expected to be from particle misidentification (misID). We assume that protons can always be identified above threshold and that neutrons leave no detectable signature in the detector. In addition, we require no charge ID capabilities from the detector and assume that the interaction vertex can always be reconstructed. Under these assumptions, we have incorporated three misidentifications which will affect our analysis, and give our naive estimates for their rates in Tab.~\ref{tab:misIDlist}. Any other particle pairs are assumed to be distinguishable from each other when needed.
%
\renewcommand{\arraystretch}{1.2}
\begin{table}[t]
\centering 
\begin{tabular}{|c c|}
\hline\hline
\bf misID & \bf Rate \\
\hline\hline
%
$\gamma$ as $e^\pm$ & 0.05 \\
\hline
\multirow{2}{*}{$\gamma$ as $e^+e^-$} & 0.1 (w/ vertex)  \\
%\cline{2-2}
 & 1 (no vertex + overlapping)  \\
\hline
$\pi^\pm$ as $\mu^\pm$ & 0.1 \\
\hline\hline
\end{tabular}
\caption{\label{tab:misIDlist} Assumed misID rates for various particles in a LAr detector. We take these values to be constant in energy.}
\end{table}

The requirement of no hadronic activity helps constrain the possible background processes, but one is still left with significant events with invisible hadronic activity and other coherent neutrino-nucleus scatterings. These are then reduced by choosing appropriate cuts on physical observables, exploring the discrepancies between our signal and the background. In our GENIE analysis, we include all events that have final-states identical to trident, or that could be interpreted as a trident final-state considering our proposed misID scenarios. Our dominant sources of background for $\mu^+ \mu^-$ tridents are $\nu_\mu$-initiated charged-current events with an additional charged pion in the final-state ($\nu_\mu$CC$1\pi^\pm$). For $e^+e^-$ tridents, the most important processes are neutral current scattering with a $\pi^0$ (NC$\pi^0$), while for mixed $e^\pm \mu^\mp$ tridents, the $\nu_\mu$-initiated charged-current events with a final-state $\pi^0$ (CC$\pi^0$) dominate the backgrounds. In each case, the pion is misidentified to mimic the true trident final-state. Other relevant topologies include charm production, CC$\gamma$ and $\nu_e$CC$\pi^\pm$. For a detailed discussion of these backgrounds processes we refer the reader to \refapp{app:backgrounds}.

%%%%%%%%%%%%%%%%%%%%%%%%%%%%%%%%%%%%%%%%%%%%%%%%%%%%
\subsubsection{\label{sec:DUNE_bg_rates}Estimates for the DUNE ND}

In this section we provide estimates for the total background for each trident final-state for the DUNE ND. The number of total inclusive CC interactions in the 50 t detector due to neutrinos of all flavours is calculated to be $5.18 \times 10^8$. We scale our background event numbers to match this, and argue that one has to reach suppressions of order $10^{-6} - 10^{-5}$ to have a chance to observe trident events. Whenever our cuts remove all background events from our sample, we assume the true background rate is one event per $1.1\times10^6$ $\nu$ interactions and scale it to the appropriate number of events in the ND, applying the misID rate whenever relevant. Within our framework, this provides a conservative estimate as the true background is expected to be smaller.

Our estimates are shown in \reftab{tab:DUNE_ND_NU_BG}. We start with the total number of background candidates $\rm N_B^{\mathrm{misID}}$, using only the naive misID rates shown in \reftab{tab:misIDlist}. These are much larger than the trident rates we expect, by at least 2 orders of magnitude. Next, we veto any hadronic activity at the interaction vertex, obtaining $\rm N_B^{\mathrm{had}}$. We emphasize that this veto also affects the diffractive tridents in a non-trivial way, and therefore we remain agnostic about the hadronic signature of these. 
%
Finally, one can look at the kinematical distributions of coherent trident in \refsec{subsec:kine} and try to estimate optimal one dimensional cuts for the DUNE ND based on the kinematics of the final-state charged leptons. This is a simple way to explore the striking differences between the peaked nature of our signal and the smoother background. In a real experimental setting it is desirable to have optimization methods for isolating signal from background, preferably with a multivariate analyses. However, even in our simple analysis, cutting on the small angles to the beamline and the low invariant masses of our trident signal can achieve the desired background suppressions. For the $\mu^+\mu^-$ tridents we show the effect of our cuts in \reffig{fig:bkg_flow}. The cuts are defined to be $m^2_{\mu^+ \mu^-} < 0.2 \ \mathrm{GeV}^2$, $\Delta \theta < 20^\circ$, $\theta_\pm < 15^\circ$. The kinematics is very similar in the other trident channels, with slightly less forward distributions for electrons. For the $e^+ e^-$ channel we take  $m^2_{e^+ e^-} < 0.1 \ \mathrm{GeV}^2$, $\Delta \theta < 40^\circ$ and $\theta_\pm < 20^\circ$. 
%
The asymmetry between the positive and negative charged leptons is visible in the distributions, where the latter tends to be more energetic. This feature was not explored in our cuts, as it is not significant enough to further improve background discrimination. In the mixed flavour tridents, however, one sees a much more pronounced asymmetry. The muon tends to carry most of the energy and be more forward than the electron, which can make the search for this channel more challenging due to the softness of the electron in the high energy event. Nevertheless, the low invariant masses and forward profiles can still serve as powerful tool for background discrimination, provided the event can be well reconstructed. We assume that is the case here and use the following cuts on the background:  $m^2_{e^\pm \mu^\mp} < 0.1 \  \mathrm{GeV}^2$, $\Delta \theta < 20^\circ$, $\theta_e < 40^\circ$ and $\theta_\mu < 20^\circ$. When performing kinematical cuts, we also include the effects of detection thresholds after smearing. For a discussion on the impact of these thresholds on the trident signal see \refsec{subsec:kine}. 
%

The resulting signal efficiencies due to our cuts and thresholds are shown in the last two columns of \reftab{tab:DUNE_ND_NU_BG}. One can see that these are all $ \approx 50\%$ or greater for our coherent samples, whilst all background numbers remain much below the trident signal. The diffractive samples are also somewhat more affected by our cuts than the coherent ones. If one is worried about the contamination of coherent events by diffractive ones, then the kinematics of the charged leptons alone can help reduce this, independently of the hadronic energy deposition of the events. For instance, in the case where all $\mu^+\mu^-$ diffractive events appear with no hadronic signature, then after our cuts the diffractive contribution is reduced from $41\%$ to $15\%$ of the total trident signal. This reduction is, however, also subject to large uncertainties coming from nuclear effects. In summary, the set of results above are encouraging, suggesting that the signal of coherent-like trident production is sufficiently unique to allow for its search at near detectors despite naively large backgrounds. 

%
\begin{table}[t]	
	\begin{center}
    \resizebox{\textwidth}{!}{
		\begin{tabular}{|clllll|}
		\hline \hline
		\bf Channel& $\bf N^{\mathrm{misID}}_{\mathrm{B}} / N_{\mathrm{CC}}$        & $ \bf N^{\mathrm{had}}_{\mathrm{B}} / N_{\mathrm{CC}}$ & $\bf N^{\mathrm{kin}}_{\mathrm{B}} / N_{\mathrm{CC}}$ & ${\epsilon_{\mathrm{sig}}^{\mathrm{coh}}}$ &
${\epsilon_{\mathrm{sig}}^{\mathrm{dif}}}$ \footnotemark\\	\hline \hline
		$e^{\pm}\mu^{\mp}$ & $1.67\  (1.62) \times 10^{-4}$ & $2.68\  (4.31) \times 10^{-5}$ & $4.40 \ (3.17) \times 10^{-7}$ & $ 0.61 \ (0.61)$ & $ 0.39 \ (0.39)$\\
		$e^+e^-$ & $2.83 \ (4.19)\times 10^{-4}$ & $1.30 \ (2.41) \times 10^{-4}$ &  $6.54 \ (14.1) \times 10^{-6}$ & $ 0.48 \ (0.47)$ & $ 0.21 \ (0.21)$\\
		$\mu^+\mu^-$ & $2.66 \ (2.73)\times 10^{-3}$ & $10.4 \ (9.75)\times 10^{-4}$ & $3.36 \ (3.10)\times10^{-8}$ & $0.66 \ (0.67)$ & $0.17 \ (0.16)$\\\hline\hline
		\end{tabular}
    }
	\end{center}
	\caption{\label{tab:DUNE_ND_NU_BG} Reduction of backgrounds at the DUNE ND in neutrino (antineutrino) mode and its impact on the signal for each distinguishable trident final-state. $\mathbf{N^{\mathrm{misID}}_{\mathrm{B}}}$ stands for total backgrounds to trident after only applying misID rates, $\mathbf{N^{\mathrm{had}}_{\mathrm{B}}}$ are the backgrounds after the hadronic veto, and $\mathbf{N^{\mathrm{kin}}_{\mathrm{B}}}$ reduce the latter with detection thresholds and kinematical cuts (see text for the cuts chosen). These quantities are normalized to the total number of CC interactions in the ND $\mathbf{N_{\mathrm{CC}}}$ (flavour inclusive). We also show the impact of our detection thresholds and kinematical cuts on the trident signal via efficiencies for coherent only ($\epsilon_{\mathrm{sig}}^{\mathrm{coh}}$) and diffractive only samples ($\epsilon_{\mathrm{sig}}^{\mathrm{dif}}$). We do not cut on the hadronic activity of diffractive events.}
\end{table}

\footnotetext{Despite the fact that many diffractive events will likely deposit hadronic energy in the detector, we quote the efficiency of our cuts on diffractive events with no assumptions on their hadronic signature.}

\begin{figure}[t]
\centering
\includegraphics[width = 0.75\textwidth]{figs/SigvsBkg.pdf}
 \caption{Signal and background distributions in invariant mass. The total background events (blue) include the misID rates in table \reftab{tab:misIDlist}. We apply consecutive cuts on the background, starting with cuts on the separation angle $\Delta \theta$ (red), both charged lepton angles to the beamline ($\theta_+$ and $\theta_-$) (orange) and the invariant mass $m^2_{\mu^+ \mu^-}$ . We show the signal samples before and after all the cuts in dashed black and filled black, respectively. \label{fig:bkg_flow}}
\end{figure}
%
Finally, we comment on some of the limitations of our analysis. The low rate of trident events calls for a more careful evaluation of other subdominant processes that could be easily be overlooked. For channels involving electrons, it is possible that de-excitation photons and internal bremsstrahlung become a source of background, as these also produce very soft EM showers, none of which are implemented in GENIE. The question of reconstruction of these soft EM showers, accompanied either by a high energy muon or by another soft EM shower also would have to be addressed, especially in the latter case where a trigger for these soft events would have to be in place. A more complete analysis is also needed for treating the decay products of charged pions and muons produced in neutrino interactions, as well as rare meson decay channels (like the Dalitz decay of neutral pions $\pi^0 \to \gamma e^+ e^-$). Cosmic ray events are not expected to be a problem due to the requirement of a vertex and a correlation with the beam for trident events. Perhaps even more exotic processes, such as the production of three final-state charged leptons ($\nu_{\alpha} (\overline{\nu}_{\alpha}) + \mathcal{H} \to  \ell_\alpha^- (\ell_\alpha^+) + \ell_\beta^+ + \ell_{\beta}^- + \mathcal{H^\prime}$), can also become relevant. For instance, radiative trimuon production \cite{Smith:1977nx} can potentially serve as a background to dimuon tridents if one of the muons is undetected. Similarly, $\mu e e$ production would fake a dielectron (mixed) trident signature if the muon (an electron) is missed. We are not aware of any estimates for the rate of these processes at the DUNE ND, but we note that their rate can be comparable to trident production at energies above $30$ GeV \cite{Albright:1978mg}. Improvements on our analysis should come from the collaboration's sophisticated simulations, allowing for a better quantification of hadronic activity, more realistic misID rates and more accurate detector responses.  

%%%%%%%%%%%%%%%%%%%%%%%%%%%%%%%%%%%%%%%%%%%%%%%%%%%%
\section{Trident Events in Other Near Detector Facilities}

\label{sec:others}
The search for neutrino trident production events certainly benefits from the capabilities of LAr technologies but need not be limited to it. In this section we study neutrino trident production rates at non-LAr experiments which have finished data taking or are still running: the on-axis near detector of T2K (INGRID), the near detectors of MINOS and NO$\nu$A and the MINER$\nu$A experiment. We calculate the total number of trident events as in Eq.~(\ref{eq:nevents}), taking into account the fact that some detectors are made of composite material. We summarize in Tab.~\ref{tab:others} the details of all non-LAr detectors considered in this section. We limit ourselves to a discussion of the total rates in the fiducial volume, but remark that a careful consideration of each detector is needed in order to assess their true potential to detect a trident signal. For instance, requirements about low energy EM shower reconstruction, hadronic activity measurements and event containment would have to be met to a good degree in order for the detector to be competitive. 

%%
%%
\begin{table}[t]
\begin{center}
\scalebox{0.72}{
\begin{tabular}{|c|c|c|c|c|c|}
%\hline
\hline\hline
\bf Experiment& \bf Material & \bf Baseline (m) & \bf Exposure (POT) & \bf Fiducial Mass (t) & \bf $\mathbf{E_\nu}$ (GeV)\\\hline
INGRID  & Fe & 280  & $3.9\times10^{21}$ [$10^{22}$] T2K-I [T2K-II] & 99.4 & $0-4$ \\\hline
MINOS[+] & Fe and C &1040  & $10.56(3.36)[9.69]\times 10^{20}$  & 28.6& $0-20$ \\\hline
NO$\nu$A & ${\rm{C_2 H_3 Cl}} $ and $ {\rm{C H_2}}$  & 1000 &$8.85(6.9)~[36(36) ]\times10^{20}$ [NO$\nu$A-II] &231 & $0-20$ \\\hline
MINER$\nu$A & ${\rm{CH,H_2O}}, {\rm{Fe,Pb,C}}$ & 1035  & $12(12)\times 10^{20}$ &7.98 & $0-20$ \\\hline\hline
\end{tabular}}
\end{center}
\caption{\label{tab:others} Summary of the non-LAr detector set-up and values used in our calculations. The POT numbers are given for a neutrino (antineutrino) beam. For T2K-I and II 
neutrino and antineutrino beams have the same exposure.}
\end{table}

%%%%%%%%%%%%%%%%%%%%%%%%%%%%%%%%%%%%%%%%%%%%%%%%%%%%
\subsection{INGRID}
\label{subsec:INGRID}
INGRID, the on-axis near detector of the T2K experiment, is located 280 m from the beam source. It consists of 14 identical iron modules, each with a mass of $7.1$~t, resulting in a total fiducial mass of $99.4$ t~\cite{Abe:2011xv}. The modules are spread over a range of angles between $0^\circ$ and $1.1^\circ$ with respect to the beam axis. The currently approved T2K exposure is $(3.9+3.9)\times 10^{21}$ 
POT in neutrino + antineutrino modes (T2K-I), with the goal to increase it to a total exposure of $(1+1)\times 10^{22}$ POT in the second phase of the experiment (T2K-II) \cite{Abe:2016tez}. Hence we expect approximately $2.6$ times more trident events for T2K-II. 

We use the on-axis neutrino mode flux spectra at the INGRID module-3 from Ref.~\cite{Abe:2015biq}, as shown on the top of the first panel of Fig.~\ref{fig:others}. The flux contribution for each neutrino flavour and  energy range is listed in Table 1 of Ref.~\cite{Abe:2015biq}. The total neutrino flux flavour composition at module-3 is 92.5\% $\nu_\mu$, 5.8\% $\overline\nu_\mu$,  1.5\% $\nu_e$ and 0.2\%  $\overline\nu_e$. We assume here that the fluxes at the other 13 modules are the same as at module-3. Although this is not exactly correct it should provide a reasonable estimate of the total rate.

Under these assumptions we show the total number of trident events we calculated for INGRID in 
the first (second) column of Tab.~\ref{tab:otherrates} for T2K-I (T2K-II) exposure.
We predict about 600 (1600) events for the mixed, 290 (735) events for the dielectron and 45 (115) 
events for the dimuon channel for T2K-I (T2K-II). These numbers, although less than 
those expected at the DUNE ND, are already very significant and worth further consideration. We expect, however, that the main challenge will be the reconstruction of final state electrons in these iron detectors.

\begin{figure}[t]
\centering
\includegraphics[width=\textwidth]{figs/NLAr_f+xsec.pdf}
\caption{\label{fig:others}
Energy distribution of the neutrino fluxes at the position of the detector (top plot) and 
corresponding total trident production cross sections (bottom plot) for:
INGRID~\cite{Abe:2015biq} (first panel), MINOS ND~\cite{fluxes:nonLAr} (second panel), NO$\nu$A ND\cite{fluxes:nonLAr} (third panel)
and MINER$\nu$A\cite{fluxes:nonLAr} (fourth panel).
The cross sections show here for the composite detectors are normalized by the total number of 
atoms.}
\end{figure}


\begin{table}[t]
\begin{center}
\scalebox{0.8}{
\begin{tabular}{|cccccccc|}
\hline\hline
		\bf Channel & \bf T2K-I& \bf T2K-II & \bf MINOS & \bf MINOS+ & \bf NO$\nu$A-I & \bf NO$\nu$A-II & \bf MINER$\nu$A \\ \hline \hline
		$\nu_\mu\to\nu_e e^+ \mu^-$& $538$ &$1379$ &$179~(25)$&$688$ &$71~(14)$ &$291~(73)$& $140~(13)$ \\
        &$49$ &$126$ &$21~(3)$ &$82$ &$21~(4)$ &$86~(21)$ & $30~(3)$\\\hline
        $\overline\nu_\mu\to\overline\nu_e e^- \mu^+$&$23$ &$58$ &$42~(31)$&$38$ &$10~(57)$ &$41~(296)$ & $8~(89)$\\
        &$2$ &$5$ &$5~(4)$&$5$ &$3~(17)$ &$12~(88)$& $2~(19)$\\\hline
	$\nu_e\to\nu_\mu e^- \mu^+$& $2$ &$6$ &$1~(0.2)$&$4$ &$2~(0.5)$&$8~(3)$ & $1~(0.09)$\\
        &$0.3$ &$1$ &$0.3~(0.04)$ &$0.8$&$0.9~(0.2)$ &$4~(1)$& $0.3~(0.03)$\\\hline
	$\overline\nu_e\to\overline\nu_\mu e^+ \mu^-$& $0.2$ &$0.6$ &$0.4~(0.3)$&$0.4$ &$0.5~(0.9)$ &$2~(5)$& $0.06~(0.5)$\\
    &$0.04$ &$0.1$ &$0.08~(0.06)$ &$0.08$&$0.2~(0.4)$ &$0.8~(2)$& $0.02~(0.2)$\\\hline
    \hline\hline
    {$\rm{Total} \ e^\pm \mu^\mp$}& $563$ &$1444$ &$222~(56)$ &$730$&$83~(72)$ &$340~(374)$& $149~(102)$ \\
    &$52$ &$132$ &$27~(7)$&$88$ &$25~(22)$ &$102~(114)$& $32~(22)$\\\hline
    \hline
    		$\nu_\mu\to\nu_\mu e^+ e^-$& $257$ &$659$ &$48~(5)$ &$44$&$22~(3)$ &$90~(16)$& $35~(3)$ \\
        &$9$ &$23$ &$3~(0.4)$ &$3$&$3~(0.6)$ &$00$& $4~(0.4)$\\\hline
        $\overline\nu_\mu\to\overline\nu_\mu e^- e^+$& $10$ &$26$ &$9~(8)$&$9$ &$2~(16)$ &$8~(83)$& $2~(23)$\\
        &$0.4$ &$1$ &$0.7~(0.5)$ &$0.7$&$0.4~(3)$ &$2~(15)$& $0.2~(3)$\\\hline
	$\nu_e\to\nu_e e^- e^+$&$9$ &$24$ &$3~(0.3)$ &$8$&$3~(0.9)$ &$12~(5)$& $2~(0.2)$ \\
    &$0.3$ &$0.8$ &$0.2~(0.03)$ &$0.6$&$0.7~(0.2)$ &$3~(1)$& $0.2~(0.02)$\\\hline
	$\overline\nu_e\to\overline\nu_e e^+ e^-$&$0.9$ &$2$ &$0.7~(0.6)$&$0.7$ &$0.8~(2)$ &$3~(10)$& $0.1~(0.9)$ \\
    &$0.03$ &$0.08$ &$0.06~(0.04)$ &$0.05$&$0.2~(0.3)$ &$0.8~(1)$& $0.01~(0.1)$\\\hline
    \hline\hline
    ${\rm{Total}}\  e^+ e^-$& $277$ &$711$ &$61~(15)$ &$62$&$29~(22)$ &$119~(114)$& $39~(27)$\\
    &$10$ &$25$ &$4~(1)$ &$4$&$4~(4)$ &$16~(21)$& $4~(3)$\\\hline
    \hline    
    		$\nu_\mu\to\nu_\mu \mu^+ \mu^-$& $29$ &$73$ &$21~(3)$ &$81$&$7~(2)$ &$28~(11)$& $17~(2)$ \\
        &$15$ &$38$ &$8~(1)$ &$33$&$7~(2)$ &$29~(10)$& $12~(1)$\\\hline
        $\overline\nu_\mu\to\overline\nu_\mu \mu^- \mu^+$& $1$ &$3$ &$5~(3)$&$5$ &$1~(7)$ &$4~(35)$& $1~(11)$\\
        &$0.7$ &$2$ &$2~(1)$ &$2$&$1~(6)$ &$4~(30)$& $0.7~(8)$\\\hline
 $\nu_e\to\nu_e \mu^+ \mu^-$    &$0.09$ &$0.2$ &$0.09~(0.01)$&$0.3$ &$0.1~(0.04)$ &$0.4~(0.2)$& $0.06~(0.007)$\\  
 &$0.04$ &$0.1$ &$0.03~(0.004)$ &$0.1$&$0.1~(0.03)$ &$0.4~(0.1)$& $0.03~(0.004)$\\\hline
        $\overline\nu_e\to\overline\nu_e \mu^+ \mu^-$&$0.01$ &$0.03$ &$0.03~(0.02)$ &$0.03$&$0.04~(0.06)$&$0.2~(0.3)$ & $0.004~(0.03)$\\
        &$0.004$ &$0.01$ &$0.01~(0.009)$ &$0.01$&$~0.03(0.05)$  &$0.1~(0.3)$ & $0.003~(0.02)$\\\hline
        \hline \hline
    ${\rm{Total}} \ \mu^+ \mu^-$ &$30$ &$76$ &$26~(6)$&$86$ &$9~(9)$ &$37~(47)$ & $18~(13)$ \\
    &$16$ &$40$ &$10~(2)$ &$35$&$8~(8)$ &$34~(36)$ & $13~(9)$\\
    \hline\hline
\end{tabular}}
\end{center}
\caption{\label{tab:otherrates}Total number of \textbf{coherent} (top row) and \textbf{diffractive} (bottom row) trident events expected at different non-LAr detectors for each channel. The numbers in parentheses are for the antineutrino running mode, when present. These calculations consider a detection efficiency of 100\%.}
\end{table}

%%%%%%%%%%%%%%%%%%%%%%%%%%%%%%%%%%%%%%%%%%%%%%%%%%%%
\subsection{MINOS/MINOS+ Near Detector}
\label{subsec:MINOS}
The MINOS near detector is a magnetized, coarse-grained tracking calorimeter, made primarily of steel and plastic scintillator. Placed $1.04$~km away from the NuMI target at Fermilab \cite{Aliaga:2016oaz}, it weighs $980$~t and is similar to the far detector in design. In our analysis, we assume a similar fiducial volume cut to the standard $\nu_\mu$ CC analyses, namely a fiducial mass of $28.6$~t made
of $80\%$ of iron and $20\%$ of carbon \cite{Boehm:2009zz}.

%
The experiment ran from 2005 till 2012 in the low energy (LE) configuration of the NuMI beam ($E_\nu^{\rm peak} \approx 3$ GeV) and collected $10.56\times 10^{20} ~(3.36\times 10^{20})$ POT in the neutrino (antineutrino) beam \cite{Aurisano}. The successor to MINOS, MINOS+, ran with the same detectors subjected to the medium energy (ME) configuration of the NuMI beam ($E_\nu^{\rm peak} \approx 7$ GeV) from 2013 to 2016, and has collected $9.69\times 10^{20}$ POT in the neutrino mode.
%
To calculate the trident event rates we use the fluxes taken from Ref.~\cite{fluxes:nonLAr}. 
The flavour composition at MINOS ND is 89\% (18\%) $\nu_\mu$ and 10\% (81\%) $\overline \nu_\mu$ for the neutrino (antineutrino) beam and about 1\%  $\nu_e+\overline\nu_e$ for either beam mode. We assume that the MINOS+ neutrino flux is identical to the one at the MINER$\nu$A experiment (see section \ref{subsec:MINERvA}).
These fluxes and total trident production cross sections are shown on the second  panel of Fig.~\ref{fig:others}.

Due to the multi-component material of the detector, the corresponding cross sections that 
enter in Eq.~(\ref{eq:nevents}) are:
\begin{equation}
\sigma_{\rm \nu X}^{\rm{MINOS}}= \sum_{i=\rm Fe,C} f_i \, \sigma_{\rm \nu X}^{i}\,,
\end{equation}
where $f_i$ is the number of nuclei $i$ over the total number of nuclei in the detector.
As a reference, the weighted cross sections, normalized by the total number of atoms, is 
also shown in Fig.~\ref{fig:others}. 

We report the total number of trident events for MINOS ND in Tab.~\ref{tab:otherrates}. 
Although the cross section for iron is about two times 
larger than for argon and the neutrino fluxes similar, the number of trident events at MINOS ND is much smaller than the expected one at DUNE ND due to a lower exposure and fiducial mass.
%
We predict that about 250 (63) mixed, 65 (16) dielectron and 36 (8)   dimuon trident events
were produced at this detector with the neutrino (antineutrino) LE NuMI beam. 
%
The rates are expected to be larger for MINOS+, as it benefits from the larger energies of the ME NuMI beam configuration and has similar number of POT to MINOS in neutrino mode. In total, we predict about 820 mixed, 66 dielectron and 121 dimuon trident events.

The stringent cut on the fiducial volume assumed here implies a reduction from the $980$~t near detector bulk mass to $28.6$~t. This cut can be relaxed, depending on the signature considered, and may significantly enhance the rates we quote. A careful analysis of trident signatures outside the fiducial volume would be necessary, but we point out that our rates can increase by at most a factor of $\approx30$.

%%%%%%%%%%%%%%%%%%%%%%%%%%%%%%%%%%%%%%%%%%%%%%%%%%%%
\subsection{NO$\nu$A Near Detector}
\label{subsec:NOvA}

The NO$\nu$A near detector is a fine grained low-Z liquid-scintillator detector placed off-axis from the NuMI beam at a distance of $1$~km. Its total mass is $330$~t, with almost $70$\% of it active mass ($231$~t). In this analysis we assume all of this active mass to also be fiducial. The detector is mainly made of 70\% mineral oil (CH$_2$) and 30\% of PVC (${\rm{C_2H_3Cl}}$) \cite{Wang:Biao}. A total exposure of $8.85 \, (6.9)\times10^{20}$ POT has been collected in the neutrino (antineutrino) beam mode prior to 2018~\cite{sanchez_mayly_2018_1286758}. 

The NO$\nu$A ND neutrino fluxes (taken from Ref. \cite{fluxes:nonLAr}) peak at slightly lower energies than the MINOS or MINER$\nu$A ones, $E_\nu^{\rm peak} \approx 2$ GeV, and are shown in the third panel of Fig. \ref{fig:others}. The flavour composition 
is 91\% (11\%) $\nu_\mu$ and 8\% (88\%) $\overline \nu_\mu$ in the neutrino (antineutrino) 
mode and about 1\% $\nu_e+\overline\nu_e$ in each mode.

Here the cross sections entering in Eq.~(\ref{eq:nevents}) are calculated as
%
\begin{eqnarray}
\sigma^{\rm{NO\nu A}}_{\rm \nu X}=\sum_{i=\rm C, Cl, H} f_i \sigma_{\rm \nu X}^{i} \,,
\end{eqnarray}
%
where $f_i$ is the number of nuclei $i$ over the total number of nuclei in the detector.
As a reference, the weighted cross sections, normalized by the total number of atoms, is shown 
in Fig.~\ref{fig:others}. 

In Tab.~\ref{tab:otherrates} we show our predictions for the number of trident events at 
NO$\nu$A ND. 
%
Comparing NO$\nu$A and MINOS, we see that while NO$\nu$A ND has a fiducial mass 
almost 8 times larger, the flux times total cross section at MINOS ND is at least two orders of magnitude larger than at NO$\nu$A ND, especially above 4 GeV (see Fig. \ref{fig:others}), making the rates at MINOS ND larger than the rates at NO$\nu$A ND. 
%

NO$\nu$A is planning to collect a total exposure of $36\,(36)\times10^{20}$ POT in the neutrino (antineutrino) mode (NO$\nu$A-II) \cite{sanchez_mayly_2018_1286758,NovaII}, making the expected rates almost $4.1 (5.2)$ times larger (shown in Tab.~\ref{tab:otherrates}). In this case the expected dimuons and mixed events at MINOS+ would be at least two times larger than NO$\nu$A-II. On the other hand, for NO$\nu$A-II there will be two times more dielectron events given the much higher exposure. 

%%%%%%%%%%%%%%%%%%%%%%%%%%%%%%%%%%%%%%%%%%%%%%%%%%%%
\subsection{MINER$\nu$A}
\label{subsec:MINERvA}

The multi-component MINER$\nu$A detector was mainly designed to measure neutrino and antineutrino interaction cross sections with different nuclei in the 1-20 GeV range of energy~\cite{MINERvA:2017}. The detector is located at $1.035$~km from the NuMI target. We assume a fiducial mass of about $8$~t, with a composition of 75\% CH, 9\% Pb, 8\% Fe, 6\% H$_2$O and $2\%$ C. The experiment has collected $12\times10^{20}$ POT in the neutrino mode and is planning to reach the same exposure in the antineutrino mode by 2019, both using the medium energy flux of NuMI beam configuration (shown in fourth panel of Fig.~\ref{fig:others}). We do not include the low energy runs, as these have lower number of POT and lower neutrino energies. The neutrino (antineutrino) beam is composed of 95\% (7\%) $\nu_\mu$ and 
4\% (92\%) $\overline \nu_\mu$, both beams have  about $1\%$ of $\nu_e+\overline\nu_e$.

For MINER$\nu$A the cross sections  in Eq.~(\ref{eq:nevents}) are calculated as
\begin{eqnarray}
\sigma^{\rm MINER \nu A}_{\rm \nu X}=\sum_{i=\rm C, Cl, H, Pb, Fe,O} f_i \, \sigma_{\rm \nu X}^{i} \,,
\end{eqnarray}
where $f_i$ is the number of nuclei $i$ over the total number of nuclei in the detector.
As a reference, the weighted cross sections, normalized by the total number of atoms, is shown in
Fig.~\ref{fig:others}. 

The total number of trident events we estimate for MINER$\nu$A are listed in Tab.~\ref{tab:otherrates}. As expected, these are lower than MINOS+, as the latter has a larger fiducial mass. MINER$\nu$A, however, benefits from its fine grained technology and its dedicated design for cross section measurements.

%%%%%%%%%%%%%%%%%%%%%%%%%%%%%%%%%%%%%%%%%%%%%%%%%%%%
\section{Conclusions}

\label{sec:conc}
Neutrino trident events are predicted by the SM, however, only $\overline{\nu}_\mu$ initiated dimuon tridents have been observed in small numbers, typically fewer than 100 events. This will change in the near future thanks to the current and future generations of precision neutrino scattering and oscillation experiments, which incorporate state-of-the-art detectors located at short distances from intense neutrino sources. 
%
In this work we discuss the calculation of the neutrino trident cross section for all flavours and hadronic targets, and provide estimates for the number and distributions of events at 9 current or future neutrino detectors: five detectors based on the new LAr technology (SBND, $\mu$BooNE, ICARUS, DUNE ND and $\nu$STORM ND) as well as four more conventional detectors (INGRID, MINOS ND, NO$\nu$A ND and MINER$\nu$A). The search for tridents, however, need not be exclusive to near detectors of accelerator neutrino experiments. As pointed out by the authors of Ref.~\cite{Ge2017}, atmospheric neutrino experiments can also look for these processes, benefiting from the increase of the cross section at large energies. 

We have stressed the need for a full four-body phase space calculation of the trident cross sections without using the EPA. This approximation has been employed in recent calculations and can lead to overestimations of the cross section by 200\% or more at the peak neutrino energies relevant for many accelerator neutrino experiments.
%
Moreover, we show why the EPA is not applicable for computing trident cross sections, and provide the first quantitative assessment of this breakdown for coherent and diffractive hadronic regimes. 
%
We find that the breakdown of the approximation is most severe for processes with electrons in the final-state and for diffractive scattering of all final state flavours. 
%
For coherent dimuon production, the approximation can give a reasonable result at large neutrino energies. This is due to the nuclear form factors that serendipitously suppress those regions of phase space where the EPA is least applicable. We also demonstrated that the best results in this channel are achieved when applying artificial cuts to the phase space.
%
However, even in this case, at energies relevant for the above experiments, the EPA can artificially suppress the coherent scattering contribution and increase the diffractive one giving rise to an incorrect rate and distributions of observable quantities. 
%
For instance, the invariant mass of the charged lepton pair $m^2_{\ell \ell}$ and their angular separation $\Delta \theta$ are more uniformly distributed for diffractive than for coherent trident scattering. Using the correct distributions is crucial to correctly disentangle the signal from the background by cutting on these powerful discriminators.

Our calculations show that DUNE ND is the future detector with the highest neutrino trident statistics, more than 6000 mixed events, 11\% produced by diffractive scattering, more than 1900 dielectron events, 5\%  produced by diffractive scattering and about 750 dimuon events,
almost 34\% of those produced by a diffractive process. Making use of our efficiencies (see \reftab{tab:DUNE_ND_NU_BG}), assuming an ideal background suppression and neglecting systematic uncertainties, we quote the statistical uncertainty on the coherent-like flux averaged cross section for the DUNE ND. We do this for coherent only events and, in brackets, for coherent plus diffractive events, yielding
%
\[\frac{\delta \langle \sigma^{e^\pm\mu^\mp} \rangle}{\langle \sigma^{e^\pm\mu^\mp} \rangle} =  1.8\% \, (1.6\%), \quad \frac{\delta \langle \sigma^{e^+e^-} \rangle}{\langle \sigma^{e^+e^-} \rangle} =  3.4\% \,(3.3\%) \quad \mathrm{and} \quad \frac{\delta \langle\sigma^{\mu^+\mu^-} \rangle}{\langle \sigma^{\mu^+\mu^-} \rangle} =  5.5\% \,(5.1\%).\]
%
In this optimistic framework we expect the true statistical uncertainty on coherent-like tridents to lie between the two numbers quoted, depending on how many diffractive events contribute to the coherent-like event sample. This impressive precision would provide unprecedented knowledge of the trident process and the nuclear effects governing the interplay between coherent and diffractive regimes. We emphasize, however, that given these small values for the relative uncertainties, the trident cross section will likely be dominated by systematic uncertainties from detector response and backgrounds which are not modelled here. 

For DUNE ND, we have studied the distribution of observables which could help distinguish trident events from the background. We have estimated the background for each trident channel via a Monte Carlo simulation using GENIE, and identified the dominant contributions arising primarily from particle misidentification.  
%
We conclude that reaching background rates of the order 
${\cal O}(10^{-6}-10^{-5})$ times the CC rate is necessary to observe trident events at DUNE ND, and given the distinctive kinematic behaviour of the trident signal a simple cut-based GENIE-level analysis suggests that this is an attainable goal in a LAr TPC. 

Existing facilities may also be able to make a neutrino trident measurement at their near detectors. Despite not including reconstruction efficiencies nor an indication of the impact of backgrounds, we find that the largest trident statistics is available at INGRID, the T2K on-axis near detector. We predict about 660 (1700) events for the mixed flavour, 300 (770) events for the dielectron and 50 (130) events for the dimuon channel for T2K-I (T2K-II). The more fine-grained near detector of MINOS and MINOS+ is also expected to have collected a significant numbers of events during its run. As such, the very first measurement of neutrino trident production of mixed and dielectron channels may be at hand.


%%%%%%%%%%%%%%%%%%%%%%%%%%%%%%%%%%%%%%%%%%%%%%%%%%%%
\section{\label{app:formfactors}Form Factors}

In the coherent regime, we use a Woods-Saxon (WS) form factor due to its success in reproducing the experimental data \cite{Fricke:1995zz,Jentschura2009}. The WS form factor is the Fourier transform of the nuclear charge distribution, defined as 
%
\begin{equation}
 \rho(r) = \frac{\rho_0}{1+\exp\left(\dfrac{r - r_0}{a}\right)} \, ,
\end{equation}
%
where we take $r_0 = 1.126 \, A^{1/3}$ fm and $a = 0.523$ fm. One can then calculate the WS form factor as
%
\begin{equation}
 F(Q^2) = \frac{1}{\int \rho(r) \, \dd^3r}  \int \rho(r) \, \exp\left( -i \vec{q} \vdot \vec{r}\right) d^3r \, .
\end{equation}
%
Here we use an analytic expression for the symmetrized Fermi function \cite{Anni1994,Sprung1997} instead of calculating the WS form factor numerically. This symmetrized form is found to agree very well with the full calculation and reads
%
\begin{equation}
  F(Q^2) =  \frac{3 \pi  a}{r_0^2 + \pi^2 a^2} \frac{\pi a \coth{(\pi Q a)} \sin{(Q r_0)} - r_0 \cos{(Q r_0)} }{Q r_0 \sinh{(\pi Q a)}}\, .
\end{equation}
%

In the diffractive regime, we work with the functions $H_1^{\rm N}(Q^2)$ and $H_2^{\rm N}(Q^2)$, which depend on the Dirac and Pauli form factors of the nucleon ${\rm N}$ as follows
%
\begin{equation}
H_1^{\rm N}(Q^2)= |F_1^{\rm N}(Q^2)|^2 - \tau |F_2^{\rm N}(Q^2)|^2\, , \quad \mathrm{and} \quad H_2^{\rm N}(Q^2) = \left| F_1^{\rm N}(Q^2) + F_2^{\rm N}(Q^2)\right|^2 \, ,
\end{equation}
%
where $\tau = -Q^2/4M^2$. The form factors $F_1^{\rm N}(Q^2)$ and $F_2^{\rm N}(Q^2)$ can be related to the usual Sachs electric $G_\mathrm{E}$ and magnetic $G_{\mathrm{M}}$ form factors. These have a simple dipole parametrization
%
\begin{align}
G^{\rm N}_E(Q^2) =& F^{\rm N}_1 (Q^2) + \tau F^{\rm N}_2 (Q^2) = \begin{cases}
                                              0,\, &\mathrm{if }\, {\rm N} = n,\\
                                              G_D(Q^2),\, &\mathrm{if }\, {\rm N} = p,
                                              \end{cases} \\
G^{\rm N}_M(Q^2) =& F^{\rm N}_1 (Q^2) + F^{\rm N}_2 (Q^2) = \begin{cases}
                                              \mu_n  \, G_D(Q^2),\, &\mathrm{if }\, {\rm N} = n,\\
                                              \mu_p  \, G_D(Q^2),\, &\mathrm{if }\, {\rm N} = p,
                                              \end{cases}
\end{align}
%
where $\mu_{p,n}$ is the nucleon magnetic moment in units of the nuclear magneton and $G_D(Q^2) = (1 + Q^2/M_V^2)^{-2}$ is a simple dipole form factor with $M_V = 0.84$ GeV.
%


%%%%%%%%%%%%%%%%%%%%%%%%%%%%%%%%%%%%%%%%%%%%%
\section{Kinematical Distributions \label{app:distributions}}

\begin{figure}[t]
\centering
\includegraphics[width=0.92\textwidth]{figs/DUNE_nu_3horn_aC_Q2_thetapm.pdf}
\caption{Flux convolved neutrino trident production distributions for DUNE ND in neutrino mode in additional variables. In purple we show the coherent contribution in $^{40}$Ar and in blue the diffractive contribution from protons as targets only (including Pauli blocking). The coherent and diffractive distributions are normalized independently. \label{fig:other_dists}}
\end{figure}

%
In this section, we show additional distributions in different observables for neutrino trident production, also focused on the DUNE ND in neutrino mode, in \reffig{fig:other_dists}. While trident events are generally quite forward going, their angular behaviour is quite interesting. We consider here the angle between the charged lepton cone and the neutrino beam, $\alpha_C$, defined as 
%
\[
\cos{\alpha_C} = \frac{ (\vec{p}_3 + \vec{p}_4)  \vdot \vec{p}_1}{ |\vec{p}_3 + \vec{p}_4| |\vec{p}_1| }\, ,\]  
%
and in the individual angle of the charged lepton to the neutrino beam, $\theta$. For same flavour tridents we define $\theta$ for each charge of the visible final-state, whilst for mixed tridents we use their flavour. We also show the distribution in $Q^2 = {-q^2}$, where $q = (P - P^\prime)$, which is of particular interest when considering coherency and the impact of form factors.


%%%%%%%%%%%%%%%%%%%%%%%%%%%%%%%%
\section{Individual Backgrounds}
\label{app:backgrounds}
Here we discuss backgrounds to trident final-states in more detail. We start by motivating our misID rates shown in \reftab{tab:misIDlist}, and then discuss the dominant background processes individually.

In LAr photons can be distinguished from a single electron if their showers start displaced from the vertex (if present). Photons have a conversion length in LAr of around 18 cm, meaning $5$--$10\%$ could be expected to convert quickly enough to hinder electron-photon discrimination by this means if the resolution on the gap is from $1$--$2$ cm \cite{Acciarri:2016sli}. Once pair conversion happens, photons can be distinguished from a single electron purely by $\dd E/\dd x$ measurements in the first 1--2 cm of their showers. Motivated by the success of this method as shown at ArgoNeuT \cite{Acciarri:2016sli} and based on projections for DUNE \cite{Acciarri:2016ooe}, we assume that $5\%$ of photons would be taken as $e^\pm$ with perfect efficiency, without the need for an event vertex. Needless to say that a dedicated study for trident topologies would be necessary for a more complete study. It is worth noting that our remarks concern only the misID of a single photon for a single electron, whilst the distinction between a photon and an overlapping $e^+e^-$ pair without a vertex can be much more challenging. For this reason we take the misID rate between an overlapping $e^+e^-$ pair and a photon to be 1 in the absence of a vertex.  

Charged pions are notorious for faking long muon tracks. We estimate this misID rate as arising from through-going pions, which do not exhibit the decay kink used in their identification. We assume an interaction length of around $1$ m, meaning that about $5\%$ of particles travel $\sim3$ meters and escape the fiducial volume. Assuming that this is the most likely way a pion can spoof a muon, we estimate a naive suppression rate of $10^{-2}$. In a more complete study, it is desirable to explore the length of the muon and pion tracks inside the detector as a function of energy. The length of the contained tracks can also be an important tool for background suppression which we leave to future studies.  

%%%%%%%%%%%%%%%%%%%%%%%%%%%%%%%%%%%%%%%%%%%%%%%%%%%%
\subsection{Pion Production}

Coherent pion production in its charged ($\nu + A \to \ell^\mp + \pi^\pm + A$) and neutral ($\nu + A \to \nu + \pi^0 + A$) current version is very abundant at GeV energies. The cross section for these processes is modelled in GENIE using a modern version of the Rein-Sehgal model \cite{REIN198329,Rein:2006di}. The charged current version serves mainly as a background to $\mu^+ \mu^-$ tridents, but can also appear as a background for $e^\pm \mu^\mp$ tridents for incoming electron neutrinos or antineutrinos. It has been studied before at MiniBooNE \cite{AguilarArevalo:2010xt}, MINER$\nu$A \cite{Higuera:2014azj,Mislivec:2017qfz}, T2K \cite{Abe:2016fic,Abe:2016aoo}, and for the first time in LAr at ArgoNeuT \cite{Acciarri:2014eit}. This process has a very distict low 4-momentum transfer to the nucleus $|t|$ \cite{Higuera:2014azj}, but a much flatter distribution in invariant mass if compared to trident. The neutral current version of coherent pion production serves as a background to $e^+e^-$ tridents. This process has been studied before by the MiniBooNE \cite{AguilarArevalo:2009ww}, SciBooNE \cite{Kurimoto:2010rc} and in LAr by the ArgoNeuT collaboration \cite{Acciarri:2015ncl}. There are two possibilities for these events to fake an $e^+e^-$ trident: when one of the gammas produced in the $\pi^0$ decay is missed and the other is misIDed for an overlapping $e^+e^-$ pair, and when both photons are each misIDed for a single electron. This signature also comes with low hadronic activity, but for separated visible photons the invariant mass is a natural discriminator, as in the detector $m_{\gamma \gamma} \approx m_{\pi^0}$.

Resonant pion production can also contribute to trident backgrounds in the absence of any reconstructed protons. Resonant pion production can be larger than its coherent counterpart and is modelled in GENIE by the Rein-Sehgal model \cite{Rein:1980wg}. Its CC version was measured by MiniBooNE \cite{AguilarArevalo:2010xt}, K2K \cite{Mariani:2010ez}, MINOS \cite{Adamson:2014pgc}, and MINER$\nu$A \cite{Altinok:2017xua}. In the latter measurement one can clearly see the large number of events with undetected protons. The misIDed photon and the charged lepton invariant mass are once more flatter than the trident ones, allowing for a kinematical discrimination whenever a single photon is undetected. It is worth noting that these are some of the dominant underlying processes for pion production in GENIE, but all events leading to topologies relevant for trident are included in our analysis.
%
%%%%%%%%%%%%%%%%%%%%%%%%%%%%%%%%%%%%%%%%%%%%%%%%%%%%
\subsection{Charm Production}

Since the first observation of dimuon pairs from charm production in neutrino interaction by the HPWF experiment in 1974 \cite{Benvenuti:1975ru}, a lot has been learned about these processes (see \cite{Lellis:2004yn} for a review) in neutrino experiments. Particularly, the production of charm quarks and their subsequent weak decays into muons or electrons have been identified as a major source of background for early trident searches. At the lower neutrino energies at DUNE, however, this is expected to be a smaller yet non-negligible contribution. From our GENIE samples, we estimate that a charmed state is produced at a rate of around $10^{-4}(N_\text{CC}+N_\text{NC})$. Most of these produce either D mesons, 
$\Lambda_c$ or $\Sigma_c$ baryons. These particles decay in chains, emitting a muon with a branching ratio of around $0.1$, and are always accompanied by pions or other hadronic particles. We therefore expect these rates to be negligible with a hadronic veto, and do not consider them further. We hope, however, that future studies will address these channels in more detail.

%%%%%%%%%%%%%%%%%%%%%%%%%%%%%%%%%%%%%%%%%%%%%%%%%%%%
\subsection{CC$\gamma$ and NC$\gamma$}

The emission of a single photon alongside a CC process could be a background for $\mu e$ tridents if the photon is misIDed as a single electron. When the photon is produced in a NC event, it can be a background to overlapping $e^+e^-$ tridents. In GENIE, these topologies arise mainly due to resonance radiative decays and from the intra-nuclear processes. For this reason, it usually comes accompanied with extra hadronic activity. For hadronic resonances, we have simulated CC processes in GENIE and estimated
the multiplicities: $0.5\%$ single
photon and $1\%$ double photon emission from CC rates. Radiative photon production from the charged lepton, on the other hand, does not need to come accompanied by hadrons. It is phase space and $\alpha\approx1/137$ suppressed with respect to CCQE rates, and therefore could occur at appreciable rates compared to our signal. This contribution, however, is not included in GENIE and is absent from our samples. The rates of internal photon bremsstrahlung have been estimated before, particularly for T2K where a low-energy photon is an important background for electron
appearance searches \cite{Efrosinin:2009zz}, and as a background to the low energy events at MiniBooNE \cite{Bodek:2007wb}. De-excitation gammas from the struck nuclei can also generate CC$\gamma$ or NC$\gamma$ topologies \cite{PhysRevLett.108.052505}. These contributions for Ar are not included in GENIE, but are expected to come with a distinct energy profile, which can be tagged on.

%%%%%%%%%%%%%%%%%%%%%%%%%%%%%%%%%%%%%%%%%%%%


\chapter{New fundamental forces at DUNE}
\input{Zprime_scattering/Zprime_scattering}

\chapter{A light dark neutrino sector}
%%%%%%%%%%%%%%%%%%%%%%%%%%%%%%%%%%%%%%%%%%%%%%%%%%%%
\graphicspath{{}{dark_nus/}}

The most important evidences that the Standard Model (SM) of particle physics is incomplete are neutrino masses and mixing, and the presence of dark matter (DM) in the Universe. Both call for extensions of the SM and the possible existence of dark sectors which do not partake in SM interactions, or do so with extremely weak couplings while displaying strong ``dark" interactions~\cite{Boehm:2003hm,*Boehm:2003ha,Alexander:2016aln}.
Such sectors might exist at relatively light scales below the electroweak one, being within reach of present and future non-collider experiments. Generically, a neutral dark sector can communicate with the SM via three renormalizable portals. New neutral fermions mix with light neutrinos unless a symmetry differentiates the two, a possibility usually denoted as the neutrino portal. New vector particles can kinetically mix with the SM hypercharge, and new scalars mix with the Higgs boson through the so-called vector and scalar portals, respectively. The latter terms are generically allowed in the Lagrangian and an explanation of their smallness requires specific UV completions.  

In this article, we propose a new neutrino model with a hidden $U(1)^\prime$ gauge symmetry under which no SM fields are charged. We introduce new SM-neutral fermions, $\nu_D$ and an additional sterile neutrino $N$. The symmetry is subsequently broken by the vacuum expectation value (vev) of a complex dark scalar $\Phi$, which gives mass to the new gauge boson. For concreteness, we restrict the scale of the breaking to be below the electroweak one. 

Models with heavy neutrinos which are not completely sterile and might participate in new gauge interactions have been studied in several contexts, including $B-L$, $L_\mu-L_\tau$ and left-right symmetric models~\cite{%
Buchmuller:1991ce,% original
Khalil:2006yi,% B-L at TeV scale
Perez:2009mu,% B-L extensive discussion + X model
Khalil:2010iu,% B-L at collider and ISS
Dib:2014fua,% B-L in linear seesaw
Baek:2015mna,% mu-tau in inverse seesaw
DeRomeri:2017oxa,% Elusive B-L
Nomura:2018mwr,% U(1)_R
Brdar:2018sbk% LR symmetry low scale
}, but here we focus on the possibility of a symmetry under which no SM fields are charged~\cite{%
Okada:2014nsa,% U(1)prime
Diaz:2017edh,% Hidden with loops
Nomura:2018ibs% Hidden in linear seesaw
}. New heavy neutral fermions that feel such hidden forces, such as $\nu_D$, are referred to as \textit{dark neutrinos}, since they define a dark sector separate from the SM. Nevertheless, the dark interactions ``leak" into the SM sector via neutrino mixing, where they may dominate~\cite{Pospelov:2011ha,Batell:2016zod}. Models of this type have been invoked to generate large neutrino non-standard interactions~\cite{Farzan:2015doa,Farzan:2016wym}, generate new signals in DM experiments~\cite{Pospelov:2011ha,Pospelov:2012gm,*Pospelov:2013rha,Harnik:2012ni,McKeen:2018pbb}, weaken cosmological and terrestrial bounds on eV scale sterile neutrinos~\cite{Hannestad:2013ana,*Dasgupta:2013zpn,*Mirizzi:2014ama,*Chu:2015ipa,*Cherry:2016jol,*Chu:2018gxk,Denton:2018dqq,*Esmaili:2018qzu}, and as a potential explanation of anomalous short-baseline results at the MiniBooNE~\cite{AguilarArevalo:2007it,*Aguilar-Arevalo:2018gpe} and/or LSND~\cite{PhysRevLett.77.3082,*Aguilar:2001ty} experiments with new degrees of freedom at the MeV/GeV scale~\cite{Gninenko:2009ks,Gninenko:2010pr,Masip:2012ke,Radionov:2013mca,Ballett:2018ynz,Bertuzzo:2018itn,Arguelles:2018mtc}.

Our model presents all the three renormalizable portals to the SM. The Yukawa interactions between the leptonic doublet and $N$, and between $N$ and $\nu_D$ induce neutrino mixing. 
The gauge symmetry allows a cross-coupling term in the potential between the Higgs and the real part of the scalar, inducing mixing between the two after symmetry breaking. The broken gauge symmetry implies the existence of a light hidden gauge boson $X_\mu$, which mediates the dark neutrino interactions and generically kinetically mixes with the SM hypercharge. The set-up is self-consistent and combines the three portals into a unified picture that exhibits significantly different phenomenology with respect to each portal taken separately, as we discuss. The interplay of the different portal degrees of freedom leads to novel signatures which would have escaped searches performed to date, and that can explain long-standing anomalies. For the latter, we focus on the MiniBooNE anomaly as discussed in Ref.~\cite{Ballett:2018ynz} (see also~\cite{Bertuzzo:2018itn}) and on new neutrino scattering signatures at neutrino experiments~\cite{Arguelles:2018mtc}. We also reconsider the possibility to explain the discrepancy between the prediction and measurement of the anomalous magnetic moment of the muon ($\Delta a_\mu$)~\cite{Bennett:2006fi} via kinetic mixing~\cite{Fayet:2007ua,*Pospelov:2008zw}.

An interesting feature of the model is the generation of neutrino masses at loop-level. This requires only two key features of our setup, namely a light $Z^\prime$ and neutrino mixing, but not the vector and scalar portals. For this reason, we discuss it elsewhere~\cite{Ballett:2019cqp}.

In its minimal form, the model is not anomaly-free. We discuss how this can be cured and propose a minor extension that introduces additional dark sector neutral fermions charged under the new symmetry~\cite{Boehm:2003hm,*Boehm:2003ha}. Neutrinos, we argue, may be a window into such dark sectors, bridging the puzzles of neutrino masses and DM~\cite{Ma:2006km,Farzan:2009ji,Farzan:2010mr,Arhrib:2015dez,Cherry:2014xra,Escudero:2016tzx,Escudero:2016ksa,Batell:2017cmf,Capozzi:2017auw,Campo:2017nwh,Blennow:2019fhy}. We briefly outline the key features of a DM extension and leave a more detailed analysis to future work.

%%%%%%%%%%%%%%%%%%%%%%%%%%%%%%
%
\begin{figure}[t]
\centering\includegraphics[width=0.7\textwidth]{dark_nus/portals.pdf}
\caption[Dark neutrino model diagram.]{Schematic representation of our dark neutrino model. The dark neutrino, $\nu_{D}$ and the complex scalar $\Phi$ are the only fields charged under the new U$(1)^\prime$ gauge symmetry. The new vector boson $X_\mu$ acquires a mass after spontaneous symmetry breakig, and $N$ remains a complete singlet.
\label{fig:Tdiagrams}}
\end{figure}
%
%%%%%%%%%%%%%%%%%%%%%%%%%%%%%%



%%%%%%%%%%%%%%%%%%%%%%%%%%%%%%%%%%%%%%%%%%%%%%%%%%%%%%%%%%%%%%%%%%%%%%%
\section{The Model} 
We extend the SM gauge group with a new abelian gauge symmetry $U(1)^\prime$ with associated mediator $X_\mu$ and introduce three new singlets of the SM gauge group: a complex scalar $\Phi$, and two left-handed fermions $\nu_{D,L} \equiv \nu_{D}$ and $N_L \equiv N$. 
%As shown in \reftab{tab:fields}, t
The scalar $\Phi$ and the fermion $\nu_{D}$ are equally charged under the new symmetry, and $N$ is neutral with respect to all gauge symmetries of the model. For simplicity, we restrict our discussion to a single generation of hidden fermions. The relevant terms in the gauge-invariant Lagrangian are 
%
\begin{align} \label{eq:lagrangian}
%
\mathscr{L}  \supset  &\left(D_\mu \Phi\right)^\dagger \left(D^\mu \Phi\right) -  V(\Phi,H) \,   \nonumber\\
&  - \frac{1}{4}X^{\mu \nu} X_{\mu \nu} + \overline{N}i\slashed{\partial}N + \overline{\nu_D}i\slashed{D}\nu_D 
\nonumber\\
&- \left[y^\alpha_\nu (\overline{L_\alpha} \cdot \widetilde{H})N^c + \frac{\mu^\prime}{2}\overline{N}N^c + y_N \overline{N}\nu_D^c\Phi + \text{h.c.}\right],
\end{align}
%
where $X^{\mu\nu}$ is the field strength tensor for $X_{\mu}$, $D_\mu \equiv \left(\partial_\mu-ig^\prime X_\mu\right)$ the covariant derivative, $L_\alpha \equiv (\nu_\alpha^T, \ell_\alpha^T)^T$ the SM leptonic doublet of flavour $\alpha = e, \mu, \tau$ and $\widetilde{H} \equiv i \sigma_2 H^*$ is the charge conjugate of the SM Higgs doublet. We write $y_\nu^\alpha$ for the $L_\alpha$--$N$ Yukawa coupling, $y_N$ for the $\nu_D$--$N$ one, and $\mu^\prime$ for the Majorana mass of $N$, which is allowed by the SM and the new gauge interaction, although it breaks lepton number by 2 units.

The minimisation of the scalar potential $V(\Phi,H)$ leads the neutral component of the fields $H$ and $\Phi$ to acquire vevs $v_H$ and $v_\varphi$, respectively. The latter also generates a mass for both the new gauge boson $X_\mu$ and the real component of the scalar field $\varphi$. Although $v_\varphi$ is arbitrary, we choose it to be below the electroweak scale, $v_\varphi < v_H$, as we are interested in building a model testable at low scales.

\paragraph{Neutrino portal}
In the neutral fermion sector and after symmetry breaking, two Dirac mass terms are induced with $m_D \equiv y_\nu^\alpha v_H/\sqrt{2}$ and $\Lambda \equiv y_N v_\varphi/\sqrt{2}$.
It is useful to consider the form of the neutrino mass matrix in the single generation case to clarify its main features. For one active neutrino $\nu_\alpha$ ($\alpha= e, \mu, \tau$), it reads
%
\begin{align} \label{eq:massmatrix}
\mathscr{L}_{\rm mass} \supset
\frac{1}{2}\left (\begin{matrix} \overline{\nu}_\alpha & \overline{N} &  \overline{\nu_D} \end{matrix} \right )
\left(\begin{matrix} 
     0   &  m_D        & 0 
\\ m_D &  \mu^\prime & \Lambda 
\\   0   &  \Lambda  & 0
\end{matrix}\right)
\left (\begin{matrix} \nu_\alpha^c \\ N^c \\ \nu_D^c \end{matrix} \right) + {\rm h.c.}
\end{align}  
%
The form of this matrix appears in Inverse Seesaw (ISS)~\cite{Mohapatra:1986bd,*GonzalezGarcia:1988rw} and in Extended Seesaw (ESS)~\cite{Barry:2011wb,*Zhang:2011vh} models. In fact, it is the same matrix discussed in the so-called Minimal ISS~\cite{Dev:2012sg}, with the difference that in our case its structure is a consequence of the hidden symmetry.
After diagonalisation of the mass matrix, the two heavy neutrinos, $\nu_h$ with $h=4,5$, acquire masses. Assuming that $m_D \ll \Lambda$, we focus on two interesting limiting cases. 

In the \textit{ISS-like} limit, where $\Lambda \gg \mu^\prime$ and the two heavy neutrinos are nearly degenerate, we have 
\begin{align}
 m_5 \simeq - m_4 \simeq \Lambda ~,  \,\,  m_5-|m_4| = \mu^\prime ~,& \nonumber\,\,
 U_{\alpha 5} \simeq U_{\alpha 4} \simeq  \frac{m_D}{\sqrt{2}\Lambda}~, \\   U_{D i} \simeq \frac{m_D}{\Lambda}, \,\, U_{D5} \simeq U_{D4} \simeq \frac{1}{\sqrt{2}} ~, \,\, & U_{N5} \simeq U_{N4} \simeq \frac{1}{\sqrt{2}} ~.\nonumber
\end{align}

In the \textit{ESS-like} case, $\Lambda \ll \mu^\prime$, one neutral lepton remains very heavy and mainly in the completely neutral direction $N$, and the other acquires a small mass via the seesaw mechanism in the hidden sector. We find
%
\begin{align}
m_4 \simeq -\frac{ \Lambda^2}{\mu^\prime}~, & \,\,  m_5 \simeq \mu^\prime ~, \,\, U_{\alpha 4} \simeq U_{\alpha 5}\sqrt{\frac{m_5}{\left|m_4\right|}}  \simeq  \frac{m_D}{\Lambda}~,\nonumber \\ \,\, U_{D i} \simeq \frac{m_D}{\Lambda},  &\,\,  U_{N5} \simeq U_{D4} \simeq 1 ~, \,\,  U_{D5} \simeq U_{N4} \simeq \frac{\Lambda}{\mu^\prime} ~.\nonumber
\end{align}
%
From the discussion above, it is clear that the masses of $Z^\prime$ and $\varphi^\prime$ are typically above the heavy neutrino ones, unless we are in the ESS-like regime.

The Yukawa terms in \refeq{eq:lagrangian} induce {\em neutrino mixing} between the active (light) and heavy (sterile, dark) neutrinos. In this model, similarly to the ISS and the ESS cases, this mixing can be much larger than the typical values required in type-I seesaw extensions to explain neutrino masses, making its phenomenology more interesting. The determinant of the mass matrix in \refeq{eq:massmatrix} is zero, and so light neutrino masses vanish at tree-level and do not constrain the values of the active-heavy mixing angles. This, however, is no longer the case at one-loop level, as light neutrino masses emerge through radiative corrections from diagrams involving the $\varphi^\prime$ and $Z^\prime$ degrees of freedom~\cite{Ballett:2019cqp}.  


\paragraph{Scalar portal}
In the scalar potential, the symmetries of the model allow us to write down the following term
\begin{equation}
    V(\Phi,H) \supset \lambda_{\Phi H} \, H^\dagger H \left| \Phi\right|^2,
\end{equation}
where we identify $\lambda_{\Phi H}$ as the scalar portal coupling~\cite{Barger:2008jx}, responsible for mixing in the neutral scalar sector. If such a term exists, the scalar mass eigenstates $(h^\prime, \varphi^\prime)$ mix with the gauge eigenstates $(h, \varphi)$ 
%as $h^\prime = h \cos{\alpha}  - \varphi \sin{\alpha} $ and  $\varphi^\prime = h \sin{\alpha}  + \varphi \cos{\alpha}$,
with a mixing angle $\alpha$ defined by 
%
\begin{equation}
\tan{(2\alpha)} \equiv \frac{\lambda_{\Phi H} v_{H} v_\varphi}{\lambda_h v_{H}^2 - \lambda_\varphi v_\varphi^2},
\end{equation}
%
where $\lambda_h$ and $\lambda_\varphi$ are the quartic couplings of the Higgs and $\Phi$ scalars, respectively. 

\paragraph{Vector portal}  Similarly, mixing also arises in the neutral vector boson sector from the allowed kinetic mixing term~\cite{Holdom:1985ag}
%
\begin{equation}
 \mathscr{L} \supset - \frac{\sin{\chi}}{2} \, F^{\mu \nu} X_{\mu \nu},
\end{equation} 
%
where $F_{\mu\nu}$ is the SM hypercharge field strength. This term may be removed with a field redefinition, resulting in three mass eigenstates $\left( A,\, Z^0,\, Z^\prime\right)$, corresponding to the photon, $Z^0$-boson and the hypothetical $Z^\prime$-boson. For a light $Z^\prime$, the $Z^\prime$ coupling to SM fermions $f$ to first order in the small parameter $\chi$ is given by
%
\begin{equation}
\mathscr{L} \supset - (e\,q_f\,c_{W}) \chi \,\overline{f} \gamma^\mu f\,Z^\prime_\mu ~,
\end{equation}
%
with  $q_f$ the fermion electric charge.

The values of $\chi$ and $\lambda_{\Phi H}$ are arbitrary and could be expected to be rather large. As such, we treat them as free parameters within their allowed ranges. Here, we merely note that with our current minimal matter content, $\chi$ and $\lambda_{\Phi H}$ receive contributions at loop level from the $(\overline{L}_\alpha \cdot \widetilde{H})N^c$ and $\overline{N} \nu_D^c \Phi$ terms, which are necessarily suppressed by neutrino mixing ($\chi \propto g^\prime e |U_{\alpha h}|^2$ and $\lambda_{\Phi H} \propto |U_{\alpha h}|^2$). These values constitute a lower bound and larger values should be expected in a complete model.


%%%%%%%%%%%%%%%%%%%%%%%%%%%%%%%%%%%%%%%%%%%%%%%%%%%%%%%%%%%%%%%%%%%%%%%
%%%%%%%%%%%%%%%%%%%%%%%%%%%%%%%%%%%%%%%%%%%%%%%%%%%%%%%%%
\subsection{Portal phenomenology} \label{sec:portal_pheno}

The interplay between portal couplings and the heavy neutrinos $\nu_h$ ($h=4,5$) leads to a distinct, and possibly richer, phenomenology to what is commonly discussed in the presence of a single portal. We present here some of the most relevant signatures, devolving a longer study to future work.

\paragraph{Heavy neutrino searches} The strongest bounds on heavy neutrinos in the MeV--GeV mass range come from peak searches in meson decays~\cite{
%KEK
Yamazaki:1984sj,
%NA48/2
Artamonov:2014urb,
%NA62
CERNNA48/2:2016tdo}
and beam dump experiments~\cite{
%PS-191
Bernardi:1985ny,
%CHARM
Bergsma:1983rt,
%NA3
Badier:1986xz,
%NuTeV
Vaitaitis:1999wq, 
%BEBC
CooperSarkar:1985nh, 
%NOMAD
Astier:2001ck} looking for visible $\nu_h$ decays. 
These, however, can be weakened if the $\nu_h$ decays are sufficiently different from the case of ``standard" sterile neutrinos with SM interactions suppressed by neutrino mixing.
We now discuss how this may happen, depending on the mass hierarchy of the two heavy neutrinos and the values of neutrino and kinetic mixing. For concreteness, we focus on specific benchmark points (BP) that illustrate the key features.
In the ISS-like regime, we take $m_4/m_5=99\%$ and choose $m_4\simeq m_5 = 100$~MeV. If $\chi$ is negligible, we have that $\nu_h$ decays as in the standard sterile case via SM interactions. This is because the $\nu_5\rightarrow \nu_4 \bar{\nu}_\alpha \nu_\alpha$ decay is phase-space suppressed ($\Gamma_{\nu_5 \to \nu_4 \nu\nu} \propto \mu^{\prime\, 5}$), and because $Z^\prime$ mediated decays into three light neutrinos are negligible for small mixing, as $\Gamma_{\nu_h \to \nu\nu\nu} \propto |U_{\alpha h}|^6 m_h^5/m_{Z^\prime}^4$. If $\chi$ is sizeable, on the other hand, new visible decay channels dominate, specifically $\nu_4 \rightarrow \nu_\alpha e^+ e^-$ for this BP. The corresponding decay rate is given by 
\begin{equation}
    \Gamma(\nu_4 \to \nu_\alpha e^+ e^-)  \approx \frac{1}{2}\frac{e^2 \chi^2 g^{\prime \, 2} |U_{\alpha 4}|^2}{192 \pi^3} \frac{m_4^5}{m_{Z^\prime}^4}.
\end{equation}
Depending on the value of $\chi$ and $m_Z^\prime$ this decay can be much faster than in the SM, implying stronger constraints on the neutrino mixing parameters as discussed in Ref.~\cite{Ballett:2016opr}. For heavier masses, additional decay channels, e.g. $\nu_4 \rightarrow \nu_\alpha \mu^+ \mu^-$, would open. A feature of the model is that such channel would have the same BR as the electron one, albeit phase space suppressed. No two-body decays into neutral pseudoscalars arise due to the vector nature of the gauge coupling, unless mass mixing is introduced (see~\cite{Ilten:2018crw} for a thorough discussion of the decay products of a dark photon).
We consider also a BP in the ESS-like regime. We take $m_4 = m_5/10$. In this case, $\nu_5$ decays into 3 $\nu_4$ states very rapidly. The subsequent decays of $\nu_4$ would proceed as discussed above and would be much slower than the $\nu_5$ one, given the hierarchy of masses and the further suppression due to neutrino and/or kinetic mixing. 

For large $\chi$, peak searches and bounds on lepton number violation (LNV) from meson and tau decays may be affected~\cite{Atre:2009rg,Abada:2017jjx}. Despite simply relying on kinematics, we note that in peak searches the strict requirement of a single charged track in the detector~\cite{Artamonov:2014urb} would, in fact, veto a large fraction of new physics events if $\nu_h$ decays promptly into $\nu_\alpha e^+e^-$, for instance. In addition, LNV meson and tau decays would need to be reconsidered as the intermediate on-shell $\nu_h$ could decay dominantly via the novel NC interactions and the $\ell \pi$ and $\ell K$ final states would be absent.

\paragraph{Dark photon searches} Bounds on the vector portal come from several different sources~\cite{Curtin:2014cca,Bauer:2018onh}. Electroweak precision data and measurements of the $g-2$ of the muon and electron constrain our model~\cite{Hook:2010tw}. Major efforts at collider and beam dump experiments led to strong constraints on dark photons by searching for the production and decay of these particles. Such bounds, however, depend on the lifetime of the $Z^\prime$ and on its branching ratio (BR) into charged particles. In our model, the $Z^\prime$ decays invisibly into heavy fermions if $m_{Z^\prime} > 2 m_4$ and into light neutrinos otherwise. In the latter case, constraints would be much weaker than usually quoted with only mono-photon searches~\cite{Lees:2017lec} applying. In the former case, however, new signatures arise, where the subsequent decay of $\nu_h$ leads to multi-lepton/multi-meson events, potentially with displaced vertices and providing a very clean experimental signature. Notably, if the $Z^\prime$ decays into $\nu_h$ states that subsequently decay sufficiently fast within the detector, even the ``invisible decay" bounds will be weakened.


\paragraph{Revisiting $\Delta a_\mu$} The above possibility opens the option to explain the discrepancy between the theoretical prediction~\cite{Blum:2018mom,*Keshavarzi:2018mgv} and the experimental value~\cite{Bennett:2006fi} of the $(g-2)$ of the muon via kinetic mixing. For instance, a $1$ GeV $Z^\prime$ with $\chi = 2.2\times10^{-2}$ can explain $a_\mu$. Taking $\nu_4$ around 400 MeV (800 MeV) and $m_5 > m_{Z^\prime}$, then the $Z^\prime$ would decay into 2 $\nu_4$ ($\nu_4 \nu_\alpha$) immediately. For the quoted value of the kinetic mixing and the largest neutrino mixing allowed, these heavy fermions would further decay into $e^+ e^-$ and $\mu^+ \mu^-$ pairs plus missing energy with sub-meter decay lengths.  This region of the $\chi$ parameter space is constrained only by the BaBar $e^+e^-$ collider searches for visible~\cite{Lees:2014xha} and invisible decays~\cite{Lees:2017lec} of a standard dark photon. Both of these searches would veto the three-body decays of $\nu_4$, opening up a large region of parameter space (see Ref.~\cite{Mohlabeng:2019vrz} for a similar discussion in an inelatic DM model). Resonance searches still constrain the $Z^\prime$ BR into $e^+ e^-$ and $\mu^+ \mu^-$ which are proportional to $\chi^2$, providing a weak upper bound. In order to shorten the lifetime of $\nu_4$, we can increase mixing with the tau neutrino in order to avoid constraints from neutrino scattering. A detailed analysis to identify the viable parameter space is required and will be done elsewhere.

\paragraph{Fake rare meson decays} The $\nu_h$ states can fake leptonic decays of charged mesons $M^\pm$ and charged leptons $\ell^\pm$ through the decay chains $M^\pm \to \ell^\pm_\alpha \, (\nu_h \to \nu\, \ell^+_\beta\, \ell^-_\beta)$ and $\ell^\pm_\alpha \to \ell^\pm_\beta\, \nu\, (\nu_h \to \nu\, \ell^+ \, \ell^-)$. If the decays of $\nu_h$ are prompt, these could mimic rare SM 5-body decays, setting stringent constraints on $\Gamma_{M^\pm \to \ell^\pm_\alpha \nu_h} \propto |U_{\alpha h}|^2$. Measurements compatible with the SM prediction exist for pions~\cite{Egli:1986nk,*Grab:1986si} and kaons~\cite{Poblaguev:2002ug,*Ma:2005iv,*Peruzzo:2017qis}, where the BR are of the order of $10^{-8}$, and for muons~\cite{Bertl:1985mw} and taus~\cite{Alam:1995mt}, where the BR are around $10^{-5}$. This type of signature can also lead to displaced vertices and are complementary to peak searches. 

\paragraph{Neutrino scattering} The presence of a light vector mediator and kinetic mixing can also enhance neutrino scattering cross sections. For a hadronic target $Z$, the active neutrinos may upscatter electromagnetically into $\nu_h$, which subsequently decays into observable particles ($\nu_\alpha \, Z \to (\nu_h \to \nu\, \ell_\beta^+\,\ell_\beta^-)\, Z$). Beyond explaining MiniBooNE, see below, such upscattering signatures can also produce exotic final states in neutrino detectors such as $\mu^+\mu^-$, $\tau^+\tau^-$ and multi-meson final states.


\paragraph{MiniBooNE low energy excess} The above signatures with $\ell^\pm = e^\pm$ have been invoked as an explanation of the excess of electron-like low energy events at MiniBooNE in Ref.~\cite{Ballett:2018ynz}, where a good fit to energy and angular data is achieved with a similar model containing a single heavy neutrino with $m_4 = 140$ MeV, $m_{Z^\prime} = 1$ GeV and $\chi^2 = 5\times10^{-6}$. There, the prompt decays of $\nu_4$ were achieved by requiring large mixing with the tau flavour. In a ESS-like limit of our current model, $\nu_4$ would be dominantly produced via upscattering, decaying into $\nu_\alpha e^+e^-$ inside the detector. A dedicated analysis to understand the resulting energy and angular distribution is underway.

\paragraph{Dark scalar searches}
For the scalar portal, the coupling $\lambda_{\Phi H}$ is rather weakly bound by electroweak precision data and the measurement of the Higgs invisible decay at the level of $\lambda_{\Phi H} \lesssim 0.1$~\cite{Sirunyan:2018owy}. For processes involving $\lambda_{\Phi H}$, the physical observables are suppressed by mass insertions due to the nature of the Higgs interaction. Nevertheless, if $\varphi^\prime$ decays to $\nu_h$ states, this scalar may also lead to multi-lepton signatures inherited from $\nu_h$ decays, potentially also in the form of displaced vertices.

In the limiting case of a neutrinophilic model ($\chi=\lambda_{\Phi H}=0$), the vector and scalar particles present a challenge for detection. Nonetheless, if light, they can be searched for in meson decays~\cite{Laha:2013xua,Bakhti:2017jhm} and at neutrino experiments~\cite{Bakhti:2018avv}.

Finally, the faster decays of $\nu_h$ and its self-interactions can help ameliorate tensions with cosmological observations. We do not comment further on this, but note that great effort has been put into accommodating eV scale sterile neutrinos charged under new forces with cosmological observables~\cite{Hannestad:2013ana,Dasgupta:2013zpn,Mirizzi:2014ama,Chu:2015ipa,Cherry:2016jol,Chu:2018gxk,Song:2018zyl} (see also Ref.~\cite{Escudero:2019gzq} for an interesting discussion where the $Z^\prime$ decay to neutrinos leads to an altered expansions history of the Universe). We note that an eV sterile neutrino with relatively large mixing could be easily accommodated in our ESS framework. The eV neutrino would be mainly in the $\nu_D$ direction and would have strong hidden gauge interactions.

\section{Dark Matter} 
Given the presence of a dark sector, we can ask if the model can accommodate a DM candidate. This can be achieved introducing new fermions that do not mix with the neutrinos, in order to preserve their stability. A minimal solution would be to introduce a fermionic field $\psi_L$ which has $U(1)^\prime$ charge $1/2$. The different charges of $\psi$, $\nu_D$ and $N$ would forbid neutrino mixing. A Majorana mass term $\psi_L^T C^\dagger \psi_L$ would emerge after hidden-symmetry breaking leading to a Majorana DM candidate. 

Another minimal realisation has the advantage of being anomaly free. Following Ref.~\cite{Blennow:2019fhy}, we introduce a pair of chiral fermion fields $\psi_L$ and $\psi_R$, and charge only the latter under the $U(1)^\prime$ symmetry with the same charge as $\nu_D$. This choice ensures anomaly cancellation, and allows us to write $y_{\psi} \overline{\psi_L} \psi_R \Phi^\dagger$, which after hidden-symmetry breaking yields a Dirac mass $m_\psi$. In order to avoid $\psi_R-\nu_D$ and $\psi_L-N$ mixing, an additional $\mathbb{Z}_2$ symmetry may be imposed, under which all particles have charge $+1$, except for $\psi_L$ and $\psi_R$, which have charge $-1$.

If the scalar and vector portal couplings are small in such scenarios, DM interacts mainly with neutrinos. Direct detection bounds are then evaded, since interactions with matter are loop-suppressed. Indirect detection, on the other hand, is more promising as DM annihilation into neutrinos would dominate. For instance, take the mass of $\psi$ to be smaller than the masses of the $Z^\prime$, $\varphi^\prime$ and of both heavy neutrinos. In this case, the DM annihilation is directly into light neutrinos via $\psi \overline{\psi} \to \nu_i \nu_i$. This yields a mono-energetic neutrino line that can be looked for in large volume neutrino~\cite{Beacom:2006tt,PalomaresRuiz:2007eu} or direct detection experiments~\cite{McKeen:2018pbb}. Alternatively, if $m_\psi$ is larger than the mass of any of our new particles, then the annihilation may be predominantly into such states via $\psi \overline{\psi} \to X X$, where $X=\varphi^\prime, Z^\prime$ or $\nu_h$, which subsequently decay to light neutrinos. In this secluded realisation~\cite{Pospelov:2007mp}, the search strategy for DM can be very different since the neutrino spectrum from such annihilation is continuous~\cite{Escudero:2016ksa}. Nevertheless, neutrino-DM interactions are expected to be large and can be searched for in a variety of ways~\cite{Mangano:2006mp,Wilkinson:2014ksa,Farzan:2014gza,Campo:2017nwh,Arguelles:2017atb}.



\section{Radiative generation of neutrino masses}
Neutrino oscillations have been established by several experiments~\cite{Fukuda:1998ah,*Ahmad:2002jz,*Eguchi:2002dm}, implying small but non-vanishing neutrino masses. In the Standard Model (SM), neutrinos are strictly massless due to the absence of right-handed neutrino fields, urging for extensions of the theory. The Type-I seesaw mechanism~\cite{Minkowski:1977sc,*Mohapatra:1979ia,*GellMann:1980vs,*Yanagida:1979as,*Lazarides:1980nt,*Mohapatra:1980yp,*Schechter:1980gr,*Cheng:1980qt,*Foot:1988aq}, arguably the most popular mechanism to explain the lightness of neutrino masses, relies on the addition of at least 2 heavy right-handed neutrinos $N_R$. The large scales of $N_R$ and/or the smallness of the Yukawa couplings makes the minimal realisation of this model difficult to test. Therefore, searching for variations of the Type-I seesaw where novel and testable phenomena are present is an essential part of solving the neutrino mass puzzle~\cite{Boucenna:2014zba}. A few notable examples of such alternatives are the Inverse Seesaw (ISS)~\cite{Mohapatra:1986bd,*GonzalezGarcia:1988rw} and the Linear Seesaw (LSS)~\cite{Wyler:1982dd,*Akhmedov:1995ip,*Akhmedov:1995vm}, where the lightness of neutrino masses is explained by an approximate conservation of lepton number, and the Extended Seesaw (ESS)~\cite{Barry:2011wb,*Zhang:2011vh}, where new heavy neutral fermions generally appear at small scales.
This class of models assumes additional SM gauge neutral fermions that mix with light neutrinos, usually referred to as sterile neutrinos. These, however, need not be completely sterile and might have new gauge interactions shared with the SM fermions~\cite{%
Buchmuller:1991ce,% original
Khalil:2006yi,% B-L at TeV scale
Perez:2009mu,% B-L extensive discussion + X model
Khalil:2010iu,% B-L at collider and ISS
Dib:2014fua,% B-L in linear seesaw
Baek:2015mna,% mu-tau in inverse seesaw
DeRomeri:2017oxa,% Elusive B-L
Nomura:2018mwr,% U(1)_R
Brdar:2018sbk% LR symmetry low scale
} or not~\cite{%
Okada:2014nsa,% U(1)prime
Diaz:2017edh,% Hidden witloops
Bertuzzo:2017sbj,% nu2HDM
Nomura:2018ibs,% Hidden in linear seesaw
Bertuzzo:2018ftf% dark neutrinos PM
}. In the latter case, we refer to these new heavy fermions as \textit{dark neutrinos}. The interest in such particles arises from their novel interactions which may ``leak" into the SM sector via neutrino mixing, where they offer a variety of phenomenological and cosmological consequences. 

In this article, we consider the new minimal model introduced in Ref.~\cite{Ballett:2019pyw}.
It introduces two type of new neutral fermions, namely dark neutrinos $\nu_D$ and additional sterile neutrinos $N$. We impose a hidden $U(1)^\prime$ gauge symmetry with the associated hidden gauge boson $X_\mu$, which mediates the dark neutrino interactions. The symmetry is subsequently broken by the vacuum expectation value (vev) of a complex dark scalar $\Phi$. As discussed in Ref.~\cite{Ballett:2019pyw}, the model can exhibit a significantly different phenomenology than the case of neutrino mixing only. Beyond evading many current bounds, such dark neutrinos could explain the MiniBooNE anomaly as discussed in~\cite{Ballett:2018ynz} (see also~\cite{Bertuzzo:2018itn}) and lead to novel neutrino scattering signatures~\cite{Arguelles:2018mtc}. Bounds on dark photons might also be severely weakened. If kinematically allowed, they would mainly decay into heavy neutrinos, which may be invisible or lead to multi-lepton plus missing energy signatures. 

In this article, we discuss the generation of neutrino masses in our dark neutrino model.
Crucially, the new gauge symmetry forbids Majorana mass terms for the $\nu_D$ states and, after symmetry breaking, leads to a mass matrix similar to the one in the so-called Minimal ISS~\cite{Dev:2012sg}. As such, this symmetry-enhanced seesaw predicts vanishing light neutrino masses at tree-level. Here, we show that it induces their radiative generation via one-loop diagrams involving the new scalar and vector particles~\cite{Dev:2012sg,Zhang:2013ama,Diaz:2017edh}. After identifying the range of heavy neutrino parameters required to explain the observed light neutrino masses, we point out interesting phenomenological consequences.

% DIAGRAMS
%%%%%%%%%%%%%%%%%%%%%%%%%%%%%%%%%%%%%%%%%%%%%%%%%%%%%%%%%%%%%%%
\begin{figure}[t]
\centering\includegraphics[width=\textwidth]{dark_nus/Loop_masses.pdf}
\caption[]{\label{fig:loops}The three contributions to the neutrino self-energy arising from novel bosons in the theory.}
\end{figure}
%%%%%%%%%%%%%%%%%%%%%%%%%%%%%%%%%%%%%%%%%%%%%%%%%%%%%%%%%%%%%%%

\section{Model set-up} Following~\cite{Ballett:2019pyw}, we add two types of heavy neutral fermions to the SM, namely a dark neutrino $\nu_{D,L} \equiv \nu_{D}$ and a sterile state $N_L \equiv N$. For simplicity, we restrict the discussion to one generation in order to focus on the main features of the model.

We impose a new abelian gauge symmetry $U(1)^\prime$ with associated mediator $X_\mu$ and introduce a neutral complex scalar $\Phi$. No SM fields are charged under $U(1)^\prime$. The scalar $\Phi$ and the fermion $\nu_{D}$ carry the same $U(1)^\prime$ charge, while $N$ remains completely neutral. The gauge-invariant Lagrangian is given by
%
\begin{align} \label{eq:lagrangian}
%
\mathscr{L} =& \mathscr{L}_{\mathrm{SM}} + \left(D_\mu \Phi\right)^\dagger \left(D^\mu \Phi\right) -  V(\Phi,H) \,   \nonumber\\
&  - \frac{1}{4}X^{\mu \nu} X_{\mu \nu} + \overline{N}i\slashed{\partial}N + \overline{\nu_D}i\slashed{D}\nu_D 
 \nonumber\\
&- \left[y^\alpha_\nu (\overline{L_\alpha} \cdot \widetilde{H})N^c + \frac{\mu^\prime}{2}\overline{N}N^c + y_N \overline{N}\nu_D^c\Phi + \text{h.c.}\right],
\end{align}
%
where $X^{\mu\nu} \equiv \partial^\mu X^\nu - \partial^\nu X^\nu$, $D_\mu \equiv \left(\partial_\mu-ig^\prime X_\mu\right)$, $L_\alpha \equiv (\nu_\alpha^T, \ell_\alpha^T)^T$ is the SM leptonic doublet of flavour $\alpha = e, \mu, \tau$ and $\widetilde{H} \equiv i \sigma_2 H^*$ is the charge conjugate of the SM Higgs doublet. In the neutral fermion sector, we have Yukawa couplings $y_\nu^\alpha$ and $y_N$  responsible for $L_\alpha$-$N$ and $\nu_D$-$N$ interactions, respectively, and a Majorana mass $\mu^\prime$ for $N$. The latter term violates by two units any lepton number assignment which leaves 
the Yukawa term $L_\alpha$-$N$ invariant. As such, it plays a crucial role in the generation of light neutrino masses, as we discuss.


We are interested in the case in which both the neutral component of the fields $H$ and $\Phi$ acquire non-vanishing vevs, $v_H$ and $v_\varphi$. They induce mixing between active and heavy fermions, and give a mass to the gauge boson $X_\mu$ and to the real component of the scalar field $\varphi$. We are interested in proposing a model for neutrino masses which is testable in current and future non-collider experiments, and as such we focus on a new physics scale which is below the electroweak one, $v_\varphi < v_H$.  
In addition to the neutrino portal, this model can accommodate a vector portal arising from vector kinetic mixing term and a scalar portal coming from the cross-coupling term $H^\dagger H \Phi^\dagger \Phi$ in the potential~\cite{Ballett:2019pyw}. Kinetic mixing can be reabsorbed in a redefinition of vector fields, leading to a new gauge boson which has vector couplings to the SM fermions proportional to their electric charge. For our neutrino mass generation mechanism, the vector and scalar mixing do not play a relevant role and we set them to zero from here onward, unless otherwise specified. Regarding the vector boson, we refer to it as a $Z^\prime$, independently of kinetic mixing.


%%%%%%%%%%%%%%%%%%%%%%%%%%%%%%%%%%%%%%%%%%%%%%%%%%%%%%%%%%%%%%%%%%%%%%%
\section{Neutrino masses}  
After symmetry breaking, two Dirac mass terms are induced with $m_D \equiv y_\nu^\alpha v_H/\sqrt{2}$ and $\Lambda \equiv y_N v_\varphi/\sqrt{2}$.
For one active neutrino $\nu_\alpha$, $\alpha= e, \mu, \tau$, the mass matrix is given by
%
\begin{align} \label{eq:massmatrix}
\mathscr{L}_{\rm mass} \supset
\frac{1}{2}\left (\begin{matrix} \overline{\nu}_\alpha & \overline{N} &  \overline{\nu_D} \end{matrix} \right )
\left(\begin{matrix} 
     0   &  m_D        & 0 
\\ m_D &  \mu^\prime & \Lambda 
\\   0   &  \Lambda  & 0
\end{matrix}\right)
\left (\begin{matrix} \nu_\alpha^c \\ N^c \\ \nu_D^c \end{matrix} \right) + {\rm h.c.}
\end{align}  
%
Let us emphasize the fact that in our model the zeros in the $\nu_D$-$\nu_D$ and $\nu_\alpha$-$\nu_D$ entries are enforced by the $U(1)^\prime$ symmetry, differently from LSS and ISS models, in which these are generically assumed to be nonzero and small due to the quasi-preservation of lepton number. Here, lepton number violation (LNV) may be large, as the $\mu^\prime$ term breaks it by 2 units. Alternatively, it can be small and technically natural, leading to quasi-degenerate heavy neutrinos, see below. 

After diagonalisation of the mass matrix, the two heavy neutrinos,  $\nu_h$ ($h=4,5$), acquire masses
\[m_{4,5} = \frac{\mu^\prime \mp \sqrt{\mu^{\prime\,2} + 4 (\Lambda^2 + m_D^2) } }{2}.\]
Assuming that $m_D \ll \Lambda$, we focus on two interesting limiting cases. 

The ISS-like scenario is defined by $\Lambda \gg \mu^\prime$: the two heavy neutrinos are nearly degenerate with a mass $\Lambda$ and mass splitting $\mu^\prime$. The relevant mixing parameters are $U_{\alpha 4,5} \sim m_D/\sqrt{2} \Lambda$ and $U_{D4,5} \sim 1/\sqrt{2}$.
The ESS-like case has $\Lambda \ll \mu^\prime$: one neutral lepton remains very heavy, $m_5 \sim \mu^\prime$, and mainly in the completely neutral direction $N$, and the other acquires a small mass via the seesaw mechanism in the hidden sector with $m_4 \sim - \Lambda^2/\mu^\prime$ and $U_{D5} \sim \Lambda/ \mu^\prime$. The mixing with active neutrinos is given by $U_{\alpha 5} \sim m_D/\mu^\prime \ll U_{\alpha 4} \sim m_D/\Lambda$.

The specific form of the mass matrix in Eq.~\ref{eq:massmatrix} implies vanishing light neutrino masses at tree level, as its determinant is zero~\cite{Dev:2012sg,LopezPavon:2012zg}. This feature holds to all orders in the seesaw expansion~\cite{Grimus:2000vj,Adhikari:2010yt,LopezPavon:2012zg}. The light neutrino masses, however, are not protected by any symmetry and arise from radiative corrections (for a review of radiative neutrino mass models see, \eg, Ref.~\cite{Cai:2017jrq}). 



%%%%%%%%%%%%%%%%%%%%%%%%%%%%%%%%%%%%%%%%%%%%%%%%%%
\subsection{Radiative corrections} We now show that our model generically leads to the generation of light neutrino masses at one loop. The calculation of the radiative mass term follows Refs.~\cite{Pilaftsis:1991ug,Kniehl:1996bd} with the addition of the loops with the new boson and scalar particles shown in \reffig{fig:loops}. The self-energy of the Majorana neutrino fields is given by 
%
\begin{equation*}
 \Sigma_{ij}(\slashed{q}) = \slashed{q}P_\text{L}\Sigma^\text{L}_{ij}(\slashed{q}) + \slashed{q}P_\text{R}\Sigma^\text{L*}_{ij}(\slashed{q}) + P_\text{L}\Sigma^\text{M}_{ij}(q^2)+ P_\text{R}\Sigma^{\text{M}*}_{ij}(q^2).
\end{equation*}
%
Using the on-shell renormalization scheme, the renormalized mass matrix for the light neutrinos, massless at tree level, emerges at one-loop and is given by~\cite{Kniehl:1996bd}
%
\[    m_{ij}^\text{one-loop} = \text{Re}\left[ \Sigma^\text{M}_{ij}(0)\right], \quad  i, j <4. \]
%
The self energy can be decomposed as
%
\begin{align}
\Sigma_{ij}^\text{M}(0) = \Sigma^Z_{ij}(0)& + \Sigma^{h}_{ij}(0) + \Sigma^{G_h}_{ij}(0) \,+ \nonumber\\ &\Sigma^{Z^\prime}_{ij}(0) +
\Sigma^{\varphi^\prime}_{ij}(0) + \Sigma^{G_\varphi}_{ij}(0),
\end{align}
where $\Sigma^{Z, h, G_h}$ come from the SM particles, $Z^0$, the Higgs and the associated Goldstone boson, respectively, and $\Sigma^{Z^\prime, \varphi^\prime, G_\phi}$ are the new terms present in our model, mediated by the new gauge boson and new scalar components. From it, we write the $3\times3$ light neutrino mass matrix 
%
\begin{align}\label{eq:masses_general}
 m_{ij} = \frac{1}{4\pi^2}\sum_{k=4}^5 \Big[ & C_{ik} C_{jk} \frac{m_k^3}{m_Z^2}F(m_k^2,m_Z^2,m_h^2)  \,+ 
 \nonumber\\ &D_{ik} D_{jk} \frac{m_k^3}{m_{Z^\prime}^2}F(m_k^2,m_{Z^\prime}^2,m_{\varphi^\prime}^2) \Big], 
\end{align} 
%
where we defined coupling matrices corresponding to the SM and new physics interaction terms assuming $\chi=\lambda_{\Phi H}=0$:
%
\begin{equation} \label{eq:couplings}
C_{ik} \equiv \frac{g}{4c_W}\sum_{\alpha = e}^\tau U_{\alpha i}^*U_{\alpha k}\quad\text{and} \quad D_{ik} \equiv \frac{g^\prime}{2} U^*_{Di} U_{Dk}.
\end{equation}
Equivalent expressions can be found for non-vanishing portal couplings, but considering experimental constraints we find that these do not play a role in the neutrino mass generation. It is possible to show that in general $\sum_{k} m_k C_{ik} C_{jk} =0$ and $\sum_{k} m_k D_{ik} D_{jk} =0$ for any $i,j$. By virtue of the latter property, the loop function can be written as
%
\begin{equation} \label{eq:loop_function}
F(a,b,c) \equiv \frac{3 \, \ln{(a/b)}}{a/b - 1}  + \frac{\ln{(a/c)}}{a/c - 1}.
\end{equation}
%
Turning off the $g^\prime$ gauge coupling, we recover the expression for the Type-I seesaw case~\cite{Pilaftsis:1991ug}:
 \begin{align}  \label{eq:SM_masses}
 m_{ij} = &\frac{\alpha_W}{16\pi}\sum_{\alpha, \beta = e}^{\tau} U_{\alpha i}^\ast U_{\beta j}^\ast  U_{\alpha 5} U_{\beta 5} \frac{m_5}{m_W^2} \times\nonumber\\ & \left( m_5^2 F(m_5^2,m_Z^2,m_h^2) -  m_4^2 F(m_4^2,m_Z^2,m_h^2)\right).
 \end{align}
%
These SM corrections to neutrino masses also arise in the Minimal ISS model~\cite{Dev:2012sg,LopezPavon:2012zg}. In the latter, however, no explanation is provided as to why they dominate neutrino masses. Moreover, if we restrict the discussion to scales well below the electroweak one, $ m_5 \ll 10$~GeV, bounds on the mixing angles severely constrain the parameter space viable to generate the observed values of the masses. 

For a light $Z^\prime$, the second term in Eq.~\ref{eq:masses_general} dominates
%
\begin{align}\label{eq:BSM_masses}
m_{ij} \simeq  &\frac{g^{\prime2}}{16\pi^2} U_{D i}^{*} U_{D j}^{*} \,  U_{D5}^2 \frac{m_5}{m_{Z^\prime}^2} \times\nonumber\\  \quad \quad & \big(m_5^2 F(m_5^2,m_{Z^\prime}^2,m_{\varphi^\prime}^2) - m_4^2 F(m_4^2,m_{Z^\prime}^2,m_{\varphi^\prime}^2)\big)~.
\end{align}
%
We notice that the resulting mass matrix has only one nonzero eigenvalue. This suggests that a typical prediction of our model is a normal ordering mass spectrum, in which $m_3$ is given by this radiative mechanism and $m_2$ has another origin, for example the loops mediated by the SM gauge bosons or by additional particle content. Our simplifying assumption of one generation of hidden fermions is by no means necessary and more generations of new fermions are possible, leading to a much richer structure for the light neutrino mass matrix. The additional $\mu^\prime$ terms would not be constrained and could be at different scales, while the $\Lambda$ terms arise from the $U(1)^\prime$ breaking and are therefore constrained to be at/below $v_\varphi$. Therefore, the full model could present a combination of relatively light Majorana $\nu_h$, mainly in dark direction, some very heavy nearly-neutral neutrinos and pseudo-Dirac pairs at intermediate scales. A discussion of this extension is beyond our scope, but we note that it has interesting consequences for both the heavy and light neutrino mass spectra and mixing structure.

Working in a single family case, we derive expressions for Eq.~\ref{eq:BSM_masses} in the seesaw limit for both the ISS and ESS-like scenarios. %%%%%%%%%%%%%%%%%%%%%%%%%%%%%%%%%%%%%%%%%%%%%%%%%%%%
% ISS
In the ISS-like regime and assuming $m_{Z^\prime}, m_{\varphi^\prime} \ll \, \Lambda$, \refeq{eq:BSM_masses} simplifies to
%
\begin{align} \label{eq:ISSlimit_1}
m_3 \simeq& \frac{g^{\prime 2}}{8\pi^2} \frac{m_D^2}{m_{Z^\prime}^2} \mu^\prime  \left( {3 \ln{\frac{m_{Z^\prime}^2}{\Lambda^2}} + \ln{\frac{m_{\varphi^\prime}^2}{\Lambda^2}} - 4 }\right),
\end{align}
%
while for $m_{Z^\prime}, m_{\varphi^\prime} \gg \, \Lambda$ it reduces to 
%
\begin{align} \label{eq:ISSlimit_2}
m_3 \simeq& \frac{g^{\prime 2}}{16\pi^2} \frac{m_D^2}{\Lambda^2} \mu^\prime \left( 3 + \frac{m_{\varphi^\prime}^2}{m_{Z^\prime}^2} \right).
\end{align}
%
As it can be expected, neutrino masses are controlled by the LNV parameter $\mu^\prime$ and are enhanced with respect to the SM contribution by a factor of $(m_Z/m_{Z^\prime})^2$ in the former, or $(m_Z/\Lambda)^2$ in the latter case. 

For the ESS-like regime, taking $m_{Z^\prime}, m_{\varphi^\prime} \ll \mu^\prime$, the light neutrino mass is approximately
%
\begin{align} \label{eq:ESSlimit_1}
m_3 \simeq& \frac{g^{\prime \,2}}{16\pi^2} \frac{m_D^2}{\Lambda^2 + m_D^2} \frac{\Lambda^2}{m_{Z^\prime}^2} \mu^\prime  \left( {3 \ln{\frac{m_{Z^\prime}^2}{ \mu^{\prime 2} }} + \ln{\frac{m_{\varphi^\prime}^2}{ \mu^{\prime 2} }}}\right),
\end{align}
%
while for $m_{Z^\prime}, m_{\varphi^\prime} \gg \mu^\prime$, it is
%
\begin{align} \label{eq:ESSlimit_2}
m_3 \simeq& \frac{g^{\prime 2}}{8\pi^2} \frac{m_D^2}{\Lambda^2+m_D^2} \frac{\Lambda^2}{\mu^\prime}  \left( {3 \ln{\frac{m_{Z^\prime}^2}{\Lambda^2}} + \ln{\frac{m_{\varphi^\prime}^2}{\Lambda^2}} - 4 }\right).
\end{align}
%
In this case, the light neutrino masses are controlled mainly by $\nu_5$, and the intermediate state $\nu_4$ can be much lighter. 
%
\begin{figure*}[t]
    \centering
    \includegraphics[width=0.49\textwidth]{paper_plot_nu5_highmz.pdf}
    \includegraphics[width=0.49\textwidth]{paper_plot_nu4.pdf}
    \caption[Region of interest for neutrino mass generation in our model.]{The region of interest for neutrino mass generation in our model in the parameter space of the $\nu_5$ (left) and $\nu_4$ (right) mass states. We require $m_3 = \sqrt{\Delta m^2_{\rm atm}}$ and vary $1\%<m_4/m_5<99\%$. Our BPs are $\bigtriangleup$) $m_5 = 800$ MeV, $m_4/m_5 =99\%$, ${\circ}$) $m_5 = 150$ MeV, $m_4/m_5 =50\%$ and $\star$) $m_5 = 150$ MeV, $m_4/m_5 =12\%$. All bounds and projections displayed assume $\chi=\lambda_{\Phi H}=0$. The dashed black line shows the equivalent Type-I seesaw contribution to the light neutrino mass.\label{fig:mass_constraints}}
\end{figure*}

%%%%%%%%%%%%%%%%%%%%%%%%%%%%%%%%%%%%%%%%%%%%%%%%%%%%%%%%%%%%%%%%%%%%%%%
\section{Searching for the origin of neutrino masses} \label{sec:pure_nu_mixing}

In what follows, we discuss the experimental reach to the heavy neutrinos responsible for neutrino mass generation in our model. Since the vector and scalar portals do not contribute significantly to neutrino masses, we first restrict the study to the case $\chi=\lambda_{\Phi H}=0$. For the sake of simplicity and concreteness, we work with a single generation of light neutrinos and focus on the mixing with the muon neutrino. We emphasise that our model predicts
%
\begin{equation}
    \frac{m_4}{m_5} = - \frac{U_{\alpha 5}^2}{U_{\alpha 4}^2},
\end{equation}
%
implying that both heavy neutrinos should be searched for.
%
For a real mixing matrix one can write $\sum_i^3 U_{D i}^{2} \sim U_{\mu 4}^{2}$ and $U_{D 5}^{2} \sim 1$ for small $U_{\mu 4}$. Using these relations and \refeq{eq:masses_general}, we plot the region of interest for neutrino mass generation in \reffig{fig:mass_constraints}. We require $m_3 = 
\sqrt{\Delta m^2_{\text{atm}}} \sim 0.05~\text{eV}$ and vary $m_4/m_5$ from $1\%$ (ESS-like) to $99\%$ (ISS-like). For the hidden sector parameters, we fix $m_{Z^\prime}=1$~GeV, $m_{\varphi^\prime} = 2$ GeV and $g^\prime = 1$. By decreasing (increasing) the mass of the $Z^\prime$, it is possible to shift the band to smaller (larger) values of the mixing angles, although for values smaller than a few hundred MeV, the neutrino masses have a very mild dependence on $m_{Z^\prime}$ (Eqs.~\ref{eq:ISSlimit_2} and~\ref{eq:ESSlimit_2}). Increasing $m_4/m_5$ to values closer to $100\%$ (\ie\,, decreasing $\mu^\prime$ below $m_5/100$) shifts the top of the band to larger values of mixing angle and asymptotically recovers lepton number as a symmetry. Although this possibility appears excluded for $m_{Z^\prime} = 1$ GeV, it can be achieved by lowering the mass of the mediator particles. For instance, for $m_{Z^\prime} = m_{\varphi^\prime}/2 = 100$ MeV and $m_5 < 100$ MeV, we find that values as small as $\mu^\prime \gtrsim 10^{-3} m_5$ are not covered by the grey region in \reffig{fig:mass_constraints}.
Values of $m_4/m_5 < 1\%$ have no effect in the parameter space of $\nu_5$, since in that limit the $\nu_5$ state (mostly in the $N$ direction) dominates the loop contribution.

The region labelled as excluded in \reffig{fig:mass_constraints} is composed of bounds from peak searches~\cite{
%KEK
Yamazaki:1984sj,
%NA48/2
Artamonov:2014urb,
%NA62
CERNNA48/2:2016tdo}, beam dump \cite{
%PS-191
Bernardi:1985ny,
%CHARM
Bergsma:1983rt,
%NA3
Badier:1986xz,
%NuTeV
Vaitaitis:1999wq, 
%BEBC
CooperSarkar:1985nh, 
%NOMAD
Astier:2001ck} 
%
and collider experiments~\cite{
Abreu:1996pa,%DELPHI
Akrawy:1990zq,%OPAL
Sirunyan:2018mtv% CMS 2018
}. Current and future neutrino experiments can also cover a large region of parameter space with $m_h \lesssim 2$ GeV. For instance, we show the sensitivity of the Short-Baseline Neutrino program (SBN)~\cite{Ballett:2016opr} and of the Deep Underground Neutrino Experiment (DUNE) near detector~\cite{Ballett:2018fah,Ballett:2019bgd} to heavy neutrinos in decay-in-flight searches. We also show the reach of the NA62 Kaon factory operating in beam dump mode~\cite{Drewes:2018irr}, and the dedicated beam dump experiment Search for Hidden Particles (SHiP)~\cite{Bonivento:2013jag,Alekhin:2015byh}, which will cover a much larger region of parameter space from $400$ MeV to $\lesssim6$ GeV. All bounds and sensitivities shown do not take into account the new invisible decays of the heavy neutrinos. Searches that rely on the visible decay products of the heavy neutrinos need to be revisited if the $\nu_h$ can decay invisibly or if new channels mediated by the vector (and/or scalar) portal dominate. In particular, faster decays of $\nu_h$ can shift decay-in-flight bounds to lower values of mixing angles, as discussed in detail in Ref.~\cite{Ballett:2016opr}. Peak searches apply as shown provided $\nu_h$ does not decay immediately via neutral-current channels with visible charged particles.

Let us first consider the case of subdominant vector and scalar portals. Compared to the ``standard" sterile neutrino case, in which $\nu_h$ have only SM interactions suppressed by neutrino mixing, the new neutral-current interaction can enhance the $\nu_h$ decays into light and heavy neutrinos. A comprehensive analysis is beyond the scope of this article and we focus on three benchmark points (BP) shown in \reffig{fig:mass_constraints} to exemplify the most characteristic properties. The BP represented as a triangle ($\bigtriangleup$) corresponds to $m_5 = 800$ MeV and $m_4/m_5 = 99\%$. In this case, the two heavy states are very degenerate in mass and decay like a ``standard" sterile neutrino via $|U_{\mu 4}|^2$-suppressed SM charge- and neutral-current interactions. The channel $\nu_5 \to \nu_4 \nu_\alpha \overline{\nu}_\alpha$ via the $Z^\prime$ is phase space suppressed and becomes relevant only for larger mass splittings. The invisible $\nu_4$ decay mediated by the $Z^\prime$ is subdominant as it scales as $|U_{\mu 4}|^6$ and becomes important only for larger values of the mixing angles.

For the next BPs we fix $m_5 = 150$ MeV. If we take $m_4/m_5 = 50\%$, as we do for the BP represented by the circle ($\circ$), $\nu_5$ will predominantly decay to $\nu_4 \nu_\alpha \overline{\nu}_\alpha$ due to the $Z^\prime$ contribution (provided $|U_{\mu 5}|^2\gtrsim \left( m_{Z^\prime}/m_{Z} \right)^4$). Consequently, the best candidate for detection is the $\nu_4$ via the SM weak decays $\nu_4 \to \nu_\alpha e^+e^-$. The values of the mixing angles for this BP, $|U_{\mu 4}|^2 \sim 3\times 10^{-7}$ and $|U_{\mu 5}|^2 \sim 10^{-7}$, are within reach of the SBN and DUNE experiments. 
For a larger mass hierarchy, e.g. $m_4/m_5 = 12\%$, see star BP ($\star$), the $Z^\prime$ mediated decay $\nu_5 \to \nu_4 \overline{\nu_4} \nu_4$ dominates, inducing a large $\nu_4$ population in addition to the states already produced in the beam. The intermediate state $\nu_4$ can further decay as in the previous case into $\nu_4 \to \nu_\alpha e^+e^-$. For the mixing angles we are considering, $|U_{\mu 4}|^2 \sim 10^{-6}$ and $|U_{\mu 5}|^2 \sim 10^{-7}$, DUNE will be able to test this BP. Similar considerations apply to the case where $m_5 > m_4 + m_{Z^\prime}$, where now the $Z^\prime$ can be produced on-shell in the $\nu_5$ decay. The behaviour of $\nu_4$ is as discussed above. If $ m_{Z^\prime} < m_4$, then both heavy neutrinos predominantly decay into neutrinos and the $Z^\prime$, which presents a challenge for detection as it produces mainly light neutrinos.

Experimental detection of the $Z^\prime$ and $\varphi^\prime$ particles in the absence of kinetic and scalar mixing is also daunting. Nevertheless, they can be searched for in the kinematics of charged particles from meson decays~\cite{Laha:2013xua,Bakhti:2017jhm}. Another strategy is to search for the neutrino byproducts of the decay of a $Z^\prime$ produced at accelerator neutrino facilities~\cite{Bakhti:2018avv}. 

If the vector (and scalar) portals are non-negligible, the phenomenology could be significantly richer, as discussed in \cite{Ballett:2019pyw}. In particular, $Z^\prime$-mediated decays into $\nu_\alpha e^+ e^-$, and $\nu_\alpha \mu^+ \mu^-$ if kinematically allowed, could dominate even for tiny values of $\chi^2$. For instance, for the circle BP, $\chi^2 $ as low as $10^{-8}$ would make the above decays the main channels. Pseudo-scalar final states are suppressed due to the vector nature of the $Z^\prime$. The scalar portal is expected to give subdominant contributions due to the small Higgs-electron Yukawa coupling, although decay chains with intermediate $\nu_4$ states may become relevant. Finally, cosmological bounds on heavy neutrino in the 10 MeV -- GeV scale may be weakened as they would decay well before Big Bang Nucleosynthesis~\cite{Dolgov:2000jw} (see also the discussion in Ref.~\cite{Hannestad:2013ana,*Dasgupta:2013zpn,*Mirizzi:2014ama,*Chu:2015ipa,*Cherry:2016jol,*Chu:2018gxk,*Song:2018zyl})%The BR would have a very different structure compared to the standard neutrino and vector portals. By looking for the decay channels, one would be able to, at least partially, disentangle the neutrino, vector and scalar contributions.

We have focused on the mixing with muon neutrinos as these provide one of the most sensitive avenue to test the model. In the electron sector, direct bounds on the active-heavy mixing are similar, with peak searches from $\pi^\pm$ decay being most relevant below $\approx 100$ MeV. For cases with large LNV, heavy neutrinos can dominate neutrinoless double beta decay~\cite{LopezPavon:2012zg}, and this sets the strongest constraints in the parameter space. The tau sector is relatively poorly constrained, so greater freedom exists if such entries are relevant for neutrino mass generation. 




\chapter{Tests of the MiniBooNE anomaly}
\graphicspath{{}{miniboone/figs/}{miniboone/}{Diagrams/}}

Anomalies in short-baseline accelerator and reactor experiments~\cite{Athanassopoulos:1996jb,Aguilar:2001ty,AguilarArevalo:2007it,Aguilar-Arevalo:2018gpe} are yet to have satisfying explanations. Minimal extensions of the three-neutrino framework to explain the anomalies introduce the so-called sterile neutrino states, which do not participate in Standard Model (SM) interactions in order to agree with measurements of the Z-boson invisible decay width~\cite{ALEPH:2010aa}. Unfortunately, these minimal scenarios are disfavoured as they fail to explain all data~\cite{Collin:2016aqd, Capozzi:2016vac, Dentler:2018sju}. This has led the community to explore non-minimal scenarios. Along this direction, we have already study a well-motivated neutrino-mass model that can also explain the short-baseline anomalies in Chapter 5. In this chapter, we will focus on the phenomenological realisations of dark neutrinos that have been proposed as an explanation of the anomalous observation of $\nu_e$-like events in MiniBooNE~\cite{Aguilar-Arevalo:2018gpe}.

MiniBooNE is a mineral oil Cherenkov detector located in the Booster Neutrino Beam (BNB), at Fermilab~\cite{AguilarArevalo:2008yp,AguilarArevalo:2008qa}. From data collected between 2002 to 2017, the experiment has observed an excess of $\nu_e$-like events that is currently in tension with the standard three-neutrino prediction at a level of $4.7 \sigma$~\cite{Aguilar-Arevalo:2018gpe}. While it is possible that the excess is fully or partially due to systematic uncertainties or SM backgrounds~(see, \textit{e.g.},~\cite{AguilarArevalo:2008rc,Aguilar-Arevalo:2012fmn,Hill:2010zy}), many Beyond the Standard Model (BSM) explanations have been put forth. These new physics (NP) scenarios typically require the existence of new particles, which can: participate in short-baseline oscillations~\cite{Murayama:2000hm,Strumia:2002fw,Barenboim:2002ah, GonzalezGarcia:2003jq,Barger:2003xm,Sorel:2003hf,Barenboim:2004wu, Zurek:2004vd, Kaplan:2004dq, Pas:2005rb,deGouvea:2006qd,Schwetz:2007cd, Farzan:2008zv,Hollenberg:2009ws,Nelson:2010hz,Akhmedov:2010vy, Diaz:2010ft,Bai:2015ztj, Giunti:2015mwa,Papoulias:2016edm, Moss:2017pur,Carena:2017qhd}, change the neutrino propagation in matter~\cite{Liao:2016reh, Liao:2018mbg,Asaadi:2017bhx,Doring:2018cob}, be produced in the beam or in the detector and its surroundings~\cite{Gninenko:2009ks,Gninenko:2010pr,Dib:2011jh,McKeen:2010rx,Masip:2012ke, Masip:2011qb,Gninenko:2012rw,Magill:2018jla}. These models either increase the conversion of muon- to electron-neutrinos or produce electron-neutrino-like signatures in the detector, where in the latter category one typically exploits the fact that the LSND and MiniBooNE are Cherenkov detectors that cannot distinguish between electrons and photons. Although it is possible to consider MiniBooNE explanations that have little to no theoretical motivation, recent models~\cite{Bertuzzo:2018itn,Bertuzzo:2018ftf,Ballett:2018ynz} are motivated by neutrino-mass generation via hidden interactions in the heavy neutrino sector. In particular, the common feature of these models is the upscattering into a heavy neutrino, usually with tens to hundreds of MeV in mass, which subsequently decays into a pair of electrons. If collimated, this pair of electrons can fake a single-electron signature. 

Our main contributin is introducing new techniques to probe models that rely on the ambiguity between photons and electrons to explain the MiniBooNE observation, using the dark neutrino model from~\cite{Bertuzzo:2018itn,Bertuzzo:2018ftf} as a benchmark scenario. 
Our analysis relies on neutrino-electron scattering measurements~\cite{Auerbach:2001wg,Deniz:2009mu,Bellini:2011rx,Park:2013dax,Valencia:2019mkf,Park:2015eqa,Valencia-Rodriguez:2016vkf,DeWinter:1989zg,Geiregat:1992zv,Vilain:1994qy}. This process is currently used to normalize the neutrino fluxes, due to its well-understood cross section, and has been a fertile ground for light NP searches~\cite{Pospelov:2017kep,Lindner:2018kjo,Magill:2018tbb}. Here, however, we expand the capability of these measurements to probe BSM-produced photon-like signatures, by developing a new analysis using previously neglected sideband data. Our technique is complementary to recent searches for coherent single-photon topologies~\cite{Abe:2019cer}.
Since the upscattering process has a threshold of tens to hundreds of MeV, we focus on two high-energy neutrino experiments: \minerva~\cite{Park:2013dax,Valencia:2019mkf,Park:2015eqa,Valencia-Rodriguez:2016vkf}, a scintillator detector in the Neutrinos at the Main Injector (NuMI) beamline at Fermilab, and CHARM-II~\cite{DeWinter:1989zg,Geiregat:1992zv,Vilain:1994qy}, a segmented calorimeter detector at CERN along the Super Proton Synchrotron (SPS) beamline. These experiments are complementary in the range of neutrino energies they cover and have different background composition. In all cases a relevant sideband measurement exists, allowing us to take advantage of the excellent particle reconstruction capabilities of \minerva and the precise measurements at CHARM-II to constrain NP.
%

\section{Dark Neutrinos at MiniBooNE}

%%% DIAGRAM %%%
\begin{figure}[t!]
    \centering
    \includegraphics[width=0.9\textwidth]{Dark_neutrinos.pdf}
    \caption[Dark neutrino signal at MiniBooNE.]{The dark neutrino signal at MiniBooNE. We show the two phenomenological realisations of the dark neutrino model with a heavy (top) and light (bottom) mediator. In the heavy case, large $|U_{\tau4}|^2$ is required to shorten the $\nu_4$ lifetime.\label{fig:diagram}}
\end{figure}
%%%%%%%%%%%%%%%%%%%%%%%%%
 We limit our discussion to the minimal version of the model that could explain the MiniBooNE excess. This contains at least one Dirac heavy neutrino\footnote{Models with the decay of Majorana particles will lead to greater tension with the angular distribution at MiniBooNE due to their isotropic nature~\cite{Formaggio:1998zn,Balantekin:2018ukw}.}, $\nu_D$, charged under a new U$(1)^\prime$ gauge group, which is part of the particle content and gauge structure needed for mass generation. The dark sector is connected to the SM in two ways: through kinetic mixing between the new gauge boson and hypercharge, and through neutrino mass mixing. We start by specifying the kinetic part of the NP Lagrangian
%
\begin{equation}
\mathscr{L}_{\rm kin} \supset
\;\; \frac{1}{4} \hat{Z}^{\prime}_{\mu \nu} \hat{Z}^{\prime \mu \nu} + \frac{\sin{\chi}}{2} \hat{Z}^{\prime}_{\mu \nu} \hat{B}^{\mu \nu} + \frac{m_{\hat{Z}^\prime}^2}{2} \hat{Z}^{\prime \mu} \hat{Z}^\prime_{\mu},
\end{equation}
%
where $\hat{Z}^{\prime \mu}$ stands for the new gauge boson field, $\hat{Z}^{\prime \mu\nu}$ its field strength, and $\hat{B}^{\mu \nu}$ the hypercharge field strength. After usual field redefinitions~\cite{Chun:2010ve}, we arrive at the physical states of the theory. Working at leading order in $\chi$ and assuming $m_{Z^\prime}^2/m_{Z}^2$ to be small, we can specify the relevant interaction Lagrangian as
%
\begin{equation}
\mathscr{L}_{\rm int} \supset \;\;g_D \overline{\nu}_D \gamma_\mu \nu_D Z^{\prime \mu}
 + e \varepsilon Z'^{\mu}J^{\rm EM}_{\mu},
\end{equation}
%
where $J^{\rm EM}_{\mu}$ is the SM electromagnetic current, $g_D$ is the U$(1)^\prime$ gauge coupling assumed to be $\mathcal{O}(1)$, and $\varepsilon \equiv c_{\rm w} \chi$, with $c_{\rm w}$ being the cosine of the weak angle. Additional terms would be present at higher orders in $\chi$ and mass mixing with the SM $Z$ is also possible, though severely constrained. 
After electroweak symmetry breaking, the dark neutrino $\nu_D$ is a superposition of neutrino mass states. The flavor and mass eigenstates are related via 
\begin{equation}
    \nu_\alpha = \sum^{4}_{i=1} U_{\alpha i}\nu_{i}, \quad (\alpha=e,\mu,\tau,D),
\end{equation}
where $U$ is a $4\times4$ unitary matrix. It is expected that $|U_{\alpha 4}|$ is small for $\alpha = e, \mu, \tau$, but $|U_{D4}|$ can be of $\mathcal{O}(1)$~\cite{Parke:2015goa,Collin:2016aqd}. The choice of $m_4$ and $m_{Z^\prime}$ has important consequences for the allowed decays of the new particle content. We focus on the case in which $m_4 > m_{Z^\prime}$, where the two body $\nu_4 \to \nu_\alpha Z^\prime$ decay is allowed. In addition, the mass of the new gauge boson is kept below $\sim100$ MeV, making the decay into $e^+e^-$ pairs the dominant channel. Decay into a pair of neutrinos is possible, but is subdominant provided neutrino mixing is small. 
%%

\subsection{Signature and region of interest}

The heavy neutrino is produced from an active flavour state upscattering on a nuclear target $A$, $\nu_\alpha A \to \nu_4 A$. The upscattering cross section is proportional to $\alpha_D \alpha_\textsc{qed}\varepsilon^2 |U_{\alpha 4}|^2$, dominated by $|U_{\mu 4}|$ since all current accelerator neutrino beams are composed mainly of muon neutrinos. This production can happen off the whole nucleus in a coherent way or off individual nucleons. For $m_{Z^\prime} \lesssim 100$ MeV, the production will be mainly coherent, but for heavier masses, such as the ones considered in~\cite{Ballett:2018ynz}, incoherent upscattering dominates. In Fig.~\ref{fig:cross_section}, we show the NP cross section at the benchmark point of~\cite{Bertuzzo:2018itn} and compare it with the quasi-elastic cross section. By superimposing the cross section on the neutrino fluxes of \minerva and MiniBooNE, we make it explicit that the larger energies at \minerva and CHARM-II are ideal to produce $\nu_4$. Once produced, $\nu_4$ predominantly decays into a neutrino and a dielectron pair, $\nu_4 \to \nu_\alpha e^+ e^-$, either via an on-shell~\cite{Bertuzzo:2018itn} or off-shell~\cite{Ballett:2018ynz} $Z^\prime$ depending on the choice of $m_4$ and $m_{Z^\prime}$. In this work, we restrict our discussion to the $m_4 > m_{Z^\prime}$ case, where the upscattering is mainly coherent and is followed by a chain of prompt two body decays $\nu_4 \to \nu_\alpha (Z^\prime \to e^+ e^-)$. The on-shell $Z^\prime$ is required to decay into an overlapping $e^+e^-$ pair, setting a lower bound on its mass of a few MeV. Experimentally, however, $m_{Z^\prime} > 10$ MeV for $e \epsilon \sim 10^{-4}$ to avoid beam dump constraints~\cite{Bauer:2018onh}.
Increasing $m_{Z^\prime}$ increases the ratio of incoherent to coherent events, and makes the electron pair less overlapping.
Even though we focus on overlapping $e^+e^-$ pairs, we note that a significant fraction of events would appear as well-separated showers or as a pair of showers with large energy asymmetry, similarly to neutral current (NC) $\pi^0$ events. The asymmetric events also contribute to the MiniBooNE excess and offer a different target for searches in $\nu-e$ scattering data.

A fit to the neutrino energy spectrum at MiniBooNE was performed in ~\cite{Bertuzzo:2018itn} and is reproduced in~\reffig{fig:final_plot}. We have performed our own fit to the MiniBooNE energy spectrum using the data release from~\cite{Aguilar-Arevalo:2018gpe}, and our results agree with~\cite{Bertuzzo:2018itn}. This fit leads to preferred values of $m_4$ close to 100 MeV and $|U_{\mu 4 }| \sim 10^{-4}$. Unfortunately, this energy-only fit neglects the distribution of the excess events as a function of their angle $\theta$ with respect to the beam.
This is important, as the total observed excess contains only $\approx 50\%$ of the events in the most forward bin ($0.8 < \cos{\theta} < 1.0$), with a statistical uncorrelated uncertainty of 5\% on this quantity.  
 
As was recently pointed out in~\cite{Jordan:2018qiy}, few NP scenarios can reproduce the angular distribution of the MiniBooNE excess. Among these are models where new unstable particles are produced in inelastic collisions in the detector, such as the present case.
Here, large $\theta$ can be achieved by tweaking the mass of the heavy neutrino; the signal becomes less forward as $\nu_4$ becomes heavier.
To show this, we use our dedicated Monte Carlo (MC) simulation to asses the values of $m_4$ preferred by MiniBooNE data~\footnote{Since the released MiniBooNE data do not provide the correlation between angle and energy, and their associated systematics, an energy-angle fit is not possible.}. For $m_{Z^\prime} = 30$ MeV and $m_4 = 100$, $200$, and $400$ MeV, we find that 98\%, 87\%, and 70\% of the NP events would lie in the most forward bin, respectively. 
The latter, as expected, is close to the benchmark point of~\cite{Bertuzzo:2018itn}. Thus the relevant region for the MiniBooNE angular distribution is $m_4 \gtrsim 400$ MeV for $m_{Z^\prime} = 30$ MeV.
 
\section{Dark Neutrinos in Neutrino-Electron Scattering}
%%%%%%%%%%%%% Cross Section %%%
\begin{figure}[t!]
    \centering
    \includegraphics[width=0.65\textwidth]{cross_sections.pdf}
%     \caption[Upscattering total cross section.]{The quasi-elastic cross section for $6p^+$ is shown as a function of the neutrino energy (solid black line). Similarly the coherent, out of a carbon atom, and the diffractive NP contributions for the benchmark point of~\cite{Bertuzzo:2018itn} are shown as solid and dashed blue lines, respectively. In the background, the light gray shaded region is the Booster Neutrino Beam (BNB) flux shape, while the light golden region is the Neutrinos at the Main Injector (NuMI) low-energy neutrino-mode flux.}{\label{fig:cross_section}}
    \caption[Upscattering total cross section.]{The quasi-elastic cross section on Carbon ($6p^+$) is shown as a function of the neutrino energy (solid black line). The coherent (solid blue) and incoherent (dashed blue) scattering NP cross sections are also shown for the benchmark point of~\cite{Bertuzzo:2018itn}. In the background, we show the BNB flux of $\nu_\mu$ at MiniBooNE (light gray), and the NuMI beam neutrino flux at MINER$\nu$A for the LE (light golden) and ME (light blue) runs in neutrino mode.\label{fig:cross_section}}
\end{figure}
%%%%%%%%%%%%%%%%%%%%%%%%%%%%%%%%%%%%%%%%%%%%%%%%%%

Our goal is to develop new techniques to probe dark neutrino models in neutrino-electron scattering measurements. Our analysis showcases a generic way to look for models that rely on the ambiguity between photons and electrons to explain the MiniBooNE observation. Due to the electron-like nature of the excess, neutrino-electron scattering measurements~\cite{Auerbach:2001wg, Deniz:2009mu,Bellini:2011rx,Park:2013dax,Vilain:1994qy} provide the kind of signature one would look for. Although these measurements have been shown to provide powerful constraints on light NP~\cite{Pospelov:2017kep,Lindner:2018kjo,Magill:2018tbb}, the unique photon-like topology of the signatures we consider requires us to go beyond the final processed sample quoted by the experiments and make use of sideband measurements to constrain them.
Since the typical heavy neutrino mass is in the hundreds-of-MeV regime, we focus on two high-energy neutrino experiments: \minerva~\cite{Park:2013dax,Park:2015eqa,Valencia-Rodriguez:2016vkf} and CHARM-II~\cite{DeWinter:1989zg,Geiregat:1992zv,Vilain:1994qy}. These experiments are complementary in neutrino energy and background composition. In both cases we make use of sideband measurements, taking advantage of the excellent particle reconstruction capabilities of \minerva and the precise measurements at CHARM-II to constrain NP. In Fig.~\ref{fig:cross_section}, we show the cross section at the benchmark point of~\cite{Bertuzzo:2018ftf} and compare it with the quasi-elastic cross section. By superimposing the cross section on the neutrino fluxes of \minerva and MiniBooNE, we make it explicit that the larger energies at \minerva and CHARM-II are ideal to probe these models. 


\section{Simulation Details}
%
\begin{figure}[t]
 \includegraphics[width = 0.9\textwidth]{MiniBooNE_tests.pdf}
 \caption[Diagram of the sidebands in neutrino-electron scattering analyses.]{A schematic representation of the relative number of events in sideband regions of neutrino-electron scattering analyses.}
\end{figure}
%
We generate events distributed according to the upscattering cross section for the process $\nu_\mu A \to \nu_4 A$, where $A$ is a nuclear target. Here, we only discuss upscattering on nuclei, as the number of elastic scattering on protons is much smaller at these $Z^\prime$ masses (see \reffig{fig:cross_section}). We then implement the chain of two-body decays: $\nu_4 \to \nu_\mu Z^\prime$ followed by $Z^\prime \to e^+ e^-$. To go from our MC output to the predicted experimental signature, we perform three procedures. First, we smear the energy and angles of the $e^+$ and $e^-$ originating from the decay of the $Z^\prime$ according to detector dependent Gaussian energy and angular resolutions. Next, we select all events with an overlapping $e^+e^-$ pair, which is assumed to be reconstructed as a single electromagnetic (EM) shower. This guarantees that the events behave like a photon shower inside the detector~\footnote{For MiniBooNE, we also include events that are highly asymmetric in energy, \textit{i.e.}, $E_{\pm} > 30$ MeV and $E_{\mp} < 30$ MeV, where the most energetic shower defines the angle with respect to the beam.)}. Finally, for \minerva and CHARM-II, these samples are subject to analysis-dependent kinematical cuts to determine if they contribute to the $\nu-e$ scattering sample. Detector resolutions, requirements for the dielectron pair to be overlapping, and analysis-dependent cuts are summarized in \reftab{tab:parameters}. We now list the experimental parameters used in our simulations for each individual detector. 
%% DIAGRAM %%%
% \begin{figure}[t]
%     \centering
%     \includegraphics[width=0.49\textwidth]{diagram.pdf}
%     \caption{{\textit{Illustration of heavy neutrino production.}} Left: production of the heavy mass state via upscattering. Center: Decay of the heavy neutrino into a light neutrino and a gauge boson. Right: Decay of a gauge boson into a pair of electrons that produce the experimental signature.\label{fig:diagram}}
% \end{figure}

\paragraph{CHARM-II} The CHARM-II experiment is simulated using the CERN West Area Neutrino Facility (WANF) wide band beam ~\cite{Vilain:1998uw}. The total number of POT is $2.5 \times 10^{19}$ for the $\nu$ and $\overline{\nu}$ run combined. We assume glass to be the main detector material (SiO$_2$), such that we can treat neutrino scattering off an average target with $\langle Z\rangle=11$ and $\langle A \rangle = 20.7$~\cite{DeWinter:1989zg,Vilain:1993sf}. The fiducial volume in our analysis is confined to a transverse area of $320$cm$^2$ (corresponding to a fiducial mass of $547$t) and the detection efficiency is taken to be $76\%$ (efficiency for $\pi^0$ sample is quoted at $82\%$~\cite{Vilain:1992wx}). We reproduce the total number of $\nu-e$ scattering events with $3$ GeV $< E_{\rm vis} <24$ GeV, namely $2677+2752$, to within a few percent level when setting the number of POTs in $\nu$ mode to be $1.69$ of that in the $\overline{\nu}$ mode~\cite{Geiregat:1991md}. We assume a flux uncertainty of $\sigma_\alpha = 4.7\%$ for neutrino, and $\sigma_\alpha = 5.2\%$ for antineutrino beam~\cite{Vilain:1992wx}. The background uncertainty is constrained to be $\sigma_\beta = 3\%$ using the data with $E_{\rm vis} \theta^2 > 30$ MeV, where the number of new physics events is negligible.

\paragraph{MINER$\nu$A} For our MINER$\nu$A simulation, we use the LE and ME NuMI neutrino fluxes~\cite{AliagaSoplin:2016shs}. The total number of POT is $3.43\times 10^{20}$ for LE data, and $11.6\times10^{20}$ for ME data. The detector is assumed to be made of CH, with a fiducial mass of $6.10$t and detection efficiencies of $73\%$~\cite{Parke:2015goa,Valencia:2019mkf}. We assume a flux uncertainty of $\sigma_\alpha = 10\%$ for both the LE and ME modes~\cite{Aliaga:2016oaz}. Due to the tuning performed in the sideband of interest, the uncertainties on the background rate are much larger. For the LE, we take $\sigma_\beta = 30\%$, while for the ME data  $\sigma_\beta = 50\%$. Although tuning is significant for the coherent $\pi^0$ production sample, the overall rate of backgrounds in the sideband with large $dE/dx$ does not vary by more than $20\%$ ($40\%$) in the LE (ME) tuning.  

\paragraph{MiniBooNE} To simulate MiniBooNE, we use the Booster Neutrino Beam (BNB) fluxes from Ref.~\cite{AguilarArevalo:2008yp}. Here, we only discuss the neutrino run, although the predictions for the antineutrino run are very similar. We assume a total of $12.84 \times 10^{20}$ POT in neutrino mode. The fiducial mass of the detector is taken as $450$t of CH$_2$. In order to apply detector efficiencies, we compute the reconstructed neutrino energy under the assumption of CCQE scattering
\begin{align}
 E_\nu^{CCQE} = \frac{E_{\rm vis} m_p}{m_p - E_{\rm vis} (1 - \cos{\theta}) },
\end{align}
where $E_{\rm vis} = E_{e^+} + E_{e^-}$ is the total visible energy after smearing. Under this assumption, we can apply the efficiencies provided by the MiniBooNE collaboration~\cite{Aguilar-Arevalo:2012fmn}.

\renewcommand{\arraystretch}{1.2}
\begin{table*}[t]
    \centering
    \begin{tabular}{|lp{5.2cm}p{2.8cm}p{3cm}|}
        \hline       
        Experiment & Detector Resolution&Overlapping & Analysis Cuts  \\
        \hline        \hline
        MiniBooNE & & & \\\hline
         & $\sigma_E/E = 12\%$ \newline $\sigma_\theta = 4{^\circ}$ & $E_{+} > 30$ MeV \newline $E_{-} > 30$ MeV \newline $\Delta \theta_{\pm} < 13^\circ$ &        N/A
        \\        \hline

        MINER$\nu$A & & & \\\hline
        &  $\sigma_E/E = 6\%/\sqrt{E_e/{\rm GeV}} + 3.4\%$ \newline $\sigma_\theta = 1{^\circ}$ & $E_{+} > 30$ MeV \newline $E_{-} > 30$ MeV \newline $\Delta \theta_{\pm} < 8^\circ$  & $E_{\rm vis} > 0.8$ GeV \newline $E_{\rm vis} \theta^2 < 3.2$ MeV
        \newline $Q^2_{\rm rec} < 0.02$ GeV$^2$
        \\        \hline

        CHARM-II & & & \\\hline       
         & $\sigma_E/E = 9\%/\sqrt{E/{\rm GeV}} +  11\%$ \newline $\sigma_\theta/{\rm mrad} =  \frac{27 (E/{\rm GeV})^2 +14}{\sqrt{E/{\rm GeV}}} + 1$ & $E_{+} > 30$ MeV \newline $E_{-} > 30$ MeV \newline $\Delta \theta_{\pm} < 4^\circ$ & $E_{\rm vis} > 3$ GeV \newline $E_{\rm vis} < 24$ GeV\newline $E_{\rm vis} \theta^2 < 28$ MeV
          \\
    \hline      
    \end{tabular}
    \caption{Experimental resolution, condition for dielectrons to be reconstructed as overlapping EM showers and analysis cuts for the detectors studied in this chapter.}
    \label{tab:parameters}
\end{table*}


\section{Kinematics of the Signal}

As an important check of our calculation and of the explanation of the MiniBooNE excess within the model of interest, we plot the MiniBooNE neutrino data from 2018~\cite{Aguilar-Arevalo:2018gpe} against our MC prediction in Fig.~\ref{fig:MB_distributions}. We do this for three different new physics parameter choices. We set $m_{Z^\prime} = 30$ MeV, $\alpha \epsilon^2 = 2\times10^{-10}$ and $\alpha_D = 1/4$ for all points, but vary $|U_{\mu 4}|^2$ and $m_4$ so that the final number of excess events predicted by the model at MiniBooNE equals 334. Then, we repeat this process fixing $m_4 = 100$ and $420$ MeV, varying $m_{Z^\prime}$. This shows that the impact of the $Z^\prime$ mass on the angular distribution is minimal.
%
\begin{figure}[h!]
    \centering
    \includegraphics[width=0.49\textwidth]{Enu_reco.pdf}
    \includegraphics[width=0.49\textwidth]{Theta_reco.pdf}\\
    \includegraphics[width=0.49\textwidth]{Enu_reco_m4=100mev.pdf}
    \includegraphics[width=0.49\textwidth]{Theta_reco_m4=100mev.pdf}\\
    \includegraphics[width=0.49\textwidth]{Enu_reco_m4=400mev.pdf}
    \includegraphics[width=0.49\textwidth]{Theta_reco_m4=400mev.pdf}
    \caption[New physics prediction and data at MiniBooNE.]{Data and new physics prediction for the reconstructed neutrino energy at MiniBooNE under the assumption of CCQE scattering (\textbf{left}), and for the cosine of the angle between the visible EM and the neutrino beam (\textbf{right}). We fix couplings so that the total number of events at MiniBooNE equals 334.~\label{fig:MB_distributions}}
\end{figure}
%

To verify that the new physics signal is important in neutrino-electron studies, we also plot kinematical distributions for the benchmark point (BP) for different detectors. This corresponds to $m_{Z^\prime} = 30$ MeV, $\alpha \epsilon^2 = 2\times10^{-10}$, $\alpha_D = 1/4$, $|U_{\mu 4}|^2 = 9\times10^{-7}$ and $m_4 = 420$ MeV. The interesting variables are the energy asymmetry of the dielectron pair 
\begin{equation}
    |E_{\rm asym}| = \frac{|E_+ - E_-|}{E_+ + E_-},
\end{equation}
as well as the separation angle $\Delta \theta_{e^+e^-}$ between the two electrons. These variables are plotted in Fig.~\ref{fig:other_distributions} at MC truth level, before any smearing or selection takes place. We also plot the total reconstructed energy $E_{\rm vis} = E_{e^+} + E_{e^-}$ and the quantity $E_{\rm vis} \theta^2$, where $\theta$ stands for the angle formed by the reconstructed EM shower and the neutrino beam. $E_{\rm vis}$ and $\theta$ are computed after smearing, but before the selection into overlapping pairs takes place.

%
\begin{figure}[h]
    \centering
    \includegraphics[width=0.49\textwidth]{Easym_v1.pdf}
    \includegraphics[width=0.49\textwidth]{DeltaTheta_v1.pdf}
    \\
    \includegraphics[width=0.49\textwidth]{Evis_v1.pdf}
    \includegraphics[width=0.49\textwidth]{Etheta2_v1.pdf}
    \caption[Dark neutrino kinematical distributions.]{Kinematical distributions for the new physics events at CHARM-II, MINER$\nu$A LE and MiniBooNE for the BP. We show the energy asymmetry (\textbf{top left}), the electron separation angles (\textbf{top right}), both at MC truth level. We also show reconstructed ({after smearing}) total visible energy $E_{\rm vis}$ (\textbf{bottom left}) and $E_{\rm vis} \theta^2$ (\textbf{bottom right}).\label{fig:other_distributions}}
\end{figure}


  \subsection{MINER$\nu$A and CHARM-II Analyses}
Neutrino-electron scattering measurements predicate their cuts in the following core ideas: no hadronic activity near the interaction vertex, small opening angle from the beam, $E_e \theta^2 \lesssim 2 m_e$, and the requirement that the measured energy deposition, $dE/dx$, be consistent with that of a single electron. For the NP events, when the coherent process dominates and the mass of the $Z^\prime$ is small, the first two conditions are often satisfied. However, the requirement of a single-electron-like energy deposition removes a significant fraction of the new-physics induced events. This presents a challenge, as the NP events are mostly overlapping electron pairs and will potentially be removed by the $dE/dx$ cut.
In order to circumvent this problem, we perform our analysis not at the final-cut level, but at an intermediate one. This is done differently for CHARM-II and MINER$\nu$A: the CHARM-II experiment provides data as a function of $E_e \theta^2$ without the $dE/dx$ cut, and \minerva provides data as a function of the measured $dE/dx$ after analysis cuts on $E_e \theta^2$.

We have developed our own MC simulation for candidate electron pair events in MiniBooNE, MINER$\nu$A and CHARM-II (see the Supplemental Material for more details on detector resolutions, precise signal definition and resulting distributions). We only consider the coherent part of the cross section to avoid hadronic-activity cuts, which is conservative. We also select only events with small energy asymmetries and small opening electron angles.
When required, we assume the mean $dE/dx$ in plastic scintillator to follow the same shape as the NC $\pi^0$ prediction. Our prediction for new physics events for the BP point is show in Fig. \ref{fig:NP_events} on top of the \minerva ME and CHARM-II data and MC prediction. This includes all analysis cuts, which we describe below.   
%We calculate the mean $dE/dx$ in plastic scintillators~\cite{NIST:2018} according to~\cite{Leo:1987kd,Tanabashi:2018oca}. 

The CHARM-II analysis is mostly based on Fig. 1 of~\cite{Vilain:1994qy}. This sample is shown as a function of $E\theta^2$ and does not have any cuts on $dE/dx$. It contains all events with shower energies between $3$ and $24$ GeV, and our final cut on $E\theta^2$ is fixed at $28$ MeV. For \minerva, the event selection is identical for the LE and ME analyses~\cite{Park:2015eqa,Valencia:2019mkf}. The minimum shower energy required is $0.8$ GeV in order to remove the $\pi^0$ background and have reliable angular and energy reconstruction. Events are kept only when they meet the following angular separation criterion: $E_e \theta^2 < 3.2\times 10^{-3}~{\rm ~GeV ~rad^2}$. A final cut is applied, ensuring $dE/dx < 4.5~{\rm MeV} / 1.7~{\rm cm}$. The \minerva analyses use the data outside the previous $dE/dx$ cut to constrain backgrounds. This sideband is defined by all events with $E_e\theta^2 > 5 \times 10^{-3} {\rm ~GeV ~rad^2}$ and $dE/dx < 20~{\rm MeV}/1.7~{\rm cm}$. Using this sideband measurement, the collaboration tunes their backgrounds by ($0.76$, $0.64$, $1.0$) for ($\nu_e$CCQE, $\nu_\mu$NC, $\nu_\mu$CCQE) processes in the LE mode. Our LE analysis uses the data shown in Fig. 3 of~\cite{Park:2015eqa} where all the cuts are applied except for the final $dE/dx$ cut. In our final event selection, we require that the sum of the energy deposited be more than $4.5$ MeV$/ 1.7$ cm, compatible with an $e^+e^-$ pair and yielding an efficiency of $90\%$.

The \minerva ME data contains an excess in the region of large $dE/dx$~\cite{Valencia:2019mkf}, where the NP events would lie. This excess is attributed to NC $\pi^0$ events, and grows with the shower energy. With normalization factors as large as 1.7, the collaboration tunes primarily the NC $\pi^0$ prediction in an energy dependent way. After tuning, the total NC $\pi^0$ sample corresponds to $20\%$ of the total number of events before the $dE/dx$ cut.

To place our limits, we perform a rate-only analysis by means of a $\chi^2$ test statistic (detailed in the Supplemental Material). We incorporate uncertainties in background size and flux normalization as nuisance parameters with Gaussian constraint terms. For the neutrino-electron scattering and BSM signal, we allow the normalization to scale proportionally to the same flux uncertainty parameter. 
The background term also scales with the flux-uncertainty parameter but has an additional nuisance parameter to account for its unknown size. We obtain our constraint as a function of heavy neutrino mass $m_4$, and mixing $|U_{\mu 4}|$ assuming a $\chi^2$ with two degrees of freedom~\cite{Tanabashi:2018oca}.

In our nominal \minerva LE (ME) analysis, we allow for 10\% uncertainty on the flux~\cite{Aliaga:2016oaz}, and 30\% (40\%) uncertainty on the background motivated by the amount of tuning performed on the original backgrounds. Note that the nominal background predictions in the \minerva LE (ME) analysis overpredicts (underpredicts) the data before tuning, and that tuning parameters are measured at the 3\% (5\%) level~\cite{Park:2013dax,Valencia:2019mkf}.
We also perform a background-ignorant analysis in which we assume 100\% uncertainty for the background normalization, which changes our conclusions by only less than a factor of two. This emphasizes the robustness of our \minerva bound, since the NP typically overshoots the low number of events in the sideband. For the benchmark point of~\cite{Bertuzzo:2018itn}, we predict a total signal of 232 (4240) events for \minerva LE (ME).

For CHARM-II, the NP signal lies mostly in a region with small $E\theta^2$. Thus, we constrain backgrounds using the data from $28 < E\theta^2 < 60$ MeV rad$^2$. This sideband measurement constrains the normalization of the backgrounds in the signal region at the level of $3\%$.
The extrapolation of the shape of the background to the signal region introduces the largest uncertainty in our analysis. For this reason, we raise the uncertainty of the background normalization from $3\%$ to a conservative $10 \%$ when setting the limits. Flux uncertainties are assumed to be $4.7\%$ and $5.2\%$ for neutrino and antineutrino mode~\cite{Allaby:1987bb}, respectively, and are applicable to the new-physics signal, $\nu-e$ scattering prediction, and backgrounds. 
Uncertainties in the $\nu-e$ scattering cross sections are expected to be sub-dominant and are neglected in the analysis~\cite{deGouvea:2006hfo}. For CHARM-II, the NP also yields too many events in the signal region, namely $\approx 2.2\times10^{5}$ events for the benchmark point of~\cite{Bertuzzo:2018itn} in antineutrino mode. If we lower $|U_{\mu4}| = 10^{-4}$ and $m_4 = 100$ MeV, CHARM-II would still have $\approx 3 \times 10^3$ new physics events. 

We have performed our own fit to the MiniBooNE energy spectrum using the data release from~\cite{Aguilar-Arevalo:2018gpe}, and our results agree with~\cite{Bertuzzo:2018itn}. The data release, however, only contains information about the neutrino energy and baseline distance. Thus, the re-weighting procedure for the model of interest can only be performed approximately. A proper analysis can be performed only if true and reconstructed electron angles and energies per simulated event are given.  
%
\begin{figure}[h]
    \centering
    \includegraphics[width=0.75\textwidth]{both_cartoon.pdf}
    \caption[New physics signal in neutrino-electron scattering data.]{Neutrino-electron scattering data in $dE/dx$ at \minerva (top) and in $E\theta^2$ at CHARM-II (bottom). Error bars are too small to be seen. For both experiments, we show the $\nu-e$ signal and total background prediction quoted (after tuning at MINER$\nu$A), as well as the NP prediction (divided by 10 at CHARM-II). The cuts in our analysis our shown as vertical lines. \label{fig:NP_events}}
\end{figure}


%
\begin{figure}[h]
    \centering
    \includegraphics[width=0.7\textwidth]{bounds.pdf}
%     \caption[New constraints on dark neutrinos.]{The MiniBooNE region of interest from~\cite{Bertuzzo:2018itn}, only fitted to the energy distribution, is shown as closed yellow (orange) regions for one (three) sigma C.L. The benchmark point, chosen to provide a good angular distribution fit, is shown as a black star. Exclusion from heavy neutrino searches is shown as a hatched background. Our new constraints at 90\% C.L. using \minerva are shown in blue for our nominal 30\% background normalization uncertainty (solid) and conservative case of 100\% background uncertainty (dashed). Our CHARM-II bound is shown in cherry red, where the 3\% background normalization from the sideband is shown as a solid curve and the conservative 10\% case as a dashed curve. The solid vertical black line at 100 MeV signals the point where 90\% of NP events lie in the most forward bin in the MiniBooNE angular distribution, and the dashed one where 60\% of events do so. Other relevant assumed parameters are shown above the plot; changing them does not change our conclusion.}
    \caption[New constraints on dark neutrinos.]{The fit to the MiniBooNE energy distribution from~\cite{Bertuzzo:2018itn} is shown as closed yellow (orange) region for one (three) sigma C.L., together with the benchmark point (${\bf\odot}$). Our constraints are shown at 90\% C.L. for \minerva LE in blue (solid -- 30\% background normalization uncertainty, dashed -- conservative 100\% case), for \minerva ME in cyan (solid -- 40\% background normalization uncertainty, dashed -- conservative 100\% case), and for CHARM-II in red (solid -- 3\% background normalization from the sideband constraint, dashed -- conservative 10\% case). Vertical lines show the percentage of excess events at MiniBooNE that lie in the most forward angular bin. Exclusion from heavy neutrino searches is shown as a hatched background. Other relevant assumed parameters are shown above the plot; changing them does not alter our conclusion.\label{fig:final_plot}}
\end{figure}
%

\section{Results and Prospects}
The resulting limits on dark neutrinos in neutrino-electron scattering experiments are shown in the $|U_{\mu 4}|$ vs $m_4$ plane at 90\% confidence level (CL) in~\reffig{fig:final_plot}. The MiniBooNE fit from~\cite{Bertuzzo:2018itn} is shown, together with vertical lines indicating the percentage of events at MiniBooNE that populate the most forward angular bin.
We have chosen the same values of $\varepsilon$, $\alpha_D$, and $m_{Z^\prime}$ as used in~\cite{Bertuzzo:2018itn}, and shown their benchmark point ($m_4 = 420$ MeV and $|U_{\mu 4}|^2 = 9 \times 10^{-7}$) as a dotted circle. For these parameters, we can conclude that a good angular distribution at MiniBooNE is in large tension with neutrino-electron scattering data.
We note that the MiniBooNE event rate scales identically to our signal rate in all the couplings, and the dependence on $m_{Z^\prime}$ is subleading due to the typical momentum transfer to the nucleus, provided $m_{Z^\prime} \lesssim 100$ MeV .
This implies that changing the values of these parameters does not modify the overall conclusions of our work. In addition, for this realization of the model, larger $m_{Z^\prime}$ implies larger values of $m_4$, increasing the tension between the MiniBooNE fit and our bounds.
Our results from \minerva and CHARM-II are mutually reinforcing given that 
they impose similar constraints for $m_4 \lesssim 200 $ MeV. For larger masses, the kinematics of the signal becomes less forward and the production thresholds start being important. This explains the upturns visible in our bounds, where we observe it first in \minerva and later in CHARM-II as we increase $m_4$, since CHARM-II has higher beam energy. Finally, we emphasize that our analysis can be adapted to other models, such as the dark neutrino realisation of~\cite{Ballett:2018ynz} and scenarios with heavy neutrinos with dipole interactions~\cite{Magill:2018jla}. For the former, however, we do not expect our bounds to constrain the region of parameter space where the MiniBooNE explanation is viable, since most of the signal at MiniBooNE contains hadronic activity which would be visible at \minerva and CHARM-II.

In the near future, our new analysis strategy could be used in the up-coming \minerva ME results on antineutrino-electron scattering. The NP cross section, being the same for neutrino and antineutrinos, is thus more prominent on top of backgrounds.
This class of analyses will also greatly benefit from improved calculations and measurements of coherent $\pi^0$ production and single-photon emitting processes. This is particularly important given the excess seen in the \minerva ME analysis.
A complementary result can also be obtained by neutrino-electron scattering measurements at NO$\nu$A, which will sample a different kinematic regime as its off-axis beam peaks at lower energies and expects fewer NC~$\pi^0$ events per ton.
Beyond neutrino-electron scattering, the BSM signatures we consider could be lurking in current measurements of $\pi^0$ production, \textit{e.g.}, at MINOS~\cite{Adamson:2016hyz} and MINER$\nu$A~\cite{Wolcott:2016hws}~\footnote{This $\nu_e$CCQE measurement by \minerva observes a significant excess of single photon-like showers attributed to diffractive $\pi^0$ events. These are abundant in similar realizations of this NP model~\cite{Ballett:2018ynz}.}, and in analyses like the single photon search performed by T2K~\cite{Abe:2019cer}.
To summarize, a variety of measurements are underway to further lay siege to this explanation of the MiniBooNE observation and, simultaneously, start probing testable neutrino mass generation models, as well as other similar NP signatures. It is clear that understanding neutrino cross sections will be crucial as we move forward.


\chapter{The future of neutrino precision physics}
%%%%%%%%%%%%%%%%%%%%%%%%%%%%%%%%%%%%%%%%%%%%%%%%%%%%%
\graphicspath{{}{tridentSM/figs/}{tridentSM/}}

\section{Introduction}
\label{sec:intro}

The Standard Model (SM) has been confronted with a variety of experimental data and has so far emerged as an impressive phenomenological description of nature, except in the neutrino sector. The observation of neutrino flavour oscillations by solar, atmospheric, reactor and accelerator neutrino experiments over the last 50 years has revealed the existence of neutrino mass and flavour mixing, making necessary the first significant extension of the SM.

The precise determination of the neutrino mixing parameters as well as the search for the neutrino mass ordering and leptonic CP violation drive both present and future accelerator neutrino experiments. To accomplish these tasks, these experiments rely on state-of-the-art near detectors, made of heavy materials, located a few hundred meters downstream of the neutrino source and subjected to a high intensity beam. Their main purpose is to ensure high precision measurements at a far detector by reducing the systematic uncertainties related to neutrino fluxes, charged-current (CC) and neutral-current (NC) cross sections and backgrounds.  
%
The high beam luminosity they are subjected to (about $10^{21}$ protons on target) and their relatively large fiducial mass of high-$Z$ materials (typically 100~ton) make these detectors ideal places to investigate rare neutrino-nucleus interactions ($\sigma\lesssim 10^{-44}$~cm${}^2$), such as neutrino trident scattering. 

Trident events are processes predicted by the SM as the result of (anti)neutrino-nucleus scattering with the production of a charged lepton pair \cite{Czyz:1964zz,Lovseth:1971vv,Fujikawa:1971nx,Brown:1971qr,Koike:1971tu}, $\pbar{\nu}_{\alpha}+{\cal{H}} \to \pbar{\nu}_{\alpha \,{\rm{or}} \, \kappa(\beta)} + \ell_{\beta}^- + \ell_\kappa^+ +{\cal{H}}$, $\{\alpha,\beta,\kappa\}\in \{e,\mu,\tau\}$\footnote{Throughout the manuscript we will consider ${\alpha,\beta, \kappa}$ as flavour indexes.} where $\cal{H}$ denotes a hadronic target. Depending on the (anti)neutrino and charged lepton flavours in the final-state, the process will be mediated by the $Z^0$ boson, $W$ boson or both. Coherent interactions between (anti)neutrinos and the atomic nuclei are expected to dominate these processes as long as the momentum transferred $Q$ is significantly smaller than the inverse of the nuclear size \cite{Czyz:1964zz}. For larger momentum transfers diffractive and deep-inelastic scattering become increasingly relevant \cite{Magill:2016hgc}.
%
Although this process exists for all combinations of same-flavour or mixed flavour charged-lepton final-states, to this day only the $\nu_\mu$-induced dimuon mode, $\pbar{\nu}_\mu + {\cal{H}} \to \pbar{\nu}_\mu  + \mu^+ + \mu^- + {\cal{H}}$, has been observed. The first measurement of this trident signal performed by CHARM II~\cite{Geiregat:1990gz} is also the one with the largest statistics: 55 signal events in a beam of neutrinos and antineutrinos with $\langle E_\nu \rangle \approx 20$ GeV. Other measurements by CCFR~\cite{Mishra:1991bv} and NuTeV~\cite{Adams:1998yf} at larger energies soon followed.

As the measurement of trident events may provide a sensitive test of the weak sector~\cite{Brown:1973ih} as well as placing constraints on physics beyond the SM~\cite{Mishra:1991bv,Gaidaenko:2000hg,Altmannshofer:2014pba,Kaneta:2016uyt,Ge2017,Magill:2017mps,Falkowski:2018dmy} it is relevant to investigate how to probe it further at current and future neutrino experiments. Atmospheric neutrinos, for instance, may provide a feasible measurement of the dimuon channel, as pointed out in \refref{Ge2017}\footnote{The authors of \refref{Ge2017} have performed the full calculation of the trident process and made their code publicly available.}. Other trident modes were also re\-cog\-ni\-zed to be relevant by the authors of Ref.~\cite{Magill:2016hgc} who calculated the cross sections for trident production in all possible flavour combinations and estimated the number of events expected for the DUNE and SHiP experiments. They used the Equivalent Photon Approximation (EPA)~\cite{Belusevic:1987cw} to compute the cross section in the coherent and diffractive regimes of the scattering. The EPA, however, is known to breakdown for final state electrons~\cite{Kozhushner:1962aa, Shabalin:1963aa, Czyz:1964zz} leading, as we will demonstrate here, to an overestimation of the cross section that in some cases is by more than 200\%. 

In this work, we present a unified treatment of the coherent and diffractive trident calculation beyond the EPA for all modes. We then compute the number and distribution of events expected in each mode at various near detectors, devoting particular attention to the case of liquid argon (LAr) detectors, as they are expected to lead the field of precision neutrino scattering measurements over the next few decades thanks to their excellent tracking and calorimetry capabilities. Finally, we address the likely backgrounds that may hinder these experimental searches --- a question that we believe to be of utmost importance given the rarity of the process, and one that has been omitted in earlier sensitivity studies \cite{Magill:2016hgc,Altmannshofer:2014pba}. 

This paper is organized as follows. In Sec.~\ref{sec:xsec}, we explain how to correctly calculate the trident SM cross sections, comparing our results to the EPA and explicitly showing the breakdown of this approximation. In Sec.~\ref{sec:LAr}, we discuss the trident event rates and kinematic distributions at the near detectors of several present and future neutrino oscillation experiments based on LAr technology: the three detectors of the Short-Baseline Neutrino (SBN) Program at Fermilab~\cite{SBNproposal} and the near detector for the long-baseline Deep Underground Neutrino Experiment (DUNE)~\cite{Acciarri:2016ooe,DUNECDRvolII}, also located at Fermilab. We also consider the potential gains from an optimistic future facility: a 100 t LAr detector subject to the novel low-systematics neutrino beam of the Neutrinos from STORed Muons ($\nu$STORM) project~\cite{Soler:2015ada,nuSTORM2017}. 
%
We discuss the sources of background events at these facilities, providing a GENIE-level analysis \cite{Andreopoulos2009} of how to reduce these backgrounds and assessing the impact they are expected to have on the trident measurement. 
%
In Sec.~\ref{sec:others}, we discuss other near detectors that use more conventional technologies: the Interactive Neutrino GRID (INGRID)~\cite{Abe:2011xv,Abe:2015biq,Abe:2016fic,Abe:2016tez}, the on-axis iron near detector for T2K at J-PARC, as well as three detectors at the Neutrino at the Main Injector (NuMI) beamline at Fermilab, the one for the Main INjector ExpeRiment $\nu$-A (MINER$\nu$A)~\cite{Altinok:2017xua,MINERvA:2017} and the near detectors for the Main Injector Oscillation Search (MINOS)~\cite{Adamson:2014pgc,AlpernBoehm} and the Numi Off-axis $\nu_e$ Appearance (NO$\nu$A) experiment~\cite{Wang:Biao,sanchez_mayly_2018_1286758}. 
%
Finally, in Sec.~\ref{sec:conc} we  present our last remarks and conclusions.

%%%%%%%%%%%%%%%%%%%%%%%%%%%%%%%%%%%%%%%%%%%%%%%%%%%%
\section{Trident Production Cross Section}\label{sec:xsec}

In this section we consider neutrino trident production in the SM, defined as the process where a (anti)neutrino scattering off a hadronic system ${\cal H}$ produces a pair of same-flavour or mixed flavour charged leptons 
%
\begin{equation}
\pbar{\nu}_{\alpha}(p_1) \,+\, {\cal H}(P) \,\to\, \pbar{\nu}_{\alpha\, {\rm or}\, \kappa(\beta)} (p_2) \,+\, \ell_\beta^- (p_4)  \,+\, \ell_\kappa^+ (p_3) \,+\, {\cal H}(P^\prime),
\label{eq:indices}
\end{equation}
%
where $\beta (\kappa)$ corresponds to the flavour index of the negative (positive) charged lepton in both neutrino and antineutrino cases. 
%
Neutrino trident scattering can be divided into three regimes depending on the nature of the hadronic target: coherent, diffractive and deep inelastic, when the neutrino scatters off the nuclei, nucleons and quarks, respectively. At the energies relevant for neutrino oscillation experiments, the deep inelastic scattering contribution amounts at most to 1\% of the total trident production cross section \cite{Magill:2016hgc} and we will not consider it further.

The cross section for trident production has been calculated before in the literature, both in the context of the $V-A$ theory~\cite{Czyz:1964zz,Lovseth:1971vv,Fujikawa:1971nx} and in the SM~\cite{Brown:1973ih}, while the EPA treatment was developed in Refs.~\cite{Kozhushner:1962aa,Shabalin:1963aa,Belusevic:1987cw}. Most calculations have focused on the coherent channels \cite{Czyz:1964zz,Lovseth:1971vv,Fujikawa:1971nx,Brown:1973ih,Belusevic:1987cw} but the diffractive process has been considered in \cite{Czyz:1964zz,Lovseth:1971vv}. More recently, calculations using the EPA have been performed for coherent scattering with a dimuon final-state \cite{Altmannshofer:2014pba}, and for all combinations of hadronic targets and flavours of final-states in \cite{Magill:2016hgc}. While the EPA is expected to agree reasonably well with the full calculation for coherent channels with dimuon final-states, the assumptions of this approximation are invalid for the coherent process with electrons in the final-state \cite{Kozhushner:1962aa,Shabalin:1963aa,Czyz:1964zz}. 
%
For this reason, we perform the full $2\to 4$ calculation without the EPA in a manner applicable to any hadronic target, following a similar approach to Refs.~\cite{Czyz:1964zz,Lovseth:1971vv}. Our treatment of the cross section allows us to quantitatively assess the breakdown of the EPA in both coherent and diffractive channels for all final-state flavours, an issue we come back to in Sec.~\ref{sec:EPAbreakdown}.

We write the total cross section for neutrino trident production off a nucleus ${\cal N}$ with $Z$ protons and $(A-Z)$ neutrons as the sum
%
\begin{equation}
\sigma_\mathrm{\nu {\cal N}} = \sigma_\mathrm{\nu c} +\sigma_\mathrm{\nu d}\, ,
\end{equation}
where $\sigma_\mathrm{\nu c}$ ($\sigma_\mathrm{\nu d}$) is the coherent (diffractive) part of the cross section. 
%
%%%%%%%%%%%% SM DIAGRAMS %%%%%%%%%%%%%%%%
\unitlength = 0.9mm
\begin{figure}[t]
% \centering\includegraphics[width=\textwidth]{tridentSM/figs/SM-trident.pdf}
\centering\includegraphics[width=\textwidth]{tridentSM/figs/Neutrino_trident_production.pdf}
\caption{Diagrams contributing to the neutrino trident process in the four-point interaction limit of the Standard Model.  
\label{fig:Tdiagrams}}
\end{figure}

%%%%%%%%%%%%%%%%%%%%%%%%%%%%%%%%%%%%%%%%%
%
The relevant diagrams for these processes in the coherent or diffractive 
regimes involve the boson $Z^0$, $W$ or both mediators, depending on the particular mode. In the four-point interaction limit, depicted in \reffig{fig:Tdiagrams}, these reduce to only two contributions\footnote{An additional diagram involving a $WW\gamma$ vertex has also been neglected, since it is of order $1/M_W^4$.}, one where the photon couples to the negatively and one to the positively charged lepton. 
In Table \ref{tab:tridentmodes} we present the processes that will be considered in this work as well as the SM contributions present in each. Although our formalism applies also to processes with final-state $\tau$ leptons, the increased threshold makes them irrelevant for the experiments of interest in this study and we do not consider them further.
%
The trident amplitude for a coherent (${\rm X=c}$) or diffractive (${\rm X=d}$) scattering regime can be written as
%
\begin{equation}
i \mathcal{M} = \mathrm{L}^\mu (\{p_i\},q) \, \frac{-ig_{\mu \nu}}{q^2} \, \mathrm{H}_{\rm X}^{\nu}(P,P^{\prime})\, ,
\end{equation}
%
where $\{p_i\}=\{p_2,p_3,p_4\}$ is the set of outgoing leptonic momenta. $ \mathrm{L}^\mu (\{p_i\},q)$ is the total leptonic amplitude 
%
\begin{align}
\mathrm{L}^\mu & \equiv - \frac{ie G_F}{\sqrt{2}}[\bar{u}(p_2)\gamma^\tau(1-\gamma_5)u(p_1)] \times 
\bar{u}(p_4)\left[\gamma_\tau(V_{\alpha\beta\kappa}-A_{\alpha\beta\kappa}\gamma_5)\frac{1}{(\slashed{q}-\slashed{p}_3-m_3)}\gamma^\mu \right . \nonumber \\ 
& \left . + \gamma^\mu \frac{1}{(\slashed{p}_4-\slashed{q}-m_4)} \gamma_\tau (V_{\alpha\beta\kappa}-A_{\alpha\beta\kappa}\gamma_5) \right] v(p_3)\, ,
\label{eq:Lmu}
\end{align}
%
and $\mathrm{H}_{\rm X}^{\nu}(P,P^{\prime})$ is the total hadronic amplitude
\begin{align}
H_{\rm X}^\nu  &\equiv \langle {\cal H}(P) \vert J_\mathrm{E.M.}^\nu (q^2)\vert {\cal H}(P^\prime)\rangle\, ,
\label{eq:Hmu}
\end{align}
%
with  $q \equiv P - P^\prime$ denoting the transferred momentum, $m_3$ ($m_4$) the positively (negatively) charged lepton mass, $V_{\alpha\beta\kappa}(A_{\alpha\beta\kappa})\equiv g_{V}^{\beta}(g_A^{\beta})\delta_{\beta\kappa}+\delta_{\alpha\beta} \,(\beta=\alpha \, \mathrm{or} \; \kappa)$ the vector (axial) couplings, depending on the channel and have labels in accordance to Eq. (\ref{eq:indices}), and ${J}^\nu_{\rm{E.M.}} (q^2)$ the electromagnetic current for the hadronic system ${\cal H}$ (a nucleus or a nucleon).
%
\begin{table}[t]
\begin{center}
\begin{tabular}{|cc|}
\toprule\toprule
\bf (Anti)Neutrino &  \bf SM Contributions \\
\midrule\midrule
$\pbar{\nu}_\mu\, {\cal H} \to \pbar{\nu}_\mu\, \mu^- \mu^+\,  {\cal H}$  & CC + NC \\
$\pbar{\nu}_\mu\, {\cal H} \to \pbar{\nu}_e\,  e^\pm \mu^\mp\, {\cal H}$  & CC\\
$\pbar{\nu}_\mu\, {\cal H} \to \pbar{\nu}_\mu\,  e^- e^+\, {\cal H}$ & NC\\
$\pbar{\nu}_e\, {\cal H} \to \pbar{\nu}_e\,  e^- e^+\, {\cal H}$ &  CC + NC \\
$\pbar{\nu}_e\, {\cal H} \to \pbar{\nu}_\mu\,  \mu^\pm e^\mp\, {\cal H}$ & CC \\
$\pbar{\nu}_e\, {\cal H} \to \pbar{\nu}_e\,  \mu^- \mu^+\, {\cal H}$ & NC \\
\bottomrule
\bottomrule
\end{tabular}
\end{center}
\caption{\label{tab:tridentmodes} (Anti)Neutrino trident processes considered in this paper.}
\end{table}
%

We can write the differential cross section as
\begin{align}
\frac{\dd^2 \sigma_{\nu  {\rm X}}}{\dd Q^2 \dd \hat{s}}= \frac{1}{32  \pi^2(s-M_{\cal H}^2)^2}\frac{\mathrm{H}_{\rm X}^{\mu\nu}\mathrm{L}_{\mu\nu}}{Q^4}\, ,
\end{align}
where $s = (p_1 + P)^2$, $\hat{s} \equiv 2\,(p_1 \vdot q)$, $Q^2 = -q^2$ and $M_{\cal H}$ is the mass of the hadronic target. We have also introduced the hadronic tensor $\mathrm{H}_{\rm X}^{\mu \nu}$
%
\begin{align}
\mathrm{H}_{\rm X}^{\mu\nu} &\equiv \overline{\sum_{\rm{spins}}}  \left(\mathrm{H}_{\rm X}^\mu\right)^* \mathrm{H}_{\rm X}^\nu.
%
\end{align}
The two scattering regimes in which the hadronic tensor is computed will be discussed in more detail in Sec.~\ref{sec:had}. The leptonic tensor, $\mathrm{L}^{\mu \nu}$, integrated over the phase space of the three final-state leptons, $\dd^{3} \Pi \left(p_1 + q; \{p_i\}\right)$, and merely summed over final and initial spins is given by
%
\begin{equation}
\mathrm{L}^{\mu \nu} (p_1, q) \equiv  \int \dd ^{3} \Pi \left(p_1 + q; \{p_i\}\right) \left( \sum_{\rm{spins}} \left(  \mathrm{L}^\mu \right)^*  \mathrm{L}^\nu  \right)\, .
\label{eq:Lmunu}
\end{equation}
We can use $\mathrm{L}^{\mu \nu}$ to define two scalar functions, one related to the longitudinal ($\mathrm{L}_{\mathrm{L}}$) and the other to the transverse ($\mathrm{L}_{\mathrm{T}}$) polarization of the exchanged photon
\begin{equation}
\mathrm{L}_{\mathrm{T}} = -\frac{1}{2}\left( g^{\mu \nu} - \frac{4Q^2}{\hat{s}^2} p_1^\mu p_1^\nu \right) \mathrm{L}_{\mu \nu}, \quad \mathrm{and} \quad \mathrm{L_{L}} =  \frac{4Q^2}{\hat{s}^2} p_1^\mu p_1^\nu \mathrm{L}_{\mu \nu}\, .\label{eq:LT_LL}
\end{equation}
%
This allows us to write the differential cross section as a sum of a longitudinal and a transverse contribution \cite{Hand:1963bb} as follows
%
\begin{align}
\frac{\dd^2 \sigma_{\nu  {\rm X}}}{\dd Q^2 \dd \hat{s}} &= \frac{1}{32 \pi^2} \frac{1}{\hat{s}\,Q^2} \left [ h_{\rm X}^\mathrm{T}(Q^2, \hat{s}) \, \sigma^\mathrm{T}_{\nu \gamma} (Q^2, \hat{s}) + h_{\rm X}^\mathrm{L}(Q^2, \hat{s}) \, \sigma^\mathrm{L}_{\nu \gamma} (Q^2, \hat{s}) \right] \, ,\label{eq:full_diff_xsec}
\end{align}
%
where we have defined two functions for the flux of longitudinal and transverse virtual photons 
%
\begin{subequations}
\label{eq:splitting_function}
\begin{align}
%
h_{\rm X}^\mathrm{T}(Q^2, \hat{s})  &\equiv \frac{2}{(E_\nu M_{\cal H})^2} \left[ p_{1 \mu} p_{1\nu} -\frac{\hat{s}^2}{4 Q^2} \, g_{\mu \nu} \right]\mathrm{H}_{\rm X}^{\mu \nu} , \quad \text{and} \\ \quad
h_{\rm X}^\mathrm{L}(Q^2, \hat{s})  &\equiv \frac{1}{(E_\nu M_{\cal H})^2} \, p_{1\mu} p_{1\nu}\, \mathrm{H}_{\rm X}^{\mu \nu}\, ,
%
\end{align}
%
\end{subequations}
%
and two leptonic neutrino-photon cross sections associated with them\footnote{Note that we include a factor of $1/2$ in $\sigma^\mathrm{T}_{\nu \gamma}$ to match the polarization averaging of the on-shell cross section: $\sigma_{\nu \gamma}^{\rm on-shell} = \frac{1}{2 \hat{s}} \left( \overline{\sum}_r (\epsilon_r^\mu)^* \epsilon^\nu_r \, {\rm L}_{\mu\nu} \right) \big\vert_{Q^2=0} = \frac{1}{4 \hat{s}} \left( - g^{\mu\nu} L_{\mu\nu}\right) \big\vert_{Q^2=0} = \frac{{\rm L_T}}{2 \hat{s}}\big\vert_{Q^2=0} = \sigma_{\nu\gamma}^\text{T}(0,\hat{s})$.}
%
\begin{equation}
\sigma^\mathrm{T}_{\nu \gamma} (Q^2, \hat{s})  = \frac{\mathrm{L_T}}{2 \hat{s}}\, , \quad \mathrm{and} \quad
\sigma^\mathrm{L}_{\nu \gamma} (Q^2, \hat{s})  = \frac{\mathrm{L_L}}{\hat{s}}\, .
\end{equation}
%
The kinematically allowed region in the $(Q^2,\hat{s})$ plane can be obtained by considering the full four-body phase space, as in~\cite{Czyz:1964zz,Lovseth:1971vv,Fujikawa:1971nx}. The limits for such physical region are given by
\begin{subequations}\label{eq:qslimts}
	\begin{align}
		Q_{\rm min}^2&=\frac{M_{\cal H} \hat{s}^2}{2E_\nu(2E_\nu M_{\cal H}-\hat{s})},&\  
        Q_{\rm max}^2&=\hat{s}-m_L^2,\label{eq:qlimts}\\
        \hat{s}_{\rm min}&=\frac{E_\nu}{2E_\nu + M_{\cal H}}\left[m_L^2+2E_\nu M_{\cal H} -\Delta\right]&\  
        \hat{s}_{\rm max}&=\frac{E_\nu}{2E_\nu + M_{\cal H}}\left[m_L^2+2E_\nu M_{\cal H} +\Delta\right],\label{eq:shatlimts}
	\end{align}
\end{subequations}
%
with $m_L\equiv m_3+m_4$, and
%
\begin{align*}
	\Delta \equiv \sqrt{(2E_\nu M_{\cal H}-m_L^2)^2-4M_{\cal H}^2 m_L^2}\,.
\end{align*}
%
Let us emphasize that \refeq{eq:full_diff_xsec} is an exact decomposition, and does not rely on any approximation of the process. In the following section, we will show how to calculate the flux functions $h_{\rm X}^\mathrm{T}$ and $h_{\rm X}^\mathrm{L}$ from Eq.~\ref{eq:splitting_function} in different scattering regimes. The total cross section for the process can then be computed by finding $\sigma^\mathrm{L}_{\nu \gamma}$ and $\sigma^\mathrm{T}_{\nu \gamma}$ from Eqs.~(\ref{eq:Lmu}), (\ref{eq:Lmunu}) and (\ref{eq:LT_LL}) and integrating over all allowed values of $Q^2$ and $\hat{s}$. Note that $\sigma^\mathrm{L}_{\nu \gamma}$ and $\sigma^\mathrm{T}_{\nu \gamma}$ are universal functions for a given leptonic process and need only to be computed once.

%%%%%%%%%%%%%%%%%%%%%%%%%%%%%%%%%%%%%%%%%%%%%%%%%%%%%%%%%%%%%%
\subsection{Hadronic Scattering Regimes}\label{sec:had}

Depending on the magnitude of the virtuality of the photon, $Q = \sqrt{-q^2}$, the hadronic current can contribute in different ways to the trident process. Thus, given the decomposition in \refeq{eq:full_diff_xsec}, the change in the hadronic treatment translates to computing the flux factors $h_{\rm X}^\mathrm{T}$ and $h_{\rm X}^\mathrm{L}$ for each scattering regime.  From those flux factors, $\sigma_{\nu\mathrm{c}}$ and $\sigma_{\nu\mathrm{d}}$ can be calculated.

%%%%%%%%%%%%%%%%%%%%%%%%%%%%%%%%%%%%%%%%%%%%%%%%%%
\subsubsection{Coherent Regime (${\rm H}^{\mu \nu}_{\rm c}$)}

In the coherent scattering regime the incoming neutrino interacts with the whole nucleus without resolving its substructure. For this to occur frequently, we need small values of $Q$. Despite the relatively large neutrino energies in contemporary neutrino beams, this is still allowed for trident.

In this regime, the hadronic tensor $\mathrm{H}^{\mu\nu}_\mathrm{c}$ for a ground state spin-zero nucleus of charge $Z e$ can be written in terms 
of the nuclear electromagnetic form factor $F(Q^2)$, discussed in more detail in Appendix~\ref{app:formfactors}, as
%
\begin{equation}
\mathrm{H}^{\mu \nu}_\mathrm{c} =  4Z^2 e^2 \left| F (Q^2)\right|^2 \left(P^\mu - \frac{q^\mu}{2}\right) \left(P^\nu - \frac{q^\nu}{2}\right).
\end{equation}
%
From Eq.~\ref{eq:splitting_function}, we find that the transverse and longitudinal flux functions for the coherent regime are
%
\begin{subequations}\label{eq:hcoh}
\begin{align}
h^\mathrm{T}_\mathrm{c}(Q^2, \hat{s})  &=  8 Z^2 e^2   \left(1 - \frac{\hat{s}}{2E_\nu M} - \frac{\hat{s}^2}{4 E_\nu^2 Q^2}\,\right) |F (Q^2)|^2\, , \\
h^\mathrm{L}_\mathrm{c}(Q^2, \hat{s})  &=  4 Z^2 e^2  \left(1 - \frac{\hat{s}}{4E_\nu M}\right)^2 |F (Q^2)|^2\, ,
\end{align}
\end{subequations}
where $E_\nu$ is the energy of the incoming neutrino and $M$ is the nuclear mass. For a fixed value of $\hat{s}$ in the physical region, the $h^{\rm T}_{\rm c}$ flux function becomes zero at $Q_{\rm min}$ while the longitudinal component does not. This different behaviour can be seen explicitly in their definitions, Eqs. \eqref{eq:hcoh}, as the terms in the parenthesis in $h^{\rm T}_{\rm c}$ cancel each other at $Q_{\rm min}$. This does not occur for $h^{\rm L}_{\rm c}$ since the physical values of $\hat{s}$ are always smaller than $E_\nu M$ in this hadronic regime. 
Due to this fact, $Q_{\rm min}$, which according to  Eq.\ \eqref{eq:qlimts} depends on  both the  neutrino energy and target material, can be approximated to
\begin{align*}
	Q_{\rm min} \approx \frac{\hat{s}}{2E_{\nu}},
\end{align*}
which only depends on the incoming neutrino energy. On the other hand, as $Q$ becomes large, the flux functions $h^{T,L}$ become quite similar, $h^{\rm T}_{\rm c}\approx 2 h^{\rm L}_{\rm c}$, and favour small values of $\hat{s}$. After some critical value of the virtuality $Q$, $h^{\rm T, L}_{\rm c}$ become negligible due to the nuclear form factor. The $Q$ value at which this happens depends on the target material, but not on the incoming neutrino energy. For instance, in the case of an Ar target
the flux functions basically vanish for $Q\gtrsim 250$ MeV.

The final cross sections for coherent neutrino trident production on Argon can be seen in \reffig{fig:coh_xsec}. Despite thresholds being important for the behaviour of these cross sections for GeV neutrino energies, we can see that mixed channels quickly become the most important due to their CC nature. At large energies one can then rank the cross sections from largest to smallest as CC, CC+NC, and NC only channels. Nevertheless, one must be aware of the fact that the cross sections are dominated by low $Q^2$ even at large energies, leading to large effects due to the final-state lepton masses as discussed in \cite{Magill:2016hgc}.

\begin{figure}[t]
\centering
\includegraphics[width=\textwidth]{figs/Xsec_4PS_coh.pdf}
\caption{Cross sections for coherent neutrino trident production on $^{40}$Ar (left) and  $^{208}$Pb (right) normalized to $\sigma_0 =  Z^2\, 10^{-44}$ cm$^2$. The full (dashed) lines correspond to the scattering of an incoming $\nu_\mu$ ($\nu_e$) produced by the NC (light-blue), CC (purple), and CC+NC (orange) SM interactions. \label{fig:coh_xsec}}
\end{figure}


%%%%%%%%%%%%%%%%%%%%%%%%%%%%%%%%%%%%%%%%%%%%%%%%%%%%
\subsubsection{Diffractive Regime ($\mathrm{H}^{\mu\nu}_\mathrm{d}$)}

At larger $Q^2$, the neutrino interacts with the individual nucleons of the nucleus. In this diffractive regime $\mathrm{H}^{\mu\nu}_\mathrm{d}$ is given by the sum of the contributions of the two types of nucleons: protons ($\mathrm{N=p}$) and neutrons ($\mathrm{N=n}$), so
\begin{equation}
\mathrm{H}^{\mu \nu}_\mathrm{d} (P, P^\prime) = Z\, \mathrm{H}^{\mu \nu}_\mathrm{p} (P, P^\prime)+
(A-Z)\,\mathrm{H}^{\mu \nu}_\mathrm{n}(P, P^\prime)\, ,
\end{equation}
where each $\mathrm{H}^{\mu \nu}_\mathrm{N}$ is the square of the matrix element of the nucleon electromagnetic current summed over final and averaged over initial spins. Neglecting second class currents, the matrix elements take the form
\begin{equation}
\bra{\mathrm{N}(P^\prime)} {J}^\mu_{\rm{E.M.}} (Q^2) \ket{\mathrm{N} (P) } = e \, \overline{u}_\mathrm{N} (P^\prime) \left[ \gamma^\mu F^\mathrm{N}_1(Q^2) - i \frac{\sigma^{\mu \nu} q_{\nu}}{2 M_{\rm N}} F^\mathrm{N}_2(Q^2) \right] u_\mathrm{N}(P)\, ,
\end{equation}
%
with $F^\mathrm{N}_{1,2}(Q^2)$ the Dirac and Pauli form factors, respectively. The hadronic tensors are then given by \cite{Kniehl:1990iv}
%
\begin{equation}
\mathrm{H}^{\mu \nu}_\mathrm{N} = e^2 \left[ 4 \, H_1^\mathrm{N}(Q^2) \left(P^\mu - \frac{q^\mu}{2}\right)\left(P^\nu - \frac{q^\nu}{2}\right) - H_2^\mathrm{N}(Q^2) \left( Q^2 g^{\mu \nu} + q^\mu q^\nu \right) \right]\, ,
\end{equation}
%
where the  $H_1^\mathrm{N}(Q^2)$ and $H_2^\mathrm{N}(Q^2)$ form factors, functions of $F^\mathrm{N}_{1,2}(Q^2)$, are given in Appendix~\ref{app:formfactors}. The flux functions in the diffractive regime can then be calculated as
\begin{subequations}\label{eq:dcoh}
\begin{align}
h^\mathrm{T}_\mathrm{N}(Q^2, \hat{s})  &=  8 \, e^2 \left[ \left(1 - \frac{\hat{s}}{2E_\nu M_{\rm N}} - \frac{\hat{s}^2}{4 E_\nu^2 Q^2 }\,\right) H_1^\mathrm{N}(Q^2) + \frac{\hat{s}^2}{8E_\nu^2 M_{\rm N}^2}  H_2^\mathrm{N}(Q^2)\right ]\, ,\label{eq:hTdiff}\\
%
h^\mathrm{L}_\mathrm{N}(Q^2, \hat{s})  &=  4 e^2 \, \left[ \left(1-\frac{\hat{s}}{4 E_\nu M_{\rm N}} \right)^2 H_1^\mathrm{N}(Q^2)  - \frac{\hat{s}^2}{16 E_\nu^2 M_{\rm N}^2} H_2^\mathrm{N}(Q^2)\right]\, .\label{eq:hLdiff}
\end{align}
\end{subequations}
%
In the case of the proton, the flux functions $h^{\rm T, L}_{\rm p}$ have some unique features given the presence of both electric and magnetic contributions. Specifically, the transverse function is non-zero at $Q=Q_{\rm min}$ for a fixed $\hat{s}$, due to the additional term proportional to $H_2^{\rm p}$. Indeed, for large values of $\hat{s}$, the $H_2^{\rm p}$ term dominates the transverse function. An opposite behaviour occurs for the longitudinal component. There, the $H_1^{\rm p}$ term dominates over the second term for all physical values of $\hat{s}$, $Q$, and for any incoming neutrino energy. On the other hand, the flux functions of the neutron, which have only 
the magnetic moment contribution, have somewhat different characteristics. While 
$h^{\rm T}_{\rm n}$ behaves similarly to $h^{\rm T}_{\rm p}$, that is, it is dominated by the second term for large values of $\hat{s}$, $h^{\rm L}_{\rm n}$ is zero at $Q_{\rm min}$ due to the exact cancellation between the $H_{1,2}^{\rm n}$ terms. This cancellation is not evident from Eq.\ ~\eqref{eq:hLdiff}; however, simplifying the longitudinal component for the neutron case, one finds
\begin{align*}
	h^\mathrm{L}_\mathrm{n}(Q^2, \hat{s})  &=4 e^2 \left(1+\frac{Q^2}{4M_{\rm n}^2}\right)\frac{Q^2}{4 M_{\rm N}^2}\left( 1 - \frac{\hat{s}}{2 E_\nu M_{\rm N}} - \frac{\hat{s}^2}{4 E_\nu^2 Q^2} \right) \left| F^\mathrm{n}_2(Q^2) \right|^2,
\end{align*}
which is zero for $Q=Q_{\rm min}$. Also, this shows why $h^{\rm L}_{\rm p}$ does not 
vanish at $Q_{\rm min}$ since there we have the additional contribution of the electric component. 

When the neutrino interacts with an individual nucleon inside the nucleus, one must be aware of the nuclear effects at play. One such effect is Pauli blocking, a suppression of neutrino-nucleon interactions due to the Pauli exclusion principle. Modelling the nucleus as an ideal Fermi gas of protons and neutrons, one can take Pauli blocking effects into account by requiring that the hit nucleon cannot be in a state which is already occupied \cite{Brown:1971qr}. This requirement is implemented in our calculations by a simple replacement of the differential diffractive cross section
\begin{align*}
\frac{\dd^2 \sigma_{\nu  \mathrm{d}}}{\dd Q^2 \dd \hat{s}}\to f (|\vec{q}|) \, \frac{\dd^2 \sigma_{\nu  \mathrm{d}}}{\dd Q^2 \dd \hat{s}},
\end{align*}
where $|\vec{q}|$ is the magnitude of the transferred three-momentum in the lab frame. In particular, following \cite{Brown:1971qr}, assuming an equal density of neutrons and protons, we have
%
\begin{equation}
f (|\vec{q}|) = \begin{cases} \displaystyle
                    \frac{3}{2} \frac{|\vec{q}|}{2 \, k_F} - \frac{1}{2} \left( \frac{|\vec{q}|}{2 \, k_F} \right)^3 ,\, &\mathrm{if }\;\; |\vec{q}| < 2\, k_F\, ,\\
                    1,\, &\mathrm{if }\;\; |\vec{q}| \geq 2 \, k_F\, ,
                \end{cases}
\end{equation}
%
where $k_F$ is the Fermi momentum of the gas, taken to be $235$ MeV. This is a rather low value of $k_F$ and the assumption of equal density of neutrons and protons must be taken with care for heavy nuclei. We refrain from trying to model any additional nuclear effects as we believe that this is the dominant effect on the total diffractive rate, particularly when requiring no hadronic activity in the event. The net result is a reduction of the diffractive cross section by about $50\%$ for protons and $20\%$ for neutrons. Unless clearly stated otherwise, we always include Pauli blocking in our calculations.

Our final cross sections for this regime can be seen in \reffig{fig:dif_xsec}. One can clearly see that the neutron contribution is subdominant, and that, up to factors of $Z^2$, the proton one is comparable to the coherent cross section. Note that now the typical values of $Q^2$ are much larger than in the coherent regime and the impact of the final-state lepton masses is much smaller. 

\begin{figure}[t]
\centering\includegraphics[width=\textwidth]{figs/Xsec_4PS_diff.pdf}
\caption{Cross sections for diffractive neutrino trident production on neutrons (left) and protons (right), including Pauli blocking effects as described in the text, normalized to $\sigma_0 =  10^{-44}$ cm$^2$. The full (dashed) lines correspond to the scattering of an incoming $\nu_\mu$ ($\nu_e$) produced by the NC (light-blue), CC (purple), and CC+NC (orange) SM interactions. \label{fig:dif_xsec}}
\end{figure}


%%%%%%%%%%%%%%%%%%%%%%%%%%%%%%%%%%%%%%%%%%%%%%%%%%%%
\subsection{Breakdown of the EPA \label{sec:EPAbreakdown}}

In order to understand the breakdown of the EPA in the neutrino trident case, let us first remind briefly the reader about the Weizs\"acker--Williams method of equivalent photons in Quantum Electrodynamics (QED)~\cite{vonWeizsacker:1934nji,Williams:1934ad}, and the main reason for its validity in that theory. The EPA, first introduced by E.\ Fermi~\cite{Fermi:1924tc}, is based on a simple principle: when an ultra-relativistic particle $P_i$ approaches a charged system $C_s$, like a nucleus, it will perceive the electromagnetic fields as nearly transverse, similar to the fields of a pulse of radiation, {\it i.e.},  as an on-shell photon. Therefore, it is possible to obtain an approximate total cross section for the inelastic scattering process producing a set of final particles $P_f$, $\sigma_{\rm t}(P_i + C_s \to P_f + C_s)$, by computing the scattering of the incoming particle with a real photon integrated over the energy spectrum of the off-shell photons,
%
\begin{align}
	\sigma_{\rm t}(P_i + C_s \to P_f + C_s)\approx\int\, dP(Q^2,\hat{s})\,\sigma_\gamma(P_i + \gamma \to P_f; \hat{s}, Q^2 = 0),
\end{align}
where the photo-production cross section for the process $P_i + \gamma \to P_f$, 
$\sigma_\gamma(P_i + \gamma \to P_f; \hat{s}, Q^2 = 0)$, 
depends on the center-of-mass energy of the $P_i$--photon system, $\sqrt{\hat{s}}$. Here $dP(Q^2,\hat{s})$ corresponds to the energy spectrum of the virtual photons, that is, the probability of emission of a virtual photon with transferred four-momentum $Q^2$ resulting in an center-of-mass energy $\sqrt{\hat{s}}$.
For trident scattering off a nuclear target, this probability can be approximated by~\cite{Belusevic:1987cw,Altmannshofer:2014pba}
\begin{align}\label{eq:GenEPA}
	dP(Q^2,\hat{s})=\frac{Z^2e^2}{4\pi^2}|F (Q^2)|^2\,\frac{d\hat{s}}{\hat{s}}\,\frac{dQ^2}{Q^2}\, .
\end{align}
A crucial fact in QED is that the cross section $\sigma_\gamma^{\rm QED}(P_i + \gamma \to P_f; \hat{s},0)$ is inversely proportional to $\hat{s}$,
\begin{align*}
	\sigma_\gamma^{\rm QED}(P_i + \gamma \to P_f; \hat{s},0) \propto \frac{1}{\hat{s}}\,.
\end{align*}
We see clearly that small values of $\hat{s}$ and consequently of the transferred four-momentum $Q^2$ dominate the cross section. Hence, the on-shell contribution is much more significant 
than the off-shell one, so the EPA will be valid and give the correct cross  section 
estimate for any QED process. 

Now, let us consider the case of neutrino trident production. In this case, the equivalent-photon cross section in the four-point interaction limit has a completely opposite dependence on the center-of-mass energy; it is \emph{proportional} to $\hat{s}$,
\begin{align*}
	\sigma_\gamma^{\rm FL}(P_i + \gamma \to P_f; \hat{s}, 0)\propto G_{\rm F}^2\, \hat{s}\, .
\end{align*}
This dependence is a manifestation of the unitarity violation in the Fermi theory. Therefore, we can see that for weak processes larger values of $\hat{s}$, and, consequently, larger values of $Q^2$ are more significant \cite{Kozhushner:1962aa, Shabalin:1963aa}.  The EPA is then generally not valid for the neutrino trident production, as the virtual photon contribution dominates over the real one. Nevertheless, one may wonder if there is a situation  in which the EPA can give a reasonable estimate for a neutrino trident process. 
As noticed in the early literature \cite{Kozhushner:1962aa, Shabalin:1963aa}, the presence of the nuclear form factor introduces a cut in the transferred momentum which, in turn, makes the EPA applicable for the specific case of the dimuon channel in the coherent regime. Let us discuss this in more detail. 

Recalling our exact decomposition, \refeq{eq:full_diff_xsec}, it is necessary to consider two assumptions for implementing the EPA \cite{Kozhushner:1962aa}:
%
\begin{enumerate} 
%
\item The longitudinal polarization contribution to the cross section can be neglected, i.e., $\sigma_{\nu\gamma}^\mathrm{L}(Q^2,\hat{s})\approx 0$;
%
\item The transverse polarization contribution to the cross section can be taken to be on-shell, i.e., $\sigma^\text{T}_{\nu\gamma}(Q^2,\hat{s}) \approx \sigma^\text{T}_{\nu\gamma}(0,\hat{s})$. 
%
\end{enumerate}
%
Assuming for now that these approximations hold, we can find a simplified expression for the coherent neutrino-target process, described by Eqs.~(\ref{eq:full_diff_xsec}) and (\ref{eq:hcoh}), in terms of the photon-neutrino cross section\footnote{An analogous expression can be obtained for the diffractive regime from Eq.~(\ref{eq:dcoh}).}:
%
\begin{align}     
\sigma_\text{EPA} = \frac{Z^2e^2}{4\pi^2}\int_{m_L^2}^{\hat{s}_{\rm max}} \frac{d\hat{s}}{\hat{s}}\,
\sigma^\mathrm{T}_{\nu\gamma}(0,\hat{s})
\int_{(\hat{s}/2E_\nu)^2}^{Q^2_{\rm max}}\frac{|F (Q^2)|^2}{Q^4} \left[ Q^2(1-y) - M_{\cal H}^2y^2\right]dQ^2\, , 
\end{align}
%
where we introduced the fractional change of the nucleus energy $y$, defined as $\hat{s} = (s-M_{\cal H}^2)y$, and the integration limits can be obtained from \eqref{eq:qslimts} after considering that $m_L^2\ll E_\nu M_{\cal H}$. Keeping only the leading terms in the small parameter $y$ \cite{Belusevic:1987cw}, we recover the EPA applied to the neutrino trident case
%
\begin{align} \label{eq:EPA_bad}
\sigma_\text{EPA} = \int \sigma^\mathrm{T}_{\nu\gamma}(0,\hat{s}) \, dP(Q^2,\hat{s})\, ,
\end{align}
%
where $dP(Q^2,\hat{s})$ is given in Eq.~(\ref{eq:GenEPA}). The EPA in the form of \refeq{eq:EPA_bad} has been used in trident calculations for the coherent dimuon channel \cite{Altmannshofer:2014pba} as well as for coherent mixed- and electron-flavour trident modes and diffractive trident modes \cite{Magill:2016hgc}.  Using our decomposition, we can explicitly compute both $\sigma^\mathrm{L}_{\nu \gamma}$ and $\sigma^\mathrm{T}_{\nu \gamma}$ and verify if the EPA conditions are satisfied for any channel and, if they are not, quantify the error introduced by making this approximation. For that purpose, we will compare the results of the full calculation, \refeq{eq:full_diff_xsec}, with the EPA results, \refeq{eq:EPA_bad}, by computing the following ratios in the physical region of the $(Q,\hat{s})$ plane,
\begin{align}\label{eq:ratios}
		\frac{\sigma^{\rm L}(Q^2,\hat{s})\,h_{\rm c}^{\rm L}(Q^2,\hat{s})}{\sigma^{\rm T}(Q^2,\hat{s})\,h_{\rm c}^{\rm T}(Q^2,\hat{s})}\, , \quad \frac{\sigma^\mathrm{T}_{\nu\gamma}(Q^2,\hat{s})}{\sigma^\mathrm{T}_{\nu\gamma}(0,\hat{s})}\, .
\end{align} 
The first ratio in Eq.\ \eqref{eq:ratios} will indicate where the longitudinal contribution can be neglected compared to the transverse one; while, the second ratio will show where the transverse contribution behaves as an on-shell photon. 

As an illustration of the general behaviour, we show in Fig.\ \ref{fig:4PSvsEPA} those ratios 
of cross sections for an incoming $\nu_\mu$ of fixed energy $E_\nu=3$ GeV colliding coherently with an $^{40}$Ar target, for the dielectron (left panels), mixed  (middle panels) and dimuon  (right panels)
channels. On the top panels of Fig.\ \ref{fig:4PSvsEPA} we see that the longitudinal component can be neglected for $Q\lesssim m_\alpha$, for the dielectron and dimuon channels, $\alpha=e,\mu$, while in the mixed case there is a much less pronounced hierarchy between the transverse and longitudinal components. On the bottom panels we have the comparison between on-shell and off-shell transverse photo-production cross sections. Again, we find that the EPA is only valid for $Q \lesssim m_\alpha$ for the dielectron and dimuon channels. For the mixed case, there is only a very small region in $Q < 10^{-2}$\,GeV for which the off-shell transverse cross section is comparable to the on-shell one. This relative suppression of the off-shell cross section can be understood by noticing that $Q$ enters the lepton propagators, suppressing the process for $Q \gtrsim m_\alpha$. For mixed channels it is then the smallest mass scale ($m_e$) that dictates the fall-off of the matrix element in $Q$, whilst the heaviest mass ($m_\mu$) defines the phase space boundaries, rendering most of this phase space incompatible with the EPA assumptions.   
%
\begin{figure}[t]
\centering
\includegraphics[width=\textwidth]{figs/4PS_vs_EPA.pdf}%
\caption{\label{fig:4PSvsEPA} Comparison between the full calculation of the trident production 
coherent cross section and the EPA in the kinematically allowed region of the $(Q,\hat{s})$ plane for an incoming $\nu_\mu$ with fixed energy $E_\nu=3$ GeV colliding with an $^{40}$Ar target. 
The left, middle and right panels correspond to the dielectron, mixed and dimuon final-states, respectively. The top panels correspond to the comparison between the longitudinal and transverse contributions while the bottom ones show the ratio between the transverse cross sections computed for an specific value of $Q$ with the cross section for an on-shell photon. The thick black dashed lines correspond to the cut in the $Q^2$ integration at $\Lambda_{\rm QCD}^2/ A^{2/3}$, and the shadowed region around these lines account for a variation of $20\%$ in the value of this cut. The purple dashed lines are for $Q=m_\alpha$, $\alpha=e,\mu$ for the unmixed cases.}
\end{figure}

These results explicitly show that the EPA is, in principle, not suitable for any neutrino trident process as it can overestimate the cross section quite substantially by treating the photo-production cross section at large $Q^2$ as on-shell. However, as previously mentioned, in the coherent regime the nuclear form factor introduces a strong suppression for large values of $Q^2$. In general, this dominates the behaviour of the cross sections for values of $Q^2$ smaller than the purely kinematic limit, $Q^2_{\rm max}$, and of the order of $\Lambda_{\rm QCD}/ A^{1/3}\approx 0.06$ GeV for coherent scattering on $^{40}$Ar. In the dimuon case, the latter scale happens to be smaller than the charged lepton masses, implying that the region where the EPA breaks down is heavily suppressed due to the nuclear form factor. The same cannot be said about coherent trident channels involving electrons, as the nuclear form factor suppression happens for much larger values of $Q$ than the EPA breakdown. Furthermore, for diffractive scattering the nucleon form factors suppress the cross sections only for much larger $Q$ values, $Q\approx 0.8$ GeV. The effective range of integration then includes a significant region where the EPA assumptions are invalid, leading to an overestimation of the diffractive cross section for every process regardless of the flavours of their final-state charged leptons. 

%
\begin{figure}[t]
	\centering
	\includegraphics[width=\textwidth]{figs/XSec_ratio.pdf}%
	\caption{\label{fig:comparison4PS_EPA} 
    Ratio $\mathcal{R}$ of the trident cross section calculated using 
    the EPA to the full four-body calculation. 
    Left panel: Ratio in the coherent regime on $^{40}$Ar. The full curves correspond to the central value of $Q_{\rm cut}$, and the upper (lower) boundary corresponds to a choice 100 times larger ($20\%$ smaller). 
Right  panel: Ratio in the diffractive regime for scattering on protons, where the full curves corresponds to the central value of $1.0$ GeV, and the upper (lower) boundary corresponds to a choice 100 times larger ($20\%$ smaller); we have taken the lower limit in the integration on $Q$ to match the choice of the coherent regime and we do not include Pauli blocking in these curves. A guide to the eye at $\mathcal{R} = 1$ is also shown.}
\end{figure}

In some calculations, artificial cuts have been imposed on the range of $Q^2$, affecting the validity of the EPA. In Ref. \cite{Magill:2016hgc}, it is claimed that to avoid double counting between different regimes, an artificial cut must be imposed, lowering the upper limit of integration in $Q^2$. Ref.~\cite{Magill:2016hgc} chooses a value of $Q^{\rm cut}_{\rm max} = \Lambda_{\rm QCD}/ A^{1/3}$ in the coherent regime (black thick dashed lines in Fig.\ \ref{fig:4PSvsEPA}), and $Q^{\rm cut}_{\rm min}= {\rm max}\left( \Lambda_{\rm QCD}/ A^{1/3}, \hat{s}/2E_\nu\right)$ and $Q^{\rm cut}_{\rm max} = 1.0$ GeV in the diffractive regime. We believe that no such cut is required on physical grounds\footnote{It should be noted that the coherent and diffractive regimes have different phase space boundaries and that the form factors should guarantee their independence.}, and their presence will impact the EPA cross section quite dramatically. Let us first consider the dimuon case in the coherent regime, where the EPA assumptions hold reasonably well in the relevant parts of phase space. By introducing a value for $Q^{\rm cut}_{\rm max}$ we would be decreasing the total relevant phase space for the process, reducing the total cross section. Therefore, despite the EPA tendency to overestimate the cross section in this channel, an artificial cut in $Q^2$ can actually lead to an underestimation of the cross section. In the electron channels, where the EPA breakdown is much more dramatic, we can expect that the overestimation of the cross section by the EPA is reduced by the cut $Q^{\rm cut}_{\rm max}$. In fact, one way to improve the EPA for the dielectron channel is to artificially cut on the $Q^2$ integral around the region where the ap\-pro\-xi\-ma\-tion breaks down \cite{Frixione:1993yw}. This cut does then improve the coherent EPA calculation by decreasing the overestimation of the cross section. However, an energy independent cut cannot provide a good estimate of the cross section over all values of $E_\nu$. To illustrate our point and to quantify the errors induced by the EPA, we show on the left panel of \reffig{fig:comparison4PS_EPA} the ratio $\mathcal{R}$ of the trident cross section calculated using the EPA with an artificial cut at $Q^2_\text{cut}$, as performed in \cite{Magill:2016hgc}, to the full calculation used in this work as a function of the incoming neutrino energy:
%
 \begin{equation}
 \mathcal{R} = \frac{\sigma_{\rm EPA} (E_\nu) \vert_{Q_{\rm cut}}}{\sigma_{\rm 4PS} (E_\nu)}\,.
 \end{equation}
%
 In this plot we vary the artificial cut on $Q^2$ around the choice of \cite{Magill:2016hgc} (shown as the central dashed line) in two ways. First we reduce it by $20 \%$, and then increase it by a large factor, recovering the case with no $Q^2$ cut. From this, our conclusions about the validity of the approximation are confirmed, and it becomes evident that the trident coherent cross section is very sensitive to the choice of $Q^2_{\mathrm{cut}}$. In particular, the EPA with all the assumptions that lead to \refeq{eq:EPA_bad} and the absence of a $Q^2$ cut can lead to an overestimation of all trident channels, including the dimuon one. Once the cut is implemented, however, the approximation becomes better for the dimuon channel, but still unacceptable for the electron ones. It is also clear that an energy independent cut cannot give the correct cross section at all energies. This is particularly troublesome for detectors subjected to a neutrino flux covering a wide energy range such as the near detectors for DUNE and  MINOS or MINER$\nu$A. Moreover, \refeq{eq:EPA_bad} fails at low energies, and generally, overestimates the coherent cross sections by at least  200\%. At these energies, one must be wary of the additional approximations in \refeq{eq:EPA_bad} regarding the integration limits and the small $y$ limit.     

On the right panel of \reffig{fig:comparison4PS_EPA} we illustrate what happens in the diffractive regime, where the nucleon form factors impact the cross section at much larger values of $Q^2$ and have a slower fall-off. We see that the diffractive cross section is dramatically overestimated over the full range of $E_\nu$ considered and for any trident mode. The discrepancy is particularly important for $E_\nu \lesssim$ 5 GeV and larger than in the coherent regime by at least an order of magnitude\footnote{There are some differences in the treatment of the hadronic system between the EPA calculation in \cite{Magill:2016hgc} and the one presented here. However, these differences are of the order 10\% to 20\%. Note also that we do not implement any Pauli blocking when calculating $\mathcal{R}$ to avoid ambiguities over the choice of the range of $Q^2$.}. We also see that the cuts on $Q^2$ impact the EPA calculation much less dramatically, and that its use is unlikely to yield the correct result.

Given these problems with both coherent and diffractive cross section calculations due to the breakdown of the EPA for trident production, in what follows we will use the complete four-body calculation.

%%%%%%%%%%%%%%%%%%%%%%%%%%%%%%%%%%%%%%%%%%%%%%%%%%%%
\subsection{Coherent versus Diffractive Scattering in Trident Production}
\label{subsec:cohdiff}

%
\begin{figure}[t]
	\centering
	\includegraphics[width=\textwidth]{figs/Ratio_CDvsT.pdf}%
	\caption{\label{fig:RatioCDvsT} On the left (right) panel we show the
    ratio of the coherent (full lines) and the diffractive (dashed lines) contributions to the total trident cross section for an incoming flux of $\nu_\mu$($\nu_e$) as a function of $E_\nu$ for an 
    $^{40}$Ar target.}
\end{figure}
%

Let us  now comment on the significance of  the coherent and diffractive contributions to the total cross for the different trident channels. 
In Fig.\ \ref{fig:RatioCDvsT} we present the ratio of the coherent and the diffractive scattering cross sections to the total cross section for an $^{40}$Ar target for an incoming $\nu_\mu$ (left) and $\nu_e$ (right)  neutrino. We can see that the coherent regime dominates at all neutrino energies when there is an electron in the final-state, especially in the dielectron case. 
This can be explained by noting that the $Q^2$ necessary to create an electron pair is smaller than the one needed to create a muon; thus, coherent scattering is more likely to occur for this mode. Conversely, 
as one needs larger momentum transferred to produce a muon (either accompanied by an electron 
or another muon) the  diffractive regime becomes more likely in these modes, as we can explicitly 
see in Fig.\ \ref{fig:RatioCDvsT}. 
Because of this effect the diffractive contribution is $\lesssim$ 10\%, except for the 
dimuon channel where it can be between $30$ and $40$\% in most of the energy region.
Furthermore, when we compare the two incoming types of neutrinos, we see that for an incoming $\nu_\mu$ the diffractive contribution is larger than the coherent one in the range $0.3\ {\rm GeV}\lesssim E_\nu \lesssim 0.8$ GeV, while for an incoming $\nu_e$ this never happens. 
This difference can be explained by the fact that CC and NC contributions  are simultaneously present for the scattering of an initial $\nu_\mu$ creating a muon pair, whereas
for an initial $\nu_e$ creating a muon pair, we will only have the NC contribution, see Table \ref{tab:tridentmodes}.

An important difference between the coherent and diffractive regimes will be in their hadronic signatures in the detector. Neutrino trident production is usually associated with zero hadronic energy at the vertex, a feature that proved very useful in reducing backgrounds in previous measurements. Whilst this is a natural assumption for the coherent regime, it need not be the case in the diffractive one. In fact, in the latter it is likely that the struck nucleon is ejected from the nucleus in a significant fraction of events with $Q$ exceeding the nuclear binding energy
%
\footnote{The peak of our diffractive $Q^2$ distributions happens at around $Q \approx 300$ MeV, much beyond the typical binding energy for Ar (see \refapp{app:distributions}). Without Pauli suppression, however, we expect this value to drop.}. Since the dominant diffractive contribution comes from scattering on protons, these could then be visible in the detector if their energies are above threshold. On the other hand, the struck nucleon is subject to many nuclear effects which may significantly affect the hadronic signature, such as interactions of the struck nucleon in the nuclear medium as well as reabsorption. Our calculation of Pauli blocking, for example, shows large suppressions ($\sim 50\%$) precisely in the low $Q^2$ region, usually associated with no hadronic activity. This then raises the question of how well one can predict the hadronic signatures of diffractive events given the difficulty in modelling the nuclear environment. We therefore do not commit to an estimate of the number of diffractive events that would have a coherent-like hadronic signature, but merely point out that this might introduce additional uncertainties in the calculation, especially in the $\mu^+ \mu^-$ channel where the diffractive contribution is comparable to the coherent one. Finally, from now on we will refer to the number of trident events with no hadronic activity as coherent-like, where this number can range from coherent only to coherent plus all diffractive events. 



%%%%%%%%%%%%%%%%%%%%%%%%%%%%%%%%%%%%%%%%%%%%%%%%%%%%
\section{Trident Events in LAr Detectors}
\label{sec:LAr}

In this section we calculate the total number of expected  trident events for some present and future LAr detectors with different fiducial masses, total exposures and beamlines. In Table~\ref{tab:LAr} we specify the values used for each set-up and in Fig.~\ref{fig:LAr} we show the total production cross section for each neutrino trident mode of Table~\ref{tab:tridentmodes}
as well as the neutrino fluxes as a function of $E_\nu$ at the position of each experiment.  

%%%%%%%%%%%%%%%%%%%%%%%%%%%%%%%%%%%%%%%%%%%%%%%%%%%%
\subsection{Event Rates}
\label{subsec:rates}

The total number of trident events, $N^{\text{\Neptune}}_{\rm X}$, expected for a given trident mode at any detector is written as  
\begin{eqnarray}
\label{eq:nevents}
N^{\text{\Neptune}}_{\rm X}={\rm Norm}\times\int dE_\nu \, \sigma_{\nu {\rm X}}(E_\nu) \frac{d\phi_{\nu}(E_\nu)}{dE_\nu}\epsilon(E_\nu)\,,
\end{eqnarray}
where $\sigma_{\nu {\rm X}}$ can be the trident total (${\rm X}={\cal N}$), coherent ($\mathrm{X=c}$) or diffractive ($\mathrm{X=d}$) cross sections 
for a given mode, $\phi_{\nu}$ is the flux of the incoming neutrino and $\epsilon(E_\nu)$ is the efficiency of detection of the charged leptons. In the calculations of this section, we assume an efficiency of $100\%$\footnote{See \refsec{subsec:kine} for a discussion on the detection efficiencies for trident events and backgrounds.}.
%
The normalization is calculated as 
 %
$${\rm Norm}= {\rm{Exposure}}~[{\rm{POT}}] \times \frac{{\rm{Fiducial~Detector~Mass}\times N_A}}{m_{\rm T}} \left[{\rm{target~particles}}\right],$$
%%
where $m_{\rm T}$ is the molar mass of the target particle and $N_A$ is Avogadro's number.
%
Two features of the cross sections are important for the event rate calculation: 
threshold effects, especially for channels involving muons in the final-state,
and cross section's growth with energy. In particular, we expect higher trident event rates for experiments with higher energy neutrino beams. 

We start our study with the three detectors of the SBN program, one of which, $\mu$BooNE, is already installed and taking data at Fermilab. These three LAr time projection chamber detectors are located along the Booster Neutrino Beam line which is by now a well-understood source, having the focus of active research for over 15 years. 
%
Although the number of trident events expected in these detectors is rather low, they may offer one of the first opportunities to study trident events in LAr, as well as to better understand their backgrounds in this medium and to devise improved analysis techniques.
%
After that we study the proposed near detector for DUNE. This turns out to be the most important LAr detector for trident production since it will provide the highest number of events in both neutrino and antineutrino modes. 
%
Finally, having in mind the novel flavour composition of neutrino beams from muon facilities, we investigate trident rates at a 100~t LAr detector for the $\nu$STORM project. This last facility could offer a very well understood neutrino beam with as many electron neutrinos as muon antineutrinos from muon decays, creating new possibilities for trident scattering measurements.
%%
%%
\begin{table}[t]
\begin{center}
\scalebox{0.9}{
\begin{tabular}{|c|c|c|c|c|}
%\hline
\hline\hline
\bf Experiment& \bf Baseline (m) & \bf Total Exposure (POT) & \bf Fiducial Mass (t) & \bf $\mathbf{E_\nu}$ (GeV)\\\hline\hline
SBND & 110 & $6.6\times 10^{20}$ & 112 & $0-3$\\\hline
$\mu$BooNE & 470 & $1.32\times 10^{21}$ & 89& $0-3$\\\hline
ICARUS & 600 & $6.6\times 10^{20}$ & 476 & $0-3$\\\hline
DUNE & 574  & $12.81~(12.81)\times 10^{21}$& 50 & $0-40$ \\\hline
$\nu$STORM & 50  & $10^{21}$ & 100& $0-6$\\\hline\hline
\end{tabular}}
\end{center}
\caption{\label{tab:LAr} Summary of the LAr detectors set-up and values assumed in our calculations.
The POT numbers are given for a neutrino (antineutrino) beam.}
\end{table}

\begin{figure}[t]
\centering
\includegraphics[width=1.\textwidth]{figs/LAr_f+xsec.pdf}
\caption{\label{fig:LAr}Energy distribution of the neutrino fluxes at the position of 
the LAr detectors DUNE (top left, \cite{DUNE:flux}), SBND (top right,\cite{SBNproposal}) and $\nu$STORM (bottom left, \cite{nuSTORM2017}) and of the cross sections for the various trident modes (bottom right). The fluxes at $\mu$BooNE and ICARUS are similar to the one shown for SBND when normalized over distance.}
\end{figure}

%%%%%%%%%%%%%%%%%%%%%%%%%%%%%%%%%%%%%%%%%%%%%%%%%%%%
\subsubsection{The SBN Program}
\label{subsubsec:SBND}

The SBN Program at Fermilab is a joint endeavour by three collaborations ICARUS, $\mu$BooNE and 
SBND to perform searches for eV-sterile neutrinos and study neutrino-Ar cross sections \cite{SBNproposal}. As can be seen in Tab.~\ref{tab:LAr}, SBND has the shortest baseline (110 m) and therefore the largest neutrino fluxes (shown in Fig.~\ref{fig:LAr} and taken from Fig. 3 of \cite{SBNproposal}). The largest detector, ICARUS, is also the one with the longest baseline (600 m) and consequently subject to the lowest neutrino fluxes.
%
The ratio between the fluxes at the different detectors are  $\phi_{\mu\rm{BooNE}}/\phi_{\rm{SBND}}=5$\% and $\phi_{\rm{ICARUS}}/\phi_{\rm{SBND}}=3$\%.
%
The neutrino beam composition is about 93\% of $\nu_\mu$,  6\% of $\overline\nu_\mu$ and  
$1\%$ of $\nu_e+\overline{\nu}_e$. 

Considering the difference in fluxes and the total number of targets in each of these 
detectors, one can estimate the following ratios of trident events: 
${N^\text{\Neptune}_{\mu\rm{BooNE}}}/{N^\text{\Neptune}_{\rm{SBND}}}\sim 8$\% and ${N^\text{\Neptune}_{\rm{ICARUS}}}/{N^\text{\Neptune}_{\rm{SBND}}}\sim 10$\%. Unfortunately, 
since the fluxes are peaked at a rather low energy ($E_\nu \lesssim 1$ GeV), where the trident  
cross sections are still quite small ($\lesssim 10^{-42}$ cm$^2$) we expect very few 
trident events produced.
%
The exact number of trident events for those detectors according to our calculations is 
presented in Tab.~\ref{tab:LArrates}. For each trident channel the first (second) row
shows the number of coherent (diffractive) events. As expected, less than a total 
of 20 events across all channels can be detected by SBND, and a negligible rate of events is expected at $\mu$BooNE and ICARUS. 

%%%%%%%%%%%%%%%%%%%%%%%%%%%%%%%%%%%%%%%%%%%%%%%%%%%%
\subsubsection{DUNE Near Detector}
\label{subsubsec:DUNE}

\begin{table}[t]
\begin{center}
\scalebox{0.9}{
\begin{tabular}{|cccccc|}
\hline\hline
		\bf Channel & \bf SBND& \bf $\mu$BooNE & \bf ICARUS & \bf DUNE ND &\bf  $\nu$STORM ND \\ \hline \hline
		$\nu_\mu\to\nu_e e^+ \mu^-$& $10$ &$0.7$ &$1$ &$2844 ~ (235)$ & $159$ \\
        &$1$ &$0.08$ &$0.1$ &$369 ~ (33)$& $18$\\\hline
        $\overline\nu_\mu\to\overline\nu_e e^- \mu^+$&$0.4$ &$0.02$ &$0.04$ &$122~(2051)$ & $23$\\
        &$0.04$ &$0.003$ &$0.004$ &$16~(262)$ & $3$\\\hline
	$\nu_e\to\nu_\mu e^- \mu^+$& $0.05$ &$0.003$ &$0.004$ &$22~(7)$ & $9$\\
        &$0.008$ &$0.0005$ &$0.0008$ &$5~(1)$ & $2$\\\hline
	$\overline\nu_e\to\overline\nu_\mu e^+ \mu^-$& $0.005$ &$0.0003$ &$0.0005$ &$5~(14)$ & $-$\\
    &$0.001$ &$0.0001$ &$0.0001$ &$1~(3)$ & $-$\\\hline
    \hline\hline
    {$\rm{Total} \ e^\pm \mu^\mp$}& $10$ &$0.7$ &$1$ &$2993~(2307)$ & $191$ \\
    &$1$ &$0.1$ &$0.1$ &$391~(299)$ & $23$\\\hline
    \hline
		$\nu_\mu\to\nu_\mu e^+ e^-$& $6$ &$0.4$ &$0.7$ &$913~(58)$ & $73$ \\
        &$0.2$ &$0.04$ &$0.02$ &$57~(5)$ & $3$\\\hline
        $\overline\nu_\mu\to\overline\nu_\mu e^- e^+$& $0.2$ &$0.01$ &$0.02$ &$34~(695)$ & $9$\\
        &$0.01$ &$0.001$ &$0.002$ &$2~(41)$ & $0.5$\\\hline
	$\nu_e\to\nu_e e^- e^+$&$0.2$ &$0.01$ &$0.02$ &$50~(13)$ & $32$ \\
    &$0.01$ &$0.001$ &$0.002$ &$4~(1)$ & $2$\\\hline
	$\overline\nu_e\to\overline\nu_e e^+ e^-$&$0.02$ &$0.001$ &$0.002$ &$10~(34)$ & $-$ \\
    &$0.0009$ &$0.0001$ &$0.0002$ &$1~(2)$ & $-$\\\hline
    \hline\hline
    ${\rm{Total}}\  e^+ e^-$& $6$ &$0.4$ &$0.7$ &$1007~(800)$ & $114$\\
    &$0.2$ &$0.0$ &$0.02$ &$64~(49)$ & $6$\\\hline
    \hline    
		$\nu_\mu\to\nu_\mu \mu^+ \mu^-$& $0.4$ &$0.03$ &$0.04$ &$271~(32)$ & $9$ \\
        &$0.3$ &$0.03$ &$0.04$ &$135~(14)$ & $5$\\\hline
        $\overline\nu_\mu\to\overline\nu_\mu \mu^- \mu^+$& $0.01$ &$0.001$ &$0.001$ &$14~(177)$ & $2$\\
        &$0.01$ &$0.0009$ &$0.001$ &$7~(93)$ & $1$\\\hline
 $\nu_e\to\nu_e \mu^+ \mu^-$    &$0.002$ &$0.0001$ &$0.0001$ &$1~(0.5)$ & $0.4$\\  
 &$0.001$ &$0.0001$ &$0.0001$ &$0.5~(0.2)$ & $0.2$\\\hline
        $\overline\nu_e\to\overline\nu_e \mu^+ \mu^-$&$0.0002$ &$0.0000$ &$0.0000$ &$0.3~(0.9)$ & $-$\\
        &$0.0001$ &$0.0000$ &$0.0000$ &$0.1~(0.3)$ & $-$\\\hline
        \hline\hline
    ${\rm{Total}} \ \mu^+ \mu^-$ &$0.4$ &$0.0$ &$0.0$ &$286~(210)$ & $11$ \\
    &$0.3$ &$0.0$ &$0.0$ &$143~(108)$ & $6$\\
    \hline\hline
\end{tabular}}
\end{center}
\caption{\label{tab:LArrates}Total number of \textbf{coherent} (top row) and \textbf{diffractive} (bottom row) trident events expected at different LAr experiments for a given channel.
The numbers in parentheses are for the antineutrino running mode, when present. These calculations  
considered a detector efficiency of 100\%. }
\end{table}

The DUNE experiment will operate with neutrino as well as antineutrino LBNF beams produced by 
directing a 1.2 MW beam of protons onto a fixed target \cite{Acciarri:2016ooe,DUNECDRvolII}. 
The design of the near detector is not finalised, but the current designs favour a mixed technology  detector combining a LAr TPC with a larger tracker module.  In this work, we will assume that DUNE ND is a LAr detector located at $574$ m from the target with a fiducial mass of 50~t \cite{WeberTalk}. As the trident event rate scales with the density of the target, any tracker module will not significantly influence the total event rate, and does not feature in our estimates; although, its presence is assumed to improve reconstruction of final-state muons. Our estimates can be easily scaled for the final design by using \refeq{eq:nevents}.

For the first 6 years of data taking (3 years in the neutrino plus 3 years in the antineutrino 
mode) the collaboration expects $1.83\times 10^{21}$~POT/year with  a plan to upgrade the beam after the 6th year for 2 extra years in each beam mode  with double exposure, making a total of $1.83 \times(3+2\times2)\times 10^{21}~{\rm{POT}}$ for each mode \cite{DUNE:exposure}. We will 
assume the total 10-year exposure in our calculations.
%
. as the relevant fluxes at the DUNE ND location (see Fig.~\ref{fig:LAr}). The beam composition of the neutrino (antineutrino) beam is about 96\% $\nu_\mu$ ($\overline\nu_\mu$), 4\%  $\overline\nu_\mu$ ($\nu_\mu$) and 1\% $\nu_e+\overline\nu_e$.
 
The number of trident events for DUNE ND can be found in Tab.~\ref{tab:LArrates}. 
The numbers in parentheses correspond to antineutrino beam mode.
Note that although the trident cross sections are the same 
for neutrinos and antineutrinos, the fluxes are a bit lower for the antineutrino beam, as a consequence we predict a lower event rate for this beam\footnote{A similar difference will apply to the processes constituting the background to the trident process, although there is an additional suppression in many channels due to the lower antineutrino cross sections.}.
%
Due to the much higher energy and wider energy range of the neutrino fluxes at DUNE ND, as compared to the SBN detectors, DUNE can observe a considerable number of trident events, about 300 times the number of trident events expected for SBND just in the neutrino mode. Moreover, the subdominant component of 
each beam mode will also contribute to the signal. For example, we expect to observe $2051$ trident events in the $\overline{\nu}_\mu\to\overline{\nu}_e e^- \mu^+$ channel in the antineutrino mode. However, we also expect 
$235$ events in the $\nu_\mu\to\nu_e e^+ \mu^-$ channel produced by 
the subdominant component of $\nu_\mu$ in the antineutrino beam.
%
We have considered 100\% detection efficiency here, however, we will see in Sec.~\ref{subsec:bck} that after implementing hadronic vetos, detector thresholds and kinematical cuts to substantially reduce the background we expect an efficiency of about 47\%-65\% on coherent tridents, depending on the channel (see Tab.~\ref{tab:DUNE_ND_NU_BG}).

The mixed flavour trident channel is the one with the highest statistics (more than 6000 events adding 
neutrino and antineutrino beam modes), 11\% of which are produced by diffractive scattering. The dielectron channel comes next with a total of a bit more than 1900 events, 5\%  of which are produced by diffractive scattering. Although the  dimuon channel is the less copious one, with only about 
750 events produced, almost 34\% of these events are produced by a diffractive process.
This can be understood by recalling our discussions in Sec.~\ref{subsec:cohdiff}.

Finally, we note that a dedicated high-energy run at DUNE has been mooted, to be undertaken after the full period of data collecting for the oscillation analysis. Thanks to the higher energies of the beam, this has the potential to see a significant number of neutrino tridents, provided it can collect enough POTs.  

%%%%%%%%%%%%%%%%%%%%%%%%%%%%%%%%%%%%%%%%%%%%%%%%%%%%
\subsubsection{$\nu$STORM}
\label{subsubsec:nuSTORM}
In this section we study the trident rates for a possible LAr detector for the proposed 
$\nu$STORM experiment \cite{Soler:2015ada,nuSTORM2017}. The $\nu$STORM facility 
is based on a neutrino factory-like design and has the goal to search for sterile neutrinos and study neutrino nucleus cross sections \cite{Adey:2014rfv}. Although this proposal is in its early days, $\nu$STORM has the potential to make cross section measurements with unprecedented precision. In its current design, $120$-GeV protons are used to produce pions from a fixed target with the pions subsequently decaying into muons and neutrinos. The muons are captured in a storage ring and during repeated passes around the ring they decay to produce neutrinos.
%
Consequently, the storage ring is an intense source of three types of neutrino
flavours: $\nu_\mu$ from $\pi^+$ and $K^+$ decays, which will be more than $99\%$ of the total flux, $\nu_e$ and $\overline\nu_\mu$ from recirculated muon decays which will comprise less than $1\%$ of the total flux. An important point, however, is that the neutrinos coming from the pion and kaon decays can be separated by event timing from the ones produced by the stored muons. This distinction allows the $\nu_\mu$ flux to be studied almost independently from the $\overline{\nu}_\mu$ and $\nu_e$ flux. In addition, it implies after the initial flash of meson-derived events, that the flux consists of as many electron neutrinos as muon antineutrinos. We will assume a LAr detector for $\nu$STORM at a baseline of 50\,m with 100\,t of fiducial mass with an exposure of $10^{21}$ POT. The neutrino fluxes, assuming 
a central $\mu^+$ momentum of $3.8$~GeV/c in the storage ring, are taken from Ref.~\cite{nuSTORM2017} and are 
shown in Fig.~\ref{fig:LAr}.

In Tab.~\ref{tab:LArrates}, we show the results of our calculations for $\nu$STORM. 
More than $97\%$ of the events from the incoming $\nu_\mu$ are from pion decays and only less 
than $3\%$ from kaon decays. Since we only consider the decay of mesons with positive charges and we expect neutral and wrong charge contamination to be small, we do not have trident events from incoming $\overline\nu_e$.
%
The total number of mixed flavour, dielectron and dimuon channel events is, respectively,
214, 120 and 17, much less than what can be achieved at the larger neutrino energies available at the DUNE ND. The novel flavour structure of the beam does enhance the contribution of $\nu_e$ induced tridents with respect to the $\pbar{\nu}_\mu$ ones, but this contribution only becomes dominant for the $e^+e^-$ tridents in the muon decay events. Finally, we emphasize that the experimental design parameters for $\nu$STORM are far from definite. Increasing the energy of stored muons and the size of the detector are both viable options which could significantly enhance the rates we present.

%
%%%%%%%%%%%%%%%%%%%%%%%%%%%%%%%%%%%%%%%%%%%%%%%%%%%%
\subsection{Kinematical Distributions at DUNE ND}
%
\label{subsec:kine}

In this section we explore the trident signal in more detail, showing some relevant kinematical distributions for coherent and diffractive events. For concreteness, and due to its large number of events, we choose to focus on the DUNE ND, only commenting slightly on the signal at the lower energies of SBN and $\nu$STORM. The observables we calculate are the invariant mass of the charged leptons $m^2_{\ell^+ \ell^-}$, their separation angle $\Delta \theta$ and their individual energies $E_\pm$. The flux convolved distributions of these observables are shown for the DUNE ND in neutrino mode in \reffig{fig:DUNE_ND_dist}. In these plots, we sum all trident channels with a given undistinguishable final-state proportionally to their rates, although $\nu_\mu$ initiated processes always dominate. The coherent and diffractive contributions are shown separately and on the same axes, but we do not worry about their relative normalization. Other potentially interesting quantities are the angle between the cone formed by the two charged leptons and the beam, $\alpha_C$, and the angle of each charged lepton with respect to the beam direction, $\theta_\pm$.  These additional observables are explored in \refapp{app:distributions}. We also report the distributions of the momentum transfer to the hadronic system, $Q^2$. Although this is not a directly measurable quantity, it is a strong discriminant between the coherent and diffractive processes. We do not present the antineutrino distributions here, but they are qualitatively similar.

Perhaps one of the most valuable tools for background suppression in the measurement of the $\mu^+\mu^-$ trident signal at CHARM~II, CCFR and NuTeV \cite{Geiregat:1990gz,Mishra:1991bv,Adams:1998yf} was the smallness of the invariant mass $m^2_{\ell^+ \ell^-}$. This feature, shown here on the top row of \reffig{fig:DUNE_ND_dist}, is also present at lower energies, where the distributions become even more peaked at lower values; although, the diffractive events tend to be have a more uniform distribution in this variable. This is also true for the angular separation $\Delta \theta$, where coherent dimuon tridents tends to be quite collimated, with $90\%$ of events having $\Delta \theta < 20^\circ$, whilst diffractive ones are less so, with only $47\%$ of events surviving the cut. This difference is much less pronounced for mixed and dielectron channels, where only half of our coherent events obey $\Delta \theta < 20^\circ$, when $37\%$ of diffractive events do so.

%
\begin{figure}[H]
\centering
\includegraphics[width=\textwidth]{figs/DUNE_nu_3horn_mll_theta_E.pdf}
\caption{Flux convolved neutrino trident production distributions for DUNE ND in neutrino mode. In purple we show the coherent contribution in $^{40}$Ar and in blue the diffractive contribution from protons as targets only (including Pauli blocking). The coherent and diffractive distributions are normalized independently. The relative importance of each contribution as a function of $E_\nu$
can be seen in Fig.~\ref{fig:RatioCDvsT}.
%
\label{fig:DUNE_ND_dist}}
\end{figure}
%


An interesting feature of same flavour tridents induced by a neutrino (antineutrino) is that the negative (positive) charged lepton tends to be slightly more energetic than its counterpart, whilst for mixed tridents muons tend to carry away most of the energy. These considerations are also reflected in the angular distributions. The most energetic particle is also the more forward one. For instance, in mixed neutrino induced tridents, $\sim 80 \%$ of the $\mu^-$ are expected to be within $10^\circ$ of the beam direction, whilst only $\sim 35 \%$ of their $e^+$ counterparts do so (see \refapp{app:distributions} for additional distributions).


Finally, we mention that detection thresholds can also be important for trident channels with electrons in the final-state. Assuming, for example, a detection threshold for muons and electromagnetic (EM) showers of 30 MeV in LAr, we end up with efficiencies of (99\%, 71\%, 77\%, 86\%) for ($\mu^+ \mu^-$, $e^+ e^-$, $e^+ \mu^-$, $e^- \mu^+$) coherent tridents. These efficiencies become (96\%, 91\%, 93\%, 96\%) for diffractive tridents, dropping for $\mu^+\mu^-$ and increasing for all others. For comparison, at the lower neutrino energies of SBND and assuming the same detection thresholds, the efficiencies for coherent and diffractive tridents are slightly lower, (97\%, 57\%, 67\%, 77\%) and (90\%, 81\%, 85\%, 90\%) respectively.

%%%%%%%%%%%%%%%%%%%%%%%%%%%%%%%%%%%%%%%%%%%%%%%%%%%%
\subsection{Background Estimates for Neutrino Trident in LAr}
\label{subsec:bck}

The study of any rare process is a struggle against both systematic uncertainties in the event rates and unavoidable background processes. True dilepton signatures are naturally rare in neutrino scattering experiments, but with modest rates of particle misidentification a non-trivial background arises. In this section we estimate the background to trident processes in LAr and its impact on the trident measurement. We perform our analysis only for DUNE ND, in neutrino and antineutrino mode, but our results are expected to be broadly applicable to other LAr detectors. We have generated a sample of $1.1 \times 10^6$ background events using GENIE \cite{Andreopoulos2009} for incident electron and muon flavour neutrinos and antineutrinos. It is worth noting, however, that this event sample will in fact be smaller than the total number of neutrino interactions expected in the DUNE ND. 
%
Our goal, therefore, will be to demonstrate that with modest analysis cuts background levels can be suppressed significantly such that they become comparable to or smaller than the signals we are looking for. In the absence of events that satisfy our background definition, we argue that the frequency of that type of event is less than one in $1.1\times 10^6$ interactions of the corresponding initial neutrino.  

To account for misreconstruction in the detector, we implement resolutions as a gaussian smear around the true MC energies and angles. We assume relative energy resolutions as $\sigma/E = 15\%/\sqrt{E}$ for $e/\gamma$ showers and protons, and $6\%/\sqrt{E}$ for charged pions and muons. Angular resolutions are assumed to be $1^\circ$ for all particles (proton angles are never smeared in our analysis). The detection thresholds are a crucial part of the analysis, since for many channels one ends up with very soft electrons. We take thresholds to be $30$ MeV for muons and $e/\gamma$ showers kinetic energy, $21$ MeV for protons and $100$ MeV for $\pi^{\pm}$ \cite{DUNECDRvolII}.

%%%%%%%%%%%%%%%%%%%%%%%%%%%%%%%%%%%%%%%%%%%%%%%%%%%%
\subsubsection{Background Candidates}
\label{subsubsec:misID}
We focus on three final-state charged lepton combinations: $\mu^+\mu^-$, $\mu^\pm e^\mp$ and $e^+e^-$. Genuine production of these states is possible in background processes, but usually rare, deriving from meson resonances or other prompt decays. The majority of the background is expected to be from particle misidentification (misID). We assume that protons can always be identified above threshold and that neutrons leave no detectable signature in the detector. In addition, we require no charge ID capabilities from the detector and assume that the interaction vertex can always be reconstructed. Under these assumptions, we have incorporated three misidentifications which will affect our analysis, and give our naive estimates for their rates in Tab.~\ref{tab:misIDlist}. Any other particle pairs are assumed to be distinguishable from each other when needed.
%
\renewcommand{\arraystretch}{1.2}
\begin{table}[t]
\centering 
\begin{tabular}{|c c|}
\hline\hline
\bf misID & \bf Rate \\
\hline\hline
%
$\gamma$ as $e^\pm$ & 0.05 \\
\hline
\multirow{2}{*}{$\gamma$ as $e^+e^-$} & 0.1 (w/ vertex)  \\
%\cline{2-2}
 & 1 (no vertex + overlapping)  \\
\hline
$\pi^\pm$ as $\mu^\pm$ & 0.1 \\
\hline\hline
\end{tabular}
\caption{\label{tab:misIDlist} Assumed misID rates for various particles in a LAr detector. We take these values to be constant in energy.}
\end{table}

The requirement of no hadronic activity helps constrain the possible background processes, but one is still left with significant events with invisible hadronic activity and other coherent neutrino-nucleus scatterings. These are then reduced by choosing appropriate cuts on physical observables, exploring the discrepancies between our signal and the background. In our GENIE analysis, we include all events that have final-states identical to trident, or that could be interpreted as a trident final-state considering our proposed misID scenarios. Our dominant sources of background for $\mu^+ \mu^-$ tridents are $\nu_\mu$-initiated charged-current events with an additional charged pion in the final-state ($\nu_\mu$CC$1\pi^\pm$). For $e^+e^-$ tridents, the most important processes are neutral current scattering with a $\pi^0$ (NC$\pi^0$), while for mixed $e^\pm \mu^\mp$ tridents, the $\nu_\mu$-initiated charged-current events with a final-state $\pi^0$ (CC$\pi^0$) dominate the backgrounds. In each case, the pion is misidentified to mimic the true trident final-state. Other relevant topologies include charm production, CC$\gamma$ and $\nu_e$CC$\pi^\pm$. For a detailed discussion of these backgrounds processes we refer the reader to \refapp{app:backgrounds}.

%%%%%%%%%%%%%%%%%%%%%%%%%%%%%%%%%%%%%%%%%%%%%%%%%%%%
\subsubsection{\label{sec:DUNE_bg_rates}Estimates for the DUNE ND}

In this section we provide estimates for the total background for each trident final-state for the DUNE ND. The number of total inclusive CC interactions in the 50 t detector due to neutrinos of all flavours is calculated to be $5.18 \times 10^8$. We scale our background event numbers to match this, and argue that one has to reach suppressions of order $10^{-6} - 10^{-5}$ to have a chance to observe trident events. Whenever our cuts remove all background events from our sample, we assume the true background rate is one event per $1.1\times10^6$ $\nu$ interactions and scale it to the appropriate number of events in the ND, applying the misID rate whenever relevant. Within our framework, this provides a conservative estimate as the true background is expected to be smaller.

Our estimates are shown in \reftab{tab:DUNE_ND_NU_BG}. We start with the total number of background candidates $\rm N_B^{\mathrm{misID}}$, using only the naive misID rates shown in \reftab{tab:misIDlist}. These are much larger than the trident rates we expect, by at least 2 orders of magnitude. Next, we veto any hadronic activity at the interaction vertex, obtaining $\rm N_B^{\mathrm{had}}$. We emphasize that this veto also affects the diffractive tridents in a non-trivial way, and therefore we remain agnostic about the hadronic signature of these. 
%
Finally, one can look at the kinematical distributions of coherent trident in \refsec{subsec:kine} and try to estimate optimal one dimensional cuts for the DUNE ND based on the kinematics of the final-state charged leptons. This is a simple way to explore the striking differences between the peaked nature of our signal and the smoother background. In a real experimental setting it is desirable to have optimization methods for isolating signal from background, preferably with a multivariate analyses. However, even in our simple analysis, cutting on the small angles to the beamline and the low invariant masses of our trident signal can achieve the desired background suppressions. For the $\mu^+\mu^-$ tridents we show the effect of our cuts in \reffig{fig:bkg_flow}. The cuts are defined to be $m^2_{\mu^+ \mu^-} < 0.2 \ \mathrm{GeV}^2$, $\Delta \theta < 20^\circ$, $\theta_\pm < 15^\circ$. The kinematics is very similar in the other trident channels, with slightly less forward distributions for electrons. For the $e^+ e^-$ channel we take  $m^2_{e^+ e^-} < 0.1 \ \mathrm{GeV}^2$, $\Delta \theta < 40^\circ$ and $\theta_\pm < 20^\circ$. 
%
The asymmetry between the positive and negative charged leptons is visible in the distributions, where the latter tends to be more energetic. This feature was not explored in our cuts, as it is not significant enough to further improve background discrimination. In the mixed flavour tridents, however, one sees a much more pronounced asymmetry. The muon tends to carry most of the energy and be more forward than the electron, which can make the search for this channel more challenging due to the softness of the electron in the high energy event. Nevertheless, the low invariant masses and forward profiles can still serve as powerful tool for background discrimination, provided the event can be well reconstructed. We assume that is the case here and use the following cuts on the background:  $m^2_{e^\pm \mu^\mp} < 0.1 \  \mathrm{GeV}^2$, $\Delta \theta < 20^\circ$, $\theta_e < 40^\circ$ and $\theta_\mu < 20^\circ$. When performing kinematical cuts, we also include the effects of detection thresholds after smearing. For a discussion on the impact of these thresholds on the trident signal see \refsec{subsec:kine}. 
%

The resulting signal efficiencies due to our cuts and thresholds are shown in the last two columns of \reftab{tab:DUNE_ND_NU_BG}. One can see that these are all $ \approx 50\%$ or greater for our coherent samples, whilst all background numbers remain much below the trident signal. The diffractive samples are also somewhat more affected by our cuts than the coherent ones. If one is worried about the contamination of coherent events by diffractive ones, then the kinematics of the charged leptons alone can help reduce this, independently of the hadronic energy deposition of the events. For instance, in the case where all $\mu^+\mu^-$ diffractive events appear with no hadronic signature, then after our cuts the diffractive contribution is reduced from $41\%$ to $15\%$ of the total trident signal. This reduction is, however, also subject to large uncertainties coming from nuclear effects. In summary, the set of results above are encouraging, suggesting that the signal of coherent-like trident production is sufficiently unique to allow for its search at near detectors despite naively large backgrounds. 

%
\begin{table}[t]	
	\begin{center}
    \resizebox{\textwidth}{!}{
		\begin{tabular}{|clllll|}
		\hline \hline
		\bf Channel& $\bf N^{\mathrm{misID}}_{\mathrm{B}} / N_{\mathrm{CC}}$        & $ \bf N^{\mathrm{had}}_{\mathrm{B}} / N_{\mathrm{CC}}$ & $\bf N^{\mathrm{kin}}_{\mathrm{B}} / N_{\mathrm{CC}}$ & ${\epsilon_{\mathrm{sig}}^{\mathrm{coh}}}$ &
${\epsilon_{\mathrm{sig}}^{\mathrm{dif}}}$ \footnotemark\\	\hline \hline
		$e^{\pm}\mu^{\mp}$ & $1.67\  (1.62) \times 10^{-4}$ & $2.68\  (4.31) \times 10^{-5}$ & $4.40 \ (3.17) \times 10^{-7}$ & $ 0.61 \ (0.61)$ & $ 0.39 \ (0.39)$\\
		$e^+e^-$ & $2.83 \ (4.19)\times 10^{-4}$ & $1.30 \ (2.41) \times 10^{-4}$ &  $6.54 \ (14.1) \times 10^{-6}$ & $ 0.48 \ (0.47)$ & $ 0.21 \ (0.21)$\\
		$\mu^+\mu^-$ & $2.66 \ (2.73)\times 10^{-3}$ & $10.4 \ (9.75)\times 10^{-4}$ & $3.36 \ (3.10)\times10^{-8}$ & $0.66 \ (0.67)$ & $0.17 \ (0.16)$\\\hline\hline
		\end{tabular}
    }
	\end{center}
	\caption{\label{tab:DUNE_ND_NU_BG} Reduction of backgrounds at the DUNE ND in neutrino (antineutrino) mode and its impact on the signal for each distinguishable trident final-state. $\mathbf{N^{\mathrm{misID}}_{\mathrm{B}}}$ stands for total backgrounds to trident after only applying misID rates, $\mathbf{N^{\mathrm{had}}_{\mathrm{B}}}$ are the backgrounds after the hadronic veto, and $\mathbf{N^{\mathrm{kin}}_{\mathrm{B}}}$ reduce the latter with detection thresholds and kinematical cuts (see text for the cuts chosen). These quantities are normalized to the total number of CC interactions in the ND $\mathbf{N_{\mathrm{CC}}}$ (flavour inclusive). We also show the impact of our detection thresholds and kinematical cuts on the trident signal via efficiencies for coherent only ($\epsilon_{\mathrm{sig}}^{\mathrm{coh}}$) and diffractive only samples ($\epsilon_{\mathrm{sig}}^{\mathrm{dif}}$). We do not cut on the hadronic activity of diffractive events.}
\end{table}

\footnotetext{Despite the fact that many diffractive events will likely deposit hadronic energy in the detector, we quote the efficiency of our cuts on diffractive events with no assumptions on their hadronic signature.}

\begin{figure}[t]
\centering
\includegraphics[width = 0.75\textwidth]{figs/SigvsBkg.pdf}
 \caption{Signal and background distributions in invariant mass. The total background events (blue) include the misID rates in table \reftab{tab:misIDlist}. We apply consecutive cuts on the background, starting with cuts on the separation angle $\Delta \theta$ (red), both charged lepton angles to the beamline ($\theta_+$ and $\theta_-$) (orange) and the invariant mass $m^2_{\mu^+ \mu^-}$ . We show the signal samples before and after all the cuts in dashed black and filled black, respectively. \label{fig:bkg_flow}}
\end{figure}
%
Finally, we comment on some of the limitations of our analysis. The low rate of trident events calls for a more careful evaluation of other subdominant processes that could be easily be overlooked. For channels involving electrons, it is possible that de-excitation photons and internal bremsstrahlung become a source of background, as these also produce very soft EM showers, none of which are implemented in GENIE. The question of reconstruction of these soft EM showers, accompanied either by a high energy muon or by another soft EM shower also would have to be addressed, especially in the latter case where a trigger for these soft events would have to be in place. A more complete analysis is also needed for treating the decay products of charged pions and muons produced in neutrino interactions, as well as rare meson decay channels (like the Dalitz decay of neutral pions $\pi^0 \to \gamma e^+ e^-$). Cosmic ray events are not expected to be a problem due to the requirement of a vertex and a correlation with the beam for trident events. Perhaps even more exotic processes, such as the production of three final-state charged leptons ($\nu_{\alpha} (\overline{\nu}_{\alpha}) + \mathcal{H} \to  \ell_\alpha^- (\ell_\alpha^+) + \ell_\beta^+ + \ell_{\beta}^- + \mathcal{H^\prime}$), can also become relevant. For instance, radiative trimuon production \cite{Smith:1977nx} can potentially serve as a background to dimuon tridents if one of the muons is undetected. Similarly, $\mu e e$ production would fake a dielectron (mixed) trident signature if the muon (an electron) is missed. We are not aware of any estimates for the rate of these processes at the DUNE ND, but we note that their rate can be comparable to trident production at energies above $30$ GeV \cite{Albright:1978mg}. Improvements on our analysis should come from the collaboration's sophisticated simulations, allowing for a better quantification of hadronic activity, more realistic misID rates and more accurate detector responses.  

%%%%%%%%%%%%%%%%%%%%%%%%%%%%%%%%%%%%%%%%%%%%%%%%%%%%
\section{Trident Events in Other Near Detector Facilities}

\label{sec:others}
The search for neutrino trident production events certainly benefits from the capabilities of LAr technologies but need not be limited to it. In this section we study neutrino trident production rates at non-LAr experiments which have finished data taking or are still running: the on-axis near detector of T2K (INGRID), the near detectors of MINOS and NO$\nu$A and the MINER$\nu$A experiment. We calculate the total number of trident events as in Eq.~(\ref{eq:nevents}), taking into account the fact that some detectors are made of composite material. We summarize in Tab.~\ref{tab:others} the details of all non-LAr detectors considered in this section. We limit ourselves to a discussion of the total rates in the fiducial volume, but remark that a careful consideration of each detector is needed in order to assess their true potential to detect a trident signal. For instance, requirements about low energy EM shower reconstruction, hadronic activity measurements and event containment would have to be met to a good degree in order for the detector to be competitive. 

%%
%%
\begin{table}[t]
\begin{center}
\scalebox{0.72}{
\begin{tabular}{|c|c|c|c|c|c|}
%\hline
\hline\hline
\bf Experiment& \bf Material & \bf Baseline (m) & \bf Exposure (POT) & \bf Fiducial Mass (t) & \bf $\mathbf{E_\nu}$ (GeV)\\\hline
INGRID  & Fe & 280  & $3.9\times10^{21}$ [$10^{22}$] T2K-I [T2K-II] & 99.4 & $0-4$ \\\hline
MINOS[+] & Fe and C &1040  & $10.56(3.36)[9.69]\times 10^{20}$  & 28.6& $0-20$ \\\hline
NO$\nu$A & ${\rm{C_2 H_3 Cl}} $ and $ {\rm{C H_2}}$  & 1000 &$8.85(6.9)~[36(36) ]\times10^{20}$ [NO$\nu$A-II] &231 & $0-20$ \\\hline
MINER$\nu$A & ${\rm{CH,H_2O}}, {\rm{Fe,Pb,C}}$ & 1035  & $12(12)\times 10^{20}$ &7.98 & $0-20$ \\\hline\hline
\end{tabular}}
\end{center}
\caption{\label{tab:others} Summary of the non-LAr detector set-up and values used in our calculations. The POT numbers are given for a neutrino (antineutrino) beam. For T2K-I and II 
neutrino and antineutrino beams have the same exposure.}
\end{table}

%%%%%%%%%%%%%%%%%%%%%%%%%%%%%%%%%%%%%%%%%%%%%%%%%%%%
\subsection{INGRID}
\label{subsec:INGRID}
INGRID, the on-axis near detector of the T2K experiment, is located 280 m from the beam source. It consists of 14 identical iron modules, each with a mass of $7.1$~t, resulting in a total fiducial mass of $99.4$ t~\cite{Abe:2011xv}. The modules are spread over a range of angles between $0^\circ$ and $1.1^\circ$ with respect to the beam axis. The currently approved T2K exposure is $(3.9+3.9)\times 10^{21}$ 
POT in neutrino + antineutrino modes (T2K-I), with the goal to increase it to a total exposure of $(1+1)\times 10^{22}$ POT in the second phase of the experiment (T2K-II) \cite{Abe:2016tez}. Hence we expect approximately $2.6$ times more trident events for T2K-II. 

We use the on-axis neutrino mode flux spectra at the INGRID module-3 from Ref.~\cite{Abe:2015biq}, as shown on the top of the first panel of Fig.~\ref{fig:others}. The flux contribution for each neutrino flavour and  energy range is listed in Table 1 of Ref.~\cite{Abe:2015biq}. The total neutrino flux flavour composition at module-3 is 92.5\% $\nu_\mu$, 5.8\% $\overline\nu_\mu$,  1.5\% $\nu_e$ and 0.2\%  $\overline\nu_e$. We assume here that the fluxes at the other 13 modules are the same as at module-3. Although this is not exactly correct it should provide a reasonable estimate of the total rate.

Under these assumptions we show the total number of trident events we calculated for INGRID in 
the first (second) column of Tab.~\ref{tab:otherrates} for T2K-I (T2K-II) exposure.
We predict about 600 (1600) events for the mixed, 290 (735) events for the dielectron and 45 (115) 
events for the dimuon channel for T2K-I (T2K-II). These numbers, although less than 
those expected at the DUNE ND, are already very significant and worth further consideration. We expect, however, that the main challenge will be the reconstruction of final state electrons in these iron detectors.

\begin{figure}[t]
\centering
\includegraphics[width=\textwidth]{figs/NLAr_f+xsec.pdf}
\caption{\label{fig:others}
Energy distribution of the neutrino fluxes at the position of the detector (top plot) and 
corresponding total trident production cross sections (bottom plot) for:
INGRID~\cite{Abe:2015biq} (first panel), MINOS ND~\cite{fluxes:nonLAr} (second panel), NO$\nu$A ND\cite{fluxes:nonLAr} (third panel)
and MINER$\nu$A\cite{fluxes:nonLAr} (fourth panel).
The cross sections show here for the composite detectors are normalized by the total number of 
atoms.}
\end{figure}


\begin{table}[t]
\begin{center}
\scalebox{0.8}{
\begin{tabular}{|cccccccc|}
\hline\hline
		\bf Channel & \bf T2K-I& \bf T2K-II & \bf MINOS & \bf MINOS+ & \bf NO$\nu$A-I & \bf NO$\nu$A-II & \bf MINER$\nu$A \\ \hline \hline
		$\nu_\mu\to\nu_e e^+ \mu^-$& $538$ &$1379$ &$179~(25)$&$688$ &$71~(14)$ &$291~(73)$& $140~(13)$ \\
        &$49$ &$126$ &$21~(3)$ &$82$ &$21~(4)$ &$86~(21)$ & $30~(3)$\\\hline
        $\overline\nu_\mu\to\overline\nu_e e^- \mu^+$&$23$ &$58$ &$42~(31)$&$38$ &$10~(57)$ &$41~(296)$ & $8~(89)$\\
        &$2$ &$5$ &$5~(4)$&$5$ &$3~(17)$ &$12~(88)$& $2~(19)$\\\hline
	$\nu_e\to\nu_\mu e^- \mu^+$& $2$ &$6$ &$1~(0.2)$&$4$ &$2~(0.5)$&$8~(3)$ & $1~(0.09)$\\
        &$0.3$ &$1$ &$0.3~(0.04)$ &$0.8$&$0.9~(0.2)$ &$4~(1)$& $0.3~(0.03)$\\\hline
	$\overline\nu_e\to\overline\nu_\mu e^+ \mu^-$& $0.2$ &$0.6$ &$0.4~(0.3)$&$0.4$ &$0.5~(0.9)$ &$2~(5)$& $0.06~(0.5)$\\
    &$0.04$ &$0.1$ &$0.08~(0.06)$ &$0.08$&$0.2~(0.4)$ &$0.8~(2)$& $0.02~(0.2)$\\\hline
    \hline\hline
    {$\rm{Total} \ e^\pm \mu^\mp$}& $563$ &$1444$ &$222~(56)$ &$730$&$83~(72)$ &$340~(374)$& $149~(102)$ \\
    &$52$ &$132$ &$27~(7)$&$88$ &$25~(22)$ &$102~(114)$& $32~(22)$\\\hline
    \hline
    		$\nu_\mu\to\nu_\mu e^+ e^-$& $257$ &$659$ &$48~(5)$ &$44$&$22~(3)$ &$90~(16)$& $35~(3)$ \\
        &$9$ &$23$ &$3~(0.4)$ &$3$&$3~(0.6)$ &$00$& $4~(0.4)$\\\hline
        $\overline\nu_\mu\to\overline\nu_\mu e^- e^+$& $10$ &$26$ &$9~(8)$&$9$ &$2~(16)$ &$8~(83)$& $2~(23)$\\
        &$0.4$ &$1$ &$0.7~(0.5)$ &$0.7$&$0.4~(3)$ &$2~(15)$& $0.2~(3)$\\\hline
	$\nu_e\to\nu_e e^- e^+$&$9$ &$24$ &$3~(0.3)$ &$8$&$3~(0.9)$ &$12~(5)$& $2~(0.2)$ \\
    &$0.3$ &$0.8$ &$0.2~(0.03)$ &$0.6$&$0.7~(0.2)$ &$3~(1)$& $0.2~(0.02)$\\\hline
	$\overline\nu_e\to\overline\nu_e e^+ e^-$&$0.9$ &$2$ &$0.7~(0.6)$&$0.7$ &$0.8~(2)$ &$3~(10)$& $0.1~(0.9)$ \\
    &$0.03$ &$0.08$ &$0.06~(0.04)$ &$0.05$&$0.2~(0.3)$ &$0.8~(1)$& $0.01~(0.1)$\\\hline
    \hline\hline
    ${\rm{Total}}\  e^+ e^-$& $277$ &$711$ &$61~(15)$ &$62$&$29~(22)$ &$119~(114)$& $39~(27)$\\
    &$10$ &$25$ &$4~(1)$ &$4$&$4~(4)$ &$16~(21)$& $4~(3)$\\\hline
    \hline    
    		$\nu_\mu\to\nu_\mu \mu^+ \mu^-$& $29$ &$73$ &$21~(3)$ &$81$&$7~(2)$ &$28~(11)$& $17~(2)$ \\
        &$15$ &$38$ &$8~(1)$ &$33$&$7~(2)$ &$29~(10)$& $12~(1)$\\\hline
        $\overline\nu_\mu\to\overline\nu_\mu \mu^- \mu^+$& $1$ &$3$ &$5~(3)$&$5$ &$1~(7)$ &$4~(35)$& $1~(11)$\\
        &$0.7$ &$2$ &$2~(1)$ &$2$&$1~(6)$ &$4~(30)$& $0.7~(8)$\\\hline
 $\nu_e\to\nu_e \mu^+ \mu^-$    &$0.09$ &$0.2$ &$0.09~(0.01)$&$0.3$ &$0.1~(0.04)$ &$0.4~(0.2)$& $0.06~(0.007)$\\  
 &$0.04$ &$0.1$ &$0.03~(0.004)$ &$0.1$&$0.1~(0.03)$ &$0.4~(0.1)$& $0.03~(0.004)$\\\hline
        $\overline\nu_e\to\overline\nu_e \mu^+ \mu^-$&$0.01$ &$0.03$ &$0.03~(0.02)$ &$0.03$&$0.04~(0.06)$&$0.2~(0.3)$ & $0.004~(0.03)$\\
        &$0.004$ &$0.01$ &$0.01~(0.009)$ &$0.01$&$~0.03(0.05)$  &$0.1~(0.3)$ & $0.003~(0.02)$\\\hline
        \hline \hline
    ${\rm{Total}} \ \mu^+ \mu^-$ &$30$ &$76$ &$26~(6)$&$86$ &$9~(9)$ &$37~(47)$ & $18~(13)$ \\
    &$16$ &$40$ &$10~(2)$ &$35$&$8~(8)$ &$34~(36)$ & $13~(9)$\\
    \hline\hline
\end{tabular}}
\end{center}
\caption{\label{tab:otherrates}Total number of \textbf{coherent} (top row) and \textbf{diffractive} (bottom row) trident events expected at different non-LAr detectors for each channel. The numbers in parentheses are for the antineutrino running mode, when present. These calculations consider a detection efficiency of 100\%.}
\end{table}

%%%%%%%%%%%%%%%%%%%%%%%%%%%%%%%%%%%%%%%%%%%%%%%%%%%%
\subsection{MINOS/MINOS+ Near Detector}
\label{subsec:MINOS}
The MINOS near detector is a magnetized, coarse-grained tracking calorimeter, made primarily of steel and plastic scintillator. Placed $1.04$~km away from the NuMI target at Fermilab \cite{Aliaga:2016oaz}, it weighs $980$~t and is similar to the far detector in design. In our analysis, we assume a similar fiducial volume cut to the standard $\nu_\mu$ CC analyses, namely a fiducial mass of $28.6$~t made
of $80\%$ of iron and $20\%$ of carbon \cite{Boehm:2009zz}.

%
The experiment ran from 2005 till 2012 in the low energy (LE) configuration of the NuMI beam ($E_\nu^{\rm peak} \approx 3$ GeV) and collected $10.56\times 10^{20} ~(3.36\times 10^{20})$ POT in the neutrino (antineutrino) beam \cite{Aurisano}. The successor to MINOS, MINOS+, ran with the same detectors subjected to the medium energy (ME) configuration of the NuMI beam ($E_\nu^{\rm peak} \approx 7$ GeV) from 2013 to 2016, and has collected $9.69\times 10^{20}$ POT in the neutrino mode.
%
To calculate the trident event rates we use the fluxes taken from Ref.~\cite{fluxes:nonLAr}. 
The flavour composition at MINOS ND is 89\% (18\%) $\nu_\mu$ and 10\% (81\%) $\overline \nu_\mu$ for the neutrino (antineutrino) beam and about 1\%  $\nu_e+\overline\nu_e$ for either beam mode. We assume that the MINOS+ neutrino flux is identical to the one at the MINER$\nu$A experiment (see section \ref{subsec:MINERvA}).
These fluxes and total trident production cross sections are shown on the second  panel of Fig.~\ref{fig:others}.

Due to the multi-component material of the detector, the corresponding cross sections that 
enter in Eq.~(\ref{eq:nevents}) are:
\begin{equation}
\sigma_{\rm \nu X}^{\rm{MINOS}}= \sum_{i=\rm Fe,C} f_i \, \sigma_{\rm \nu X}^{i}\,,
\end{equation}
where $f_i$ is the number of nuclei $i$ over the total number of nuclei in the detector.
As a reference, the weighted cross sections, normalized by the total number of atoms, is 
also shown in Fig.~\ref{fig:others}. 

We report the total number of trident events for MINOS ND in Tab.~\ref{tab:otherrates}. 
Although the cross section for iron is about two times 
larger than for argon and the neutrino fluxes similar, the number of trident events at MINOS ND is much smaller than the expected one at DUNE ND due to a lower exposure and fiducial mass.
%
We predict that about 250 (63) mixed, 65 (16) dielectron and 36 (8)   dimuon trident events
were produced at this detector with the neutrino (antineutrino) LE NuMI beam. 
%
The rates are expected to be larger for MINOS+, as it benefits from the larger energies of the ME NuMI beam configuration and has similar number of POT to MINOS in neutrino mode. In total, we predict about 820 mixed, 66 dielectron and 121 dimuon trident events.

The stringent cut on the fiducial volume assumed here implies a reduction from the $980$~t near detector bulk mass to $28.6$~t. This cut can be relaxed, depending on the signature considered, and may significantly enhance the rates we quote. A careful analysis of trident signatures outside the fiducial volume would be necessary, but we point out that our rates can increase by at most a factor of $\approx30$.

%%%%%%%%%%%%%%%%%%%%%%%%%%%%%%%%%%%%%%%%%%%%%%%%%%%%
\subsection{NO$\nu$A Near Detector}
\label{subsec:NOvA}

The NO$\nu$A near detector is a fine grained low-Z liquid-scintillator detector placed off-axis from the NuMI beam at a distance of $1$~km. Its total mass is $330$~t, with almost $70$\% of it active mass ($231$~t). In this analysis we assume all of this active mass to also be fiducial. The detector is mainly made of 70\% mineral oil (CH$_2$) and 30\% of PVC (${\rm{C_2H_3Cl}}$) \cite{Wang:Biao}. A total exposure of $8.85 \, (6.9)\times10^{20}$ POT has been collected in the neutrino (antineutrino) beam mode prior to 2018~\cite{sanchez_mayly_2018_1286758}. 

The NO$\nu$A ND neutrino fluxes (taken from Ref. \cite{fluxes:nonLAr}) peak at slightly lower energies than the MINOS or MINER$\nu$A ones, $E_\nu^{\rm peak} \approx 2$ GeV, and are shown in the third panel of Fig. \ref{fig:others}. The flavour composition 
is 91\% (11\%) $\nu_\mu$ and 8\% (88\%) $\overline \nu_\mu$ in the neutrino (antineutrino) 
mode and about 1\% $\nu_e+\overline\nu_e$ in each mode.

Here the cross sections entering in Eq.~(\ref{eq:nevents}) are calculated as
%
\begin{eqnarray}
\sigma^{\rm{NO\nu A}}_{\rm \nu X}=\sum_{i=\rm C, Cl, H} f_i \sigma_{\rm \nu X}^{i} \,,
\end{eqnarray}
%
where $f_i$ is the number of nuclei $i$ over the total number of nuclei in the detector.
As a reference, the weighted cross sections, normalized by the total number of atoms, is shown 
in Fig.~\ref{fig:others}. 

In Tab.~\ref{tab:otherrates} we show our predictions for the number of trident events at 
NO$\nu$A ND. 
%
Comparing NO$\nu$A and MINOS, we see that while NO$\nu$A ND has a fiducial mass 
almost 8 times larger, the flux times total cross section at MINOS ND is at least two orders of magnitude larger than at NO$\nu$A ND, especially above 4 GeV (see Fig. \ref{fig:others}), making the rates at MINOS ND larger than the rates at NO$\nu$A ND. 
%

NO$\nu$A is planning to collect a total exposure of $36\,(36)\times10^{20}$ POT in the neutrino (antineutrino) mode (NO$\nu$A-II) \cite{sanchez_mayly_2018_1286758,NovaII}, making the expected rates almost $4.1 (5.2)$ times larger (shown in Tab.~\ref{tab:otherrates}). In this case the expected dimuons and mixed events at MINOS+ would be at least two times larger than NO$\nu$A-II. On the other hand, for NO$\nu$A-II there will be two times more dielectron events given the much higher exposure. 

%%%%%%%%%%%%%%%%%%%%%%%%%%%%%%%%%%%%%%%%%%%%%%%%%%%%
\subsection{MINER$\nu$A}
\label{subsec:MINERvA}

The multi-component MINER$\nu$A detector was mainly designed to measure neutrino and antineutrino interaction cross sections with different nuclei in the 1-20 GeV range of energy~\cite{MINERvA:2017}. The detector is located at $1.035$~km from the NuMI target. We assume a fiducial mass of about $8$~t, with a composition of 75\% CH, 9\% Pb, 8\% Fe, 6\% H$_2$O and $2\%$ C. The experiment has collected $12\times10^{20}$ POT in the neutrino mode and is planning to reach the same exposure in the antineutrino mode by 2019, both using the medium energy flux of NuMI beam configuration (shown in fourth panel of Fig.~\ref{fig:others}). We do not include the low energy runs, as these have lower number of POT and lower neutrino energies. The neutrino (antineutrino) beam is composed of 95\% (7\%) $\nu_\mu$ and 
4\% (92\%) $\overline \nu_\mu$, both beams have  about $1\%$ of $\nu_e+\overline\nu_e$.

For MINER$\nu$A the cross sections  in Eq.~(\ref{eq:nevents}) are calculated as
\begin{eqnarray}
\sigma^{\rm MINER \nu A}_{\rm \nu X}=\sum_{i=\rm C, Cl, H, Pb, Fe,O} f_i \, \sigma_{\rm \nu X}^{i} \,,
\end{eqnarray}
where $f_i$ is the number of nuclei $i$ over the total number of nuclei in the detector.
As a reference, the weighted cross sections, normalized by the total number of atoms, is shown in
Fig.~\ref{fig:others}. 

The total number of trident events we estimate for MINER$\nu$A are listed in Tab.~\ref{tab:otherrates}. As expected, these are lower than MINOS+, as the latter has a larger fiducial mass. MINER$\nu$A, however, benefits from its fine grained technology and its dedicated design for cross section measurements.

%%%%%%%%%%%%%%%%%%%%%%%%%%%%%%%%%%%%%%%%%%%%%%%%%%%%
\section{Conclusions}

\label{sec:conc}
Neutrino trident events are predicted by the SM, however, only $\overline{\nu}_\mu$ initiated dimuon tridents have been observed in small numbers, typically fewer than 100 events. This will change in the near future thanks to the current and future generations of precision neutrino scattering and oscillation experiments, which incorporate state-of-the-art detectors located at short distances from intense neutrino sources. 
%
In this work we discuss the calculation of the neutrino trident cross section for all flavours and hadronic targets, and provide estimates for the number and distributions of events at 9 current or future neutrino detectors: five detectors based on the new LAr technology (SBND, $\mu$BooNE, ICARUS, DUNE ND and $\nu$STORM ND) as well as four more conventional detectors (INGRID, MINOS ND, NO$\nu$A ND and MINER$\nu$A). The search for tridents, however, need not be exclusive to near detectors of accelerator neutrino experiments. As pointed out by the authors of Ref.~\cite{Ge2017}, atmospheric neutrino experiments can also look for these processes, benefiting from the increase of the cross section at large energies. 

We have stressed the need for a full four-body phase space calculation of the trident cross sections without using the EPA. This approximation has been employed in recent calculations and can lead to overestimations of the cross section by 200\% or more at the peak neutrino energies relevant for many accelerator neutrino experiments.
%
Moreover, we show why the EPA is not applicable for computing trident cross sections, and provide the first quantitative assessment of this breakdown for coherent and diffractive hadronic regimes. 
%
We find that the breakdown of the approximation is most severe for processes with electrons in the final-state and for diffractive scattering of all final state flavours. 
%
For coherent dimuon production, the approximation can give a reasonable result at large neutrino energies. This is due to the nuclear form factors that serendipitously suppress those regions of phase space where the EPA is least applicable. We also demonstrated that the best results in this channel are achieved when applying artificial cuts to the phase space.
%
However, even in this case, at energies relevant for the above experiments, the EPA can artificially suppress the coherent scattering contribution and increase the diffractive one giving rise to an incorrect rate and distributions of observable quantities. 
%
For instance, the invariant mass of the charged lepton pair $m^2_{\ell \ell}$ and their angular separation $\Delta \theta$ are more uniformly distributed for diffractive than for coherent trident scattering. Using the correct distributions is crucial to correctly disentangle the signal from the background by cutting on these powerful discriminators.

Our calculations show that DUNE ND is the future detector with the highest neutrino trident statistics, more than 6000 mixed events, 11\% produced by diffractive scattering, more than 1900 dielectron events, 5\%  produced by diffractive scattering and about 750 dimuon events,
almost 34\% of those produced by a diffractive process. Making use of our efficiencies (see \reftab{tab:DUNE_ND_NU_BG}), assuming an ideal background suppression and neglecting systematic uncertainties, we quote the statistical uncertainty on the coherent-like flux averaged cross section for the DUNE ND. We do this for coherent only events and, in brackets, for coherent plus diffractive events, yielding
%
\[\frac{\delta \langle \sigma^{e^\pm\mu^\mp} \rangle}{\langle \sigma^{e^\pm\mu^\mp} \rangle} =  1.8\% \, (1.6\%), \quad \frac{\delta \langle \sigma^{e^+e^-} \rangle}{\langle \sigma^{e^+e^-} \rangle} =  3.4\% \,(3.3\%) \quad \mathrm{and} \quad \frac{\delta \langle\sigma^{\mu^+\mu^-} \rangle}{\langle \sigma^{\mu^+\mu^-} \rangle} =  5.5\% \,(5.1\%).\]
%
In this optimistic framework we expect the true statistical uncertainty on coherent-like tridents to lie between the two numbers quoted, depending on how many diffractive events contribute to the coherent-like event sample. This impressive precision would provide unprecedented knowledge of the trident process and the nuclear effects governing the interplay between coherent and diffractive regimes. We emphasize, however, that given these small values for the relative uncertainties, the trident cross section will likely be dominated by systematic uncertainties from detector response and backgrounds which are not modelled here. 

For DUNE ND, we have studied the distribution of observables which could help distinguish trident events from the background. We have estimated the background for each trident channel via a Monte Carlo simulation using GENIE, and identified the dominant contributions arising primarily from particle misidentification.  
%
We conclude that reaching background rates of the order 
${\cal O}(10^{-6}-10^{-5})$ times the CC rate is necessary to observe trident events at DUNE ND, and given the distinctive kinematic behaviour of the trident signal a simple cut-based GENIE-level analysis suggests that this is an attainable goal in a LAr TPC. 

Existing facilities may also be able to make a neutrino trident measurement at their near detectors. Despite not including reconstruction efficiencies nor an indication of the impact of backgrounds, we find that the largest trident statistics is available at INGRID, the T2K on-axis near detector. We predict about 660 (1700) events for the mixed flavour, 300 (770) events for the dielectron and 50 (130) events for the dimuon channel for T2K-I (T2K-II). The more fine-grained near detector of MINOS and MINOS+ is also expected to have collected a significant numbers of events during its run. As such, the very first measurement of neutrino trident production of mixed and dielectron channels may be at hand.


%%%%%%%%%%%%%%%%%%%%%%%%%%%%%%%%%%%%%%%%%%%%%%%%%%%%
\section{\label{app:formfactors}Form Factors}

In the coherent regime, we use a Woods-Saxon (WS) form factor due to its success in reproducing the experimental data \cite{Fricke:1995zz,Jentschura2009}. The WS form factor is the Fourier transform of the nuclear charge distribution, defined as 
%
\begin{equation}
 \rho(r) = \frac{\rho_0}{1+\exp\left(\dfrac{r - r_0}{a}\right)} \, ,
\end{equation}
%
where we take $r_0 = 1.126 \, A^{1/3}$ fm and $a = 0.523$ fm. One can then calculate the WS form factor as
%
\begin{equation}
 F(Q^2) = \frac{1}{\int \rho(r) \, \dd^3r}  \int \rho(r) \, \exp\left( -i \vec{q} \vdot \vec{r}\right) d^3r \, .
\end{equation}
%
Here we use an analytic expression for the symmetrized Fermi function \cite{Anni1994,Sprung1997} instead of calculating the WS form factor numerically. This symmetrized form is found to agree very well with the full calculation and reads
%
\begin{equation}
  F(Q^2) =  \frac{3 \pi  a}{r_0^2 + \pi^2 a^2} \frac{\pi a \coth{(\pi Q a)} \sin{(Q r_0)} - r_0 \cos{(Q r_0)} }{Q r_0 \sinh{(\pi Q a)}}\, .
\end{equation}
%

In the diffractive regime, we work with the functions $H_1^{\rm N}(Q^2)$ and $H_2^{\rm N}(Q^2)$, which depend on the Dirac and Pauli form factors of the nucleon ${\rm N}$ as follows
%
\begin{equation}
H_1^{\rm N}(Q^2)= |F_1^{\rm N}(Q^2)|^2 - \tau |F_2^{\rm N}(Q^2)|^2\, , \quad \mathrm{and} \quad H_2^{\rm N}(Q^2) = \left| F_1^{\rm N}(Q^2) + F_2^{\rm N}(Q^2)\right|^2 \, ,
\end{equation}
%
where $\tau = -Q^2/4M^2$. The form factors $F_1^{\rm N}(Q^2)$ and $F_2^{\rm N}(Q^2)$ can be related to the usual Sachs electric $G_\mathrm{E}$ and magnetic $G_{\mathrm{M}}$ form factors. These have a simple dipole parametrization
%
\begin{align}
G^{\rm N}_E(Q^2) =& F^{\rm N}_1 (Q^2) + \tau F^{\rm N}_2 (Q^2) = \begin{cases}
                                              0,\, &\mathrm{if }\, {\rm N} = n,\\
                                              G_D(Q^2),\, &\mathrm{if }\, {\rm N} = p,
                                              \end{cases} \\
G^{\rm N}_M(Q^2) =& F^{\rm N}_1 (Q^2) + F^{\rm N}_2 (Q^2) = \begin{cases}
                                              \mu_n  \, G_D(Q^2),\, &\mathrm{if }\, {\rm N} = n,\\
                                              \mu_p  \, G_D(Q^2),\, &\mathrm{if }\, {\rm N} = p,
                                              \end{cases}
\end{align}
%
where $\mu_{p,n}$ is the nucleon magnetic moment in units of the nuclear magneton and $G_D(Q^2) = (1 + Q^2/M_V^2)^{-2}$ is a simple dipole form factor with $M_V = 0.84$ GeV.
%


%%%%%%%%%%%%%%%%%%%%%%%%%%%%%%%%%%%%%%%%%%%%%
\section{Kinematical Distributions \label{app:distributions}}

\begin{figure}[t]
\centering
\includegraphics[width=0.92\textwidth]{figs/DUNE_nu_3horn_aC_Q2_thetapm.pdf}
\caption{Flux convolved neutrino trident production distributions for DUNE ND in neutrino mode in additional variables. In purple we show the coherent contribution in $^{40}$Ar and in blue the diffractive contribution from protons as targets only (including Pauli blocking). The coherent and diffractive distributions are normalized independently. \label{fig:other_dists}}
\end{figure}

%
In this section, we show additional distributions in different observables for neutrino trident production, also focused on the DUNE ND in neutrino mode, in \reffig{fig:other_dists}. While trident events are generally quite forward going, their angular behaviour is quite interesting. We consider here the angle between the charged lepton cone and the neutrino beam, $\alpha_C$, defined as 
%
\[
\cos{\alpha_C} = \frac{ (\vec{p}_3 + \vec{p}_4)  \vdot \vec{p}_1}{ |\vec{p}_3 + \vec{p}_4| |\vec{p}_1| }\, ,\]  
%
and in the individual angle of the charged lepton to the neutrino beam, $\theta$. For same flavour tridents we define $\theta$ for each charge of the visible final-state, whilst for mixed tridents we use their flavour. We also show the distribution in $Q^2 = {-q^2}$, where $q = (P - P^\prime)$, which is of particular interest when considering coherency and the impact of form factors.


%%%%%%%%%%%%%%%%%%%%%%%%%%%%%%%%
\section{Individual Backgrounds}
\label{app:backgrounds}
Here we discuss backgrounds to trident final-states in more detail. We start by motivating our misID rates shown in \reftab{tab:misIDlist}, and then discuss the dominant background processes individually.

In LAr photons can be distinguished from a single electron if their showers start displaced from the vertex (if present). Photons have a conversion length in LAr of around 18 cm, meaning $5$--$10\%$ could be expected to convert quickly enough to hinder electron-photon discrimination by this means if the resolution on the gap is from $1$--$2$ cm \cite{Acciarri:2016sli}. Once pair conversion happens, photons can be distinguished from a single electron purely by $\dd E/\dd x$ measurements in the first 1--2 cm of their showers. Motivated by the success of this method as shown at ArgoNeuT \cite{Acciarri:2016sli} and based on projections for DUNE \cite{Acciarri:2016ooe}, we assume that $5\%$ of photons would be taken as $e^\pm$ with perfect efficiency, without the need for an event vertex. Needless to say that a dedicated study for trident topologies would be necessary for a more complete study. It is worth noting that our remarks concern only the misID of a single photon for a single electron, whilst the distinction between a photon and an overlapping $e^+e^-$ pair without a vertex can be much more challenging. For this reason we take the misID rate between an overlapping $e^+e^-$ pair and a photon to be 1 in the absence of a vertex.  

Charged pions are notorious for faking long muon tracks. We estimate this misID rate as arising from through-going pions, which do not exhibit the decay kink used in their identification. We assume an interaction length of around $1$ m, meaning that about $5\%$ of particles travel $\sim3$ meters and escape the fiducial volume. Assuming that this is the most likely way a pion can spoof a muon, we estimate a naive suppression rate of $10^{-2}$. In a more complete study, it is desirable to explore the length of the muon and pion tracks inside the detector as a function of energy. The length of the contained tracks can also be an important tool for background suppression which we leave to future studies.  

%%%%%%%%%%%%%%%%%%%%%%%%%%%%%%%%%%%%%%%%%%%%%%%%%%%%
\subsection{Pion Production}

Coherent pion production in its charged ($\nu + A \to \ell^\mp + \pi^\pm + A$) and neutral ($\nu + A \to \nu + \pi^0 + A$) current version is very abundant at GeV energies. The cross section for these processes is modelled in GENIE using a modern version of the Rein-Sehgal model \cite{REIN198329,Rein:2006di}. The charged current version serves mainly as a background to $\mu^+ \mu^-$ tridents, but can also appear as a background for $e^\pm \mu^\mp$ tridents for incoming electron neutrinos or antineutrinos. It has been studied before at MiniBooNE \cite{AguilarArevalo:2010xt}, MINER$\nu$A \cite{Higuera:2014azj,Mislivec:2017qfz}, T2K \cite{Abe:2016fic,Abe:2016aoo}, and for the first time in LAr at ArgoNeuT \cite{Acciarri:2014eit}. This process has a very distict low 4-momentum transfer to the nucleus $|t|$ \cite{Higuera:2014azj}, but a much flatter distribution in invariant mass if compared to trident. The neutral current version of coherent pion production serves as a background to $e^+e^-$ tridents. This process has been studied before by the MiniBooNE \cite{AguilarArevalo:2009ww}, SciBooNE \cite{Kurimoto:2010rc} and in LAr by the ArgoNeuT collaboration \cite{Acciarri:2015ncl}. There are two possibilities for these events to fake an $e^+e^-$ trident: when one of the gammas produced in the $\pi^0$ decay is missed and the other is misIDed for an overlapping $e^+e^-$ pair, and when both photons are each misIDed for a single electron. This signature also comes with low hadronic activity, but for separated visible photons the invariant mass is a natural discriminator, as in the detector $m_{\gamma \gamma} \approx m_{\pi^0}$.

Resonant pion production can also contribute to trident backgrounds in the absence of any reconstructed protons. Resonant pion production can be larger than its coherent counterpart and is modelled in GENIE by the Rein-Sehgal model \cite{Rein:1980wg}. Its CC version was measured by MiniBooNE \cite{AguilarArevalo:2010xt}, K2K \cite{Mariani:2010ez}, MINOS \cite{Adamson:2014pgc}, and MINER$\nu$A \cite{Altinok:2017xua}. In the latter measurement one can clearly see the large number of events with undetected protons. The misIDed photon and the charged lepton invariant mass are once more flatter than the trident ones, allowing for a kinematical discrimination whenever a single photon is undetected. It is worth noting that these are some of the dominant underlying processes for pion production in GENIE, but all events leading to topologies relevant for trident are included in our analysis.
%
%%%%%%%%%%%%%%%%%%%%%%%%%%%%%%%%%%%%%%%%%%%%%%%%%%%%
\subsection{Charm Production}

Since the first observation of dimuon pairs from charm production in neutrino interaction by the HPWF experiment in 1974 \cite{Benvenuti:1975ru}, a lot has been learned about these processes (see \cite{Lellis:2004yn} for a review) in neutrino experiments. Particularly, the production of charm quarks and their subsequent weak decays into muons or electrons have been identified as a major source of background for early trident searches. At the lower neutrino energies at DUNE, however, this is expected to be a smaller yet non-negligible contribution. From our GENIE samples, we estimate that a charmed state is produced at a rate of around $10^{-4}(N_\text{CC}+N_\text{NC})$. Most of these produce either D mesons, 
$\Lambda_c$ or $\Sigma_c$ baryons. These particles decay in chains, emitting a muon with a branching ratio of around $0.1$, and are always accompanied by pions or other hadronic particles. We therefore expect these rates to be negligible with a hadronic veto, and do not consider them further. We hope, however, that future studies will address these channels in more detail.

%%%%%%%%%%%%%%%%%%%%%%%%%%%%%%%%%%%%%%%%%%%%%%%%%%%%
\subsection{CC$\gamma$ and NC$\gamma$}

The emission of a single photon alongside a CC process could be a background for $\mu e$ tridents if the photon is misIDed as a single electron. When the photon is produced in a NC event, it can be a background to overlapping $e^+e^-$ tridents. In GENIE, these topologies arise mainly due to resonance radiative decays and from the intra-nuclear processes. For this reason, it usually comes accompanied with extra hadronic activity. For hadronic resonances, we have simulated CC processes in GENIE and estimated
the multiplicities: $0.5\%$ single
photon and $1\%$ double photon emission from CC rates. Radiative photon production from the charged lepton, on the other hand, does not need to come accompanied by hadrons. It is phase space and $\alpha\approx1/137$ suppressed with respect to CCQE rates, and therefore could occur at appreciable rates compared to our signal. This contribution, however, is not included in GENIE and is absent from our samples. The rates of internal photon bremsstrahlung have been estimated before, particularly for T2K where a low-energy photon is an important background for electron
appearance searches \cite{Efrosinin:2009zz}, and as a background to the low energy events at MiniBooNE \cite{Bodek:2007wb}. De-excitation gammas from the struck nuclei can also generate CC$\gamma$ or NC$\gamma$ topologies \cite{PhysRevLett.108.052505}. These contributions for Ar are not included in GENIE, but are expected to come with a distinct energy profile, which can be tagged on.

%%%%%%%%%%%%%%%%%%%%%%%%%%%%%%%%%%%%%%%%%%%%


\chapter{Conclusions}
Particle physics is at a very important moment of its history. The Standard Model has surprised us with its unprecedent accuracy in describing particle data, but a few mysteries remain. Perhaps neutrino masses and dark matter are to the Standard Model what blackbody radiation was to classical physics in the beginning of the 20th century, a scientific revolution on the wait. 

% \appendixpageoff
\begin{appendices}
% \let\clearpage\relax
\chapter{Phase space}\label{app:phase_space}
In this appendix we derive some key results for the phase space treatment we use in calculating cross sections and decay rates. We begin with the factorization of $N$-final state phase-space factors into $N-2$ two-body ones. In general, the $N$-body phase space can be written as 

\begin{equation}
 \dd \Phi_{N} (P,\{p_i\}) = (2 \pi)^4 \delta^4 (P - \sum_i^N p_i) \, \prod_i^{N}\, \frac{\dd^3 p_i}{(2\pi)^3 2 E_i},
\end{equation}
where $\{p_i\} = p_1, \ldots, p_N$. Focusing on the 1-2 subsystem with total momentum $p_{12} = p_1 + p_2$, we can write

\begin{align}
  \dd \Phi_{N} (P,\{p_i\}) =& \int \dd^4 p_{12} \, \delta^4 (p_{12} - p_1 - p_2) \, (2 \pi)^4 \delta^4 (P - \sum_i^N p_i) \, \prod_i^{N}\, \frac{\dd^3 p_i}{(2\pi)^3 2 E_i}  \nonumber\\
 =& \int \dd^4 p_{12}\, \dd \Phi_2 (p_{12}, p_1, p_2) \, (2 \pi)^4 \delta^4 (P - p_{12} - \sum_{i=3}^N p_i) \, \prod_{i=3}^{N}\, \frac{\dd^3 p_i}{(2\pi)^3 2 E_i}\nonumber\\
 =& \int \dd^4 p_{12} \, \dd m_{12}^2 \, \delta(p_{12}^2 - m_{12}^2) \, \dd \Phi_2 (p_{12}, p_1, p_2)\nonumber\\ &\hspace{20ex} \times  (2 \pi)^4 \delta^4 (P - p_{12} - \sum_{i=3}^N p_i) \, \prod_{i=3}^{N}\, \frac{\dd^3 p_i}{(2\pi)^3 2 E_i},
\end{align}
which from 

\[\int \dd^4 p_{12} \, \delta(p_{12}^2 - m_{12}^2) = \int \dd^4 p_{12} \, \frac{\delta (E_{12} - \sqrt{m_{12}^2 + |\vec{p}_{12}|^2})}{2\sqrt{m_{12}^2 + |\vec{p}_{12}|^2} } = \int \frac{\dd^3 p_{12}}{2 E_{12}}, \]
yields the final results
\begin{equation}
 \dd \Phi_{N} (P, p_1, \ldots, p_N) = \frac{\dd m_{12}^2}{2 \pi} \dd \Phi_2 (p_{12}, p_1, p_2) \dd \Phi_N (P, p_{12}, p_3,\ldots,p_N). 
\end{equation}

This result not only lets us factorize any resonant features in phase space, but also provides a recipe to tackle the kinematics of any process in terms of a series of 2-body problems, which are much simpler. The 2-body phase-space factors and associated four-momenta in the respective center-of-mass (CM) frame can always be written as

\begin{align}
\dd \Phi_2 (p_{12}, p_1, p_2) &= \frac{{\lambda^{1/2} \left( 1, m_1^2/E_{12}^{ {\rm CM} \, 2}, m_2^2/E_{12}^{ {\rm CM} \, 2} \right)} }{32 \pi^2 } \dd \Omega^{\rm CM},\nonumber\\
p_{12} &= \left(E_{12}^{ {\rm CM}}, \vec{0} \right),\nonumber\\
p_{1}  &= \left(\frac{E_{12}^{ {\rm CM} \, 2}+m_1^2-m_{2}^2}{2E_{12}^{ {\rm CM} \, 2}}, |\vec{p}_1| \sin{\theta} \cos{\phi}, |\vec{p}_1| \sin{\theta} \sin{\phi} , |\vec{p}_1| \cos{\theta} \right),\nonumber\\
p_{2}  &= \left(\frac{E_{12}^{ {\rm CM} \, 2}+m_{2}^2 - m_1^2}{2E_{12}^{ {\rm CM} \, 2}}, -\vec{p}_1 \right),
\end{align}
where $\lambda(a,b,c) = (a - b -c)^2 - 4 b c$ is the K\"all\'en function. Now, the problem is reduced to finding the CM frame of every $p_{ij}$ subsystem, and the transformation between all such frames, if necessary. Of course, the dependence of the matrix element on the kinematics makes certain phase space parametrization better than others, making each problem unique. Lab variables, for instance, are the standard parametrization for DIS scattering as they preserve crucial physical intuition of the process at hand.

\paragraph{Dark-bremsstrahlung} Here we take the Dark-Bremsstrahlung (DB) process discussed in \refsec{...} as an example of a 3-body phase space. In this process, the neutrino scatters off the nuclear target with momentum exchange $Q^2=()$.

\paragraph{Neutrino trident production} Now we derive a phase space parametrization for neutrino trident production in terms of the momentum transfer $K^2 =  2 p_1 \vdot p_2$. This is important if one wants to change variables to smooth out the integrand at low $M_{Z^\prime}$ masses. We follow the calculation in \cite{Czyz1964} and \cite{Ballett:2018uuc}, and proceed to define $K^2$ as one of the integration variables. The relevant Lorentz invariant phase space for the $2\to3$ leptonic part of the cross section is given by
%
\begin{align}
\int \dd^3 & \Pi_{\mathrm{LIPS}} = \nonumber\int \frac{\dd \vec{p_2} }{(2\pi)^32 E_2} \frac{ \dd \vec{p_3} } {(2\pi)^3 2 E_3} \frac{\dd \vec{p_4}}{(2\pi)^32 E_4} \\& (2\pi)^4\delta^{(4)} (p_1 + q - p_2 - p_3 - p_4).
\end{align}
%
Following \cite{Czyz1964} we start by working in the frame $\vec{p_1} + \vec{q} - \vec{p_3} = 0$, putting $\vec{p_1}$ along the $\hat{z}$ direction instead. The delta function can be integrated with the $\vec{p_4}$ and $|\vec{p_2}|$ integrals, such that 
%
\begin{equation}
 \int \frac{\dd \vec{p_2} }{2 E_2} \frac{\dd \vec{p_4}}{2 E_4} \, \delta^{(4)} (p_1 + q - p_2 - p_3 - p_4) =  \int \frac{ |\vec{p_2}|}{4 W_c} \, \frac{1}{E_1 E_2} \dd K^2 \, \dd \phi_2,
\end{equation}
%
where we defined
%
\begin{align}
&|\vec{p_2}| = (W_c^2 - m_1^2)/2W_c, \nonumber\\\nonumber &W_c = q^0 + E_1 - E_3, \\ &K^2 = 2 E_1 E_2 (1 - \cos{\theta_2}).
\end{align}
%
Since we conserve energy and momentum in this frame, we can take $-1 \leq \cos{\theta_2} \leq 1$ and $0 \leq \phi_2 \leq 2 \pi$. The remaining $\vec{p_3}$ integral can be performed with the variables defined in \cite{Czyz1964} to yield
%
\begin{equation}
 \int \frac{\dd \vec{p_3} }{2 E_3}  =  \int \frac{2 \pi}{\hat{s}} \, \dd x_5 \, \dd x_3,
\end{equation}
%
where a trivial azimuthal angle was integrated over. Their limits are more easily found in the frame $\vec{p_1} + \vec{q} = 0$, with $\vec{q}$ along the $\hat{z}$ direction. Finally, our main result is given by
%
\begin{equation}
\int \dd^3 \Pi_{\mathrm{LIPS}} = \frac{1}{(2\pi)^4} \int \frac{|\vec{p_2}|}{4 W_c} \, \frac{1}{\hat{s}} \, \frac{1}{E_1 E_2} \, \dd x_5 \, \dd x_3 \, \dd K^2 \, \dd \phi_2 .
\end{equation}
%
There remains two non-trivial integrations to be performed to obtain the full 4-body phase space cross section, namely the ones over $q^2$ and $\hat{s}$. The substitutions suggested in \cite{Lovseth1971} for these two invariants are still convenient, and we make use of them in our numerical integrations.


\chapter{Form factors}\label{app:form_factors}

%%%%%%%%%%%%%%%%%%%%%%%%%%%%%%%%%%%%%%%%%%%%%%%%%%

In the coherent regime, we use a Woods-Saxon (WS) form factor due to its success in reproducing the experimental data \cite{Fricke:1995zz,Jentschura2009}. The WS form factor is the Fourier transform of the nuclear charge distribution, defined as 
%
\begin{equation}
 \rho(r) = \frac{\rho_0}{1+\exp\left(\dfrac{r - r_0}{a}\right)} \, ,
\end{equation}
%
where we take $r_0 = 1.126 \, A^{1/3}$ fm and $a = 0.523$ fm. One can then calculate the WS form factor as
%
\begin{equation}
 F(Q^2) = \frac{1}{\int \rho(r) \, \dd^3r}  \int \rho(r) \, \exp\left( -i \vec{q} \vdot \vec{r}\right) d^3r \, .
\end{equation}
%
Here we use an analytic expression for the symmetrized Fermi function \cite{Anni1994,Sprung1997} instead of calculating the WS form factor numerically. This symmetrized form is found to agree very well with the full calculation and reads
%
\begin{equation}
  F(Q^2) =  \frac{3 \pi  a}{r_0^2 + \pi^2 a^2} \frac{\pi a \coth{(\pi Q a)} \sin{(Q r_0)} - r_0 \cos{(Q r_0)} }{Q r_0 \sinh{(\pi Q a)}}\, .
\end{equation}
%

In the diffractive regime, we work with the functions $H_1^{\rm N}(Q^2)$ and $H_2^{\rm N}(Q^2)$, which depend on the Dirac and Pauli form factors of the nucleon ${\rm N}$ as follows
%
\begin{equation}
H_1^{\rm N}(Q^2)= |F_1^{\rm N}(Q^2)|^2 - \tau |F_2^{\rm N}(Q^2)|^2\, , \quad \mathrm{and} \quad H_2^{\rm N}(Q^2) = \left| F_1^{\rm N}(Q^2) + F_2^{\rm N}(Q^2)\right|^2 \, ,
\end{equation}
%
where $\tau = -Q^2/4M^2$. The form factors $F_1^{\rm N}(Q^2)$ and $F_2^{\rm N}(Q^2)$ can be related to the usual Sachs electric $G_\mathrm{E}$ and magnetic $G_{\mathrm{M}}$ form factors. These have a simple dipole parametrization
%
\begin{align}
G^{\rm N}_E(Q^2) =& F^{\rm N}_1 (Q^2) + \tau F^{\rm N}_2 (Q^2) = \begin{cases}
                                              0,\, &\mathrm{if }\, {\rm N} = n,\\
                                              G_D(Q^2),\, &\mathrm{if }\, {\rm N} = p,
                                              \end{cases} \\
G^{\rm N}_M(Q^2) =& F^{\rm N}_1 (Q^2) + F^{\rm N}_2 (Q^2) = \begin{cases}
                                              \mu_n  \, G_D(Q^2),\, &\mathrm{if }\, {\rm N} = n,\\
                                              \mu_p  \, G_D(Q^2),\, &\mathrm{if }\, {\rm N} = p,
                                              \end{cases}
\end{align}
%
where $\mu_{p,n}$ is the nucleon magnetic moment in units of the nuclear magneton and $G_D(Q^2) = (1 + Q^2/M_V^2)^{-2}$ is a simple dipole form factor with $M_V = 0.84$ GeV.
%


%%%%%%%%%%%%%%%%%%%%%%%%%%%%%%%%%%%%%%%%%%%%%%%%%%
\section{Weak form factor}
\label{app:formfactors}

Here we show our weak hadronic current used in the dark-bremsstrahlung calculation. Similarly to the electromagnetic case, we write the weak hadronic current for a spin-0 nucleus with $Z$ protons and $N$ neutrons as
%
\begin{align}
 {\rm H}_{\rm W}^\mu &= \bra{\mathcal{H}(k_3)} J^\mu_{\rm W} (Q^2) \ket{\mathcal{H}(k_b)} \nonumber\\&= Q_{\rm W} (k_b+k_3)^\mu F_{\rm W}(Q^2),
\end{align}
where $Q_{\rm W} = (1-4 s_{\rm w}^2)Z - N$ and $F_{\rm W}(Q^2)$ stands for the weak form factor of the nucleus. We implement the Helm form factor as in \cite{Duda:2006uk}, defined as  
\begin{equation}
|F(Q^2)|^2 = \left( \frac{3 j_1(QR)}{QR}\right)^2 e^{-Q^2 s^2},
\end{equation}
where $j_1 (x)$ stands for the spherical Bessel function of the first kind, $s = 0.9$ fm and $R = 3.9$~fm for $^{40}$Ar.
%%%%%%%%%%%%%%%%%%%%%%%%%%%%%%%%

\chapter{Trident distributions}\label{app:trident_distributions}
%%%%%%%%%%%%%%%%%%%%%%%%%%%%%%%%%%%%%%%%%%%%%%%%%%%%
\graphicspath{{}{appendices/figs/}{appendices/}}
\subsection{Kinematical Distributions at DUNE ND}
%
\label{subsec:kine}

In this section we explore the trident signal in more detail, showing some relevant kinematical distributions for coherent and diffractive events. For concreteness, and due to its large number of events, we choose to focus on the DUNE ND, only commenting slightly on the signal at the lower energies of SBN and $\nu$STORM. The observables we calculate are the invariant mass of the charged leptons $m^2_{\ell^+ \ell^-}$, their separation angle $\Delta \theta$ and their individual energies $E_\pm$. The flux convolved distributions of these observables are shown for the DUNE ND in neutrino mode in \reffig{fig:DUNE_ND_dist}. In these plots, we sum all trident channels with a given undistinguishable final-state proportionally to their rates, although $\nu_\mu$ initiated processes always dominate. The coherent and diffractive contributions are shown separately and on the same axes, but we do not worry about their relative normalization. Other potentially interesting quantities are the angle between the cone formed by the two charged leptons and the beam, $\alpha_C$, and the angle of each charged lepton with respect to the beam direction, $\theta_\pm$.  These additional observables are explored in \refapp{app:distributions}. We also report the distributions of the momentum transfer to the hadronic system, $Q^2$. Although this is not a directly measurable quantity, it is a strong discriminant between the coherent and diffractive processes. We do not present the antineutrino distributions here, but they are qualitatively similar.

Perhaps one of the most valuable tools for background suppression in the measurement of the $\mu^+\mu^-$ trident signal at CHARM~II, CCFR and NuTeV \cite{Geiregat:1990gz,Mishra:1991bv,Adams:1998yf} was the smallness of the invariant mass $m^2_{\ell^+ \ell^-}$. This feature, shown here on the top row of \reffig{fig:DUNE_ND_dist}, is also present at lower energies, where the distributions become even more peaked at lower values; although, the diffractive events tend to be have a more uniform distribution in this variable. This is also true for the angular separation $\Delta \theta$, where coherent dimuon tridents tends to be quite collimated, with $90\%$ of events having $\Delta \theta < 20^\circ$, whilst diffractive ones are less so, with only $47\%$ of events surviving the cut. This difference is much less pronounced for mixed and dielectron channels, where only half of our coherent events obey $\Delta \theta < 20^\circ$, when $37\%$ of diffractive events do so.

An interesting feature of same flavour tridents induced by a neutrino (antineutrino) is that the negative (positive) charged lepton tends to be slightly more energetic than its counterpart, whilst for mixed tridents muons tend to carry away most of the energy. These considerations are also reflected in the angular distributions. The most energetic particle is also the more forward one. For instance, in mixed neutrino induced tridents, $\sim 80 \%$ of the $\mu^-$ are expected to be within $10^\circ$ of the beam direction, whilst only $\sim 35 \%$ of their $e^+$ counterparts do so (see \refapp{app:distributions} for additional distributions).

Finally, we mention that detection thresholds can also be important for trident channels with electrons in the final-state. Assuming, for example, a detection threshold for muons and electromagnetic (EM) showers of 30 MeV in LAr, we end up with efficiencies of (99\%, 71\%, 77\%, 86\%) for ($\mu^+ \mu^-$, $e^+ e^-$, $e^+ \mu^-$, $e^- \mu^+$) coherent tridents. These efficiencies become (96\%, 91\%, 93\%, 96\%) for diffractive tridents, dropping for $\mu^+\mu^-$ and increasing for all others. For comparison, at the lower neutrino energies of SBND and assuming the same detection thresholds, the efficiencies for coherent and diffractive tridents are slightly lower, (97\%, 57\%, 67\%, 77\%) and (90\%, 81\%, 85\%, 90\%) respectively.

While trident events are generally quite forward going, their angular behaviour is quite interesting. We consider here the angle between the charged lepton cone and the neutrino beam, $\alpha_C$, defined as 
%
\[
\cos{\alpha_C} = \frac{ (\vec{p}_3 + \vec{p}_4)  \vdot \vec{p}_1}{ |\vec{p}_3 + \vec{p}_4| |\vec{p}_1| }\, ,\]  
%
and in the individual angle of the charged lepton to the neutrino beam, $\theta$. For same flavour tridents we define $\theta$ for each charge of the visible final-state, whilst for mixed tridents we use their flavour. We also show the distribution in $Q^2 = {-q^2}$, where $q = (P - P^\prime)$, which is of particular interest when considering coherency and the impact of form factors.

\begin{figure}[H]
\centering
\includegraphics[width=\textwidth]{figs/DUNE_nu_3horn_mll_theta_E.pdf}
\caption{Flux convolved neutrino trident production distributions for DUNE ND in neutrino mode. In purple we show the coherent contribution in $^{40}$Ar and in blue the diffractive contribution from protons as targets only (including Pauli blocking). The coherent and diffractive distributions are normalized independently. The relative importance of each contribution as a function of $E_\nu$
can be seen in Fig.~\ref{fig:RatioCDvsT}.
%
\label{fig:DUNE_ND_dist}}
\end{figure}
%
\begin{figure}[H]
\centering
\includegraphics[width=0.92\textwidth]{figs/DUNE_nu_3horn_aC_Q2_thetapm.pdf}
\caption{Flux convolved neutrino trident production distributions for DUNE ND in neutrino mode in additional variables. In purple we show the coherent contribution in $^{40}$Ar and in blue the diffractive contribution from protons as targets only (including Pauli blocking). The coherent and diffractive distributions are normalized independently. \label{fig:other_dists}}
\end{figure}
%

% \chapter{Trident rates at current facilities}\label{app:rates_other}
% \input{appendices/trident_rates_other}

\chapter{Dark neutrino self-energies}\label{app:loop_masses}
In this appendix, we compute the one-loop corrections to the light neutrino masses in our dark neutrino model, following Refs.~\cite{Kniehl:1996bd,Grimus:2002nk,AristizabalSierra:2011mn}.

We work with the on-shell (OS) renormalization scheme. This is ensured by requiring that the off-diagonal elements of the self-energy be diagonal when the external particles are on their mass shell, and that the residue of the renormalized propagator are equal to one. Note this is only applicable to the off-diagonal entries that involve at least one heavy neutrino, and that the light-light entries are all non-zero and finite at one-loop.

Assuming Majorana neutrino fields, one can write the self-energy tensor in its most general form:
%
\begin{equation}
  \Sigma_{ij} (\slashed{q}) = \slashed{q} {\rm P}_L \Sigma_{ij}^L (q^2) + \slashed{q} {\rm P}_R \Sigma_{ij}^R (q^2) +  {\rm P}_L \Sigma_{ij}^M (q^2) +  {\rm P}_R \Sigma_{ij}^{M*} (q^2),
\end{equation}
%
where by virtue of the Majorana nature the previous terms obey
\[\Sigma_{ij}^L (q^2) = \Sigma_{ij}^{R*} (q^2), \qquad \Sigma_{ij}^M (q^2) = \Sigma_{ji}^M (q^2). \]

\subsection{Self-energy}

For now, we will ignore kinetic and scalar mixing effects, as these can be show to be small given current experimental bounds. The contribution from the scalar fields $s= h^0, \varphi^0$, the goldstones $G = G_h, G_\varphi$ and the vector bosons $V = Z, Z^\prime$ are 

\begin{align*}
 -i\,\Sigma^s_{ij} (p^2) &= (-i)^2 \left(\Delta_s P_R + \Delta^*_s P_L \right)_{ik} \times\\&\qquad\qquad \int \frac{d^d k}{(2\pi)^d} \frac{i (\slashed{p} + \slashed{k} + m_k) }{(p+k)^2 - m_k^2} \frac{i}{k^2 - m_s^2} \left( \Delta_s P_R + \Delta^*_s P_L \right)_{kj},\\
 %
 -i\,\Sigma^G_{ij} (p^2) &= (-i)^2 \left(i\,\Delta_G P_R + i\,\Delta^*_G P_L \right)_{ik} \times\\&\qquad\qquad\int \frac{d^d k}{(2\pi)^d} \frac{i (\slashed{p} + \slashed{k} + m_k) }{(p+k)^2 - m_k^2} \frac{i}{k^2 - \xi_V m_V^2} \left( i\,\Delta_G P_R + i\,\Delta^*_G P_L  \right)_{kj},\\
%
 -i\,\Sigma^G_{ij} (p^2) &= -(-i)^2 \gamma^\mu \left( C_V P_L - C_V^T P_R \right)_{ik} \times \\&\qquad\qquad\int \frac{d^d k}{(2\pi)^d} \frac{i (\slashed{p} + \slashed{k} + m_k) }{(p+k)^2 - m_k^2} \frac{iP_{\mu\nu}}{k^2 - m_V^2} \gamma^\nu \left( C P_L - C^T P_R \right)_{kj},
\end{align*}
with no index summation notation. In the latter term, we defined the vector boson propagator numerator, which we rewrite as
%
\begin{align*}
\gamma^\mu P_{\mu\nu} \gamma^\nu &= \gamma^\mu \left[ g_{\mu\nu} - (1-\xi_V) \frac{k_\mu k_\nu}{k^2 - \xi_V m_V^2} \right] \gamma^\nu\\
&= d - (1-\xi_V) \frac{k^2 - m_k^2}{k^2 - \xi_V m_V^2} - \frac{m_k^2}{m_V^2} \frac{ (k^2 - \xi_V m_V^2) - (k^2 - m_V^2)}{k^2 - \xi_V m_V^2}.
\end{align*}

This allows us to write the relevant part of the self-energy as functions of the scalar two-point loop function
\begin{align}
 B_0 (l, m_a^2,m_c^2) = \mu^{2\epsilon} \int \frac{d^d k}{(2 \pi)^d} \frac{1}{(k^2-m_a^2)( (l+k)^2 - m_c^2 )},
\end{align}
such that
\begin{align}
  \Sigma^s_{ij} (0) \, P_R =& -\frac{\pi^2}{(2\pi)^4} \mu^{d-4} \left[ (\Delta_s)_{ik} m_k B_0 (0,m_k^2, m_s^2) (\Delta_s)_{kj}\right] \, P_R \\
%
 \Sigma^G_{ij} (0) \, P_R =& \frac{\pi^2}{(2\pi)^4} \mu^{d-4} \left[ (\Delta_G)_{ik} m_k B_0 (0,m_k^2, \xi_V m_V^2) (\Delta_G)_{kj}\right] \, P_R \\
%
 \Sigma^V_{ij} (0) \, P_R =& -\frac{\pi^2}{(2 \pi)^4} \mu^{d-4} \left[ (C_V)_{ik} m_k \,f( m_k^2, m_V^2, \xi m_V^2) (C_V^*)_{kj} \right] \, P_R,
\end{align}
where the rearrangement of the boson propagator allowed us to write $f( m_k^2, m_V^2, \xi m_V^2)$ as
\begin{align*}
f( m_k^2, m_V^2, \xi m_V^2) =&\, d\, B_0(0,m_k^2, m_V^2) - (1-\xi_V) B_0 (0,m_V^2, \xi_V m_V^2) +\\ & \frac{m_k^2}{m_V^2} B_0 (0, m_k^2, \xi_V m_V^2) - \frac{m_k^2}{m_V^2} B_0 (0,  m_k^2, m_V^2). 
\end{align*}

Finally, the scalar loop function is given by
\begin{align*}
 B_0 (0,m_a^2,m_b^2) &= \frac{1}{\epsilon} - \gamma_E + \ln{4 \pi} - \int_0^1 dx \ln{\frac{m_a^2-x(m_a^2 - m_b^2)}{\mu^2}}\\
 &= \frac{1}{\epsilon} - \gamma_E + \ln{4 \pi} - \frac{m_a^2}{m_b^2 - m_a^2} \left[ \ln{\frac{m_a^2}{\mu^2}}  - 1 \right] + \frac{m_b^2}{m_b^2 - m_a^2} \left[ \ln{\frac{m_b^2}{\mu^2}}  - 1 \right].
\end{align*}

The finiteness of our final result and its gauge invariance are a consequence of the two following identities
\begin{align}
 \Delta_G = \Delta_S = C \frac{\hat{m}}{m_V} + \frac{\hat{m}}{m_V}C^T, \quad \quad C \hat{m} C^T = 0.
\end{align}

For light neutrinos ($i,j=1,2,3$), the final result reads

\begin{align}
 \Sigma_{ij} P_R  = -\frac{\pi^2}{(2\pi)^4} C \hat{m}\left[ d\, B_0(0,\hat{m}^2,m_V^2) + \frac{\hat{m}^2}{m_V^2} \left( B_0 (0,\hat{m}^2,m_S^2) - B_0(0,\hat{m}^2,m_V^2) \right) \right] C^T P_R.
\end{align}

After significant algebra, and checking the cancellation of divergencies, we arrive at 
%
\begin{align}\label{eq:masses_general}
 m_{ij} = \frac{1}{4\pi^2}\sum_{k=4}^5 \Big[ & C_{ik} C_{jk} \frac{m_k^3}{m_Z^2}F(m_k^2,m_Z^2,m_h^2)  \,+\, D_{ik} D_{jk} \frac{m_k^3}{m_{Z^\prime}^2}F(m_k^2,m_{Z^\prime}^2,m_{\varphi^\prime}^2) \Big],
\end{align} 
%
where
%
\begin{equation} \label{eq:loop_function}
F(a,b,c) \equiv \frac{3 \, \ln{(a/b)}}{a/b - 1}  + \frac{\ln{(a/c)}}{a/c - 1}.
\end{equation}
%

For the full expression, involving mixing, we find ($B_0(x^2) \equiv B_0(0,\hat{m}^2,x^2)$) \textcolor{red}{needs some checking! This is only for light neutrinos}
%
\begin{align*}
%
    \Sigma^M(0) &= C\hat{m}\left\{ 4B_0(m_Z^2) -
\frac{\hat{m}^2}{m_Z^2}\left[B_0(m_Z^2) - c_{\omega^\prime}^2B_0(m_{h^\prime}^2)- s_{\omega^\prime}^2B_0(m_{\varphi^\prime}^2)\right] \right\}C^\text{T}\\
%
&~+D\hat{m}\left\{ 4B_0(m_{Z^\prime}^2) -
\frac{\hat{m}^2}{m_{Z^\prime}^2}\left[B_0(m_{Z^\prime}^2) - c_{\omega^\prime}^2B_0(m_{\varphi^\prime}^2)- s_{\omega^\prime}^2B_0(m_{h^\prime}^2)\right] \right\}D^\text{T}\\
%
&~+s_{\omega^\prime} c_{\omega^\prime} \left\{C\hat{m}\left[B_0(m_{h^\prime}^2) -B_0(m_{\varphi^\prime}^2) \right]D^\text{T}+D\hat{m}\left[ B_0(m_{h^\prime}^2) -B_0(m_{\varphi^\prime}^2) \right]C^\text{T}\right\}.
%
\end{align*}

\end{appendices}

\bibliographystyle{JHEP}
\bibliography{introduction/introduction,theory/theory,tridentSM/tridentSM,Zprime_scattering/Zprime_scattering,dark_nus/three_portals,dark_nus/nu_masses,miniboone/miniboone,appendices/loop_masses}

\end{document}
