In this appendix we derive some key results for the phase space treatment we use in calculating cross sections and decay rates. We begin with the factorization of $N$-final state phase-space factors into $N-2$ two-body ones. In general, the $N$-body phase space can be written as 

\begin{equation}
 \dd \Phi_{N} (P,\{p_i\}) = (2 \pi)^4 \delta^4 (P - \sum_i^N p_i) \, \prod_i^{N}\, \frac{\dd^3 p_i}{(2\pi)^3 2 E_i},
\end{equation}
where $\{p_i\} = p_1, \ldots, p_N$. Focusing on the 1-2 subsystem with total momentum $p_{12} = p_1 + p_2$, we can write

\begin{align}
  \dd \Phi_{N} (P,\{p_i\}) =& \int \dd^4 p_{12} \, \delta^4 (p_{12} - p_1 - p_2) \, (2 \pi)^4 \delta^4 (P - \sum_i^N p_i) \, \prod_i^{N}\, \frac{\dd^3 p_i}{(2\pi)^3 2 E_i}  \nonumber\\
 =& \int \dd^4 p_{12}\, \dd \Phi_2 (p_{12}, p_1, p_2) \, (2 \pi)^4 \delta^4 (P - p_{12} - \sum_{i=3}^N p_i) \, \prod_{i=3}^{N}\, \frac{\dd^3 p_i}{(2\pi)^3 2 E_i}\nonumber\\
 =& \int \dd^4 p_{12} \, \dd m_{12}^2 \, \delta(p_{12}^2 - m_{12}^2) \, \dd \Phi_2 (p_{12}, p_1, p_2)\nonumber\\ &\hspace{20ex} \times  (2 \pi)^4 \delta^4 (P - p_{12} - \sum_{i=3}^N p_i) \, \prod_{i=3}^{N}\, \frac{\dd^3 p_i}{(2\pi)^3 2 E_i},
\end{align}
which from 

\[\int \dd^4 p_{12} \, \delta(p_{12}^2 - m_{12}^2) = \int \dd^4 p_{12} \, \frac{\delta (E_{12} - \sqrt{m_{12}^2 + |\vec{p}_{12}|^2})}{2\sqrt{m_{12}^2 + |\vec{p}_{12}|^2} } = \int \frac{\dd^3 p_{12}}{2 E_{12}}, \]
yields the final results
\begin{equation}
 \dd \Phi_{N} (P, p_1, \ldots, p_N) = \frac{\dd m_{12}^2}{2 \pi} \dd \Phi_2 (p_{12}, p_1, p_2) \dd \Phi_N (P, p_{12}, p_3,\ldots,p_N). 
\end{equation}

This result not only lets us factorize any resonant features in phase space, but also provides a recipe to tackle the kinematics of any process in terms of a series of 2-body problems, which are much simpler. The 2-body phase-space factors and associated four-momenta in the respective center-of-mass (CM) frame can always be written as

\begin{align}
\dd \Phi_2 (p_{12}, p_1, p_2) &= \frac{{\lambda^{1/2} \left( 1, m_1^2/E_{12}^{ {\rm CM} \, 2}, m_2^2/E_{12}^{ {\rm CM} \, 2} \right)} }{32 \pi^2 } \dd \Omega^{\rm CM},\nonumber\\
p_{12} &= \left(E_{12}^{ {\rm CM}}, \vec{0} \right),\nonumber\\
p_{1}  &= \left(\frac{E_{12}^{ {\rm CM} \, 2}+m_1^2-m_{2}^2}{2E_{12}^{ {\rm CM} \, 2}}, |\vec{p}_1| \sin{\theta} \cos{\phi}, |\vec{p}_1| \sin{\theta} \sin{\phi} , |\vec{p}_1| \cos{\theta} \right),\nonumber\\
p_{2}  &= \left(\frac{E_{12}^{ {\rm CM} \, 2}+m_{2}^2 - m_1^2}{2E_{12}^{ {\rm CM} \, 2}}, -\vec{p}_1 \right),
\end{align}
where $\lambda(a,b,c) = (a - b -c)^2 - 4 b c$ is the K\"all\'en function. Now, the problem is reduced to finding the CM frame of every $p_{ij}$ subsystem, and the transformation between all such frames, if necessary. Of course, the dependence of the matrix element on the kinematics makes certain phase space parametrization better than others, making each problem unique. Lab variables, for instance, are the standard parametrization for DIS scattering as they preserve crucial physical intuition of the process at hand.

\paragraph{Neutrino trident production} Now we derive a phase space parametrization for neutrino trident production in terms of the momentum transfer $K^2 =  2 p_1 \vdot p_2$. This is important if one wants to change variables to smooth out the integrand at low $M_{Z^\prime}$ masses. We follow the calculation in \cite{Czyz1964} and \cite{Ballett:2018uuc}, and proceed to define $K^2$ as one of the integration variables. The relevant Lorentz invariant phase space for the $2\to3$ leptonic part of the cross section is given by
%
\begin{align}
\int \dd^3 & \Pi_{\mathrm{LIPS}} = \nonumber\int \frac{\dd \vec{p_2} }{(2\pi)^32 E_2} \frac{ \dd \vec{p_3} } {(2\pi)^3 2 E_3} \frac{\dd \vec{p_4}}{(2\pi)^32 E_4} \\& (2\pi)^4\delta^{(4)} (p_1 + q - p_2 - p_3 - p_4).
\end{align}
%
Following \cite{Czyz1964} we start by working in the frame $\vec{p_1} + \vec{q} - \vec{p_3} = 0$, putting $\vec{p_1}$ along the $\hat{z}$ direction instead. The delta function can be integrated with the $\vec{p_4}$ and $|\vec{p_2}|$ integrals, such that 
%
\begin{equation}
 \int \frac{\dd \vec{p_2} }{2 E_2} \frac{\dd \vec{p_4}}{2 E_4} \, \delta^{(4)} (p_1 + q - p_2 - p_3 - p_4) =  \int \frac{ |\vec{p_2}|}{4 W_c} \, \frac{1}{E_1 E_2} \dd K^2 \, \dd \phi_2,
\end{equation}
%
where we defined
%
\begin{align}
&|\vec{p_2}| = (W_c^2 - m_1^2)/2W_c, \nonumber\\\nonumber &W_c = q^0 + E_1 - E_3, \\ &K^2 = 2 E_1 E_2 (1 - \cos{\theta_2}).
\end{align}
%
Since we conserve energy and momentum in this frame, we can take $-1 \leq \cos{\theta_2} \leq 1$ and $0 \leq \phi_2 \leq 2 \pi$. The remaining $\vec{p_3}$ integral can be performed with the variables defined in \cite{Czyz1964} to yield
%
\begin{equation}
 \int \frac{\dd \vec{p_3} }{2 E_3}  =  \int \frac{2 \pi}{\hat{s}} \, \dd x_5 \, \dd x_3,
\end{equation}
%
where a trivial azimuthal angle was integrated over. Their limits are more easily found in the frame $\vec{p_1} + \vec{q} = 0$, with $\vec{q}$ along the $\hat{z}$ direction. Finally, our main result is given by
%
\begin{equation}
\int \dd^3 \Pi_{\mathrm{LIPS}} = \frac{1}{(2\pi)^4} \int \frac{|\vec{p_2}|}{4 W_c} \, \frac{1}{\hat{s}} \, \frac{1}{E_1 E_2} \, \dd x_5 \, \dd x_3 \, \dd K^2 \, \dd \phi_2 .
\end{equation}
%
There remains two non-trivial integrations to be performed to obtain the full 4-body phase space cross section, namely the ones over $q^2$ and $\hat{s}$. The substitutions suggested in \cite{Lovseth1971} for these two invariants are still convenient, and we make use of them in our numerical integrations.
