Particle physics is at a very important moment of its history. The Standard Model (SM) provides unprecedented accuracy when describing particle physics data, but it provides no explanation for neutrino masses and the existence of dark matter (DM). It is tempting to think that these phenomena are to the Standard Model what blackbody radiation was to classical physics in the beginning of the 20th century, a scientific revolution on the wait. While we cannot be sure, we continue to devise new theoretical explanations and new methods to test them. In particular, the experimental efforts in neutrino physics open new possibilities to test the connections between neutrinos and beyond the SM physics. New detector technologies, such as liquid Argon (LAr), and more powerful neutrino beams, allow us to study neutrino interactions in great detail and mimic conditions of high intensity fixed-target and beam-dump facilities. 

Rare and well-understood neutrino scattering processes offer a unique tool to test the SM weak interactions. We have shown that measuring the neutrino trident production ($\nu \, A \, \to \, \nu \, \ell \, \ell \, A$) cross section at GeV energies will be an attainable goal of near future experiments such as the Deep Underground Neutrino Experiment (DUNE). The backgrounds in LAr are expected to be manageable within our assumptions for particle identification capabilities. Our calculation for the cross section makes it explicit the poor performance of the Equivalent Photon Approximation for this process, and provides an estimate for trident rates at various current and future neutrino facilities. In Chapter 4, we assess the sensitivity of DUNE to new anomaly-free leptophilic $U(1)_{L_\alpha - L_\beta}$ groups. The new $Z^\prime$ gauge bosons can be searched for in leptonic and semi-leptonic neutrino scattering processes, such as neutrino trident and neutrino-electron scattering ($\nu \, e \to \nu \, e$). We showed that for $L_e - L_\mu$, the new boson may be searched for in neutrino-electron scattering measurements, provided the neutrino flux uncertainties are kept under control. For $L_\mu - L_\tau$, neutrino trident production of dimuon pairs can set strong bounds on the coupling and masses of the $Z^\prime$ boson, otherwise much harder to constrain since it does not couple to the first SM generation of particles. We show that if the number of non-trident background events exceeds that of the number of SM tridents, then DUNE starts to lose its ability to probe the entire parameter space able to explain the muon $(g-2)_\mu$. 

A new set of models for low energy phenomenology in neutrino experiments has been developed in Chapter 5. The model contains a new dark neutrino state $\nu_D$, charged under a hidden $U(1)^\prime$ local gauge symmetry, which in turn is broken by the vev of a new scalar $\Phi$. The setup realises all three neutral and renormalizable portals to hidden sectors: the scalar, vector and neutrino portals. With the presence of a completely neutral state $N$, this closely resembles other low-scale seesaw models like the inverse and extended seesaw. We show how the phenomenology is very different from having each portal taken individually, opening up parameter space to explain experimental anomalies such as the muon $(g-2)$, and the MiniBooNE low energy anomaly. As it turns out, the model also radiatively generates light neutrino masses, while remaining testable at the MeV scale.  

Phenomenological realizations of the dark neutrino model had already been put forward as explanations of the excess of electron-like events at MiniBooNE. As shown in Chapter 6, light $Z^\prime$ scenarios are severely constrained by neutrino-electron scattering measurements at accelerator neutrino experiments. We proposed a new technique to constrain these models by investigating sideband data in the MINER$\nu$A low energy measurement, as well as in the past measurements performed by CHARM-II. By using simplified rate analysis in sideband regions of MINER$\nu$A and CHARM-II, as well as computing the MiniBooNE angular spectrum, we showed that the region where both energy and angular distributions at MiniBooNE can be explained in this model are in severe tension with neutrino-electron scattering data. Although not all parameter space is excluded, our work highlights the importance of the coherent photon-like sidebands in neutrino-electron scattering measurements and paves the way for future analyses at MINER$\nu$A, NO$\nu$A, and DUNE, eventually.

Neutrino beams from stored muons can bring great improvements to laboratory neutrino experiments. The \nus project is a first step towards neutrino factories and muon colliders, and may provide a definite test of whether short-baseline oscillations due to eV sterile neutrinos exist. Beyond testing existing anomalies, it could provide the most stringent limits on non-unitarity of the PMNS matrix due to light sterile neutrinos. Our analysis also highlighted the peculiarities of the beam, with implications for production localisation, short-baseline CP violation, and displaying a variety of oscillation channels in a single experiment.

Neutrino physics is a field full of exciting and unexpected results. Neutrino oscillations have marked the beginning of our exploration of beyond the SM physics, but much more is yet to be learned. Beyond studying the physics of neutrino flavour, neutrino oscillation and scattering experiments are a unique tool to probe the weak interactions and to search for new particles. The theoretical models, novel measurements, and analyses techniques we have proposed in this thesis are intimately connected to the unique properties of neutrinos. Whether neutrinos are indeed a gateway to dark sectors, or just what we need to rule this possibility out, we are confident that the bright future of neutrino physics will shine light on what lies beyond the SM.