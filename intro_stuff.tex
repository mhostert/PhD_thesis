INTRO STUFF

TRIDENT

\section{Introduction}
\label{sec:intro}

The Standard Model (SM) has been confronted with a variety of experimental data and has so far emerged as an impressive phenomenological description of nature, except in the neutrino sector. The observation of neutrino flavour oscillations by solar, atmospheric, reactor and accelerator neutrino experiments over the last 50 years has revealed the existence of neutrino mass and flavour mixing, making necessary the first significant extension of the SM.

The precise determination of the neutrino mixing parameters as well as the search for the neutrino mass ordering and leptonic CP violation drive both present and future accelerator neutrino experiments. To accomplish these tasks, these experiments rely on state-of-the-art near detectors, made of heavy materials, located a few hundred meters downstream of the neutrino source and subjected to a high intensity beam. Their main purpose is to ensure high precision measurements at a far detector by reducing the systematic uncertainties related to neutrino fluxes, charged-current (CC) and neutral-current (NC) cross sections and backgrounds.  
%