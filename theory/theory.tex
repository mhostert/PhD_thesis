\graphicspath{{}{theory/}{Diagrams/}}


\section{The three neutrino paradigm}

Neutrinos are special in many ways. They are the only singlets under the broken symmetry group of the SM we know of, and the only fermion that is strictly massless in the SM. In this chapter we will explore some of the theoretical and experimental status of neutrino physics. We start with a few theoretical considerations about neutrino oscillations in vacuum and in matter, and why this phenomenom is so important. We then move on to study the most abundant neutrino interactions in oscillation experiments, as they provide a direct probe of the weak interactions. We will find that it is hard to ignore the strong force in many of these cases. Later, we will have a quick overview of the most popular models to explain the non-zero neutrino masses beyond the SM. 
% Finally, interactionsone can expect that neutrino experiments resemble the kind setup one would design to look for new dark states. Exploring this resemblence in new physics searches ``under the lightpost'', using the \emph{intesity frontier} in a opportunistic way. 

Some history --Madame Wu + Freinberg numus + Lederman + Paul Davies + 

\section{Oscillations}

Neutrino oscillations arise when a superposition of neutrino mass eigenstates is produced, propagates macroscopic distances and scatters inside a detector. Given sufficiently long baselines, neutrino flavour transitions may always occurs due to mixing, but for a non-trivial dependence on baseline distances, coherence must be preserved throughout the process. In this section, we will make these statements more precise and derive the standard formula for the oscillation probability in vacuum, $P(\nu_\alpha \to \nu_\beta)$ (matter effects are discussed in \refsec{sec:matter_effects}). This exercise can be done in multiple ways and most often derivations rely on plane-wave neutrino states. This approach leads to correct expressions for $P(\nu_\alpha \to \nu_\beta)$ in virtually all cases of interest, but it is a rather poor conceptual description of oscillations and relies on unphysical assumptions. Instead, we will derive the oscillation formula from a quantum mechanical wave-packet approach, and encounter a few conditions for oscillations to happen. More sophisticated treatments in Quantum Field Theory (QFT) have been known for some time~\cite{Cardall:1999ze,Beuthe:2001rc,Giunti:2002xg}, and their results have been shown to be directly mapped onto the wave-packet approach~\cite{Akhmedov:2010ms}. Nonetheless, neutrino oscillations are notorious for being conceptually confusing and the correct method to compute such processes is still debated in the literature~\cite{Kobach:2017osm}. We may seek consolation in the fact that a few aspects are common to all approaches, for instance, the ultra-relativistic nature of the mass states through expansions of $\sqrt{m^2 + p^2}$ and the need for momentum uncertainties in the initial and final neutrino processes.
%
\begin{figure}[t]
\centering
\includegraphics[width=\textwidth]{oscillations_diagram.pdf}
  \caption[Neutrino oscillations diagram.]{The usual set-up of an oscillation experiment. We show the source, where a process $P_\textsc{i} \to \nu_\alpha \, X_\textsc{i}$ happens (note that $P_\textsc{i}= \ell_\alpha$ is allowed and that $P_\textsc{i}$ may be a scattering process), and the detector, where $\nu_\beta \,P_i \to \ell_\beta \,X_\text{f}$. \label{fig:oscillations_diagram}}
\end{figure}
%

Our setup typical of neutrino oscillations experiments and is represented in \reffig{fig:oscillations_diagram}. We will first discuss the role of production and detection processes, and then later study the oscillations \emph{per se}. Initially, a neutrino flavour state $\nu_\beta$ is produced in a CC reaction at the source, $P_\textsc{i} \to \nu_\beta X_\textsc{i}$. More precisely, a state $\bra{f}$ is produced,
%
\begin{equation}\label{eq:flavour_source}
 \ket{f} = \hat{S} \ket{P_\textsc{i}}, \quad \hat{S} \approx  \hat{1} - i \int \dd^4 x \, H_{\rm int}^{\rm CC} (x),
\end{equation}
%
where $\hat{S}$ is the S-matrix operator approximated to first order in weak coupling and
%
\begin{equation}
 H_{\rm int}^{\rm CC} (x) = \sqrt{2} G_F\, \sum_\alpha \overline{\nu}_\alpha (x) \gamma^\mu P_L \ell_\alpha(x) \, J_\mu (x) \,+\, {\rm h.c.},
\end{equation}
%
is the interaction Hamiltanian between the neutrino current and the current $J_\mu(x)$ that describes the transition $P_\textsc{i} \to X_\textsc{i}$. After $\ell_\alpha$ and the final particles interact with the medium, $\bra{f}$ is reduced to $\bra{\ell_\alpha,\,X_\textsc{i}}\ket{f}$, which ought to ensure that the neutrino state produced is a superposition of physical states weighed by the PMNS matrix elements $U_{\alpha i}^*$ and other factors. Note that this may differ from the usual definition of a flavour state, which can be misleading in this situation as these do not span a Fock space (for a recent and illuminating discussion on this issue, see Ref.~\cite{Cozzella:2018zwm}). We want to work instead with the eigenstates of the free Hamiltonian, which are the ones we can easilly evolve in time. With this in mind, we define the following amplitude
%
\begin{equation}
 A_{\alpha k}(\vec{p},h)^P \equiv \bra{\nu_k(\vec{p},h),\ell_\beta,X_{\textsc{i}}} \hat{S}\ket{P_\textsc{i}}, \quad {\rm with} \quad A_{\alpha k}(\vec{p},h) = U^*_{\alpha k} M_{\alpha k}(\vec{p},h),
\end{equation}
%
where we factored out a mixing angle in the definition of $M_{\alpha k}$ and made the helicity index $h$ explicit. By virtue of the completeness relation with massive neutrino eigenstates, we can insert the identity in $\bra{\ell_\alpha,\,X_\textsc{i}}\ket{f}$ and define a \emph{normalized} neutrino flavour state as
%
\begin{equation}
 \ket{\nu_\alpha}^P = N_P \,\sum_{k, h} \, \int \dd^3 p\, A^P_{\alpha k} (\vec{p},h) \ket{\nu_k(\vec{p},h)}, \quad N_P^{-2} = \sum_{k,h} \int \dd^3 p \, \left|A^P_{\alpha k} (\vec{p},h)\right|^2.
\end{equation}
%
An analogous discussion holds for the detection process $\nu_\beta \, P_\textsc{f} \to \ell_\beta \, X_\textsc{f}$, where a detection flavour state $\ket{\nu_\alpha}^D$ with an amplitude for detection $A_{\alpha k}(\vec{p},h)^D$ can be defined. Before we move on to a discussion about oscillations, we want to emphasize two points. First, the normalization of the flavour state is a clear sign that we are working in a quantum mechanical description. To compute probabilities, we rely on normalized states. In a QFT description, however, the normalization is not necessary, but so is the concept of $P_\textsc{i} \to X_\textsc{i}$ in the first place. There, the full process in \reffig{fig:oscillations_diagram} can be compute directly through the use of long-distance propagators, and if the production, propagation and detection parts of the amplitude squared factorize, an object equivalent to the oscillation probability can be extracted. This factorization is implicitly assumed in our calculation. Secondly, the decay rate of the $P_\text{I}$ particle can be computed as
%
\begin{equation}
 \left|A^P\right|^2 =  \left|  \bra{\nu_\alpha (\vec{p},h),\ell_\beta, X_{\textsc{i}}} \hat{S} \ket{P_\textsc{i}}  \right|^2 = \sum_{k,h} \,|U_{\alpha k}|^2\, \int \dd^3 p \, \left|M^P_{\alpha k} (\vec{p},h) \right|^2,
\end{equation}
%
and it becomes evident that the decay rate is given by the \emph{incoherent} sum of the decay rate into different massive neutrinos. No interference is present as the states $\ket{\nu_k}$ are assumed to be orthonormal to each other. This remains true in the QFT description~\cite{Giunti:2002xg}.

Now, one is left to compute the functions $M_{\alpha k}$. This is a rather involved process, but one can show that the form of these functions resemble simple wavepackets~\cite{Akhmedov:2010ms}. In doing so, many approximations are necessary, in particular, that of ultra-relativistic neutrinos. More precisely, the most relevant assumptions are \emph{i)} flipped-helicity terms ($h=+1$ for neutrinos), suppressed by $m_k^2/E_k^2$, are ignored, \emph{ii)} all neutrinos travel in the same direction, $\vec{p} \to p$, and \emph{iii)} the production and detection processes are not sensitive to the neutrino mass differences, amounting to replacing $M_{\alpha k} \approx M_\alpha$. Under these assumptions, we are justified to take normalized gaussian wavepackets for production and detection flavour states as an \emph{ansatz},
%
\begin{equation}
 \ket{\nu_\alpha}^i =  \sum_{k}  U^*_{\alpha k} \, \int \dd p \, \psi_k^i (p) \,\ket{\nu_k({p})}, \quad%
 %
 \psi_k^i (p) = \left(2\pi\,\sigma_p^{i\, 2}\right)^{-1/4} \exp\left[ - \frac{(p-p_k)^2}{4 \sigma_p^{i\, 2}} \right],
\end{equation}
%
with $\sigma_p^i$ being the spread around the central momenta $p_k$ and $i=P,D$.

Now that the flavour states are written in terms of the eigenstates of the free Hamiltonian, we know how to evolve them and how to write the flavour transition amplitude after a time $t$ and distance $L$
%
\begin{align}
 A(\nu_\alpha \to \nu_\beta) &= \bra{\nu_\beta^D} e^{-i \hat{E} t + i \hat{P} L} \ket{\nu_\alpha^P}
 \nonumber \\ &= N \,\sum_k U_{\alpha k}^* U_{\beta k} \int \dd p \exp\left[ -i E_k(p) t + ip L - (p -p_k)^2/4 \sigma_p^2\right],
\end{align}
where $E_k(p) = \sqrt{p^2 + m_k^2}$ and N is a normalization factor coming from the normalization of the wavepackets and a single integral over $p$. We have also defined the global energy uncertainty on momentum $\sigma_p^{-2} = \left(\sigma_p^{P}\right)^{-2} + \left(\sigma_p^{D}\right)^{-2}$. This may also be related to the global uncertainty on production and detection positions through $\sigma_x \sigma_p \approx 1/2$. Finally, to integrate over the remaining $p$ integral, we can taylor expand around the central wavepacket momentum
%
\begin{equation}
 E_k(p) \approx E_k + v_k (p-p_k), \quad{\rm with}\quad v_k = \left.\frac{\partial E_k(p)}{\partial p}\right|_{p=p_k} = \frac{p_k}{E_k},  \quad E_k = \sqrt{p_k^2 + m_k^2}.
\end{equation}
%
Perfomrming the final integral over $p$, integrating over $t$ (an unmeasured quantity) and squaring the amplitude, one obtains a formula for the oscillation probability
%
\begin{equation}
 P(\nu_\alpha \to \nu_\beta) = \sum_{k,j} U_{\alpha k}^*U_{\alpha j}U_{\beta k}U_{\beta j}^* \, e^{-2\pi i L/L^{\rm osc}_{kj}} \, P^{\rm coh}_{kj} \, P^{\rm loc}_{kj},
\end{equation}
%
where we defined
%
\begin{equation}
  P^{\rm loc}_{kj} = \exp\left( -2 \pi^2 \xi^2 \left(\frac{\sigma_x}{L_{kj}^{\rm osc}} \right)^2 \right),\quad  P^{\rm coh}_{kj} = \exp\left( \frac{L \left|\Delta m^2_{kj} \right|^2}{16 E^2 \sigma_x} \right),
\end{equation}
%
and the important scales of the problem can be identified as
%
\begin{equation}
 L_{kj}^{\rm osc} = \frac{4 \pi E}{\Delta m_{kj}^2}, \quad  p_k \approx E - (1-\xi) \frac{m_k^2}{2E}, \quad E_k \approx E + \xi \frac{m_k^2}{2E},
\end{equation}
with $\xi$ measuring the deviation from ultra-relativistic behaviour.
%

For most application, $P^{\rm coh}_{kj} = P^{\rm loc}_{kj} = 1$, and one recovers the standard oscillation formulae. A more useful way of writing it is
%
\begin{align}
 P(\nu_\alpha \to \nu_\beta) =
 \delta_{\alpha\beta} &- 2\sum_{k>j} \Re{U_{\alpha k}^*U_{\alpha j}U_{\beta k}U_{\beta j}^*} \left[ 1- \cos\left( \frac{\Delta m^2_{kj} L}{2E}\right) \right] \nonumber\\
 %
 &- 2\sum_{k>j} \Im{U_{\alpha k}^*U_{\alpha j}U_{\beta k}U_{\beta j}^*} \sin \left( \frac{\Delta m^2_{kj} L}{2E}\right).
\end{align}
%

\subsection{Matter effects}\label{sec:matter_effects}

Neutrinos are neutral particles and their rare interactions allow them to propagate through matter without losing energy in collisions with the medium particles. Nevertheless, in a similar fashion to photons, neutrinos undergo coherent forward scattering, acquiring an effective refractive index in the presence of a medium. In contrast to photons, which undergo Compton scattering, neutrinos are only charged under the weak force and undergo CC and NC interactions. The weakness of these interactions at low energies implies that matter effects arise only when neutrinos have transversed sufficiently large distances or are in a sufficiently dense environment. In principle, for matter effects all that is needed is a net weak charge for the medium, provided by the neutrons in the case of the Earth. However, for such effects to be observable in neutrino oscillation experiments, the additional condition that different flavour states exhibit different interaction potential with the medium must be ensured. In the SM, this is possible due to the CC interactions that are exclusively present between electron-neutrino states and electrons in the medium.

We want to avoid the complications from the previous discussion due to coherence and focus on the effects of matter. Similar conditions to the ones we found in the previous section apply for oscilation probabilities in matter to be well-defined, and with this caveat we proceed with the plane-wave picture. By applying the same momentum approximation and assuming neutrinos to be relativistic we can write the Schr\"oedinger equation in matrix form as 
%
\begin{equation}\label{eq:matter_evolution}
 i \frac{\dd}{\dd x} \ket{\nu_\alpha} = \left[ U \frac{\hat{m}^2}{2E} U^\dagger + \hat{V}(x) \right] \ket{\nu_\alpha}, 
\end{equation}
%
where we used $H_0 \ket{\nu_\alpha} \approx U \left[ p \hat{1} + \hat{m}^2/2 p \right] U^\dagger \ket{\nu_\alpha} $ and $t \approx x$. Here $\hat{V}(x)$ is a matrix containing the interaction potential of each neutrino flavour with the background, which depends on the density of matter particles. Solving this equation is a much more complicated task than in the vacuum case. Full analytical solutions are only known in specific cases, such as when the column density of matter particles is constant. In general, this may be solved numerically for a given choice of $\hat{V}(x)$, although several perturbative expansions exist. In this sense, the problem reduces to finding the appropriate potential and solving \refeq{eq:matter_evolution}.

Various calculations of the neutrino matter potential exist, but often these rely on statistical averages over the particle background and are difficult to generalize to non-standard cases. Here, we sketch the approach of Refs.~\cite{Notzold:1987ik,Nieves:1989ez}, where the neutrino potential arises from finite temperature and finite density corrections to the neutrino dispersion relation. In particular, the dispersion relations arises from
%
\begin{equation}
 \det{\slashed{k} - \Sigma} = 0,
\end{equation}
%
guaranteeing non-trivial solutions to the Dirac equation $(\slashed{k} - \Sigma)\nu_L = 0$, with $k^\mu$ the neutrino four-momentum and $\Sigma$ its self-energy. For LH neutrino states $\nu_L$, we can write the neutrino self-energy in the most general form and make explicit the background dependent contribution as~\cite{Weldon:1982aq}
%
\begin{equation}
  \Sigma = m - \left( a_L \slashed{k} + b_L \slashed{u} + c_L [\slashed{k},\slashed{u}] \right) P_L.
\end{equation}
%
where $u$ is the 4-velocity of the medium, $a_L,b_L$ and $c_L$ are scalar functions of Lorentz invariants $w=k \cdot u$ and $\kappa=(w^2 - k^2)^{1/2}$, and $m$ is the vacuum neutrino mass. The presence of the medium introduces a preferential frame, namely the rest frame of the medium with $u = (1,0,0,0)$. Also note that in vacuum, only terms proportional to $\slashed{k}$ exist, and the pole of the neutrino propagator is unchanged. To lowest order, $g^2/m_\textsc{w}^2$, only $b_L$ contributes and it is proportional to the medium particle-antiparticle asymmetry, both statements not holding for higher order terms of the form $g^2/m_\textsc{w}^4$~\cite{DOlivo:1992lwg}. The neutrino self-energy is in fact a gauge-dependent quantity, so the physical observables of interest are the dispersion relations, $(1-a_L)(w-\kappa)- b_L = 0$ for neutrinos and $(1-a_L)(w+\kappa) - b_L = 0$ for antineutrinos. To lowest order, however, the dispersion relations are much simpler,
%
\begin{equation}
 w \approx \kappa + \frac{m^2}{2\kappa} + V_{\rm eff}, \quad V_{\rm eff} = -b_L,
\end{equation}
%
where we defined the effective potential, which for ultra-relativistic neutrinos arises precisely from the difference between the total and kinetic energy $V_{\rm eff} = w - \kappa$. This also shows us how to calculate the neutrino refractive index $n =|\vec{k}|/k^0$.
%
\begin{figure}[t]
 \includegraphics[width=\textwidth]{thermal_diagrams.pdf}
  \caption[Finite temperature corrections to $\Sigma$.]{Finite temperature and density corrections to the neutrino self-energy. These can be used to infer the effective matter potential for neutrinos.\label{fig:thermal_diagrams}}
\end{figure}

Now the problem reduces to computing $\Sigma$ in finite temperature field theory. For most applications of thermal mass calculations, replacing vacuum propagators by the thermal propagators from the real-time formalism is sufficient. In particular, the fermion thermal propagator of interest is
%
\begin{equation}
 S(P) = (\slashed{p} + m)\left[ \frac{1}{P^2 - m^2 +i\epsilon } + i 2\pi \delta(P^2 - m^2) f(P)\right],
\end{equation}
%
with $f(P) = \left\{ \exp\left[(|P\cdot u| - {\rm sgn}(P \cdot u) \,\mu_f)/T\right] + 1\right\}^{-1}$ is the occupational number of the fermions in the thermal bath of temperature $T$ and chemical potential $\mu_f$. Similar expressions exist for bosonic propagators. Finally, as an example, explicit computation of the self-energy in \reffig{fig:thermal_diagrams} yields
%
\begin{equation}
 \Sigma = -i \frac{g^2}{16 c_\textsc{w}^2}\int \frac{\dd^4 P}{(2\pi)^4} \gamma^\mu P_L \,iS(P+K) \,\gamma^\nu \, P_L \,iD_{\mu\nu} (P),  \quad D_{\mu\nu} = \frac{-g_{\mu\nu} + \frac{P_\mu P_\nu}{M_Z^2}}{P^2 - M_Z^2 + i\epsilon}.
\end{equation}
%

Putting it all together, the potential for neutrinos of flavour $\alpha$ on a background of protons, neutrons and electrons is
%
\begin{align}
 V^e_\alpha =& -\frac{G_F}{\sqrt{2}} \left( 2\delta_{\alpha e} 1-4s^2_\textsc{w} \right) \left( N_e - N_{\overline{e}} \right),\nonumber\\%
 V^p_\alpha =& \frac{G_F}{\sqrt{2}} \left( 1-4s^2_\textsc{w} \right) \left( N_p - N_{\overline{p}} \right),\nonumber\\%
 V^n_\alpha =& -\frac{G_F}{\sqrt{2}} \left(N_n - N_{\overline{n}} \right).
\end{align}
%
For antineutrino an overall minus sign is introduced. Note the $\nu_e$ potential is the only one where CC interactions contribute, and so it is the sole responsible for non-trivial flavour evolution in \refeq{eq:matter_evolution}. One may wonder about radiative corrections to these potentials in the SM and whether additional flavour non-universality can be achieved through the difference in charged-lepton masses. These effects, however, are known to be extremely small in the SM~\cite{Botella:1986wy}, where $\left(V_\tau - V_\mu\right)/V_e \approx 5 \times 10^{-5}$ for a neutral unpolarized medium like the Earth. 




\subsection{Sources}\label{sec:sources}

To find a source of neutrinos, all we have to do is to look for environments where the Weak force is prominently manifested. A natural candidate, as we have seen in the discovery of the neutrino, are nuclear reactors. Fortunately, the list does not stop there. Abundant neutrino sources include the Sun, the atmosphere, the Big-Bang, particle accelerators and astrophysical environments processes such as supernovae, active galactic nuclei and others. 

\section{Scattering}

Neutrino cross sections are an invaluable observable to understand the weak force and to search be able to study oscillation physics. Due to the purely weakly interacting nature of neutrinos, they also serve as a strong test of \emph{stronger than weak} interactions with matter, that is, new interactions with $G_X > G_F$. In fact, if sufficiently clean, neutrino scattering processes provide the strongest limits on light new mediators of masses from a few MeV to a few GeV~\cite{}. In this section, we present the standard scattering channels used in oscillation experiments, as well as other curious processes that are relatively poorly studied. 

Accelerator experiments typically produce neutrinos of a few GeV energies to achieve $\mathcal{O}(1)$ oscillation phases $\Delta m^2_{\rm atm} L/E$ within thousands of km. Drastically different energy regimes are impractical either due to diluted fluxes at longer baselines ($\Phi \propto 1/L^2$), or due to thresholds to produce muons in CC interactions ($E_\nu > m_\ell + m_\ell^2/2 m_{\mathcal{H}}$ in reactions of the type $\nu_\ell \mathcal{H} \,\to\, \ell^\pm \mathcal{H}^\prime$). Also important is the fact that such experiments rely on the neutrino flux produced in proton-on-target collisions, where charged mesons are subject to magnetic fields for focusing (see \refsec{sec:sources}). This produces neutrinos with a wide spectrum, typically referred to as wide-band-beam spectram, and are hard to model due to hadroproduction and focusing uncertainties~\cite{}. In this way, the expected neutrino event rate in neutrino detectors inherits two sources of uncertainties which are difficult to disentangle: the un-oscillated flux spectra and the neutrino-nucleus cross sections. The latter observable is subject to large nuclear effects, which require either precision data or accurate descriptions of the nuclear environment. 

Take the MiniBooNE experiment, for instance, where the average neutrino energy is $\langle E_\nu \rangle \approx 800 $ MeV. The nucleons struck by the neutrinos in the processes of interst are inside Carbon nuclei, and so their interactions with the nuclear medium are important. Even in the crudest approximation for a nucleus, that of a $T=0$ Fermi gas of free protons and neutrons, nucleons may have quite large momenta in the rest frame of the nucleus, $|p| \lesssim 250$ MeV, being comparable to the incoming neutrino energies. In addition to Fermi motion, the Pauli exclusion principle suppresses possible final state configurations in the nucleus. This is known as Pauli blocking, and typically suppresses the neutrino interactions on nuclei. On their way out of the nucleus, the nucleons may also exchange EM charges, knock-out additional particles or be absorbed in the nuclear medium. The importance of nuclear effects is perhaps most famously illustrated by the CCQE measurement at MiniBooNE~\cite{AguilarArevalo:2010zc}, where a disagreement of $20\%$ was observed between the MC prediction and the data, unless the axial mass $M_A$ was set to be $\approx1.32$ GeV, much larger than the world-average of $M_A = 1.03 \pm 0.02$ GeV. This was later understood as a missing contribution from two-particle two-hole meson exchange currents (where the neutrino interacts with a nucleon pair, rather than with an individual nucleon), shown to be as large as $30\%$ of the CCQE cross section used by the experiment~\cite{Nieves:2011yp}.

Currently several neutrino event generators exists, the most popular being GENIE~\cite{Andreopoulos:2009rq}, GiBUU~\cite{Buss:2011mx}, NEUT~\cite{Hayato:2002sd} and NuWro~\cite{Juszczak:2005zs}.

\section{Mass mechanisms}

A more ambitious task than measuring the low-energy properties of neutrinos is to understand the theoretical origins of neutrino mass and mixing. Here, the possibilities are manifold. The high-scale dynamics, typical of many neutrino mass models, is incredibly hard to test, so it is fair to say we have more neutrino mass models than ways to test them. Nevertheless, many of these models possess similar features. They may rely on the seesaw mechanism, on radiative neutrino mass generation or in extended scalar sectors. On top of that, new symmetries and fundamental forces may also be at play. We will explore a small fraction of this model space.

We have already alluded to the first possibility to introduce light neutrino masses in the SM. All that is needed are at least 2 RH neutrino fields, singlets under all SM symmetries. We shall refer to them as $N^\alpha$, where $\alpha$ is their generation index. This may be chosen to be $\alpha = e, \mu, \tau$ or any combination of 2 of these. The full new neutrino mass Lagrangian then becomes
%
\begin{equation}
 \mathscr{L}_{\nu-{\rm mass}} = \overline{N}^\alpha i\slashed{\partial} N_\alpha + y^\nu_{\alpha\beta} \left( \overline{L}^\alpha \cdot H \right) N^\beta + M_{\alpha \beta} \overline{N^c}^\alpha N^\beta,
\end{equation}
%
where $N^c = C \overline{N}^T$ with $C$ the charge conjugation matrix ($C = i \gamma^0 \gamma^2$ in the chiral basis). The middle term endows neutrinos with Dirac masses, but the latter is a new ingredient. This is the Majorana mass matrix for the new RH neutrinos, and it is allowed by the symmetries of the model. It does, however, violate any $U(1)$ symmetry associated with the fields $N$, as
%
\begin{equation}
 N \to e^{i\theta} N \,\implies\, \overline{N^c} N \to e^{2i\theta}\overline{N^c} N.
\end{equation}
%
So if $N$ are assigned lepton number, then the accidental global symmetries of the SM $B-L$ and $L$ are violated by the Majorana mass term. If we insist and set $L(N)=0$, then the Dirac mass term will, instead, explicitly break these global symmetries. This interesting observation, together with the fact that $B-L$ is anomaly-free with 3 RH neutrinos, had led to many theorists to propose Dirac neutrino mass models with gauged $B-L$. On top of that, the scale of the entries in $\bm{M}$ is not set by the Higgs vev and, therefore, may be wildly different from the EW scale. We may argue that it has to be small, since in the limit that all $M_{\alpha\beta}\to0$, the SM symmetry is enhanced. This choice is then said to be technically natural, in the t'Hooft sense. Regardless of our mechanism to prevent this term, one thing is clear, a purely Dirac neutrino mass model has to deal with the fact that the neutrino Yukawas are extremely small $y^\nu/y^t \approx 10^{-12}$. The smallness of this coupling and the reason for it being so only aggravate the flavour problem. We may be tempted to stop here, but while we continue to look for any experimental evidence that this is not the final word, we cannot rule out the many other alternatives.

In the same way that the $N$ particles admitted a Majorana mass term, we may wonder if the SM may also provide a similar term for the LH states. Clearly $SU(2)$ invariance forbids any renormalizable operator that would give rise to this term, but at $d=5$, we can write the so-called Weinberg operator
%
\begin{equation}
 \mathscr{L}_{d=5} = \frac{c_{\alpha\beta}}{\Lambda} \left( \overline{L^c} \widetilde{H}^* \right) \left( \widetilde{H}^\dagger L  \right) \to  \frac{c_{\alpha\beta}}{\Lambda} \frac{v^2}{2} \overline{\nu_L^c}^\alpha \nu_L^\beta
\end{equation}
%
where we go from the unbroken to the broken phase. This operator is, in fact, the only $d=5$ operators allowed in the SM effective field theory (SMEFT). The fact that the lowest dimension non-renormalizable operator in the SMEFT gives neutrino masses is, yet again, an indication that neutrino masses point towards BSM physics. In this case, neutrinos turn out to be Majorana. Testing this hypothesis is extremely difficult because of the smallness of $m_\nu$, any process containing lepton number violation (LNV) will be suppressed by $m_\nu^2/E^2$, where $E$ is the typical energy involved. Another way to understand this is that any process where the operator above is important must be sensitive to the neutrino mass, which we have not measured so far. Curiously, neutrino oscillations are insensitive to the nature of neutrinos, that is, the oscillation probabilities are the same for Dirac or Majorana states. The most promising search for the Majorana nature of neutrinos is neutrinoless double-beta decay ($0\nu\beta\beta$), $(A,Z)\to(A,Z+2) + e^- + e^-$, which is only allowed if neutrinos are majorana particles. 

We may be satisfied to work with this low-energy description of neutrino masses, but ultimately, our goal is to understand the ultra-violet (UV) physics. To this end, we want to find all UV-completions to the Weinberg operator. As it turns out, there are a total of 3 three-level completions, which we show in \reffig{fig:seesaw_mechanisms}. Of course, neutrino masses need not be tree-level effects, and loop-induced mechanism for generating Dirac or Majorana neutrino masses are also important. These typically explain the smallness of neutrino masses through loop suppression factors, and we present a few in \refsec{sec:radiative}.
%
\begin{figure}[t]
\centering
\includegraphics[width=\textwidth]{seesaw_mechanisms.pdf}
\caption[The tree-level UV completions of the Weinberg operator.]{The only three UV completions of the $d=5$ Weinberg operator with their respective contributions to light neutrino masses.\label{fig:seesaw_mechanisms}}
\end{figure}

\subsection{Conventional seesaws}

\paragraph{Type-I}
\paragraph{Type-II}
\paragraph{Type-III}


\subsection{Low scale seesaw variants}



\subsection{Radiative masses}\label{sec:radiative}

The most straightforward extension of the SM which can generate neutrino masses at loop level is perhaps the so-called Zee-Babu model. For a review on low-scale models see \cite{Boucenna:2014zba}.
